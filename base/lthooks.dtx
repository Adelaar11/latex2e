% \iffalse meta-comment
%
%% File: lthooks.dtx (C) Copyright 2020 Frank Mittelbach, LaTeX Team
%
% It may be distributed and/or modified under the conditions of the
% LaTeX Project Public License (LPPL), either version 1.3c of this
% license or (at your option) any later version.  The latest version
% of this license is in the file
%
%    https://www.latex-project.org/lppl.txt
%
%
% The development version of the bundle can be found below
%
%    https://github.com/FrankMittelbach/...
%
% for those people who are interested or want to report an issue.
%
%    \begin{macrocode}
\providecommand\lthooksversion{v0.9a}
\providecommand\lthooksdate{2020/07/15}
%    \end{macrocode}
%
%<*driver>
\RequirePackage[debug]{lthooks}

\documentclass{l3doc}

% bug fix fo l3doc.cls
\ExplSyntaxOn
\cs_set_protected:Npn \__codedoc_macro_typeset_one:nN #1#2
  {
    \vbox_set:Nn \l__codedoc_macro_box
      {
        \vbox_unpack_drop:N \l__codedoc_macro_box
        \hbox { \llap { \__codedoc_print_macroname:nN {#1} #2
            \MacroFont       % <----- without it the \ is in lmr10 if a link is made
            \      
        } }
      }
    \int_incr:N \l__codedoc_macro_int
  }
\ExplSyntaxOff

\EnableCrossrefs
\CodelineIndex
\begin{document}
  \DocInput{lthooks.dtx}
\end{document}
%</driver>
%
% \fi
%
%
% \long\def\fmi#1{\begin{quote}\itshape FMi: #1\end{quote}}
% \long\def\pho#1{\begin{quote}\itshape PhO: #1\end{quote}}
%
% \newcommand\hook[1]{\texttt{#1}}
%    
%
% \title{The \texttt{lthooks} package\thanks{This package has version
%    \lthooksversion\ dated \lthooksdate, \copyright\ \LaTeX\
%    Project.}}
%
% \author{Frank Mittelbach\thanks{Code improvements for speed and other goodies by Phelype Oleinik}}
%
% \maketitle
%
%
% \tableofcontents
%
% \section{Introduction}
%
%    Hooks are points in the code of commands or environments where it
%    is possible to add processing code into existing commands. This
%    can be done by different packages that do not know about each
%    other and to allow for hopefully safe processing it is necessary
%    to sort different chunks of code added by different packages into
%    a suitable processing order.
%
%    This is done by the packages adding chunks of code (via
%    \cs{AddToHook}) and labeling their code with some label by
%    default using the package name as a label.
%
%    At \verb=\begin{document}= all code for a hook is then sorted
%    according to some rules (given by \cs{DeclareHookRule}) for fast
%    execution without processing overhead. If the hook code is
%    modified afterwards (or the rules are changed),
%    a new version for fast processing is generated.
%
%    Some hooks are used already in the preamble of the document. If
%    that happens then the hook is prepared for execution (and sorted)
%    already at that point.
%
%
% \section{Package writer interface}
%
%    The hook management system is offered as a set of CamelCase
%    commands for traditional \LaTeXe{} packages (and for use in the
%    document preamble if needed) as well as \texttt{expl3} commands
%    for modern packages, that use the L3 programming layer of
%    \LaTeX{}. Behind the scenes, a single set of data structures is
%    accessed so that packages from both worlds can coexist and access
%    hooks in other packages.
%
%
%
% \subsection{\LaTeXe\ interfaces}
%
% \subsubsection{Declaring hooks and using them in code}
%
%    With two exceptions, hooks have to be declared before they can be
%    used. The exceptions are hooks in environments (i.e., executed at
%    \cs{begin} and \cs{end}) and hooks run when loading files,
%    e.g. before and after a package is loaded, etc. Their hook names
%    depend on the environment or the file name and so declaring them
%    beforehand is difficult.
%
%
% \begin{function}{\NewHook}
%   \begin{syntax}
%     \cs{NewHook} \Arg{hook}
%   \end{syntax}
%   Creates a new \meta{hook}.
%    If this is a hook provided as part of a package it is suggested
%    that the \meta{hook} name is always structured as follows:
%    \meta{package-name}\texttt{/}\meta{hook-name}. If necessary you
%    can further subdivide the name by adding more \texttt{/} parts.
%    If a hook name is already taken, an error is raised and the hook
%    is not created.
%
%    The \meta{hook} can be specified using the dot-syntax to denote
%    the current package name. See section~\ref{sec:default-label}.
% \end{function}
%
% \begin{function}{\NewReversedHook}
%   \begin{syntax}
%     \cs{NewReversedHook} \Arg{hook}
%   \end{syntax}
%     Like \cs{NewHook} declares a new \meta{hook}.
%     the difference is that the code chunks for this hook are in
%     reverse order by default (those added last are executed first).
%     Any rules for the hook are applied after the default ordering.
%     See sections~\ref{sec:order} and \ref{sec:reversed-order}
%    for further details.
%
%    The \meta{hook} can be specified using the dot-syntax to denote
%    the current package name. See section~\ref{sec:default-label}.
% \end{function}
%
%
% \begin{function}{\NewMirroredHookPair}
%   \begin{syntax}
%     \cs{NewMirroredHookPair} \Arg{hook-1} \Arg{hook-2}
%   \end{syntax}
%     A shorthand for
%    \cs{NewHook}\Arg{hook-1}\cs{NewReversedHook}\Arg{hook-2}.
%
%    The \meta{hooks} can be specified using the dot-syntax to denote
%    the current package name. See section~\ref{sec:default-label}.
% \end{function}
%



% \begin{function}{\UseHook}
%   \begin{syntax}
%     \cs{UseHook} \Arg{hook}
%   \end{syntax}
%    Execute the hook code inside a command or environment.\footnote{For
%     legacy hooks such as \hook{begindocument} it is also
%    possible to call \cs{@...hook}, e.g., \cs{@begindocumenthook},
%    but this syntax is discouraged.}
%
%    Before \verb=\begin{document}= the fast execution code for a hook
%    is not set up, so in order to use a hook there it is explicitly
%    initialized first. As that involves assignments using a hook at
%    those times is not 100\% the same as using it after
%    \verb=\begin{document}=.
%
%    The \meta{hook} \emph{cannot} be specified using the dot-syntax.
%    A leading |.| is treated literally.
% \end{function}
%
% \begin{function}{\UseOneTimeHook}
%   \begin{syntax}
%     \cs{UseOneTimeHook} \Arg{hook}
%   \end{syntax}
%    Some hooks are only used (and can be only used) in one place, for
%    example, those in \verb=\begin{document}= or
%    \verb=\end{document}=. Once we have passed that point adding to
%    the hook through a defined \cs{\meta{addto-cmd}} command (e.g.,
%    \cs{AddToHook} or \cs{AtBeginDocument}, etc.\@) would have no
%    effect (as would the use of such a command inside the hook code
%    itself). It is therefore customary to redefine
%    \cs{\meta{addto-cmd}} to simply  process its argument, i.e.,
%    essentially make it behave like \cs{@firstofone}.
%
%    \cs{UseOneTimeHook} does that: it records that the hook has been
%    consumed and any further attempt to add to it will result in
%    executing the code to be added immediately.
%
%    \fmi{Maybe add an error version as well?}
%
%    The \meta{hook} \emph{cannot} be specified using the dot-syntax.
%    A leading |.| is treated literally.
% \end{function}
%
%
% \subsubsection{Updating code for hooks}
%
% \begin{function}{\AddToHook}
%   \begin{syntax}
%     \cs{AddToHook} \Arg{hook}\oarg{label}\Arg{code}
%   \end{syntax}
%    Adds \meta{code} to the \meta{hook} labeled by \meta{label}. If
%    the optional argument \meta{label} is not provided, if \cs{AddToHook}
%    is used in a package/class, then the current
%    package/class name is used, otherwise \hook{top-level} is
%    used~(see section~\ref{sec:default-label}).
%
%    If there already exists code under the \meta{label} then the new
%    \meta{code} is appended to the existing one (even if this is a reversed hook).
%    If you want to replace existing code under the
%    \meta{label}, first apply \cs{RemoveFromHook}.
%
%    The hook doesn't have to exist for code to be added to
%    it. However, if it is not declared later then obviously the
%    added \meta{code} will never be executed.  This
%    allows for hooks to work regardless of package loading order and
%    enables packages to add to hook of other packages without
%    worrying whether they are actually used in the current document.
%    See section~\ref{sec:querying}.
%
%    The \meta{hook} and \meta{label} can be specified using the
%    dot-syntax to denote the current package name.
%    See section~\ref{sec:default-label}.
% \end{function}
%
% \begin{function}{\RemoveFromHook}
%   \begin{syntax}
%     \cs{RemoveFromHook} \Arg{hook}\oarg{label}
%   \end{syntax}
%    Removes any code labeled by \meta{label} from the \meta{hook}.
%    If the optional
%    argument \meta{label} is not provided, if \cs{AddToHook}
%    is used in a package/class, then the current
%    package/class name is used, otherwise \hook{top-level} is used.
%
%    If the optional argument is \texttt{*}, then all code chunks are
%    removed. This is rather dangerous as it drops code from other
%    packages one may not know about!
%
%    The \meta{hook} and \meta{label} can be specified using the
%    dot-syntax to denote the current package name.
%    See section~\ref{sec:default-label}.
% \end{function}
%
% \medskip
%
% A useful application for this declaration inside the document body
% is when one wants to temporarily add code to hooks and later remove
% it again, e.g.,
%\begin{verbatim}
%   \AddToHook{env/quote/before}{\small}
%   \begin{quote}
%     A quote set in a smaller typeface
%   \end{quote}
%   ...
%   \RemoveFromHook{env/quote/before}
%   ... now back to normal for further quotes
%\end{verbatim}
% Note that you can't cancel the setting with
%\begin{verbatim}
%   \AddToHook{env/quote/before}{}
%\end{verbatim}
% because that only \enquote{adds} a further empty chunk of code to
% the hook. Adding \cs{normalsize} would work but that means the hook
% then contained \cs{small}\cs{normalsize} which means to font size
% changes for no good reason.
%
% The above is only needed if one wants to typeset several quotes in a
% smaller typeface. If the hook is only needed once then
% \cs{AddToHookNext} is simpler, because it resets itself after one use.
%
%
% \begin{function}{\AddToHookNext}
%   \begin{syntax}
%     \cs{AddToHookNext} \Arg{hook}\Arg{code}
%   \end{syntax}
%    Adds \meta{code} to the next invocation of the \meta{hook}.
%    The code is executed after the normal hook code has finished and
%    it is executed only once, i.e. it is deleted after it was used.
%
%    Using the declaration is a global operation, i.e., the code is
%    not lost, even if the declaration is used inside a group and the
%    next invocation happens after the group. If the declaration is
%    used several times before the hook is executed then all code is
%    executed in the order in which it was declared.\footnotemark
%
%    The hook doesn't have to exist for code to be added to it.  This
%    allows for hooks to work regardless of package loading order.
%    See section~\ref{sec:querying}.
%
%    The \meta{hook} can be specified using the dot-syntax to denote
%    the current package name.  See section~\ref{sec:default-label}.
% \end{function}\footnotetext{There is
%    no mechanism to reorder such code chunks (or delete them).}
%
% \subsubsection{Hook names and default labels}
% \label{sec:default-label}
%
% It is best practice to use \cs{AddToHook} in packages or classes
% \emph{without specifying a \meta{label}} because then the package
% or class name is automatically used, which is helpful if rules are
% needed, and avoids mistyping the \meta{label}.
%
% Using an explicit \meta{label} is only necessary in very specific
% situations, e.g., if you want to add several chunks of code into a
% single hook and have them placed in different parts of the hook
% (by providing some rules).
%
% The other case is when you develop a larger package with several
% sub-packages. In that case you may want to use the same
% \meta{label} throughout the sub-packages in order to avoid
% that the labels change if you internally reorganize your code.
%
% It is not enforced, but highly recommended that the hooks defined by
% a package, and the \meta{labels} used to add code to other hooks
% contain the package name to easily identify the source of the code
% chunk and to prevent clashes.  This should be the standard practice,
% so this hook management code provides a shortcut to refer to the
% current package in the name of a \meta{hook} and in a \meta{label}.
% If \meta{hook} name or \meta{label} consist just of a single dot
% (|.|), or starts with a dot followed by a slash (|./|) then the dot
% denotes the \meta{default label} (usually the current package or class
% name---see~\cs{DeclareDefaultHookLabel}).
% A \enquote{|.|} or \enquote{|./|} anywhere else in a \meta{hook} or in
% \meta{label} is treated literally and is not replaced.

% For example,
% inside the package \texttt{mypackage.sty}, the default label is
% \texttt{mypackage}, so the instructions:
% \begin{verbatim}
%   \NewHook   {./hook}
%   \AddToHook {./hook}[.]{code}     % Same as \AddToHook{./hook}{code}
%   \AddToHook {./hook}[./sub]{code}
%   \DeclareHookRule{begindocument}{.}{<}{babel}
%   \AddToHook {file/after/foo.tex}{code}
% \end{verbatim}
%    are equivalent to:
% \begin{verbatim}
%   \NewHook   {mypackage/hook}[mypackage]{code}
%   \AddToHook {mypackage/hook}[mypackage]{code}
%   \AddToHook {mypackage/hook}[mypackage/sub]{code}
%   \DeclareHookRule{begindocument}{mypackage}{<}{babel}
%   \AddToHook {file/after/foo.tex}{code}                  % unchanged
% \end{verbatim}
%
% The \meta{default label} is automatically set to the name of the
% current package or class (using \cs{@currname}).  If \cs{@currname}
% is not set (because the hook command is used outside of a package, or
% the current file wasn't loaded with \cs{usepackage} or
% \cs{documentclass}), then the \texttt{top-level} is used as the
% \meta{default label}.
%
% This syntax is available in all \meta{label} arguments and most
% \meta{hook}, both in the \LaTeXe{} interface, and the \LaTeX3
% interface described in section~\ref{sec:l3hook-interface}.
%
% Note, however, that the replacement of |.| by the \meta{default label}
% takes place when the hook command is executed, so actions that are
% somehow executed after the package ends will have the wrong
% \meta{default label} if the dot-syntax is used.  For that reason,
% this syntax is not available in \cs{UseHook} (and \cs{hook_use:n})
% because the hook is most of the time used outside of the package file
% in which it was defined. This syntax is also not available in the hook
% conditionals \cs{IfHookEmptyTF} (and \cs{hook_if_empty:nTF}) and
% \cs{IfHookExistTF} (and \cs{hook_if_exist:nTF}) because these
% conditionals are used in some performance-critical parts of the hook
% management code, and because they are usually used to refer to other
% package's hooks, so the dot-syntax doesn't make much sense.
%
% In some cases, for example in large packages, one may want to separate
% it in logical parts, but still use the main package name as
% \meta{label}, then the \meta{default label} can be set using
% \cs{DeclareDefaultHookLabel}:
%
% \begin{function}{\DeclareDefaultHookLabel}
%   \begin{syntax}
%     \cs{DeclareDefaultHookLabel} \Arg{default label}
%   \end{syntax}
%   Sets the \meta{default label} to be used in \meta{label} arguments.
%   If \cs{DeclareDefaultHookLabel} is not used in the
%   current package, \cs{@currname} is used instead.  If \cs{@currname}
%   is not set, the code is assumed to be in the main document, in which
%   case \texttt{top-level} is used.
%
%   The effect of \cs{DeclareDefaultHookLabel} holds for the current
%   file, and is reset to the previous value when the file is closed.
% \end{function}
%
%
% \subsubsection{Defining relations between hook code}
%
% The default assumption is that code added to hooks by different
% packages is independent and the order in which it is executed is
% irrelevant. While this is true in many case it is  obviously false
% in many others.
%
% Before the hook management system was introduced
% packages had to take elaborate precaution to determine of some other
% package got loaded as well (before or after) and find some ways to
% alter its behavior accordingly. In addition is was often the user's
% responsibility to load packages in the right order so that code
% added to hooks got added in the right order and some cases even
% altering the loading order wouldn't resolve the conflicts.
%
% With the new hook management system it is now possible to define
% rules (i.e., relationships) between code chunks added by different
% packages and explicitly describe in which order they should be
% processed.
%
% \begin{function}{\DeclareHookRule}
%   \begin{syntax}
%     \cs{DeclareHookRule} \Arg{hook}\Arg{label1}\Arg{relation}\Arg{label2}
%   \end{syntax}
%    Defines a relation between \meta{label1} and \meta{label2} for a
%    given \meta{hook}. If \meta{hook} is \texttt{??} this defines a
%    relation for all hooks that use the two labels, i.e., that have
%    chunks of code labeled with \meta{label1} and \meta{label2}.
%    Rules specific to a given hook take precedence over default
%    rules that use \texttt{??} as the \meta{hook}.
%
%    Currently, the supported relations are the following:
%    \begin{itemize}
%
%    \item[\texttt{before} or \texttt{\string<}]
%
%      Code for \meta{label1} comes before code for \meta{label2}.
%
%    \item[\texttt{after} or \texttt{\string>}]
%      Code for \meta{label1} comes after code for \meta{label2}.
%
%    \item[\texttt{incompatible-warning}]
%
%      Only code for either \meta{label1} or \meta{label2} can appear
%      for that hook (a way to say that two packages---or parts of
%      them---are incompatible). A warning is raised if both labels
%      appear in the same hook.
%
%    \item[\texttt{incompatible-error}]
%
%      Like \texttt{incompatible-error} but instead of a warning a
%      \LaTeX{} error is raised, and the code for both labels are
%      dropped from that hook until the conflict is resolved.
%
%    \item[\texttt{removes}]
%
%      Code for \meta{label1} overwrites code for \meta{label2}. More
%      precisely, code for \meta{label2} is dropped for that
%      hook. This can be used, for example if one package is a
%      superset in functionality of another one and therefore wants to
%      undo code in some hook and replace it with its own version.
%
%    \item[\texttt{unrelated}]
%
%       The order of code for \meta{label1} and \meta{label2} is
%      irrelevant. This rule is there to undo an incorrect rule
%      specified earlier.
%
%    \end{itemize}
%
%    The \meta{hook} and \meta{label} can be specified using the
%    dot-syntax to denote the current package name.
%    See section~\ref{sec:default-label}.
% \end{function}
%
%
% \begin{function}{\ClearHookRule}
%   \begin{syntax}
%     \cs{ClearHookRule}\Arg{hook}\Arg{label1}\Arg{label2}
%   \end{syntax}
%    Syntactic sugar for saying that \meta{label1} and \meta{label2}
%    are unrelated for the given \meta{hook}.
% \end{function}
%
%
%
% \begin{function}{\DeclareDefaultHookRule}
%   \begin{syntax}
%     \cs{DeclareDefaultHookRule}\Arg{label1}\Arg{relation}\Arg{label2}
%   \end{syntax}
%   This sets up a relation between \meta{label1} and \meta{label2}
%    for all hooks unless overwritten by a specific rule for a hook.
%    Useful for cases where one package has a specific relation to
%    some other package, e.g., is \texttt{incompatible} or always
%    needs a special ordering \texttt{before} or \texttt{after}.
%    (Technically it is just a shorthand for using \cs{DeclareHookRule}
%    with \texttt{??} as the hook name.)
%
%    Declaring default rules is only supported in the document
%    preamble.\footnotemark{}
%
%    The \meta{label} can be specified using the dot-syntax to denote
%    the current package name. See section~\ref{sec:default-label}.
% \end{function}\footnotetext{Trying to do so, e.g., via
%    \cs{DeclareHookRule} with \texttt{??}  has bad side-effects and
%    is not supported (though not explicitly caught for performance
%    reasons).}
%
%
%
% \subsubsection{Querying hooks}
% \label{sec:querying}
%
% Simpler data types, like token lists, have three possible states; they
% can:
% \begin{itemize}
%   \item exist and be empty;
%   \item exist and be non-empty; and
%   \item not exist (in which case emptiness doesn't apply);
% \end{itemize}
% Hooks are a bit more complicated: they have four possible states.
% A hook may exist or not, and either way it may or may not be empty.
% This means that even a hook that doesn't exist may be non-empty.
%
% This seemingly strange state may happen when, for example, package~$A$
% defines hook \hook{A/foo}, and package $B$ adds some code to that
% hook.  However, a document may load package $B$ before package $A$, or
% may not load package $A$ at all.  In both cases some code is added to
% hook \hook{A/foo} without that hook being defined yet, thus that
% hook is said to be non-empty, whereas it doesn't exist.  Therefore,
% querying the existence of a hook doesn't imply its emptiness, neither
% does the other way around.
%
% A hook is said to be empty when no code was added to it, either to
% its permanent code pool, or to its ``next'' token list.  The hook
% doesn't need to be declared to have code added to its code pool.
% A hook is said to exist when it was declared with \cs{NewHook} or
% some variant thereof.
%
% \begin{function}[EXP]{\IfHookEmptyTF}
%   \begin{syntax}
%     \cs{IfHookEmptyTF} \Arg{hook} \Arg{true code} \Arg{false code}
%   \end{syntax}
%   Tests if the \meta{hook} is empty (\emph{i.e.}, no code was added to
%   it using either \cs{AddToHook} or \cs{AddToHookNext}), and
%   branches to either \meta{true code} or \meta{false code} depending
%   on the result.
%
%    The \meta{hook} \emph{cannot} be specified using the dot-syntax.
%    A leading |.| is treated literally.
% \end{function}
%
% \begin{function}[EXP]{\IfHookExistTF}
%   \begin{syntax}
%     \cs{IfHookExistTF} \Arg{hook} \Arg{true code} \Arg{false code}
%   \end{syntax}
%   Tests if the \meta{hook} exists (if it was created with either
%   \cs{NewHook}, \cs{NewReversedHook}, or \cs{NewMirroredHookPair}), and
%   branches to either \meta{true code} or \meta{false code} depending
%   on the result.
%
%   The existence of a hook usually doesn't mean much from the viewpoint
%   of code that tries to add/remove code from that hook, since package
%   loading order may vary, thus the creation of hooks is asynchronous
%   to adding and removing code from it, so this test should be used
%   sparingly.
%
%    The \meta{hook} \emph{cannot} be specified using the dot-syntax.
%    A leading |.| is treated literally.
% \end{function}
%
% \fmi{Would be helpful if we provide some use cases}
%
% \subsubsection{Displaying hook code}
%
%    If one has to adjust the code execution in a hook using a hook
%    rule it is helpful to get some information about the code
%    associated with a hook, its current order and the existing rules.
%
% \begin{function}{\ShowHook}
%   \begin{syntax}
%     \cs{ShowHook} \Arg{hook}
%   \end{syntax}
%   Displays information about the \meta{hook} such as
%   \begin{itemize}
%   \item
%      the code chunks (and their labels) added to it,
%   \item
%      any rules set up to order them,
%      \fmi{currently this is missing the default rules that apply,
%      guess that needs fixing}
%   \item
%      the computed order (if already defined),
%   \item
%      any code executed on the next invocation only.
%   \end{itemize}
%
%    The \meta{hook} can be specified using the dot-syntax to denote
%    the current package name. See section~\ref{sec:default-label}.
% \end{function}
%
%
% \subsubsection{Debugging hook code}
%
% \begin{function}{\DebugHookOn,\DebugHookOff}
%   \begin{syntax}
%     \cs{DebugHookOn}
%   \end{syntax}
%    Turn the debugging of hook code on or off. This displays changes
%    made to the hook data structures. The output is rather coarse and
%      not really intended for normal use.
% \end{function}
%
%
% \subsection{L3 programming layer (\texttt{expl3}) interfaces}
% \label{sec:l3hook-interface}
%
%
% This is a quick summary of the \LaTeX3 programming interfaces for
% use with packages written in \texttt{expl3}. In contrast to the
% \LaTeXe{} interfaces they always use mandatory arguments only, e.g.,
% you always have to specify the \meta{label} for a code chunk.  We
% therefore suggest to use the declarations discussed in the previous
% section even in \texttt{expl3} packages, but the choice is yours.
%
%
% \begin{function}
%   {\hook_new:n,\hook_new_reversed:n,\hook_new_pair:nn}
%   \begin{syntax}
%     \cs{hook_new:n}\Arg{hook}
%     \cs{hook_new_pair:nn}\Arg{hook-1}\Arg{hook-2}
%   \end{syntax}
%   Creates a new \meta{hook} with normal or reverse ordering of code
%    chunks. \cs{hook_new_pair:nn} creates a pair of such hooks with
%    \Arg{hook-2} being a reversed hook.
%    If a hook name is already taken, an error is raised and the hook
%    is not created.
%
%    The \meta{hook} can be specified using the dot-syntax to denote
%    the current package name. See section~\ref{sec:default-label}.
% \end{function}
%
%
%
% \begin{function}{\hook_use:n}
%   \begin{syntax}
%     \cs{hook_use:n} \Arg{hook}
%   \end{syntax}
%    Executes the \Arg{hook} code followed (if set up) by the code for next
%    invocation only, then empties that next invocation code.
%
%    The \meta{hook} \emph{cannot} be specified using the dot-syntax.
%    A leading |.| is treated literally.
% \end{function}
%
% \begin{function}{\hook_use_once:n}
%   \begin{syntax}
%     \cs{hook_use_once:n} \Arg{hook}
%   \end{syntax}
%     Changes the \Arg{hook} status so that from now on any addition to
%     the hook code is executed immediately. Then execute any
%     \Arg{hook} code already set up.
%    \fmi{better L3 name?}
%
%    The \meta{hook} \emph{cannot} be specified using the dot-syntax.
%    A leading |.| is treated literally.
% \end{function}
%
% \begin{function}{\hook_gput_code:nnn}
%   \begin{syntax}
%     \cs{hook_gput_code:nnn} \Arg{hook} \Arg{label} \Arg{code}
%   \end{syntax}
%    Adds a chunk of \meta{code} to the \meta{hook} labeled
%    \meta{label}. If the label already exists the \meta{code} is
%    appended to the already existing code.
%
%    If code is added to an external \meta{hook} (of the kernel or
%    another package) then the convention is to use the package name
%    as the \meta{label} not some internal module name or some other
%    arbitrary string.
%
%    The \meta{hook} and \meta{label} can be specified using the
%    dot-syntax to denote the current package name.
%    See section~\ref{sec:default-label}.
% \end{function}
%
% \begin{function}
%   {\hook_gput_next_code:nn}
%   \begin{syntax}
%     \cs{hook_gput_next_code:nn} \Arg{hook} \Arg{code}
%   \end{syntax}
%    Adds a chunk of \meta{code} for use only in the next invocation of the
%    \meta{hook}. Once used it is gone.
%
%    This is simpler than \cs{hook_gput_code:nnn}, the code is simply
%    appended to the hook in the order of declaration at the very end,
%    i.e., after all standard code for the hook got executed.
%
%    Thus if one needs to undo what the standard does one has to do
%    that as part of \meta{code}.
%
%    The \meta{hook} can be specified using the dot-syntax to denote
%    the current package name. See section~\ref{sec:default-label}.
% \end{function}
%


%  \begin{function}{\hook_gremove_code:nn}
%   \begin{syntax}
%     \cs{hook_gremove_code:nn} \Arg{hook} \Arg{label}
%   \end{syntax}
%    Removes any code for \meta{hook} labeled \meta{label}.
%
%    If the code for that \meta{label} wasn't yet added to the
%    \meta{hook}, an order is set so that when some code attempts to add
%    that label, the removal order takes action and the code is not
%    added.
%
%    If the second argument is \texttt{*}, then all code chunks are
%    removed. This is rather dangerous as it drops code from other
%    packages one may not know about, so think twice before using
%    that!
%
%    The \meta{hook} and \meta{label} can be specified using the
%    dot-syntax to denote the current package name.
%    See section~\ref{sec:default-label}.
% \end{function}


%  \begin{function}{\hook_gset_rule:nnnn}
%   \begin{syntax}
%     \cs{hook_gset_rule:nnnn} \Arg{hook} \Arg{label1} \Arg{relation} \Arg{label2}
%   \end{syntax}
%    Relate \meta{label1} with \meta{label2} when used in \meta{hook}.
%    See \cs{DeclareHookRule} for the allowed \meta{relation}s.
%    If \meta{hook} is \texttt{??} a default rule is specified.
%
%    The \meta{hook} and \meta{label} can be specified using the
%    dot-syntax to denote the current package name.
%    See section~\ref{sec:default-label}.
%    The dot-syntax is parsed in both \meta{label} arguments, but it
%    usually makes sense to be used in only one of them.
% \end{function}
%
% \begin{function}[pTF]{\hook_if_empty:n}
%   \begin{syntax}
%     \cs{hook_if_empty:n} \Arg{hook} \Arg{true code} \Arg{false code}
%   \end{syntax}
%   Tests if the \meta{hook} is empty (\emph{i.e.}, no code was added to
%   it using either \cs{AddToHook} or \cs{AddToHookNext}), and
%   branches to either \meta{true code} or \meta{false code} depending
%   on the result.
%
%    The \meta{hook} \emph{cannot} be specified using the dot-syntax.
%    A leading |.| is treated literally.
% \end{function}
%
% \begin{function}[pTF]{\hook_if_exist:n}
%   \begin{syntax}
%     \cs{hook_if_exist:n} \Arg{hook} \Arg{true code} \Arg{false code}
%   \end{syntax}
%   Tests if the \meta{hook} exists (if it was created with either
%   \cs{NewHook}, \cs{NewReversedHook}, or \cs{NewMirroredHookPair}), and
%   branches to either \meta{true code} or \meta{false code} depending
%   on the result.
%
%  \fmi{what are the results for generic hooks that do not need to be declared?}
%
%   The existence of a hook usually doesn't mean much from the viewpoint
%   of code that tries to add/remove code from that hook, since package
%   loading order may vary, thus the creation of hooks is asynchronous
%   to adding and removing code from it, so this test should be used
%   sparingly.
%
%    The \meta{hook} \emph{cannot} be specified using the dot-syntax.
%    A leading |.| is treated literally.
% \end{function}
%
%
% \begin{function}{\hook_debug_on:,\hook_debug_off:}
%   \begin{syntax}
%     \cs{hook_debug_on:}
%   \end{syntax}
%    Turns the debugging of hook code on or off. This displays changes
%    to the hook data.
% \end{function}
%
%
%
% \subsection{On the order of hook code execution} \label{sec:order}
%
%    Chunks of code for a \meta{hook} under different labels are supposed
%    to be independent if there are no special rules set up that
%    define a relation between the chunks. This means that you can't
%    make assumptions about the order of execution!
%
%    Suppose you have the following declarations:
%\begin{verbatim}
%    \NewHook{myhook}
%    \AddToHook{myhook}[packageA]{\typeout{A}}
%    \AddToHook{myhook}[packageB]{\typeout{B}}
%    \AddToHook{myhook}[packageC]{\typeout{C}}
%\end{verbatim}
%    then executing the hook with \cs{UseHook} will produce the
%    typeout \texttt{A} \texttt{B} \texttt{C} in that order.  In other
%    words, the execution order is computed to be \texttt{packageA},
%    \texttt{packageB}, \texttt{packageC} which you can verify with
%    \cs{ShowHook}\texttt{\{myhook\}}:
%\begin{verbatim}
%   The hook 'myhook':
%    Code chunks:
%       packageA -> \typeout {A}
%       packageB -> \typeout {B}
%       packageC -> \typeout {C}
%    Extra code next invocation:
%       ---
%    Rules:
%       ---
%    Execution order:
%       packageA, packageB, packageC
%\end{verbatim}
%    The reason is that the code chunks are internally saved in a property list
%    and the initial order of such a property list is the order in
%    which key-value pairs got added. However, that is only true if
%    nothing other than adding happens!
%
%    Suppose, or example, you want to replace the code chunk for
%    \texttt{packageA}, e.g.,
%\begin{verbatim}
%    \RemoveFromHook{myhook}[packageA]
%    \AddToHook{myhook}[packageA]{\typeout{A alt}}
%\end{verbatim}
%    then your order becomes  \texttt{packageB},
%    \texttt{packageC}, \texttt{packageA} because the label got removed
%    from the property list and then re-added (at its end).
%
%    While that may not be too surprising,  the execution order is
%    also sometimes altered if you add a redundant rule, e.g. if you specify
%\begin{verbatim}
%    \DeclareHookRule{myhook}{packageA}{before}{packageB}
%\end{verbatim}
%    instead of the previous lines we get
%\begin{verbatim}
%   The hook 'myhook':
%    Code chunks:
%       packageA -> \typeout {A}
%       packageB -> \typeout {B}
%       packageC -> \typeout {C}
%    Extra code next invocation:
%       ---
%    Rules:
%       packageA|packageB with relation before
%    Execution order (after applying rules):
%       packageA, packageC, packageB
%\end{verbatim}
%    As you can see the code chunks are still in the same order, but
%    in the execution order for the labels \texttt{packageB} and
%    \texttt{packageC} have
%    swapped places.
%    The reason is that, with the rule there are two orders that
%    satisfy it, and the algorithm for sorting happened to pick a
%    different one compared to the case without rules (where it
%    doesn't run at all as there is nothing to resolve).
%    Incidentally, if we had instead specified the redundant rule
%\begin{verbatim}
%    \DeclareHookRule{myhook}{packageB}{before}{label-3}
%\end{verbatim}
%    the execution order would not have changed.
%
%    In summary: it is not possible to rely on the order of execution
%    unless there are rules that partially or fully define the order
%    (in which you can rely on them being fulfilled).
%
%
% \subsection{The use of \enquote{reversed} hooks} \label{sec:reversed-order}
%
%    You may have wondered why you can declare a \enquote{reversed} hook
%    with \cs{NewReversedHook} and what that does exactly.
%
%    In short: the execution order of a reversed hook (without any
%    rules!) is exactly reversed to the order you would have gotten for
%    a hook declared with \cs{NewHook}.
%
%    This is helpful if you have a pair of hooks where you expect to see
%    code added that involves grouping, e.g., starting an environment
%    in the first and closing that environment in the second hook.
%    To give a somewhat contrived example\footnote{there are simpler
%    ways to achieve the same effect.}, suppose there is a package
%    adding the following:
%\begin{verbatim}
%    \AddToHook{env/quote/before}[package-1]{\begin{itshape}}
%    \AddToHook{env/quote/after} [package-1]{\end{itshape}}
%\end{verbatim}
%    As a result, all quotes will be in italics.
%    Now suppose further that the user wants the quotes also in blue
%    and therefore adds:
%\begin{verbatim}
%    \usepackage{color}
%    \AddToHook{env/quote/before}{\begin{color}{blue}}
%    \AddToHook{env/quote/after} {\end{color}}
%\end{verbatim}
%    Now if the \hook{env/quote/after} hook would be a normal hook we
%    would get the same execution order in  both hooks, namely:
%\begin{verbatim}
%    package-1, top-level
%\end{verbatim}
%    (or vice versa) and as a result, would get:
%\begin{verbatim}
%    \begin{itshape}\begin{color}{blue} ...
%    \end{itshape}\end{color}
%\end{verbatim}
%   and an error message that \verb=\begin{color}= ended by
%    \verb=\end{itshape}=.
%    With \hook{env/quote/after} declared as a reversed hook the
%    execution order is reversed and so all environments are closed in
%    the correct sequence and \cs{ShowHook} would give us the
%    following output:
%\begin{verbatim}
%   The hook 'env/quote/after':
%    Code chunks:
%       package-1 -> \end {itshape}
%       top-level -> \end {color}
%    Extra code next invocation:
%       ---
%    Rules:
%       ---
%    Execution order (after reversal):
%       top-level, package-1
%\end{verbatim}
%
%    The reversal of the execution order happens before applying any
%    rules, so if you alter the order you will probably have to alter
%    it in both hooks, not just in one, but that depends on the use case.
%
%
%
%
% \subsection{Private \LaTeX{} kernel hooks}
%
%    There are a few places where it is absolutely essential for
%    \LaTeX{} to function correctly that code is executed in a precisely
%    defined order. Even that could have been implemented with the
%    hook management (by adding various rules to ensure the
%    appropriate ordering with respect to other code added by
%    packages). However, this makes every document unnecessary
%    slow, because there has to be sorting even through the result is
%    predetermined. Furthermore it forces package writers to
%    unnecessarily add such rules if they add further code to the hook
%    (or break \LaTeX{}).
%
%    For that reason such code is not using the hook management, but
%    instead private kernel commands directly before or after a public
%    hook with the following naming
%    convention: \cs{@kernel@before@\meta{hookname}} or
%    \cs{@kernel@after@\meta{hookname}}. For example, in
%    \cs{enddocument} you find
%\begin{verbatim}
%   \UseHook{enddocument}%
%   \@kernel@after@enddocument
%\end{verbatim}
%    which means first the user/package-accessible \hook{enddocument}
%    hook is executed and then the internal kernel hook. As their name
%    indicates these kernel commands should not be altered by third-party
%    packages, so please refrain from that in the interest of
%    stability and instead use the public hook next to it.\footnote{As
%    with everything in \TeX{} there is not enforcement of this rule,
%    and by looking at the code it is easy to find out how the kernel
%    adds to them. The main reason of this section is therefore to say
%    \enquote{please don't do that, this is unconfigurable code!}}
%
%
%
% \subsection{Legacy \LaTeXe{} interfaces}
%
%  \LaTeXe{} offered a small number of hooks together with commands to
%    add to them. They are listed here and are retained for backwards
%    compatibility.
%
%  With the new hook management several additional hooks have been added
%    to \LaTeX\ and more will follow. See the next section for what
%    is already available.
%
%
% \begin{function}{\AtBeginDocument}
%   \begin{syntax}
%     \cs{AtBeginDocument} \oarg{label} \Arg{code}
%   \end{syntax}
%   If used without the optional argument \meta{label}, it works essentially
%    like before, i.e., it is adding \meta{code} to the hook
%    \hook{begindocument} 
%    (which is executed inside \verb=\begin{document}=).
%    However, all code added this way is labeled with the label
%    \hook{top-level} if done outside of a package or class or with the
%    package/class name if called inside such a file.
%
%    This way one can add further code to the hook using
%    \cs{AddToHook} or \cs{AtBeginDocument} using a different label
%    and explicitly order the code chunks as necessary, e.g., run some
%    code before or after the \hook{top-level} code.  When using the
%    optional argument the call is equivalent to running
%    \cs{AddToHook} \texttt{\{begindocument\}} \oarg{label}
%    \Arg{code}.
%
%    For important packages with known order requirement we may over
%    time add rules to the kernel (or to those packages) so that they
%    work regardless of the loading-order in the document.
% \end{function}
%
% \begin{function}{\AtEndDocument}
%   \begin{syntax}
%     \cs{AtEndDocument} \oarg{label} \Arg{code}
%   \end{syntax}
%   Like \cs{AtBeginDocument} but for the \hook{enddocument} hook.
% \end{function}
%
%
%
% \begin{function}{\AtBeginDvi}
%   \begin{syntax}
%     \cs{AtBeginDvi} \oarg{label} \Arg{code}
%   \end{syntax}
%   This hook is discussed in conjunction with the shipout hooks.
% \end{function}
%
%
%
% \subsection{\LaTeXe{} commands and environments augmented by
%    hooks}
%
%  \emph{intro to be written}
%
% \subsubsection{Generic hooks for all environments}
%
%    Every environment \meta{env} has now four associated hooks coming
%    with it:
%    \begin{description}
%    \item[\hook{env/\meta{env}/before}]
%
%       This hook is executed as part of \cs{begin} as the very first
%       action, in particular prior to starting the environment group.
%       Its scope is therefore not restricted by the environment.
%
%    \item[\hook{env/\meta{env}/begin}]
%
%       This hook is executed as part of \cs{begin} directly in front
%       of the code specific to the environment start (e.g., the
%       second argument of \cs{newenvironment}).  Its scope is the
%       environment body.
%
%    \item[\hook{env/\meta{env}/end}]
%
%       This is executed as part of \cs{end} directly in front of the
%       code specific to the end of the environment (e.g., the third
%       argument of \cs{newenvironment}).
%
%    \item[\hook{env/\meta{env}/after}]
%
%       This is executed as part of \cs{end} after the
%       code specific to the environment end and after the environment
%       group has ended.
%       Its scope is therefore not restricted by the environment.
%
%       This hook is implemented as a reversed hook so if two packages
%       add code to \hook{env/\meta{env}/before} and to
%       \hook{env/\meta{env}/after} they can add surrounding
%       environments and the order of closing them happens in the
%       right sequence.
%
%    \end{description}
%    In contrast to other hooks these hooks do not need to be declared
%    using \cs{NewHook}.
%
%    The hooks are only executed if \cs{begin}\Arg{env} and
%    \cs{end}\Arg{env} is used. If the environment code is executed
%    via low-level calls to \cs{\meta{env}} and \cs{end\meta{env}}
%    (e.g., to avoid the environment grouping) they are not
%    available. If you want them available in code using this method,
%    you would need to add them yourself, i.e., write something like
%\begin{verbatim}
%  \UseHook{env/quote/before}\quote
%      ...
%  \endquote\UseHook{env/quote/after}
%\end{verbatim}
%    to add the outer hooks, etc.
%
%
%
% \subsubsection{Hooks provided by \cs{begin}\texttt{\{document\}}}
%
%    Until 2020 \cs{begin}\texttt{\{document\}} offered exactly one
%    hook that one had to fill using \cs{AtBeginDocument}. Experience
%    has shown that this single hook in one place was not enough and
%    as part of adding the general hook management system a number of
%    additional hooks have been added at this point. The places for
%    hooks have been chosen to provide the same support as offered by
%    external packages, such as \pkg{etoolbox} and others that
%    augmented \cs{document} to gain better control.
%
%    Supported are now the following hooks:
%    \begin{description}
%
%    \item[\hook{env/document/before}]
%
%      This is the generic environment hook executed effectively
%      before \verb=\begin{document}= starts, i.e., one can think of
%      it as a hook for code at the end of the preamble section.
%
%    \item[\hook{env/document/begin}]
%
%      This is the second generic environment hook that is executed
%      after the environment has started its group. But given that for the
%      \texttt{document} environment this group is canceled there is
%      little difference to the previous one as the two are directly
%      executed one after another (the only difference is that in this
%      hook \cs{@currenvir} is now set to \texttt{document} but
%      anybody adding to this hook would know that already).
%
%
%    \item[\hook{begindocument}]
%
%      This hook is added to by \cs{AtBeginDocument} and is executed
%      after the \texttt{.aux} file as be read in and most
%      initialization are done, so they can be altered and inspected by
%      the hook code. It is followed by a small number of further
%      initializations that shouldn't be altered and are therefore
%      coming later.
%
%    \item[\hook{begindocument/end}]
%
%      This hook is executed at the end of the \cs{document} code in
%      other words at the beginning of the document body. The only
%      command that follows it is \cs{ignorespaces}.
%
%    \end{description}
%
%
%
%
% \subsubsection{Hooks provided by \cs{end}\texttt{\{document\}}}
%
%    \LaTeXe{} always provided \cs{AtEndDocument} to add code to the
%    execution of \verb=\end{document}= just in front of the code that
%    is normally executed there. While this was a big improvement over
%    the situation in \LaTeX\,2.09 it was not flexible enough for a
%    number of use cases and so packages, such as \pkg{etoolbox},
%    \pkg{atveryend} and others patched \cs{enddocument} to add
%    additional points where code could be hooked into.
%
%    Patching using packages is always problematical as leads to
%    conflicts (code availability, ordering of patches, incompatible
%    patches, etc.).  For this reason a number of additional hooks
%    have been added to the \cs{enddocument} code to allow packages
%    to add code in various places in a controlled way without the
%    need for overwriting or patching the core code.
%
%    Supported are now the following hooks:
%    \begin{description}
%
%    \item[\hook{env/document/end}] The generic hook inside \cs{end}.
%
%    \item[\hook{enddocument}]

%      The hook associated with \cs{AtEndDocument}. It is immediately
%      called after the previous hook so there could be just
%      one.\footnote{We could make \cs{AtEndDocument} just fill the
%      \hook{env/document/end} but maybe that is a bit confusing.}
%
%      When this hook is executed there may be still unprocessed
%      material (e.g., floats on the deferlist) and the hook may add
%      further material to be typeset. After it, \cs{clearpage} is
%      called to ensure that all such material gets typeset. If there
%      is nothing waiting the \cs{clearpage} has no effect.
%
%
%    \item[\hook{enddocument/afterlastpage}]
%
%      As the name indicates this hook should not receive code that
%      generates material for further pages. It is the right place to
%      do some final housekeeping and possibly write out some
%      information to the \texttt{.aux} file (which is still open at
%      this point to receive data). It is also the correct place to
%      set up any testing code to be run when the \texttt{.aux} file
%      is re-read in the next step.
%
%
%      After this hook has been executed the \texttt{.aux} file is
%      closed for writing and then read back in to do some tests
%      (e.g., looking for missing references or duplicated labels, etc.).
%
%    \item[\hook{enddocument/afteraux}]
%
%      At this point, the \texttt{.aux} file has been reprocessed and so
%      this is a possible place for final checks and display of
%      information to the user. However, for the latter you might
%      prefer the next hook, so that your information is displayed after the
%      (possibly longish) list of files if that got requested via \cs{listfiles}.
%
%    \item[\hook{enddocument/info}]
%
%      This hook is meant to receive code that write final information
%      messages to the terminal. It follows immediately after the
%      previous hook (so both could have been combined, but then
%      packages adding further code would always need to also supply
%      an explicit rule to specify where it should go.
%
%      This hook already contains some code added by the kernel (under
%      the labels \texttt{kernel/filelist} and
%      \texttt{kernel/warnings}), namely the list of files when
%      \cs{listfiles} has been used and the warnings for duplicate
%      labels, missing references, font substitutions etc.
%
%    \item[\hook{enddocument/end}]
%
%      Finally, this hook is executed just in front of the final call
%      to \cs{@{}@end}.
%
%    \end{description}
%
%
%    There is also the hook \hook{shipout/lastpage}. This hook is
%    executed as part of the last \cs{shipout} in the document to
%    allow package to add final \cs{special}'s to that page. Where
%    this hook is executed in relation to those from the above list
%    can vary from document to document. Furthermore to determine correctly
%    which of the \cs{shipout}s is the last one, \LaTeX{} needs to be run
%    several times, so initially it might get executed on the wrong
%    page. See section~\ref{sec:shipout} for where to find the details.
%
%
% \subsubsection{Hooks provided \cs{shipout} operations}
% \label{sec:shipout}
%
%    There are several hooks and mechanisms added to \LaTeX{}'s
%    process of generating pages. These are documented in
%    \texttt{ltshipout-doc.pdf} or with code in
%    \texttt{ltshipout-code.pdf}.
%
%
% \subsubsection{Hooks provided file loading operations}
%
%    There are several hooks added to \LaTeX{}'s
%    process of loading file via its high-level interfaces such as
%    \cs{input}, \cs{include}, \cs{usepackage}, etc. These are documented in
%    \texttt{ltfilehook-doc.pdf} or with code in
%    \texttt{ltfilehook-code.pdf}.
%
%

% \StopEventually{\setlength\IndexMin{200pt}  \PrintIndex  }
%
%
% \section{The Implementation}
%    \begin{macrocode}
%<@@=hook>
%    \end{macrocode}
%
%    \begin{macrocode}
%<*2ekernel>
%    \end{macrocode}
%    
%    \begin{macrocode}
\ExplSyntaxOn
%    \end{macrocode}
%
%
%  \subsection{Debugging}
%
%  \begin{macro}{\g_@@_debug_bool}
%    Holds the current debugging state.
%    \begin{macrocode}
\bool_new:N \g_@@_debug_bool
%    \end{macrocode}
%  \end{macro}
%
%  \begin{macro}{\hook_debug_on:,\hook_debug_off:}
%  \begin{macro}{\@@_debug:n}
%  \begin{macro}{\@@_debug_gset:}
%    Turns debugging on and off by redefining \cs{@@_debug:n}.
%    \begin{macrocode}
\cs_new_eq:NN \@@_debug:n \use_none:n
\cs_new_protected:Npn \hook_debug_on:
  {
    \bool_gset_true:N \g_@@_debug_bool
    \@@_debug_gset:
  }
\cs_new_protected:Npn \hook_debug_off:
  {
    \bool_gset_false:N \g_@@_debug_bool
    \@@_debug_gset:
  }
\cs_new_protected:Npn \@@_debug_gset:
  {
    \cs_gset_protected:Npx \@@_debug:n ##1
      { \bool_if:NT \g_@@_debug_bool {##1} }
  }
%    \end{macrocode}
%  \end{macro}
%  \end{macro}
%  \end{macro}

%
%
%  \subsection{Borrowing from internals of other kernel modules}
%
%
% \begin{macro}[EXP]{\@@_str_compare:nn}
%   Private copy of \cs{__str_if_eq:nn}
%    \begin{macrocode}
\cs_new_eq:NN \@@_str_compare:nn \__str_if_eq:nn
%    \end{macrocode}
% \end{macro}
%
%  \subsection{Declarations}
%
%  \begin{macro}{\l_@@_return_tl,\l_@@_tmpa_tl,\l_@@_tmpb_tl}
%    Scratch variables used throughout the package.
%    \begin{macrocode}
\tl_new:N \l_@@_return_tl
\tl_new:N \l_@@_tmpa_tl
\tl_new:N \l_@@_tmpb_tl
%    \end{macrocode}
%  \end{macro}
%
%  \begin{macro}{\g_@@_all_seq}
%    In a few places we need a list of all hook names ever defined so
%    we keep track if them in this sequence.
%    \begin{macrocode}
\seq_new:N \g_@@_all_seq
%    \end{macrocode}
%  \end{macro}
%
% \begin{macro}{\g_@@_removal_list_prop}
%   A property list to hold delayed removals.
%    \begin{macrocode}
\tl_new:N \g_@@_removal_list_tl
%    \end{macrocode}
% \end{macro}
%
% \begin{macro}{\l_@@_cur_hook_tl}
%   Stores the name of the hook currently being sorted.
%    \begin{macrocode}
\tl_new:N \l_@@_cur_hook_tl
%    \end{macrocode}
% \end{macro}
%
% \begin{macro}{\g_@@_code_temp_prop}
%   A property list to temporarily save the original one so that it
%   isn't permanently changed during sorting.
%    \begin{macrocode}
\prop_new:N \g_@@_code_temp_prop
%    \end{macrocode}
% \end{macro}
%
% \begin{macro}{\g_@@_hook_curr_name_tl,\g_@@_name_stack_seq}
%   Default label used for hook commands, and a stack to keep track of
%   packages within packages.
%    \begin{macrocode}
\tl_new:N \g_@@_hook_curr_name_tl
\seq_new:N \g_@@_name_stack_seq
%    \end{macrocode}
% \end{macro}
%
% \begin{macro}{\@@_tmp:w}
%   Temporary macro for generic usage.
%    \begin{macrocode}
\cs_new_eq:NN \@@_tmp:w ?
%    \end{macrocode}
% \end{macro}
%
% \begin{macro}{\tl_gremove_once:Nx}
%   Some variants of \pkg{expl3} functions. \fmi{should be moved to expl3}
%    \begin{macrocode}
\cs_generate_variant:Nn \tl_gremove_once:Nn { Nx }
%    \end{macrocode}
% \end{macro}
%
% \begin{macro}{\s_@@_mark}
%   Scan mark used for delimited arguments.
%    \begin{macrocode}
\scan_new:N \s_@@_mark
%    \end{macrocode}
% \end{macro}
%
%
% \subsection{Providing new hooks}
%
%  \begin{macro}{\g_@@_..._code_prop,\g_@@_..._rules_prop,
%                \g_@@_..._code_tl,\g_@@_..._next_code_tl}
%
%    Hooks have a \meta{name} and for each hook we have to provide a number of
%    data structures. These are
%    \begin{description}
%    \item[\cs{g_@@_\meta{name}_code_prop}] A property list holding the code
%    for the hook in separate chunks. The keys are by default the
%    package names that add code to the hook, but it is possible
%    for packages to define other keys. 
%
%    \item[\cs{g_@@_\meta{name}_rules_prop}] A property listing holding
%    relation info how the code chunks should be ordered within a
%    hook. This is used for debugging only. The actual rule for a
%    \meta{hook} is stored in a separate token lists named
%    \cs[no-index]{g_@@_\meta{hook}_rule_\meta{label1}\string|\meta{label2}_tl}
%    for a pair of labels.
%
%    \item[\cs{g_@@_\meta{name}_code_tl}] The code that is actually executed
%    when the hook is called in the document is stored in this token
%    list. It is constructed from the code chunks applying the
%    information.
%
%    \item[\cs{g_@@_\meta{name}_next_code_tl}] Finally there is extra code
%    (normally empty) that is used on the next invocation of the hook
%    (and then deleted). This can be used to define some special
%    behavior for a single occasion from within the document.
%
%    \end{description}
%  \end{macro}


%
%
%  \begin{macro}{\hook_new:n}
%    The \cs{hook_new:n} declaration declare a new hook and expects
%    the hook \meta{name} as its argument, e.g.,
%    \hook{begindocument}.
%    \begin{macrocode}
\cs_new_protected:Npn \hook_new:n #1
  {
    \exp_args:Nx \@@_new:n
      { \@@_parse_label_default:nn {#1} { top-level } }
  }
\cs_new_protected:Npn \@@_new:n #1 {
%    \end{macrocode}
%    We check for one of the internal data structures and if it
%    already exists we complain.
%    \begin{macrocode}
  \hook_if_exist:nTF {#1}
     { \ErrorHookExists }
%    \end{macrocode}
%    Otherwise we add the hook name to the list of all hooks and
%    allocate the necessary data structures for the new hook.
%    \begin{macrocode}
     { \seq_gput_right:Nn \g_@@_all_seq {#1}
%    \end{macrocode}
%    This is only used by the actual code of the current hook, so
%    declare it normally:
%    \begin{macrocode}
       \tl_new:c { g_@@_#1_code_tl }
%    \end{macrocode}
%    Now ensure that the base data structure for the hook exists:
%    \begin{macrocode}
       \@@_declare:n {#1}
%    \end{macrocode}
%    The \cs{g_@@_\meta{hook}_labels_clist} holds the sorted list of
%    labels (once it got sorted). This is used only for debugging.
%    \begin{macrocode}
       \clist_new:c {g_@@_#1_labels_clist}
%    \end{macrocode}
%    Some hooks should reverse the default order of code chunks. To
%    signal this we have a token list which is empty for normal hooks
%    and contains a \verb=-= for reversed hooks.
%    \begin{macrocode}
       \tl_new:c { g_@@_#1_reversed_tl }
%    \end{macrocode}
%    The above is all in L3 convention, but we also provide an
%    interface to legacy \LaTeXe{} for use in the current kernel. This
%    is done in a separate macro.
%    \begin{macrocode}
       \@@_provide_legacy_interface:n {#1}
     }
}
%    \end{macrocode}
%  \end{macro}

%
%
% \begin{macro}{\@@_declare:n}
%   This function declares the basic data structures for a hook without
%   actually declaring the hook itself.  This is needed to allow adding
%   to undeclared hooks.  Here it is unnecessary to check whether both
%   variables exist, since both are declared at the same time (either
%   both exist, or neither).
%    \begin{macrocode}
\cs_new_protected:Npn \@@_declare:n #1
  {
    \@@_if_exist:nF {#1}
      {
        \prop_new:c { g_@@_#1_code_prop }
        \tl_new:c { g_@@_#1_next_code_tl }
        \prop_new:c { g_@@_#1_rules_prop } % only for debugging
      }
  }
%    \end{macrocode}
%  \end{macro}



%  \begin{macro}{\hook_new_reversed:n}
%
%    Declare a new hook. The default ordering of code chunks is
%    reversed, signaled by setting the token list to a minus sign.
%    \begin{macrocode}
\cs_new_protected:Npn \hook_new_reversed:n #1 {
  \hook_new:n {#1}
  \tl_gset:cn { g_@@_#1_reversed_tl } { - }
}
%    \end{macrocode}
%  \end{macro}

%  \begin{macro}{\hook_new_pair:nn}
%    A shorthand for declaring a normal and a (matching) reversed hook in one go.
%    \begin{macrocode}
\cs_new_protected:Npn \hook_new_pair:nn #1#2 {
  \hook_new:n {#1}  \hook_new_reversed:n {#2}
}
%    \end{macrocode}
%  \end{macro}


% \begin{macro}{\@@_provide_legacy_interface:n}
%    The \LaTeX{} legacy concept for hooks uses with hooks the
%    following naming scheme in the code: \cs{@...hook}.
%
%    We follow this convention and insert the hook code using this
%    naming scheme in \LaTeXe{}. At least as long as this code is in a
%    package, some such hooks are already filled with data when we move
%    them over to the new scheme. We therefore insert already existing
%    code under the label \texttt{legacy} into the hook management
%    machinery and then replace the \cs{@...hook} with its counterpart
%    which is \cs{g_@@_\#1_code_tl}.\footnote{This means one extra
%    unnecessary expansion on each invocation in the document but
%    keeps the \LaTeXe{} and the L3 coding side properly separated.}
%    \begin{macrocode}
\cs_new_protected:Npn \@@_provide_legacy_interface:n #1
  {
%    \end{macrocode}
%    If the \pkg{expl3} code is run with checking on then assigning or
%    using non L3 names such as \cs{@enddocumenthook} with \pkg{expl3}
%    functions will trigger warnings so we run this code with
%    debugging explicitly suspended.
%    \begin{macrocode}
    \debug_suspend:
    \tl_if_exist:cT { @#1hook }
%    \end{macrocode}
%    Of course if the hook exists but is still empty, there is no need
%    to add anything under \texttt{legacy} or the current package name.
%    \begin{macrocode}
      {
        \tl_if_empty:cF { @#1hook }
          {
            \@@_gput_code:nxv {#1}
              { \@@_parse_label_default:Vn \c_novalue_tl { legacy } }
              { @#1hook }
          }
      }
%    \end{macrocode}
%    We need a global definition in case the declaration is done
%    inside a group (which happens below at the end of the file).
%    This is another reason why need to suspend checking, otherwise
%    \cs{tl_gset:co} would complain about \cs{@...hook} not starting
%    with \cs{g_}.
%    \begin{macrocode}
    \tl_gset:co{@#1hook}{\cs:w g_@@_#1_code_tl\cs_end:}
    \debug_resume:
  }
%    \end{macrocode}
% \end{macro}
%
% \subsection{Parsing a label}
%
% \begin{macro}[EXP]{\@@_parse_label_default:nn,\@@_parse_label_default:Vn}
%   This macro checks if a label was given (not \cs{c_novalue_tl}), and
%   if so, tries to parse the label looking for a leading \verb|.| to
%   replace for \cs{@currname}.  Otherwise \cs{@@_currname_or_default:n}
%   is used to pick \cs{@currname} or the fallback value.
%    \begin{macrocode}
\cs_new:Npn \@@_parse_label_default:nn #1 #2
  {
    \tl_if_novalue:nTF {#1}
      { \@@_currname_or_default:n {#2} }
      { \tl_trim_spaces_apply:nN {#1} \@@_parse_dot_label:nn {#2} }
  }
\cs_generate_variant:Nn \@@_parse_label_default:nn { V }
%    \end{macrocode}
% \end{macro}
%
% \begin{macro}[EXP]{\@@_parse_dot_label:nn}
% \begin{macro}[EXP]{
%     \@@_parse_dot_label:nw,
%     \@@_parse_dot_label_cleanup:w,
%     \@@_parse_dot_label_aux:nw
%   }
%   Start by checking if the label is empty, which raises an error, and
%   uses the fallback value.  If not,
%   split the label at a \verb|./|, if any, and check if no tokens are
%   before the \verb|./|, or if the only character is a \verb|.|.
%   If these requirements are fulfilled, the leading
%   \verb|.| is replaced with \cs{@@_currname_or_default:n}.  Otherwise
%   the label is returned unchanged.
%    \begin{macrocode}
\cs_new:Npn \@@_parse_dot_label:nn #1 #2
  {
    \tl_if_empty:nTF {#1}
      {
        \msg_expandable_error:nnn { hooks } { empty-label } {#2}
        #2
      }
      {
        \str_if_eq:nnTF {#1} { . }
          { \@@_currname_or_default:n {#1} }
          { \@@_parse_dot_label:nw {#2} #1 ./ \s_@@_mark }
      }
  }
\cs_new:Npn \@@_parse_dot_label:nw #1 #2 ./ #3 \s_@@_mark
  {
    \tl_if_empty:nTF {#2}
      { \@@_parse_dot_label_aux:nw {#1} #3 \s_@@_mark }
      {
        \tl_if_empty:nTF {#3}
          {#2}
          { \@@_parse_dot_label_cleanup:w #2 ./ #3 \s_@@_mark }
      }
  }
\cs_new:Npn \@@_parse_dot_label_cleanup:w #1 ./ \s_@@_mark {#1}
\cs_new:Npn \@@_parse_dot_label_aux:nw #1 #2 ./ \s_@@_mark
  { \@@_currname_or_default:n {#1} / #2 }
%    \end{macrocode}
% \end{macro}
% \end{macro}
%
% \begin{macro}[EXP]{\@@_currname_or_default:n}
%   Uses \cs{g_@@_hook_curr_name_tl} if it is set, otherwise tries
%   \cs{@currname}.  If neither is set, uses the fallback value
%   \verb|#1| (usually \texttt{top-level}).
%    \begin{macrocode}
\cs_new:Npn \@@_currname_or_default:n #1
  {
    \tl_if_empty:NTF \g_@@_hook_curr_name_tl
      {
        \tl_if_empty:NTF \@currname
          {#1}
          { \@currname }
      }
      { \g_@@_hook_curr_name_tl }
  }
%    \end{macrocode}
% \end{macro}




% \begin{macro}{\hook_gput_code:nnn}
% \begin{macro}{\@@_gput_code:nnn,\@@_gput_code:nxv,\@@_hook_gput_code_do:nnn}
%
%    With \cs{hook_gput_code:nnn}\Arg{hook}\Arg{label}\Arg{code} a
%    chunk of \meta{code} is added to an existing \meta{hook} labeled
%    with \meta{label}.
%    \begin{macrocode}
\cs_new_protected:Npn \hook_gput_code:nnn #1 #2
  {
    \exp_args:Nxx \@@_gput_code:nnn
      { \@@_parse_label_default:nn {#1} { top-level } }
      { \@@_parse_label_default:nn {#2} { top-level } }
  }
\cs_new_protected:Npn \@@_gput_code:nnn #1 #2 #3
  {
%    \end{macrocode}
%    First we check if the hook exists.
%    \begin{macrocode}
    \@@_if_marked_removal:nnTF {#1} {#2}
      { \@@_unmark_removal:nn {#1} {#2} }
      {
%    \end{macrocode}
%    First we check if the hook exists.
%    \begin{macrocode}
        \hook_if_exist:nTF {#1}
%    \end{macrocode}
%    If so we simply add (or append) the new code to the property list
%    holding different chunks for the hook. At \verb=\begin{document}=
%    this is then sorted into a token list for fast execution.
%    \begin{macrocode}
          {
            \@@_hook_gput_code_do:nnn {#1} {#2} {#3}
%    \end{macrocode}
%    However, if there is an update within the document we need to alter
%    this execution code which is done by
%    \cs{@@_update_hook_code:n}. In the preamble this does nothing.
%    \begin{macrocode}
            \@@_update_hook_code:n {#1}
          }
%    \end{macrocode}
%    
%    \begin{macrocode}
          { \@@_try_declaring_generic_hook:nnn {#1} {#2} {#3} }
      }
  }
\cs_generate_variant:Nn \@@_gput_code:nnn { nxv }
%    \end{macrocode}
%
%   This macro will unconditionally add a chunk of code to the given hook.
%    \begin{macrocode}
\cs_new_protected:Npn \@@_hook_gput_code_do:nnn #1 #2 #3
  {
%   However, first some debugging info if debugging is enabled:
%    \begin{macrocode}
    \@@_debug:n{\iow_term:x{****~ Add~ to~
                      \hook_if_exist:nF {#1} { undeclared~ }
                      hook~ #1~ (#2)
                      \on@line\space <-~ \tl_to_str:n{#3}} }
%    \end{macrocode}
%   Then try to get the code chunk labeled \verb=#2= from the hook.
%   If there's code already there, then append \verb=#3= to that,
%   otherwise just put \verb=#3=.
%    \begin{macrocode}
    \prop_get:cnNTF { g_@@_#1_code_prop } {#2} \l_@@_return_tl
      {
        \prop_gput:cno { g_@@_#1_code_prop } {#2}
          { \l_@@_return_tl #3 }
      }
      { \prop_gput:cnn { g_@@_#1_code_prop } {#2} {#3} }
  }
%    \end{macrocode}
% \end{macro}
% \end{macro}
%
% \begin{macro}{\@@_gput_undeclared_hook:nnn}
%   Often it may happen that a package $A$ defines a hook \verb=foo=,
%   but package $B$, that adds code to that hook, is loaded before $A$.
%   In such case we need to add code to the hook before its declared.
%    \begin{macrocode}
\cs_new_protected:Npn \@@_gput_undeclared_hook:nnn #1 #2 #3
  {
    \@@_declare:n {#1}
    \@@_hook_gput_code_do:nnn {#1} {#2} {#3}
  }
%    \end{macrocode}
% \end{macro}
%
% \begin{macro}{\@@_try_declaring_generic_hook:nnn}
% \begin{macro}{\@@_try_declaring_generic_next_hook:nn}
%   These entry-level macros just pass the arguments along to the
%   common \cs{@@_try_declaring_generic_hook:nNNnn} with the right
%   functions to execute when some action is to be taken.
%
%   The wrapper \cs{@@_try_declaring_generic_hook:nnn} then defers
%   \cs{hook_gput_code:nnn} if the generic hook was declared, or to
%   \cs{@@_gput_undeclared_hook:nnn} otherwise (the hook was tested for
%   existence before, so at this point if it isn't generic, it doesn't
%   exist).
%
%   The wrapper \cs{@@_try_declaring_generic_next_hook:nn} for
%   next-execution hooks does the same: it defers the code to
%   \cs{hook_gput_next_code:nn} if the generic hook was declared, or
%   to \cs{@@_gput_next_do:nn} otherwise.
%    \begin{macrocode}
\cs_new_protected:Npn \@@_try_declaring_generic_hook:nnn #1
  {
    \@@_try_declaring_generic_hook:nNNnn {#1}
      \hook_gput_code:nnn \@@_gput_undeclared_hook:nnn
  }
\cs_new_protected:Npn \@@_try_declaring_generic_next_hook:nn #1
  {
    \@@_try_declaring_generic_hook:nNNnn {#1}
      \hook_gput_next_code:nn \@@_gput_next_do:nn
  }
%    \end{macrocode}
%
% \begin{macro}{
%     \@@_try_declaring_generic_hook:nNNnn,
%     \@@_try_declaring_generic_hook_split:nNNnn
%   }
% \begin{macro}[TF]{\@@_try_declaring_generic_hook:wn}
%   \cs{@@_try_declaring_generic_hook:nNNnn} now splits the hook name
%   at the first \texttt{/} (if any) and first checks if it is a
%   file-specific hook (they require some normalization) using
%   \cs{@@_if_file_hook:wTF}. If not then check it is one of a
%   predefined set for generic names. We also split off the second
%   component to see if we have to make a reversed hook.  In either case
%   the function returns \meta{true} for a generic hook and \meta{false}
%   in other cases.
%    \begin{macrocode}
\cs_new_protected:Npn \@@_try_declaring_generic_hook:nNNnn #1
  {
    \@@_if_file_hook:wTF #1 / / \s_@@_mark
      {
        \exp_args:Ne \@@_try_declaring_generic_hook_split:nNNnn
          { \exp_args:Ne \@@_file_hook_normalise:n {#1} }
      }
      { \@@_try_declaring_generic_hook_split:nNNnn {#1} }
  }
\cs_new_protected:Npn \@@_try_declaring_generic_hook_split:nNNnn #1 #2 #3
  {
    \@@_try_declaring_generic_hook:wnTF #1 / / / \scan_stop: {#1}
      { #2 }
      { #3 } {#1}
  }
\prg_new_protected_conditional:Npnn \@@_try_declaring_generic_hook:wn
    #1 / #2 / #3 / #4 \scan_stop: #5 { TF }
  {
    \tl_if_empty:nTF {#2}
      { \prg_return_false: }
      {
        \prop_if_in:NnTF \c_@@_generics_prop {#1}
          {
            \hook_if_exist:nF {#5} { \hook_new:n {#5} }
%    \end{macrocode}
%    After having declared the hook we check the second component (for
%    file hooks) or the third component for environment hooks) and
%    if it is on the list of components for which we should have declared
%    a reversed hook we alter the hook data structure accordingly.
%    \begin{macrocode}
            \prop_if_in:NnTF \c_@@_generics_reversed_ii_prop {#2}
              { \tl_gset:cn { g_@@_#5_reversed_tl } { - } }
              {
                \prop_if_in:NnT \c_@@_generics_reversed_iii_prop {#3}
                  { \tl_gset:cn { g_@@_#5_reversed_tl } { - } }
              }
%    \end{macrocode}
%    Now that we know that the hook is declared we can add the code to it.
%    \begin{macrocode}
            \prg_return_true:
          }
          { \prg_return_false: }
      }
  }
%    \end{macrocode}
% \end{macro}
% \end{macro}
% \end{macro}
% \end{macro}
%
% \begin{macro}[pTF]{\@@_if_file_hook:w}
%   \cs{@@_if_file_hook:wTF} checks if the argument is a valid
%   file-specific hook (not, for example, |file/before|, but
%   |file/before/foo.tex|).  If it is a file-specific hook, then it
%   executes the \meta{true} branch, otherwise \meta{false}.
%
%   A file-specific hook is \texttt{file/\meta{position}/\meta{name}}.
%   If any of these parts don't exist, it is a general file hook or not
%   a file hook at all, so the conditional evaluates to \meta{false}.
%   Otherwise, it checks that the first part is |file| and that the
%   \meta{position} is in the \cs{c_@@_generics_file_prop}.
%
%   A property list is used here to avoid having to worry with catcodes,
%   because \pkg{expl3}'s file name parsing turns all characters into
%   catcode-12 tokens, which might differ from hand-input letters.
%    \begin{macrocode}
\prg_new_conditional:Npnn \@@_if_file_hook:w
    #1 / #2 / #3 \s_@@_mark { TF }
  {
    \str_if_eq:nnTF {#1} { file }
      {
        \bool_lazy_or:nnTF
            { \tl_if_empty_p:n {#3} }
            { \str_if_eq_p:nn {#3} { / } }
          { \prg_return_false: }
          {
            \prop_if_in:NnTF \c_@@_generics_file_prop {#2}
              { \prg_return_true: }
              { \prg_return_false: }
          }
      }
      { \prg_return_false: }
  }
%    \end{macrocode}
% \end{macro}
%
% \begin{macro}[EXP]{\@@_file_hook_normalise:n}
% \begin{macro}[EXP]{\@@_strip_double_slash:n,\@@_strip_double_slash:w}
%   When a file-specific hook is found, before being declared it is
%   lightly normalized by \cs{@@_file_hook_normalise:n}.  The current
%   implementation just replaces two consecutive slashes (|//|) by a
%   single one, to cope with simple cases where the user did something
%   like \verb|\def\input@path{{./mypath/}}|, in which case a hook would
%   have to be \verb|\AddToHook{file/after/./mypath//file.tex}|.
%    \begin{macrocode}
\cs_new:Npn \@@_file_hook_normalise:n #1
  { \@@_strip_double_slash:n {#1} }
\cs_new:Npn \@@_strip_double_slash:n #1
  { \@@_strip_double_slash:w #1 // \s_@@_mark }
\cs_new:Npn \@@_strip_double_slash:w #1 // #2 \s_@@_mark
  {
    \tl_if_empty:nTF {#2}
      {#1}
      { \@@_strip_double_slash:w #1 / #2 \s_@@_mark }
  }
%    \end{macrocode}
% \end{macro}
% \end{macro}





%  \begin{macro}{\c_@@_generics_prop}
%    Clist holding the generic names. We don't provide any user
%    interface to this as this is meant to be static.
%    \begin{description}
%    \item[\texttt{env}]
%      The generic hooks used in \cs{begin} and \cs{end}.
%    \item[\texttt{file}]
%      The generic hooks used when loading a file
%    \end{description}
%    \begin{macrocode}
\prop_const_from_keyval:Nn \c_@@_generics_prop
  {env=,file=,package=,class=,include=}
%    \end{macrocode}
%  \end{macro}
%
%  \begin{macro}{\c_@@_generics_reversed_ii_prop,
%                \c_@@_generics_reversed_iii_prop,
%                \c_@@_generics_file_prop}
%    Some of the generic hooks are supposed to use reverse ordering, these are
%    the following (only the second or third sub-component is checked):
%    \begin{macrocode}
\prop_const_from_keyval:Nn \c_@@_generics_reversed_ii_prop {after=,end=}
\prop_const_from_keyval:Nn \c_@@_generics_reversed_iii_prop {after=}
\prop_const_from_keyval:Nn \c_@@_generics_file_prop {before=,after=}
%    \end{macrocode}
%  \end{macro}


%  \begin{macro}{\@@_update_hook_code:n}
%    Before \verb=\begin{document}=  this does nothing, in the body it
%    reinitializes the hook code using the altered data.
%    \begin{macrocode}
\cs_new_eq:NN \@@_update_hook_code:n \use_none:n
%    \end{macrocode}
%  \end{macro}


% \begin{macro}{\hook_gremove_code:nn}
% \begin{macro}{\@@_gremove_code:nn}
%    
%    With \cs{hook_gremove_code:nn}\Arg{hook}\Arg{label} any code
%    for \meta{hook} stored under \meta{label} is removed.
%    \begin{macrocode}
\cs_new_protected:Npn \hook_gremove_code:nn #1 #2
  {
    \exp_args:Nxx \@@_gremove_code:nn
      { \@@_parse_label_default:nn {#1} { top-level } }
      { \@@_parse_label_default:nn {#2} { top-level } }
  }
\cs_new_protected:Npn \@@_gremove_code:nn #1 #2
  {
%    \end{macrocode}
%    First check that the hook code pool exists.  \cs{hook_if_exist:nTF}
%    isn't used here because it should be possible to remove code from a
%    hook before its defined (see section~\ref{sec:querying}).
%    \begin{macrocode}
    \@@_if_exist:nTF {#1}
%    \end{macrocode}
%    Then remove the chunk and run \cs{@@_update_hook_code:n} so
%    that the execution token list reflects the change if we are after
%    \verb=\begin{document}=.
%    \begin{macrocode}
      {
        \str_if_eq:nnTF {#2} {*}
          {
            \prop_gclear:c { g_@@_#1_code_prop }
            \clist_gclear:c { g_@@_#1_labels_clist } % for debugging only
          }
          {
%    \end{macrocode}
%    Check if the label being removed exists in the code pool.  If it does,
%    just call \cs{@@_gremove_code_do:nn} to do the removal, otherwise mark it
%    to be removed.
%    \begin{macrocode}
            \prop_get:cnNTF { g_@@_#1_code_prop } {#2} \l_@@_return_tl
              { \@@_gremove_code_do:nn }
              { \@@_mark_removal:nn }
                  {#1} {#2}
          }
%    \end{macrocode}
%    Finally update the code, if the hook exists.
%    \begin{macrocode}
        \hook_if_exist:nT {#1}
          { \@@_update_hook_code:n {#1} }
      }
      { \@@_mark_removal:nn {#1} {#2} }
  }
%    \end{macrocode}
%
% \begin{macro}{\@@_gremove_code_do:nn}
%    \begin{macrocode}
\cs_new_protected:Npn \@@_gremove_code_do:nn #1 #2
  {
    \prop_gremove:cn { g_@@_#1_code_prop } {#2}
%    \end{macrocode}
%    Removing the dropped label from \verb=\g_@@_#1_labels_clist= is
%    rather tricky, because that clists holds the labels as strings
%    (i.e., not ordinary text which is what we have in \verb=#2=).
%    \begin{macrocode}
    \exp_args:Nco \clist_gremove_all:Nn
      { g_@@_#1_labels_clist } { \tl_to_str:n {#2} } % for debugging only
  }
%    \end{macrocode}
% \end{macro}
% \end{macro}
% \end{macro}


%
% \begin{macro}{\@@_mark_removal:nn}
%   Marks \meta{label} (\verb=#2=) to be removed from \meta{hook}
%   (\verb=#1=).
%    \begin{macrocode}
\cs_new_protected:Npn \@@_mark_removal:nn #1 #2
  {
    \tl_gput_right:Nx \g_@@_removal_list_tl
      { \@@_removal_tl:nn {#1} {#2} }
  }
%    \end{macrocode}
% \end{macro}
%
% \begin{macro}{\@@_unmark_removal:nn}
%   Unmarks \meta{label} (\verb=#2=) to be removed from \meta{hook}
%   (\verb=#1=).  \cs{tl_gremove_once:Nx} is used rather than
%   \cs{tl_gremove_all:Nx} so that two additions are needed to cancel
%   two marked removals, rather than only one.
%    \begin{macrocode}
\cs_new_protected:Npn \@@_unmark_removal:nn #1 #2
  {
    \tl_gremove_once:Nx \g_@@_removal_list_tl
      { \@@_removal_tl:nn {#1} {#2} }
  }
%    \end{macrocode}
% \end{macro}
%
% \begin{macro}[TF]{\@@_if_marked_removal:nn}
%   Checks if the \cs{g_@@_removal_list_tl} contains the current
%   \meta{label} (\verb=#2=) and \meta{hook} (\verb=#1=).
%    \begin{macrocode}
\prg_new_protected_conditional:Npnn \@@_if_marked_removal:nn #1 #2 { TF }
  {
    \exp_args:NNx \tl_if_in:NnTF \g_@@_removal_list_tl
      { \@@_removal_tl:nn {#1} {#2} }
      { \prg_return_true: } { \prg_return_false: }
  }
%    \end{macrocode}
% \end{macro}
%
% \begin{macro}[rEXP]{\@@_removal_tl:nn}
%   Builds a token list with \verb=#1= and \verb=#2= which can only be
%   matched by \verb=#1= and \verb=#2=.
%    \begin{macrocode}
\cs_new:Npn \@@_removal_tl:nn #1 #2
  { & \tl_to_str:n {#2} $ \tl_to_str:n {#1} $ }
%    \end{macrocode}
% \end{macro}




%
%
%
% \begin{macro}{
%     \g_@@_??_rules_prop,
%     \g_@@_??_code_prop,
%     \g_@@_??_code_tl,
%     \g_@@_??_reversed_tl,
%   }
%
%    Default rules applying to all hooks are stored in this property
%    list. Initially it simply used an empty ``label'' name (not two
%    question marks). This was a bit unfortunate, because then
%    \texttt{l3doc} complains about \verb=__= in the middle of a
%    command name when trying to typeset the documentation. However
%    using a ``normal'' name such as \texttt{default} has the
%    disadvantage of that being not really distinguishable from a real
%    hook name. I now have settled for \texttt{??} which needs some
%    gymnastics to get it into the csname, but since this is used a
%    lot things should be fast, so this is not done with \texttt{c}
%    expansion in the code later on.
%
%    \cs{g_@@_??_code_tl} isn't used, but it has to be defined to trick
%    the code into thinking that \verb=??= is actually a hook.
%    \begin{macrocode}
\prop_new:c {g_@@_??_rules_prop}
\prop_new:c {g_@@_??_code_prop}
\prop_new:c {g_@@_??_code_tl}
%    \end{macrocode}
%
%    Default rules are always given in normal ordering (never in
%    reversed ordering). If such a rule is applied to a reversed
%    hook it behaves as if the rule is reversed (e.g.,
%    \texttt{after} becomes \texttt{before})
%    because those rules are applied first and then the order is reversed.
%    \begin{macrocode}
\tl_new:c {g_@@_??_reversed_tl}
%    \end{macrocode}
%  \end{macro}



%  \begin{macro}{\@@_debug_gset_rule:nnnn}
%
%    \fmi{this needs cleanup and docu correction!}
%
%    With
%    \cs{@@_debug_gset_rule:nnnn}\Arg{hook}\Arg{label1}\Arg{relation}\Arg{label2}
%    a relation is defined between the two code labels for the given
%    \meta{hook}.  The special hook \texttt{??} stands for \emph{any}
%    hook describing a default rule.
%    \begin{macrocode}
\cs_new_protected:Npn \@@_debug_gset_rule:nnnn #1#2#3#4
  {
%    \end{macrocode}
%    If so we drop any existing rules with the two labels (in case
%    there are any).
%    \begin{macrocode}
    \prop_gremove:cn{g_@@_#1_rules_prop}{#2|#4}
    \prop_gremove:cn{g_@@_#1_rules_prop}{#4|#2}
%    \end{macrocode}
%    Then  we add the new one (normalizing the input a bit, e.g., we
%    always use \texttt{before} and not \texttt{after} and
%    instead reorder the labels):
%    \begin{macrocode}
    \str_case_e:nnF {#3}
       {
         {before} { \prop_gput:cnn {g_@@_#1_rules_prop}{#2|#4}{<} }
         {after}  { \prop_gput:cnn {g_@@_#1_rules_prop}{#4|#2}{<} }
%    \end{macrocode}
%    More special rule types \ldots
%    \begin{macrocode}
         {incompatible-error}   { \prop_gput:cnn {g_@@_#1_rules_prop}{#2|#4}{xE} }
         {incompatible-warning} { \prop_gput:cnn {g_@@_#1_rules_prop}{#2|#4}{xW} }
         {removes}      { \prop_gput:cnn {g_@@_#1_rules_prop}{#2|#4}{->} }
%    \end{macrocode}
%    Undo a setting:
%    \begin{macrocode}
         {unrelated}{ \prop_gremove:cn {g_@@_#1_rules_prop}{#2|#4} 
                      \prop_gremove:cn {g_@@_#1_rules_prop}{#4|#2} }
       }
       { \ERRORunknownrule }
  }
%    \end{macrocode}
%  \end{macro}
%
%
%  \subsection{Setting rules for hooks code}
%
%  \begin{macro}{\hook_gset_rule:nnnn}
%  \begin{macro}{\@@_gset_rule:nnnn}
%
%    \fmi{needs docu correction given new implementation}
%
%    With
%    \cs{hook_gset_rule:nnnn}\Arg{hook}\Arg{label1}\Arg{relation}\Arg{label2}
%    a relation is defined between the two code labels for the given
%    \meta{hook}.  The special hook \texttt{??} stands for \emph{any}
%    hook describing a default rule.
%    \begin{macrocode}
\cs_new_protected:Npn \hook_gset_rule:nnnn #1#2#3#4
  {
    \use:x
      {
        \@@_gset_rule:nnnn
          { \@@_parse_label_default:nn {#1} { top-level } }
          { \@@_parse_label_default:nn {#2} { top-level } }
          {#3}
          { \@@_parse_label_default:nn {#4} { top-level } }
      }
  }
%    \end{macrocode}
%    
%    \begin{macrocode}
\cs_new_protected:Npn \@@_gset_rule:nnnn #1#2#3#4
  {
%    \end{macrocode}
%    First we ensure the basic data structure of the hook exists:
%    \begin{macrocode}
    \@@_declare:n {#1}
%    \end{macrocode}
%    Then we clear any previous relationship between both labels.
%    \begin{macrocode}
    \@@_rule_gclear:nnn {#1} {#2} {#4}
%    \end{macrocode}
%    Then we call the function to handle the given rule. Throw an error if the
%    rule is invalid.
%    \begin{macrocode}
    \debug_suspend:
    \cs_if_exist_use:cTF { @@_rule_#3_gset:nnn }
      {
          {#1} {#2} {#4}
        \@@_update_hook_code:n {#1}
      }
      { \ERRORunknownrule }
    \debug_resume:
    \@@_debug_gset_rule:nnnn {#1} {#2} {#3} {#4} % for debugging
  }
%    \end{macrocode}
% \end{macro}
% \end{macro}
%
% \begin{macro}{\@@_rule_before_gset:nnn, \@@_rule_after_gset:nnn,
%               \@@_rule_<_gset:nnn, \@@_rule_>_gset:nnn}
%    Then we add the new rule.  We need to normalize the rules here to
%    allow for faster processing later.  Given a pair of labels
%    $l_A$ and $l_B$, the rule $l_A>l_B$ is the same as $l_B<l_A$\fmi{}
%    said differently.  But normalizing the
%    forms of the rule to a single representation, say, $l_B<l_A$, then
%    the time spent looking for the rules later is considerably reduced.
%
%    Here we do that normalization by using \cs[no-index]{(pdf)strcmp} to
%    lexically sort labels $l_A$ and $l_B$ to a fixed order.  This order
%    is then enforced every time these two labels are used together.
%
%    Here we use \cs{@@_label_pair:nn}~\Arg{hook}~\Arg{l_A}~\Arg{l_B}
%    to build a string \texttt{$l_B$\string|$l_A$} with a fixed order, and
%    use \cs{@@_label_ordered:nnTF} to apply the correct rule to the pair
%    of labels, depending if it was sorted or not.
%    \begin{macrocode}
\cs_new_protected:Npn \@@_rule_before_gset:nnn #1#2#3
  {
    \tl_gset:cx { g_@@_#1_rule_ \@@_label_pair:nn {#2} {#3} _tl }
      { \@@_label_ordered:nnTF {#2} {#3} { < } { > } }
  }
\cs_new_eq:cN { @@_rule_<_gset:nnn } \@@_rule_before_gset:nnn
%    \end{macrocode}
%
%    \begin{macrocode}
\cs_new_protected:Npn \@@_rule_after_gset:nnn #1#2#3
  {
    \tl_gset:cx { g_@@_#1_rule_ \@@_label_pair:nn {#3} {#2} _tl }
      { \@@_label_ordered:nnTF {#3} {#2} { < } { > } }
  }
\cs_new_eq:cN { @@_rule_>_gset:nnn } \@@_rule_after_gset:nnn
%    \end{macrocode}
%  \end{macro}
%
% \begin{macro}{\@@_rule_removes_gset:nnn}
%   This rule removes (clears, actually) the code from label |#3| if
%   label |#2| is in the hook |#1|.
%    \begin{macrocode}
\cs_new_protected:Npn \@@_rule_removes_gset:nnn #1#2#3
  {
    \tl_gset:cx { g_@@_#1_rule_ \@@_label_pair:nn {#2} {#3} _tl }
      { \@@_label_ordered:nnTF {#2} {#3} { -> } { <- } }
  }
%    \end{macrocode}
%  \end{macro}
%
% \begin{macro}{
%     \@@_rule_incompatible-error_gset:nnn,
%     \@@_rule_incompatible-warning_gset:nnn,
%   }
%   These relations make an error/warning if labels |#2| and |#3| appear
%   together in hook |#1|.
%    \begin{macrocode}
\cs_new_protected:cpn { @@_rule_incompatible-error_gset:nnn } #1#2#3
  { \tl_gset:cn { g_@@_#1_rule_ \@@_label_pair:nn {#2} {#3} _tl } { xE } }
\cs_new_protected:cpn { @@_rule_incompatible-warning_gset:nnn } #1#2#3
  { \tl_gset:cn { g_@@_#1_rule_ \@@_label_pair:nn {#2} {#3} _tl } { xW } }
%    \end{macrocode}
%  \end{macro}
%
% \begin{macro}{\@@_rule_unrelated_gset:nnn, \@@_rule_gclear:nnn}
%    Undo a setting. \cs{@@_rule_unrelated_gset:nnn} doesn't need to do anything,
%    since we use \cs{@@_rule_gclear:nnn} before setting any rule.
%    \begin{macrocode}
\cs_new_protected:Npn \@@_rule_unrelated_gset:nnn #1#2#3 { }
\cs_new_protected:Npn \@@_rule_gclear:nnn #1#2#3
  { \cs_undefine:c { g_@@_#1_rule_ \@@_label_pair:nn {#2} {#3} _tl } }
%    \end{macrocode}
%  \end{macro}
%
% \begin{macro}[EXP]{\@@_label_pair:nn}
%   Ensure that the lexically greater label comes first.
%    \begin{macrocode}
\cs_new:Npn \@@_label_pair:nn #1#2
  {
    \if_case:w \@@_str_compare:nn {#1} {#2} \exp_stop_f:
           #1 | #1 %  0
    \or:   #1 | #2 % +1
    \else: #2 | #1 % -1
    \fi:
  }
%    \end{macrocode}
%  \end{macro}
%
% \begin{macro}[pTF]{\@@_label_ordered:nn}
%   Check that labels |#1| and |#2| are in the correct order (as
%   returned by \cs{@@_label_pair:nn}) and if so return true, else
%   return false.
%    \begin{macrocode}
\prg_new_conditional:Npnn \@@_label_ordered:nn #1#2 { TF }
  {
    \if_int_compare:w \@@_str_compare:nn {#1} {#2} > 0 \exp_stop_f:
      \prg_return_true:
    \else
      \prg_return_false:
    \fi:
  }
%    \end{macrocode}
%  \end{macro}
%
% \begin{macro}[EXP]{\@@_if_label_case:nnnnn}
%   To avoid doing the string comparison twice in \cs{@@_initialize_single:NNNNn}
%   (once with \cs{str_if_eq:nn} and again with \cs{@@_label_ordered:nn}),
%   we use a three-way branching macro that will compare |#1| and |#2|
%   and expand to \cs{use_i:nnn} if they are equal, \cs{use_ii:nn} if
%   |#1| is lexically greater, and \cs{use_iii:nn} otherwise.
%    \begin{macrocode}
\cs_new:Npn \@@_if_label_case:nnnnn #1#2
   {
     \cs:w use_
       \if_case:w \@@_str_compare:nn {#1} {#2}
          i \or: ii \else: iii \fi: :nnn
     \cs_end:
   }
%    \end{macrocode}
%  \end{macro}


%  \begin{macro}{\@@_initialize_all:}
%    Initialize all known hooks (at \verb=\begin{document}=), i.e.,
%    update the fast execution token lists to hold the necessary code
%    in the right  order.
%    \begin{macrocode}
\cs_new_protected:Npn \@@_initialize_all: {
%    \end{macrocode}
%    First we change \cs{@@_update_hook_code:n} which so far was a
%    no-op to now initialize one hook. This way any later updates to
%    the hook will run that code and also update the execution token
%    list.
%    \begin{macrocode}
  \cs_gset_eq:NN \@@_update_hook_code:n \@@_initialize_hook_code:n
%    \end{macrocode}
%    Now we loop over all hooks that have been defined and update each
%    of them.
%    \begin{macrocode}
  \@@_debug:n { \prop_gclear:N \g_@@_used_prop }
  \seq_map_inline:Nn \g_@@_all_seq
      {
        \@@_update_hook_code:n {##1}
      }
%    \end{macrocode}
%    If we are debugging we show results hook by hook for all hooks
%    that have data. 
%    \begin{macrocode}
  \@@_debug:n
     { \iow_term:x{^^JAll~ initialized~ (non-empty)~ hooks:}
       \prop_map_inline:Nn \g_@@_used_prop
           { \iow_term:x{^^J~ ##1~ ->~
               \exp_not:v {g_@@_##1_code_tl}~ }
           }
     }
%    
%    \end{macrocode}
%    After all hooks are initialized we change the ``use'' to just
%    call the hook code and not initialize it (as it was done in the
%    preamble.
%    \begin{macrocode}
  \cs_gset_eq:NN \hook_use:n \@@_use_initialized:n
  \cs_gset_eq:NN \@@_preamble_hook:n \use_none:n
}
%    \end{macrocode}
%  \end{macro}



%  \begin{macro}{\@@_initialize_hook_code:n}
%    Initializing or reinitializing the fast execution hook code. In
%    the preamble this is selectively done in case a hook gets used
%    and at \verb=\begin{document}= this is done for all hooks and
%    afterwards only if the hook code changes.
%    \begin{macrocode}
\cs_new_protected:Npn \@@_initialize_hook_code:n #1 {
  \@@_debug:n{ \iow_term:x{^^JUpdate~ code~ for~ hook~
                                  '#1' \on@line :^^J} }
%    \end{macrocode}
%    This does the sorting and the updates. If there aren't any code
%    chunks for the current hook, there is no point in even starting
%    the sorting routine so we make a quick test for that and in that
%    case just update \cs{g_@@_\meta{hook}_code_tl} to hold the next
%    code. If there are code chunks we call
%    \cs{@@_initialize_single:NNNNn} and pass to it ready made csnames
%    as they are needed several times inside. This way we save a bit
%    on processing time if we do that up front.
%    \begin{macrocode}
  \hook_if_exist:nT {#1}
    {
      \prop_if_empty:cTF {g_@@_#1_code_prop}
        { \tl_gset:co {g_@@_#1_code_tl}
                      {\cs:w g_@@_#1_next_code_tl \cs_end: } }
        {
%    \end{macrocode}
%    By default the algorithm sorts the code chunks and then saves the
%    result in a token list for fast execution by adding the code one
%    after another using \cs{tl_gput_right:NV}. When we sort code for
%    a reversed hook, all we have to do is to add the code chunks in
%    the opposite order into the token list. So all we have to do
%    in preparation is to change two definitions used later on.
%    \begin{macrocode}
          \@@_if_reversed:nTF {#1}
            { \cs_set_eq:NN \@@_tl_gput:NV    \tl_gput_left:NV
              \cs_set_eq:NN \@@_clist_gput:NV \clist_gput_left:NV  }
            { \cs_set_eq:NN \@@_tl_gput:NV    \tl_gput_right:NV
              \cs_set_eq:NN \@@_clist_gput:NV \clist_gput_right:NV }
%    \end{macrocode}
%
%    When sorting, some relations (namely \verb|->| \verb|<-|) need to
%    act destructively on the code property lists to remove code that
%    shouldn't appear in the sorted hook token list.
%    \begin{macrocode}
          \prop_gset_eq:Nc \g_@@_code_temp_prop { g_@@_#1_code_prop }
          \@@_initialize_single:ccccn
            { g_@@_#1_code_prop } { g_@@_#1_code_tl }
            { g_@@_#1_next_code_tl } { g_@@_#1_labels_clist }
            {#1}
          \prop_gset_eq:cN { g_@@_#1_code_prop } \g_@@_code_temp_prop
%    \end{macrocode}
%    For debug display we want to keep track of those hooks that
%    actually got code added to them, so we record that in plist. We
%    use a plist to ensure that we record each hook name only once,
%    i.e., we are only interested in storing the keys and the value is arbitrary
%    \begin{macrocode}
          \@@_debug:n{ \exp_args:NNx \prop_gput:Nnn \g_@@_used_prop {#1}{} }
        }
    }
}
%    \end{macrocode}
%  \end{macro}


%  \begin{macro}{\g_@@_used_prop}
%    All hooks that receive code (for use in debugging display).
%    \begin{macrocode}
\prop_new:N\g_@@_used_prop
%    \end{macrocode}
%  \end{macro}


%
% \begin{macro}[EXP]{\@@_tl_csname:n,\@@_seq_csname:n}
%   It is faster to pass a single token and expand it when necessary
%   than to pass a bunch of character tokens around.
%   \fmi{note to myself: verify}
%    \begin{macrocode}
\cs_new:Npn \@@_tl_csname:n #1 { l_@@_label_#1_tl }
\cs_new:Npn \@@_seq_csname:n #1 { l_@@_label_#1_seq }
%    \end{macrocode}
% \end{macro}
%

%
%
%  \begin{macro}{\l_@@_labels_seq,\l_@@_labels_int,\l_@@_front_tl,
%      \l_@@_rear_tl,\l_@@_label_0_tl}
%
%    For the sorting I am basically implementing Knuth's algorithm for
%    topological sorting as given in TAOCP volume 1 pages 263--266.
%    For this algorithm we need a number of local variables:
%    \begin{itemize}
%    \item
%       List of labels used in the current hook to label code chunks:
%    \begin{macrocode}
\seq_new:N \l_@@_labels_seq
%    \end{macrocode}
%    \item
%      Number of labels used in the current hook. In Knuth's algorithm
%      this is called $N$:
%    \begin{macrocode}
\int_new:N \l_@@_labels_int
%    \end{macrocode}
%    \item
%      The sorted code list to be build is managed using two pointers
%      one to the front of the queue and one to the rear. We model this
%      using token list pointers. Knuth calls them $F$ and $R$:
%    \begin{macrocode}
\tl_new:N \l_@@_front_tl
\tl_new:N \l_@@_rear_tl
%    \end{macrocode}
%    \item
%      The data for the start of the queue is kept in this token list,
%      it corresponds to what Don calls \texttt{QLINK[0]} but since we
%      aren't manipulating individual words in memory it is slightly
%      differently done:
%    \begin{macrocode}
\tl_new:c { \@@_tl_csname:n { 0 } }
%    \end{macrocode}
%
%    \end{itemize}
%  \end{macro}


%  \begin{macro}{\@@_initialize_single:NNNNn,\@@_initialize_single:ccccn}
%
%    \cs{@@_initialize_single:NNNNn} implements the sorting of the code
%    chunks for a hook and saves the result in the token list for fast
%    execution (\verb=#3=). The arguments are \meta{hook-code-plist},
%    \meta{hook-code-tl}, \meta{hook-next-code-tl},
%    \meta{hook-ordered-labels-clist} and \meta{hook-name} (the latter
%    is only used for debugging---the \meta{hook-rule-plist} is accessed
%    using the \meta{hook-name}).
%
%    The additional complexity compared to Don's algorithm is that we
%    do not use simple positive integers but have arbitrary
%    alphanumeric labels. As usual Don's data structures are chosen in
%    a way that one can omit a lot of tests and I have mimicked that as
%    far as possible. The result is a restriction I do not test for at
%    the moment: a label can't be equal to the number 0!  \fmi{Needs
%    checking for, just in case}
%    \begin{macrocode}
\cs_new_protected:Npn \@@_initialize_single:NNNNn #1#2#3#4#5 {
%    \end{macrocode}
%    
%    \begin{macrocode}
  \debug_suspend:
%    \end{macrocode}
%    Step T1: Initialize the data structure \ldots
%    \begin{macrocode}
  \seq_clear:N \l_@@_labels_seq
  \int_zero:N  \l_@@_labels_int
%    \end{macrocode}
%
%    Store the name of the hook:
%    \begin{macrocode}
  \tl_set:Nn \l_@@_cur_hook_tl {#5}
%    \end{macrocode}
%    
%    We loop over the property list holding the code and record all
%    labels listed there. Only rules for those labels are of interest
%    to us. While we are at it we count them (which gives us the $N$
%    in Knuth's algorithm.  The prefix |label_| is added to the variables
%    to ensure that labels named |front|, |rear|, |labels|, or |return|
%    don't interact with our code.
%    \begin{macrocode}
  \prop_map_inline:Nn #1
     {
       \int_incr:N \l_@@_labels_int
       \seq_put_right:Nn \l_@@_labels_seq {##1}
       \tl_set:cn { \@@_tl_csname:n {##1} }{0}     % the counter k for number of
                                                   % j before k rules
       \seq_clear_new:c { \@@_seq_csname:n {##1} } % sequence of successors to k
                                                   % i.e., k before j rules (stores
                                                   % the names of the j's)
     }
%    \end{macrocode}
%    Steps T2 and T3: Sort the relevant rules into the data structure\ldots
%    
%    This loop constitutes a square matrix of the labels in |#1| in the
%    vertical and the horizontal directions.  However since the rule
%    $l_A\meta{rel}l_B$ is the same as $l_B\meta{rel}^{-1}l_A$ we can cut
%    the loop short at the diagonal of the matrix (\emph{i.e.}, when
%    both labels are equal), saving a good amount of time.  The way the
%    rules were set up (see the implementation of \cs{@@_rule_before_gset:nnn}
%    above) ensures that we have no rule in the ignored side of the
%    matrix, and all rules are seen.  The rules are applied in
%    \cs{@@_apply_label_pair:nnn}, which takes the properly-ordered pair
%    of labels as argument.
%    \begin{macrocode}
  \prop_map_inline:Nn #1
    {
      \prop_map_inline:Nn #1
        {
          \@@_if_label_case:nnnnn {##1} {####1}
            { \prop_map_break: }
            { \@@_apply_label_pair:nnn {##1} {####1} }
            { \@@_apply_label_pair:nnn {####1} {##1} }
                {#5}
        }
    }
%    \end{macrocode}
%    Take a breath and take a look at the data structures that have
%    been set up:
%    \begin{macrocode}
  \@@_debug:n { \@@_debug_label_data:N #1 }
%    \end{macrocode}
%    
%
%    Step T4:
%    \begin{macrocode}
  \tl_set:Nn \l_@@_rear_tl { 0 }
  \tl_set:cn { \@@_tl_csname:n { 0 } } { 0 } % really {l_@@_label_ \l_@@_rear_tl _tl}
  \seq_map_inline:Nn \l_@@_labels_seq
      {
        \int_compare:nNnT { \cs:w \@@_tl_csname:n {##1} \cs_end: } = 0
            {
              \tl_set:cn { \@@_tl_csname:n { \l_@@_rear_tl } }{##1}
              \tl_set:Nn \l_@@_rear_tl {##1}
            }
      }
  \tl_set_eq:Nc \l_@@_front_tl { \@@_tl_csname:n { 0 } }
%    \end{macrocode}
%    
%    \begin{macrocode}
  \tl_gclear:N #2
  \clist_gclear:N #4
%    \end{macrocode}
%
%    The whole loop combines steps T5--T7:
%    \begin{macrocode}
  \bool_while_do:nn { ! \str_if_eq_p:Vn \l_@@_front_tl { 0 } }
       {
%    \end{macrocode}
%    This part is step T5:
%    \begin{macrocode}
         \int_decr:N \l_@@_labels_int
         \prop_get:NVN #1 \l_@@_front_tl \l_@@_return_tl
         \@@_tl_gput:NV #2 \l_@@_return_tl
%    \end{macrocode}
%    
%    \begin{macrocode}
         \@@_clist_gput:NV #4 \l_@@_front_tl
         \@@_debug:n{ \iow_term:x{Handled~ code~ for~ \l_@@_front_tl} }
%    \end{macrocode}
%
%    This is step T6 except that we don't use a pointer $P$ to move
%    through the successors, but instead use \verb=##1= of the mapping
%    function.
%    \begin{macrocode}
         \seq_map_inline:cn { \@@_seq_csname:n { \l_@@_front_tl } }
             {
               \tl_set:cx { \@@_tl_csname:n {##1} }
                   { \int_eval:n { \cs:w \@@_tl_csname:n {##1} \cs_end: - 1 } }
               \int_compare:nNnT { \cs:w \@@_tl_csname:n {##1} \cs_end: } = 0
                   {
                     \tl_set:cn { \@@_tl_csname:n { \l_@@_rear_tl } } {##1}
                     \tl_set:Nn \l_@@_rear_tl            {##1}
                   }
             }
%    \end{macrocode}
%    and step T7:
%    \begin{macrocode}
         \tl_set_eq:Nc \l_@@_front_tl { \@@_tl_csname:n { \l_@@_front_tl } }
%    \end{macrocode}
%
%    This is step T8: If we haven't moved the code for all labels
%    (i.e., if \cs{l_@@_labels_int} is still greater than zero) we
%    have a loop and our partial order can't be flattened out.
%    \begin{macrocode}
       }
  \int_compare:nNnF \l_@@_labels_int = 0
      {
        \iow_term:x{====================}
        \iow_term:x{Error:~ label~ rules~ are~ incompatible:}
%    \end{macrocode}
%
%    This is not really the information one needs in the error case
%    but will do for now \ldots \fmi{fix}
%    \begin{macrocode}
        \@@_debug_label_data:N #1
        \iow_term:x{====================}
      }
%    \end{macrocode}
%    After we have added all hook code to \verb=#2= we finish it off
%    with adding extra code for a one time execution. That is stored
%    in \verb=#3= but is normally empty.
%    \begin{macrocode}
  \tl_gput_right:Nn #2 {#3}
%    \end{macrocode}
%    
%    \begin{macrocode}
  \debug_resume:
}
%    \end{macrocode}
%    
%    \begin{macrocode}
\cs_generate_variant:Nn \@@_initialize_single:NNNNn {cccc}
%    \end{macrocode}
%  \end{macro}



%  \begin{macro}{\@@_tl_gput:NV,\@@_clist_gput:NV}
%    These append either on the right (normal hook) or on the left
%    (reversed hook). This is setup up in
%    \cs{@@_initialize_hook_code:n}, elsewhere their behavior is undefined.
%    \begin{macrocode}
\cs_new:Npn \@@_tl_gput:NV     {\ERROR}
\cs_new:Npn \@@_clist_gput:NV  {\ERROR}
%    \end{macrocode}
%  \end{macro}


%
%  \begin{macro}{\@@_apply_label_pair:nnn,\@@_label_if_exist_apply:nnnF}
%
%    This is the payload of steps T2 and T3 executed in the loop described
%    above. This macro assumes |#1| and |#2| are ordered, which means that
%    any rule pertaining the pair |#1| and |#2| is
%    \cs{g_@@_\meta{hook}_rule_\#1\string|\#2_tl}, and not
%    \cs{g_@@_\meta{hook}_rule_\#2\string|\#1_tl}.  This also saves a great deal
%    of time since we only need to check the order of the labels once.
%
%    The arguments here are \meta{label1}, \meta{label2}, \meta{hook}, and
%    \meta{hook-code-plist}.  We are about to apply the next rule and
%    enter it into the data structure.  \cs{@@_apply_label_pair:nnn} will
%    just call \cs{@@_label_if_exist_apply:nnnF} for the \meta{hook}, and
%    if no rule is found, also try the \meta{hook} name \verb=??=
%    denoting a default hook rule.
%
%    \cs{@@_label_if_exist_apply:nnnF} will check if the rule exists for
%    the given hook, and if so call \cs{@@_apply_rule:nnn}.
%    \begin{macrocode}
\cs_new_protected:Npn \@@_apply_label_pair:nnn #1#2#3
  {
%    \end{macrocode}
%    Extra complication: as we use default rules and local hook specific
%    rules we first have to check if there is a local rule and if that
%    exist use it. Otherwise check if there is a default rule and use
%    that.
%    \begin{macrocode}
    \@@_label_if_exist_apply:nnnF {#1} {#2} {#3}
      {
%    \end{macrocode}
%    If there is no hook-specific rule we check for a default one and
%    use that if it exists.
%    \begin{macrocode}
        \@@_label_if_exist_apply:nnnF {#1} {#2} { ?? } { }
      }
  }
\cs_new_protected:Npn \@@_label_if_exist_apply:nnnF #1#2#3
  {
    \if_cs_exist:w g_@@_ #3 _rule_ #1 | #2 _tl \cs_end:
%    \end{macrocode}
%    What to do precisely depends on the type of rule we have
%    encountered. If it is a \texttt{before} rule it will be handled by the
%    algorithm but other types need to be managed differently. All
%    this is done in \cs{@@_apply_rule:nnnN}.
%    \begin{macrocode}
      \@@_apply_rule:nnn {#1} {#2} {#3}
      \exp_after:wN \use_none:n
    \else:
      \use:nn
    \fi:
  }
%    \end{macrocode}
%  \end{macro}




%  \begin{macro}{\@@_apply_rule:nnn}
%    This is the code executed in steps T2 and T3 while looping through
%    the matrix  This is part of step T3. We are about to apply the next
%    rule and enter it into the data structure. The arguments are
%    \meta{label1}, \meta{label2}, \meta{hook-name}, and \meta{hook-code-plist}.
%    \begin{macrocode}
\cs_new_protected:Npn \@@_apply_rule:nnn #1#2#3
  {
    \cs:w @@_apply_
      \cs:w g_@@_#3_reversed_tl \cs_end: rule_
        \cs:w g_@@_ #3 _rule_ #1 | #2 _tl \cs_end: :nnn \cs_end:
      {#1} {#2} {#3}
  }
%    \end{macrocode}
% \end{macro}
%
%  \begin{macro}{\@@_apply_rule_<:nnn,\@@_apply_rule_>:nnn}
%    The most common cases are \texttt{\string<} and \texttt{\string>} so we handle
%    that first.  They are relations $\prec$ and $\succ$ in TAOCP, and
%    they dictate sorting.
%    \begin{macrocode}
\cs_new_protected:cpn { @@_apply_rule_<:nnn } #1#2#3
  {
    \@@_debug:n { \@@_msg_pair_found:nnn {#1} {#2} {#3} }
    \tl_set:cx { \@@_tl_csname:n {#2} }
       { \int_eval:n{ \cs:w \@@_tl_csname:n {#2} \cs_end: + 1 } }
    \seq_put_right:cn{ \@@_seq_csname:n {#1} }{#2}
  }
\cs_new_protected:cpn { @@_apply_rule_>:nnn } #1#2#3
  {
    \@@_debug:n { \@@_msg_pair_found:nnn {#1} {#2} {#3} }
    \tl_set:cx { \@@_tl_csname:n {#1} }
       { \int_eval:n{ \cs:w \@@_tl_csname:n {#1} \cs_end: + 1 } }
    \seq_put_right:cn{ \@@_seq_csname:n {#2} }{#1}
  }
%    \end{macrocode}
% \end{macro}
%
% \begin{macro}{\@@_apply_rule_xE:nnn,\@@_apply_rule_xW:nnn}
%   These relations make two labels incompatible within a hook.
%   |xE| makes raises an error if the labels are found in the same
%   hook, and |xW| makes it a warning.
%    \begin{macrocode}
\cs_new_protected:cpn { @@_apply_rule_xE:nnn } #1#2#3
  {
    \@@_debug:n { \@@_msg_pair_found:nnn {#1} {#2} {#3} }
    \msg_error:nnnnnn { hooks } { labels-incompatible }
      {#1} {#2} {#3} { 1 }
    \use:c { @@_apply_rule_->:nnn } {#1} {#2} {#3}
    \use:c { @@_apply_rule_<-:nnn } {#1} {#2} {#3}
  }
\cs_new_protected:cpn { @@_apply_rule_xW:nnn } #1#2#3
  {
    \@@_debug:n { \@@_msg_pair_found:nnn {#1} {#2} {#3} }
    \msg_warning:nnnnnn { hooks } { labels-incompatible }
      {#1} {#2} {#3} { 0 }
  }
%    \end{macrocode}
% \end{macro}
%
%  \begin{macro}{\@@_apply_rule_->:nnn,\@@_apply_rule_<-:nnn}
%    If we see \texttt{\detokenize{->}} we have to drop code for label
%    \verb=#3= and carry on. We could do a little better and trop
%    everything for that label since it doesn't matter where we sort
%    in the empty code. However that would complicate the algorithm a
%    lot with little gain. So we still unnecessarily try to sort it in
%    and depending on the rules that might result in a loop that is
%    otherwise resolved. If that turns out to be a real issue, we can
%    improve the code.
%
%    Here the code is removed from \cs{l_@@_cur_hook_tl} rather than
%    \verb=#3= because the latter may be \verb=??=, and the default
%    hook doesn't store any code.  Removing from \cs{l_@@_cur_hook_tl}
%    makes default rules \verb=->= and  \verb=<-= work properly.
%    \begin{macrocode}
\cs_new_protected:cpn { @@_apply_rule_->:nnn } #1#2#3
  {
    \@@_debug:n
       {
         \@@_msg_pair_found:nnn {#1} {#2} {#3}
         \iow_term:x{--->~ Drop~ '#2'~ code~ from~
           \iow_char:N \\ g_@@_ \l_@@_cur_hook_tl _code_prop ~ because~ of~ '#1' }
       }
    \prop_gput:cnn { g_@@_ \l_@@_cur_hook_tl _code_prop } {#2} { }
  }
\cs_new_protected:cpn { @@_apply_rule_<-:nnn } #1#2#3
  {
    \@@_debug:n
       {
         \@@_msg_pair_found:nnn {#1} {#2} {#3}
         \iow_term:x{--->~ Drop~ '#1'~ code~ from~
           \iow_char:N \\ g_@@_ \l_@@_cur_hook_tl _code_prop ~ because~ of~ '#2' }
       }
    \prop_gput:cnn { g_@@_ \l_@@_cur_hook_tl _code_prop } {#1} { }
  }
%    \end{macrocode}
%  \end{macro}

% \begin{macro}{
%     \@@_apply_-rule_<:nnn,
%     \@@_apply_-rule_>:nnn,
%     \@@_apply_-rule_<-:nnn,
%     \@@_apply_-rule_->:nnn,
%     \@@_apply_-rule_x:nnn,
%   }
%   Reversed rules.
%    \begin{macrocode}
\cs_new_eq:cc { @@_apply_-rule_<:nnn  } { @@_apply_rule_>:nnn }
\cs_new_eq:cc { @@_apply_-rule_>:nnn  } { @@_apply_rule_<:nnn }
\cs_new_eq:cc { @@_apply_-rule_<-:nnn } { @@_apply_rule_<-:nnn }
\cs_new_eq:cc { @@_apply_-rule_->:nnn } { @@_apply_rule_->:nnn }
\cs_new_eq:cc { @@_apply_-rule_xE:nnn  } { @@_apply_rule_xE:nnn }
\cs_new_eq:cc { @@_apply_-rule_xW:nnn  } { @@_apply_rule_xW:nnn }
%    \end{macrocode}
% \end{macro}


% \begin{macro}{\@@_msg_pair_found:nnn}
%   A macro to avoid moving this many tokens around.
%    \begin{macrocode}
\cs_new_protected:Npn \@@_msg_pair_found:nnn #1#2#3
  {
    \iow_term:x{~ \str_if_eq:nnTF {#3} {??} {default} {~normal} ~
               rule~ \@@_label_pair:nn {#1} {#2}:~
             \use:c { g_@@_#3_rule_ \@@_label_pair:nn {#1} {#2} _tl } ~ found}
  }
%    \end{macrocode}
% \end{macro}


%  \begin{macro}{\@@_debug_label_data:N}
%    
%    \begin{macrocode}
\cs_new_protected:Npn \@@_debug_label_data:N #1 {
  \iow_term:x{Code~ labels~ for~ sorting:}
  \iow_term:x{~ \seq_use:Nnnn\l_@@_labels_seq {~and~}{,~}{~and~} }  % fix name!
  \iow_term:x{^^J Data~ structure~ for~ label~ rules:}
  \prop_map_inline:Nn #1
       {
         \iow_term:x{~ ##1~ =~ \tl_use:c{ \@@_tl_csname:n {##1} }~ ->~
           \seq_use:cnnn{ \@@_seq_csname:n {##1} }{~->~}{~->~}{~->~}
         }
       }
  \iow_term:x{}
}
%    \end{macrocode}
%  \end{macro}



%  \begin{macro}{\hook_log:n}
%    This writes out information about the hook given in its argument
%    onto the terminal and the \texttt{.log} file.
%    \begin{macrocode}
\cs_new_protected:Npn \hook_log:n #1
  {
    \exp_args:Nx \@@_log:n
      { \@@_parse_label_default:nn {#1} { top-level } }
  }
\cs_new_protected:Npn \@@_log:n #1
  {
    \iow_term:x{^^JThe~ hook~ '#1':}
%    \end{macrocode}
%    
%    \begin{macrocode}
    \hook_if_exist:nF {#1}
      { \iow_term:x {~Hook~ is~ not~ declared!} }
    \@@_if_exist:nTF {#1}
      {
        \iow_term:x{~Code~ chunks:}
        \prop_if_empty:cTF {g_@@_#1_code_prop}
          { \iow_term:x{\@spaces ---} }
          {
            \prop_map_inline:cn {g_@@_#1_code_prop}
              { \iow_term:x{\@spaces ##1~ ->~ \tl_to_str:n{##2} } }
          }
%    \end{macrocode}
%    
%    \begin{macrocode}
        \iow_term:x{~Extra~ code~ next~ invocation:}
        \iow_term:x{\@spaces
          \tl_if_empty:cTF { g_@@_#1_next_code_tl }
            {---} {->~ \str_use:c{g_@@_#1_next_code_tl} } }
%    \end{macrocode}
%
%    \fmi{This is currently only displaying the local rules, but it
%         should also show the matching global rules!}
%
%    \begin{macrocode}
        \iow_term:x{~Rules:}
        \prop_if_empty:cTF {g_@@_#1_rules_prop}
          { \iow_term:x{\@spaces ---} }
          { \prop_map_inline:cn {g_@@_#1_rules_prop}
              { \iow_term:x{\@spaces ##1~ with~ relation~ ##2} }
          }
%    \end{macrocode}
%    
%    \begin{macrocode}
        \hook_if_exist:nT {#1}
          { \iow_term:x { ~Execution~ order
               \prop_if_empty:cTF {g_@@_#1_rules_prop}
                 { \@@_if_reversed:nT {#1}
                        { ~ (after~ reversal) }
                 }
                 { ~ (after~
                   \@@_if_reversed:nT {#1} {reversal~ and~}
                   applying~ rules)
                 }
               :    
              }
            \iow_term:x { \@spaces
              \clist_if_empty:cTF{g_@@_#1_labels_clist}
                 {not~ set~ yet}
                 { \clist_use:cnnn {g_@@_#1_labels_clist}
                                   { ,~ } { ,~ } { ,~ }   }
            }
          }
      }
      { \iow_term:n { ~The~hook~is~empty. } }
    \iow_term:n { }
  }
%    \end{macrocode}
%    
%  \end{macro}



%  \subsection{Specifying code for next invocation}
%
%
%
%
%
%  \begin{macro}{\hook_gput_next_code:nn}
%    
%    \begin{macrocode}
\cs_new_protected:Npn \hook_gput_next_code:nn #1
  {
    \exp_args:Nx \@@_gput_next_code:nn
      { \@@_parse_label_default:nn {#1} { top-level } }
  }
\cs_new_protected:Npn \@@_gput_next_code:nn #1 #2
  {
    \@@_declare:n {#1}
    \hook_if_exist:nTF {#1}
      { \@@_gput_next_do:nn {#1} {#2} }
      { \@@_try_declaring_generic_next_hook:nn {#1} {#2} }
  }
\cs_new_protected:Npn \@@_gput_next_do:nn #1 #2
  {
    \tl_gput_right:cn { g_@@_#1_next_code_tl }
      { #2 \tl_gclear:c { g_@@_#1_next_code_tl } }
  }
%    \end{macrocode}
%  \end{macro}
%
%
% \subsection{Using the hook}
%
% \begin{macro}{\hook_use:n}
% \begin{macro}[EXP]{\@@_use_initialized:n}
% \begin{macro}{\@@_preamble_hook:n}
%   \cs{hook_use:n} as defined here is used in the preamble, where
%   hooks aren't initialized by default.  \cs{@@_use_initialized:n} is
%   also defined, which is the non-\tn{protected} version for use within
%   the document.  Their definition is identical, except for the
%   \cs{@@_preamble_hook:n} (which wouldn't hurt in the expandable
%   version, but it would be an unnecessary extra expansion).
%
%   \cs{@@_use_initialized:n} holds the expandable definition while in
%   the preamble. \cs{@@_preamble_hook:n} initializes the hook in the
%   preamble, and is redefined to \cs{use_none:n} at |\begin{document}|.
%
%   Both versions do the same internally:  check if the hook exist as
%   given, and if so use it as quickly as possible.  If it doesn't
%   exist, the a call to \cs{@@_use:wn} checks for file hooks.
%
%   At |\begin{document}|, all hooks are initialized, and any change in
%   them causes an update, so \cs{hook_use:n} can be made expandable.
%   This one is better not protected so that it can expand into nothing
%   if containing no code. Also important in case of generic hooks that
%   we do not generate a \cs{relax} as a side effect of checking for a
%   csname. In contrast to the \TeX{} low-level
%   \verb=\csname ...\endcsname= construct \cs{tl_if_exist:c} is
%   careful to avoid this.
%    \begin{macrocode}
\cs_new_protected:Npn \hook_use:n #1
  {
    \tl_if_exist:cTF { g_@@_#1_code_tl }
      {
        \@@_preamble_hook:n {#1}
        \cs:w g_@@_#1_code_tl \cs_end:
      }
      { \@@_use:wn #1 / \s_@@_mark {#1} }
  }
\cs_new:Npn \@@_use_initialized:n #1
  {
    \tl_if_exist:cTF { g_@@_#1_code_tl }
      { \cs:w g_@@_#1_code_tl \cs_end: }
      { \@@_use:wn #1 / \s_@@_mark {#1} }
  }
\cs_new_protected:Npn \@@_preamble_hook:n #1
  { \@@_initialize_hook_code:n {#1} }
%    \end{macrocode}
% \end{macro}
% \end{macro}
% \end{macro}
%
% \begin{macro}[EXP]{\@@_use:wn}
% \begin{macro}{\@@_try_file_hook:n,\@@_if_exist_use:n}
%   \cs{@@_use:wn} does a quick check to test if the current hook is a
%   file hook: those need a special treatment.  If it is not, the hook
%   does not exist.  If it is, then \cs{@@_try_file_hook:n} is called,
%   and checks that the current hook is a file-specific hook using
%   \cs{@@_if_file_hook:wTF}.  If it's not, then it's a generic |file/|
%   hook and is used if it exist.
%
%   If it is a file-specific hook, it passes through the same
%   normalization as during declaration, and then it is used if defined.
%   \cs{@@_if_exist_use:n} checks if the hook exist, and calls
%   \cs{@@_preamble_hook:n} if so, then uses the hook.
%    \begin{macrocode}
\cs_new:Npn \@@_use:wn #1 / #2 \s_@@_mark #3
  {
    \str_if_eq:nnTF {#1} { file }
      { \@@_try_file_hook:n {#3} }
      { } % Hook doesn't exist
  }
\cs_new_protected:Npn \@@_try_file_hook:n #1
  {
    \@@_if_file_hook:wTF #1 / / \s_@@_mark
      {
        \exp_args:Ne \@@_if_exist_use:n
          { \exp_args:Ne \@@_file_hook_normalise:n {#1} }
      }
      { \@@_if_exist_use:n {#1} } % file/ generic hook (e.g. file/before)
  }
\cs_new_protected:Npn \@@_if_exist_use:n #1
  {
    \tl_if_exist:cT { g_@@_#1_code_tl }
      {
        \@@_preamble_hook:n {#1}
        \cs:w g_@@_#1_code_tl \cs_end:
      }
  }
%    \end{macrocode}
% \end{macro}
% \end{macro}
%
%  \begin{macro}{\hook_use_once:n}
%    For hooks that can and should be used only once we have a special
%    use command that remembers the hook name in
%    \cs{g_@@_execute_immediately_clist}. This has the effect that any
%    further code added to the hook is executed immediately rather
%    than stored in the hook.
%    \begin{macrocode}
\cs_new_protected:Npn \hook_use_once:n #1
  {
    \tl_if_exist:cT { g_@@_#1_code_tl }
      {
        \clist_gput_left:Nn \g_@@_execute_immediately_clist {#1}
        \hook_use:n {#1}
      }
  }
%    \end{macrocode}
%  \end{macro}

% \subsection{Querying a hook}
%
% Simpler data types, like token lists, have three possible states; they
% can exist and be empty, exist and be non-empty, and they may not
% exist, in which case emptiness doesn't apply (though
% \cs{tl_if_empty:N} returns false in this case).
%
% Hooks are a bit more complicated: they have four possible states.
% A hook may exist or not, and either way it may or may not be empty
% (even a hook that doesn't exist may be non-empty).
%
% A hook is said to be empty when no code was added to it, either to
% its permanent code pool, or to its ``next'' token list.  The hook
% doesn't need to be declared to have code added to its code pool
% (it may happen that a package $A$ defines a hook \hook{foo}, but
% it's loaded after package $B$, which adds some code to that hook.
% In this case it is important that the code added by package $B$ is
% remembered until package $A$ is loaded).
%
% A hook is said to exist when it was declared with \cs{hook_new:n} or
% some variant thereof.
%
% \begin{macro}[pTF]{\hook_if_empty:n}
%   Test if a hook is empty (that is, no code was added to that hook).
%   A hook being empty means that \emph{both} its
%   \cs[no-index]{g_@@_\meta{hook}_code_prop} and its
%   \cs[no-index]{g_@@_\meta{hook}_next_code_tl} are empty.
%    \begin{macrocode}
\prg_new_conditional:Npnn \hook_if_empty:n #1 { p , T , F , TF }
  {
    \@@_if_exist:nTF {#1}
      {
        \bool_lazy_and:nnTF
            { \prop_if_empty_p:c { g_@@_#1_code_prop } }
            { \tl_if_empty_p:c { g_@@_#1_next_code_tl } }
          { \prg_return_true: }
          { \prg_return_false: }
      }
      { \prg_return_true: }
  }
%    \end{macrocode}
% \end{macro}

% \begin{macro}[pTF]{\hook_if_exist:n}
%   A canonical way to test if a hook exists.  A hook exists if the
%   token list that stores the sorted code for that hook,
%   \cs[no-index]{g_@@_\meta{hook}_code_tl}, exists.  The property list
%   \cs[no-index]{g_@@_\meta{hook}_code_prop} cannot be used here
%   because often it is necessary to add code to a hook without knowing
%   if such hook was already declared, or even if it will ever be
%   (for example, in case the package that defines it isn't loaded).
%    \begin{macrocode}
\prg_new_conditional:Npnn \hook_if_exist:n #1 { p , T , F , TF }
  {
    \tl_if_exist:cTF { g_@@_#1_code_tl }
      { \prg_return_true: }
      { \prg_return_false: }
  }
%    \end{macrocode}
% \end{macro}

% \begin{macro}[pTF]{\@@_if_exist:n}
%   An internal check if the hook has already been declared with
%   \cs{@@_declare:n}.  This means that the hook was already used somehow
%   (a code chunk or rule was added to it), but it still wasn't declared
%   with \cs{hook_new:n}.
%    \begin{macrocode}
\prg_new_conditional:Npnn \@@_if_exist:n #1 { p , T , F , TF }
  {
    \prop_if_exist:cTF { g_@@_#1_code_prop }
      { \prg_return_true: }
      { \prg_return_false: }
  }
%    \end{macrocode}
% \end{macro}
%
% \begin{macro}[pTF]{\@@_if_reversed:n}
%   An internal conditional that checks if a hook is reversed.
%    \begin{macrocode}
\prg_new_conditional:Npnn \@@_if_reversed:n #1 { p , T , F , TF }
  {
    \if_int_compare:w \cs:w g_@@_#1_reversed_tl \cs_end: 1 < 0 \exp_stop_f:
      \prg_return_true:
    \else:
      \prg_return_false:
    \fi:
  }
%    \end{macrocode}
% \end{macro}


%  \begin{macro}{\g_@@_execute_immediately_clist}
%    List of hooks that from no on should not longer receive code.
%    \begin{macrocode}
\clist_new:N \g_@@_execute_immediately_clist
%    \end{macrocode}
%  \end{macro}
%
%  \subsection{Messages}
%
%    \begin{macrocode}
\msg_new:nnnn { hooks } { labels-incompatible }
  {
    Labels~`#1'~and~`#2'~are~incompatible
    \str_if_eq:nnF {#3} {??} { ~in~hook~`#3' } .~
    \int_compare:nNnT {#4} = { 1 }
      { The~code~for~both~labels~will~be~dropped. }
  }
  {
    LaTeX~found~two~incompatible~labels~in~the~same~hook.~
    This~indicates~an~incompatibility~between~packages.
  }
\msg_new:nnn { hooks } { empty-label }
  { Empty~code~label~\msg_line_context:.~Using~`#1'~instead. }
%    \end{macrocode}
%
%  \subsection{\LaTeXe{} package interface commands}
%


%  \begin{macro}{\NewHook,\NewReversedHook,\NewMirroredHookPair}
%    Declaring new hooks \ldots
%    \begin{macrocode}
\NewDocumentCommand \NewHook             { m }{ \hook_new:n {#1} }
\NewDocumentCommand \NewReversedHook     { m }{ \hook_new_reversed:n {#1} }
\NewDocumentCommand \NewMirroredHookPair { mm }{ \hook_new_pair:nn {#1}{#2} }
%    \end{macrocode}
%  \end{macro}

%  \begin{macro}{\AddToHook}
%    
%    \begin{macrocode}
\NewDocumentCommand \AddToHook { m o +m }
  {
    \clist_if_in:NnTF \g_@@_execute_immediately_clist {#1}
      {#3}
      { \hook_gput_code:nnn {#1} {#2} {#3} }
  }
%    \end{macrocode}
%  \end{macro}

%  \begin{macro}{\AddToHookNext}
%    
%    \begin{macrocode}
\NewDocumentCommand \AddToHookNext { m +m }
  { \hook_gput_next_code:nn {#1} {#2} }
%    \end{macrocode}
%  \end{macro}


%  \begin{macro}{\RemoveFromHook}
%    
%    \begin{macrocode}
\NewDocumentCommand \RemoveFromHook { m o }
  { \hook_gremove_code:nn {#1} {#2} }
%    \end{macrocode}
%  \end{macro}
%
% \begin{macro}{\DeclareDefaultHookLabel}
% \begin{macro}{\@@_curr_name_push:n,\@@_curr_name_pop:}
%   The token list \cs{g_@@_hook_curr_name_tl} stores the name of the
%   current package/file to be used as label for hooks.
%   Providing a consistent interface is tricky, because packages can
%   be loaded within packages, and some packages may not use
%   \cs{DeclareDefaultHookLabel} to change the default label (in which case
%   \cs{@currname} is used, if set).
%
%   To pull that off, we keep a stack that contains the default label
%   for each level of input.  The bottom of the stack contains the
%   default label for the top-level.  Since the string \verb|top-level|
%   is hardcoded, here this item of the stack is empty.  Also, since
%   we're in an input level, add \verb|lthooks| to the stack as well.
%   This stack should never go empty, so we loop through \LaTeXe's
%   file name stack, and add empty entries to \cs{g_@@_name_stack_seq}
%   for each item in that stack.  The last item is the \verb|top-level|,
%   which also gets an empty entry.
%
%   Also check for the case we're loading \texttt{lthooks} in the
%   \LaTeXe{} kernel.  In that case, \cs{@currname} isn't \verb|lthooks|
%   and just the top-level is added to the stack as an empty entry.
%    \begin{macrocode}
\str_if_eq:VnTF \@currname { lthooks }
  {
    \seq_gpush:Nn \g_@@_name_stack_seq { lthooks }
    \cs_set_protected:Npn \@@_tmp:w #1 #2 #3
      {
        \quark_if_recursion_tail_stop:n {#1}
        \seq_gput_right:Nn \g_@@_name_stack_seq { }
        \@@_tmp:w
      }
    \exp_after:wN \@@_tmp:w
      \@currnamestack
      \q_recursion_tail \q_recursion_tail
      \q_recursion_tail \q_recursion_stop
  }
  { \seq_gpush:Nn \g_@@_name_stack_seq { } }
%    \end{macrocode}
%
%   Two commands keep track of the stack: when a file is input,
%   \cs{@@_curr_name_push:n} pushes an (empty by default) label to the
%   stack:
%    \begin{macrocode}
\cs_new_protected:Npn \@@_curr_name_push:n #1
  {
    \seq_gpush:Nn \g_@@_name_stack_seq {#1}
    \tl_gset:Nn \g_@@_hook_curr_name_tl {#1}
  }
%
%    \end{macrocode}
%   and when an input is over, the topmost item of the stack is popped,
%   since the label will not be used again, and \cs{g_@@_hook_curr_name_tl}
%   is updated to the now topmost item of the stack:
%    \begin{macrocode}
\cs_new_protected:Npn \@@_curr_name_pop:
  {
    \seq_gpop:NN \g_@@_name_stack_seq \l_@@_return_tl
    \seq_get:NNTF \g_@@_name_stack_seq \l_@@_return_tl
      { \tl_gset_eq:NN \g_@@_hook_curr_name_tl \l_@@_return_tl }
      { \ERROR_should_not_happen }
  }
%    \end{macrocode}
%
%   The token list \cs{g_@@_hook_curr_name_tl} is but a mirror of the top
%   of the stack.
%
%   Now define a wrapper that replaces the top of the stack with the
%   argument, and updates \cs{g_@@_hook_curr_name_tl} accordingly.
%    \begin{macrocode}
\NewDocumentCommand \DeclareDefaultHookLabel { m }
  {
    \seq_gpop:NN \g_@@_name_stack_seq \l_@@_return_tl
    \@@_curr_name_push:n {#1}
  }
%    \begin{macrocode}
%
%   The push and pop macros are injected in \cs{@pushfilename} and
%   \cs{@popfilename} so that they correctly keep track of the label.s
%    \begin{macrocode}
% TODO! \pho{Properly integrate in the kernel}
\tl_gput_left:Nn \@pushfilename { \@@_curr_name_push:n { } }
\tl_gput_left:Nn \@popfilename { \@@_curr_name_pop: }
% TODO! \pho{Properly integrate in the kernel}
%    \end{macrocode}
% \end{macro}
% \end{macro}
%
%
%
%
%  \begin{macro}{\UseHook}
%    Avoid the overhead of \pkg{xparse} and its protection that we
%    don't want here (since the hook should vanish without trace if empty)!
%    \begin{macrocode}
\newcommand \UseHook { \hook_use:n }
%    \end{macrocode}
%  \end{macro}
%
%  \begin{macro}{\UseOneTimeHook}
%
%    \begin{macrocode}
\cs_new_protected:Npn \UseOneTimeHook { \hook_use_once:n }
%    \end{macrocode}
%  \end{macro}



%  \begin{macro}{\ShowHook}
%    
%    \begin{macrocode}
\cs_new_protected:Npn \ShowHook { \hook_log:n }
%    \end{macrocode}
%  \end{macro}

%  \begin{macro}{\DebugHookOn,\DebugHookOff}
%    
%    \begin{macrocode}
\cs_new_protected:Npn \DebugHookOn { \hook_debug_on: }
\cs_new_protected:Npn \DebugHookOff { \hook_debug_off: }
%    \end{macrocode}
%  \end{macro}



%  \begin{macro}{\DeclareHookRule}
%    
%    \begin{macrocode}
\NewDocumentCommand \DeclareHookRule { m m m m }
{ \hook_gset_rule:nnnn {#1}{#2}{#3}{#4} }
%    \end{macrocode}
%  \end{macro}

%  \begin{macro}{\DeclareDefaultHookRule}
%    This declaration is only supported before \verb=\begin{document}=.
%    \begin{macrocode}
\NewDocumentCommand \DeclareDefaultHookRule { m m m }
                    { \hook_gset_rule:nnnn {??}{#1}{#2}{#3} }
\@onlypreamble\DeclareDefaultHookRule                    
%    \end{macrocode}
%  \end{macro}
%
%  \begin{macro}{\ClearHookRule}
%    A special setup rule that removes an existing relation.
%    Basically {@@_rule_gclear:nnn} plus fixing the property list for debugging.
%    \fmi{Need an L3 interface, or maybe it should get dropped}
%    \begin{macrocode}
\NewDocumentCommand \ClearHookRule { m m m }
{ \hook_gset_rule:nnnn {#1}{#2}{unrelated}{#3} }
%    \end{macrocode}
%  \end{macro}
%
%
% \begin{macro}{\IfHookExistTF,\IfHookEmptyTF}
%    \begin{macrocode}
\NewExpandableDocumentCommand \IfHookExistTF { m }
  { \hook_if_exist:nTF {#1} }
\NewExpandableDocumentCommand \IfHookEmptyTF { m }
  { \hook_if_empty:nTF {#1} }
%    \end{macrocode}
% \end{macro}




%  \begin{macro}{\AtBeginDocument}
%    
%    \begin{macrocode}
\renewcommand\AtBeginDocument{\AddToHook{begindocument}}
%    \end{macrocode}
%  \end{macro}

%  \begin{macro}{\AtEndDocument}
%    
%    \begin{macrocode}
\renewcommand\AtEndDocument {\AddToHook{enddocument}}
%\renewcommand\AtEndDocument {\AddToHook{env/document/end}} % alternative impl
%    \end{macrocode}
%    
%  \end{macro}



%  \subsection{Set up existing \LaTeXe{} hooks}
%
%    As we are in a package calling \cs{NewHook} would label any
%    already set up hook code under the package name, but we want it
%    under the name \hook{top-level} so we pretend that \cs{@currname}
%    is empty.
%    \begin{macrocode}
\begingroup
  \def\@currname{}
%    \end{macrocode}
%    
%    \begin{macrocode}
  \NewHook{begindocument}
  \NewHook{enddocument}
%    \end{macrocode}
%    We need to initialize the mechanism at \verb=\begin{document}=
%    but obviously before everything else, so we sneak\footnote{This
%    needs to move to \cs{document} directly.}
%    \cs{@@_initialize_all:} into the \LaTeXe{} hook name.
%
%    We can't use \cs{tl_gput_left:Nn} because that complains about
%    \cs{@begindocumenthook} not starting with \cs{g_} so we do this
%    through the backdoor.
%    \begin{macrocode}
%  \tex_global:D\tl_put_left:Nn \@begindocumenthook
%      {\@@_initialize_all:}
%    \end{macrocode}
%    There aren't many other hooks at the moment:
%    \begin{macrocode}
  \NewHook{rmfamily}
  \NewHook{sffamily}
  \NewHook{ttfamily}
  \NewHook{defaultfamily}
%    \end{macrocode}
%    Not checked what this one does and whether it should be there (or
%    is a real ``hook''.
%    \begin{macrocode}
  \NewHook{documentclass}
%    \end{macrocode}
%    
%    \begin{macrocode}
\endgroup
%    \end{macrocode}
%
%
%
% \section{Generic hooks for environments}
%
%
%    \begin{macrocode}
\let\begin\relax  % avoid redeclaration message
%    \end{macrocode}
%    
%    \begin{macrocode}
\DeclareRobustCommand\begin[1]{%
  \UseHook{env/#1/before}%
  \@ifundefined{#1}%
    {\def\reserved@a{\@latex@error{Environment~#1~undefined}\@eha}}%
    {\def\reserved@a{\def\@currenvir{#1}%
        \edef\@currenvline{\on@line}%
        \@execute@begin@hook{#1}%
        \csname #1\endcsname}}%
  \@ignorefalse
  \begingroup\@endpefalse\reserved@a}
%    \end{macrocode}
%
%    Before the \cs{document} code is executed we have to first undo
%    the \cs{endgroup} as there should be none for this environment to
%    avoid that changes on top-level unnecessarily go to \TeX's
%    savestack, and we have to initialize all hooks in the hook system.
%    So we need to test for this environment name. But once it has be
%    found all this testing is no longer needed and so we redefine
%    \cs{@execute@begin@hook} to simply use the hook
%    \begin{macrocode}
\def\@execute@begin@hook #1{%
  \expandafter\ifx\csname #1\endcsname\document
    \endgroup
    \gdef\@execute@begin@hook##1{\UseHook{env/##1/begin}}%
    \@@_initialize_all:
    \@execute@begin@hook{#1}%
%    \end{macrocode}
%    If this is an environment before \verb=\begin{document}= we just
%    run the hook.
%    \begin{macrocode}
  \else
    \UseHook{env/#1/begin}%
  \fi
}    
%    \end{macrocode}
%    
%    \begin{macrocode}
\@namedef{end~}#1{%
  \UseHook{env/#1/end}%
  \csname end#1\endcsname\@checkend{#1}%
  \expandafter\endgroup\if@endpe\@doendpe\fi
  \UseHook{env/#1/after}%
  \if@ignore\@ignorefalse\ignorespaces\fi}%
%    \end{macrocode}
%    Version that fixes tlb3722 but the change should perhaps be made in
%    \pkg{tabularx} instead.
%    \begin{macrocode}
\@namedef{end~}#1{%
\romannumeral
\IfHookEmptyTF{env/#1/end}%
  {\expandafter\z@}%
  {\z@\UseHook{env/#1/end}}%
\csname end#1\endcsname\@checkend{#1}%
\expandafter\endgroup\if@endpe\@doendpe\fi
\UseHook{env/#1/after}%
\if@ignore\@ignorefalse\ignorespaces\fi}%
%    \end{macrocode}
%    
%    
%
%    We provide 4 high-level hook interfaces directly, the others only when
%    etoolbox is loaded
%    \begin{macrocode}
\newcommand\AtBeginEnvironment[1]    {\AddToHook{env/#1/begin}}
\newcommand\AtEndEnvironment[1]      {\AddToHook{env/#1/end}}
\newcommand\BeforeBeginEnvironment[1]{\AddToHook{env/#1/before}}
\newcommand\AfterEndEnvironment[1]   {\AddToHook{env/#1/after}}
%    \end{macrocode}
%    
%    
%    
%    
%    
%    
%    
%    
%    
%    
%
% \section{Generic hooks for file loads}
%
%
%
%
%
%    
%
% \section{Hooks in \cs{begin} document}
%
%    Can't have \texttt{@{}@} notation here as this is \LaTeXe{} code
%    \ldots{} and makes for puzzling errors if the double \texttt{@}
%    signs get substituted.
%    \begin{macrocode}
%<@@=>
\ExplSyntaxOff
%    \end{macrocode}
%
%    The \hook{begindocument} hook was already set up earlier, here is now
%    the additional one (which was originally from the \pkg{etoolbox}
%    package under the name \texttt{afterpreamble}.
%    \begin{macrocode}
\NewHook{begindocument/end}
%    \end{macrocode}
%
%

%  \begin{macro}{\document}
%    
%    \begin{macrocode}
\def\document{%
%    \end{macrocode}
%    We do cancel the grouping as part of the \cs{begin} handling
%    (this is now done inside \cs{begin} instead) so that the
%    \hook{env/\meta{env}/begin} hook is not hidden inside \cs{begingroup}
%    \texttt{...} \cs{endgroup}.
%    \begin{macrocode}
%  \endgroup
%    \end{macrocode}
%    
%    \begin{macrocode}
  \@kernel@after@env@document@begin
%    \end{macrocode}
%
% Added hook to load \textsf{l3backend} code:
%    \begin{macrocode}
  \@expl@sys@load@backend@@
  \ifx\@unusedoptionlist\@empty\else
    \@latex@warning@no@line{Unused global option(s):^^J%
            \@spaces[\@unusedoptionlist]}%
  \fi
  \@colht\textheight
  \@colroom\textheight \vsize\textheight
  \columnwidth\textwidth
  \@clubpenalty\clubpenalty
  \if@twocolumn
    \advance\columnwidth -\columnsep
    \divide\columnwidth\tw@ \hsize\columnwidth \@firstcolumntrue
  \fi
  \hsize\columnwidth \linewidth\hsize
  \begingroup\@floatplacement\@dblfloatplacement
    \makeatletter\let\@writefile\@gobbletwo
    \global \let \@multiplelabels \relax
    \@input{\jobname.aux}%
  \endgroup
  \if@filesw
    \immediate\openout\@mainaux\jobname.aux
    \immediate\write\@mainaux{\relax}%
  \fi
  \process@table
  \let\glb@currsize\@empty  % Force math initialization.
  \normalsize
  \everypar{}%
  \ifx\normalsfcodes\@empty
    \ifnum\sfcode`\.=\@m
      \let\normalsfcodes\frenchspacing
    \else
      \let\normalsfcodes\nonfrenchspacing
    \fi
  \fi
  \ifx\document@default@language\m@ne
    \chardef\document@default@language\language
  \fi
  \@noskipsecfalse
  \let \@refundefined \relax
%    \end{macrocode}
%    
%    \begin{macrocode}
%  \let\AtBeginDocument\@firstofone
%  \@begindocumenthook
  \UseOneTimeHook{begindocument}%
  \@kernel@after@begindocument
%    \end{macrocode}
%    
%    \begin{macrocode}
  \ifdim\topskip<1sp\global\topskip 1sp\relax\fi
  \global\@maxdepth\maxdepth
  \global\let\@begindocumenthook\@undefined
  \ifx\@listfiles\@undefined
    \global\let\@filelist\relax
    \global\let\@addtofilelist\@gobble
  \fi
  \gdef\do##1{\global\let ##1\@notprerr}%
  \@preamblecmds
  \global\let \@nodocument \relax
  \global\let\do\noexpand
%    \end{macrocode}
%    
%    \begin{macrocode}
  \UseOneTimeHook{begindocument/end}%
  \ignorespaces}
%    \end{macrocode}
%    
%    \begin{macrocode}
\let\@kernel@after@begindocument\@empty
%    \end{macrocode}
%  \end{macro}
%
%
%  \begin{macro}{\@kernel@after@env@document@begin,\@kernel@hook@begindocument}
%    
%    \begin{macrocode}
\edef \@kernel@after@env@document@begin{%
  \let\expandafter\noexpand\csname
       g__hook_env/document/begin_code_tl\endcsname
  \noexpand\@empty}
%    \end{macrocode}
%    \begin{macrocode}
\let\@kernel@hook@begindocument\@empty
%    \end{macrocode}
%    
%  \end{macro}
%
%
% \section{Hooks in \cs{enddocument}}
%
%
%    
%    The \hook{enddocument} hook was already set up earlier, here are now
%    the additional ones:
%    \begin{macrocode}
\NewHook{enddocument/afterlastpage}
\NewHook{enddocument/afteraux}
\NewHook{enddocument/info}
\NewHook{enddocument/end}
%    \end{macrocode}



%  \begin{macro}{\enddocument}
%    
%    
%    \begin{macrocode}
\def\enddocument{%
   \UseHook{enddocument}%
   \@kernel@after@enddocument
   \@checkend{document}%
   \clearpage
   \UseHook{enddocument/afterlastpage}%
   \@kernel@after@enddocument@afterlastpage
   \begingroup
     \if@filesw
       \immediate\closeout\@mainaux
       \let\@setckpt\@gobbletwo
       \let\@newl@bel\@testdef
       \@tempswafalse
       \makeatletter \@@input\jobname.aux
     \fi
     \UseHook{enddocument/afteraux}%
%    \end{macrocode}
%    Next hook is expect to contain only code for writing info
%    messages on the terminal.
%    \begin{macrocode}
     \UseHook{enddocument/info}%
   \endgroup
   \UseHook{enddocument/end}%
   \deadcycles\z@\@@end}
%    \end{macrocode}
%    The two kernel hooks above are used by the shipout code.   
%    \begin{macrocode}
\let\@kernel@after@enddocument\@empty
\let\@kernel@after@enddocument@afterlastpage\@empty
%    \end{macrocode}
%  \end{macro}
%
%
%
%  \begin{macro}{\@enddocument@kernel@warnings}
%    
%    \begin{macrocode}
\def\@enddocument@kernel@warnings{%
   \ifdim \font@submax >\fontsubfuzz\relax
     \@font@warning{Size substitutions with differences\MessageBreak
                up to \font@submax\space have occurred.\@gobbletwo}%
   \fi
   \@defaultsubs
   \@refundefined
   \if@filesw
     \ifx \@multiplelabels \relax
       \if@tempswa
         \@latex@warning@no@line{Label(s) may have changed.
             Rerun to get cross-references right}%
       \fi
     \else
       \@multiplelabels
     \fi
   \fi
}
%    \end{macrocode}
%    
%    \begin{macrocode}
\AddToHook{enddocument/info}[kernel/filelist]{\@dofilelist}
\AddToHook{enddocument/info}[kernel/warnings]{\@enddocument@kernel@warnings}
\DeclareHookRule{enddocument/info}{kernel/filelist}{before}{kernel/warnings}
%    \end{macrocode}
%  \end{macro}
%
%
% \subsection{Adjusting at \pkg{atveryend} interfaces}
%
%    With the new hook management all of \pkg{atveryend} is taken care
%    of.
%
%    We therefore prevent the package from loading:
%    \begin{macrocode}
\expandafter\let\csname ver@atveryend.sty\endcsname\fmtversion
%    \end{macrocode}
%
%
%    Here are new definitions for its interfaces now pointing to the
%    hooks in \cs{enddocument}
%    \begin{macrocode}
\newcommand\AfterLastShipout  {\AddToHook{enddocument/afterlastpage}}
\newcommand\AtVeryEndDocument {\AddToHook{enddocument/afteraux}}
%    \end{macrocode}
%    Next one is a bit of a fake, but the result should normally be as
%    expected. If not one needs to add a rule to sort the code chunks
%    in \hook{enddocument/info}.
%    \begin{macrocode}
\newcommand\AtEndAfterFileList{\AddToHook{enddocument/info}}
%    \end{macrocode}
%    
%    \begin{macrocode}
\newcommand\AtVeryVeryEnd     {\AddToHook{enddocument/end}}
%    \end{macrocode}

%  \begin{macro}{\BeforeClearDocument}
%    This one is the only one we don't implement or rather don't have
%    a dedicated hook in the code.
%    \begin{macrocode}
\ExplSyntaxOn
\newcommand\BeforeClearDocument[1]
  { \AtEndDocument{#1}
    \@DEPRECATED{BeforeClearDocument \tl_to_str:n{#1}}
  }
\cs_new:Npn\@DEPRECATED #1
   {\iow_term:x{======~DEPRECATED~USAGE~#1~==========}}
%    \end{macrocode}
%    
%    \begin{macrocode}
\ExplSyntaxOff
%    \end{macrocode}
%  \end{macro}
%
%    \begin{macrocode}
%</2ekernel>
%    \end{macrocode}

%
% \section{A package version of the code for testing}
%

%    \begin{macrocode}
%<*package>
%    \end{macrocode}



%    \begin{macrocode}
\RequirePackage{xparse}
\ProvidesExplPackage{lthooks}{\lthooksdate}{\lthooksversion}
                    {Hook management interface for LaTeX2e}
%    \end{macrocode}
%
%
% \subsection{Core hook management code (kernel part)}
%
%    This should run in older formats so we can't use
%    \cs{IfFormatAtLeastTF} right now.
%    \begin{macrocode}
\@ifl@t@r\fmtversion{2020/10/01}
                    {}
                    {% \iffalse meta-comment
%
%% File: lthooks.dtx (C) Copyright 2020 Frank Mittelbach, LaTeX Team
%
% It may be distributed and/or modified under the conditions of the
% LaTeX Project Public License (LPPL), either version 1.3c of this
% license or (at your option) any later version.  The latest version
% of this license is in the file
%
%    https://www.latex-project.org/lppl.txt
%
%
% The development version of the bundle can be found below
%
%    https://github.com/FrankMittelbach/...
%
% for those people who are interested or want to report an issue.
%
%    \begin{macrocode}
\providecommand\lthooksversion{v0.9a}
\providecommand\lthooksdate{2020/07/15}
%    \end{macrocode}
%
%<*driver>
\RequirePackage[debug]{lthooks}

\documentclass{l3doc}

% bug fix fo l3doc.cls
\ExplSyntaxOn
\cs_set_protected:Npn \__codedoc_macro_typeset_one:nN #1#2
  {
    \vbox_set:Nn \l__codedoc_macro_box
      {
        \vbox_unpack_drop:N \l__codedoc_macro_box
        \hbox { \llap { \__codedoc_print_macroname:nN {#1} #2
            \MacroFont       % <----- without it the \ is in lmr10 if a link is made
            \      
        } }
      }
    \int_incr:N \l__codedoc_macro_int
  }
\ExplSyntaxOff

\EnableCrossrefs
\CodelineIndex
\begin{document}
  \DocInput{lthooks.dtx}
\end{document}
%</driver>
%
% \fi
%
%
% \long\def\fmi#1{\begin{quote}\itshape FMi: #1\end{quote}}
% \long\def\pho#1{\begin{quote}\itshape PhO: #1\end{quote}}
%
% \newcommand\hook[1]{\texttt{#1}}
%    
%
% \title{The \texttt{lthooks} package\thanks{This package has version
%    \lthooksversion\ dated \lthooksdate, \copyright\ \LaTeX\
%    Project.}}
%
% \author{Frank Mittelbach\thanks{Code improvements for speed and other goodies by Phelype Oleinik}}
%
% \maketitle
%
%
% \tableofcontents
%
% \section{Introduction}
%
%    Hooks are points in the code of commands or environments where it
%    is possible to add processing code into existing commands. This
%    can be done by different packages that do not know about each
%    other and to allow for hopefully safe processing it is necessary
%    to sort different chunks of code added by different packages into
%    a suitable processing order.
%
%    This is done by the packages adding chunks of code (via
%    \cs{AddToHook}) and labeling their code with some label by
%    default using the package name as a label.
%
%    At \verb=\begin{document}= all code for a hook is then sorted
%    according to some rules (given by \cs{DeclareHookRule}) for fast
%    execution without processing overhead. If the hook code is
%    modified afterwards (or the rules are changed),
%    a new version for fast processing is generated.
%
%    Some hooks are used already in the preamble of the document. If
%    that happens then the hook is prepared for execution (and sorted)
%    already at that point.
%
%
% \section{Package writer interface}
%
%    The hook management system is offered as a set of CamelCase
%    commands for traditional \LaTeXe{} packages (and for use in the
%    document preamble if needed) as well as \texttt{expl3} commands
%    for modern packages, that use the L3 programming layer of
%    \LaTeX{}. Behind the scenes, a single set of data structures is
%    accessed so that packages from both worlds can coexist and access
%    hooks in other packages.
%
%
%
% \subsection{\LaTeXe\ interfaces}
%
% \subsubsection{Declaring hooks and using them in code}
%
%    With two exceptions, hooks have to be declared before they can be
%    used. The exceptions are hooks in environments (i.e., executed at
%    \cs{begin} and \cs{end}) and hooks run when loading files,
%    e.g. before and after a package is loaded, etc. Their hook names
%    depend on the environment or the file name and so declaring them
%    beforehand is difficult.
%
%
% \begin{function}{\NewHook}
%   \begin{syntax}
%     \cs{NewHook} \Arg{hook}
%   \end{syntax}
%   Creates a new \meta{hook}.
%    If this is a hook provided as part of a package it is suggested
%    that the \meta{hook} name is always structured as follows:
%    \meta{package-name}\texttt{/}\meta{hook-name}. If necessary you
%    can further subdivide the name by adding more \texttt{/} parts.
%    If a hook name is already taken, an error is raised and the hook
%    is not created.
%
%    The \meta{hook} can be specified using the dot-syntax to denote
%    the current package name. See section~\ref{sec:default-label}.
% \end{function}
%
% \begin{function}{\NewReversedHook}
%   \begin{syntax}
%     \cs{NewReversedHook} \Arg{hook}
%   \end{syntax}
%     Like \cs{NewHook} declares a new \meta{hook}.
%     the difference is that the code chunks for this hook are in
%     reverse order by default (those added last are executed first).
%     Any rules for the hook are applied after the default ordering.
%     See sections~\ref{sec:order} and \ref{sec:reversed-order}
%    for further details.
%
%    The \meta{hook} can be specified using the dot-syntax to denote
%    the current package name. See section~\ref{sec:default-label}.
% \end{function}
%
%
% \begin{function}{\NewMirroredHookPair}
%   \begin{syntax}
%     \cs{NewMirroredHookPair} \Arg{hook-1} \Arg{hook-2}
%   \end{syntax}
%     A shorthand for
%    \cs{NewHook}\Arg{hook-1}\cs{NewReversedHook}\Arg{hook-2}.
%
%    The \meta{hooks} can be specified using the dot-syntax to denote
%    the current package name. See section~\ref{sec:default-label}.
% \end{function}
%



% \begin{function}{\UseHook}
%   \begin{syntax}
%     \cs{UseHook} \Arg{hook}
%   \end{syntax}
%    Execute the hook code inside a command or environment.\footnote{For
%     legacy hooks such as \hook{begindocument} it is also
%    possible to call \cs{@...hook}, e.g., \cs{@begindocumenthook},
%    but this syntax is discouraged.}
%
%    Before \verb=\begin{document}= the fast execution code for a hook
%    is not set up, so in order to use a hook there it is explicitly
%    initialized first. As that involves assignments using a hook at
%    those times is not 100\% the same as using it after
%    \verb=\begin{document}=.
%
%    The \meta{hook} \emph{cannot} be specified using the dot-syntax.
%    A leading |.| is treated literally.
% \end{function}
%
% \begin{function}{\UseOneTimeHook}
%   \begin{syntax}
%     \cs{UseOneTimeHook} \Arg{hook}
%   \end{syntax}
%    Some hooks are only used (and can be only used) in one place, for
%    example, those in \verb=\begin{document}= or
%    \verb=\end{document}=. Once we have passed that point adding to
%    the hook through a defined \cs{\meta{addto-cmd}} command (e.g.,
%    \cs{AddToHook} or \cs{AtBeginDocument}, etc.\@) would have no
%    effect (as would the use of such a command inside the hook code
%    itself). It is therefore customary to redefine
%    \cs{\meta{addto-cmd}} to simply  process its argument, i.e.,
%    essentially make it behave like \cs{@firstofone}.
%
%    \cs{UseOneTimeHook} does that: it records that the hook has been
%    consumed and any further attempt to add to it will result in
%    executing the code to be added immediately.
%
%    \fmi{Maybe add an error version as well?}
%
%    The \meta{hook} \emph{cannot} be specified using the dot-syntax.
%    A leading |.| is treated literally.
% \end{function}
%
%
% \subsubsection{Updating code for hooks}
%
% \begin{function}{\AddToHook}
%   \begin{syntax}
%     \cs{AddToHook} \Arg{hook}\oarg{label}\Arg{code}
%   \end{syntax}
%    Adds \meta{code} to the \meta{hook} labeled by \meta{label}. If
%    the optional argument \meta{label} is not provided, if \cs{AddToHook}
%    is used in a package/class, then the current
%    package/class name is used, otherwise \hook{top-level} is
%    used~(see section~\ref{sec:default-label}).
%
%    If there already exists code under the \meta{label} then the new
%    \meta{code} is appended to the existing one (even if this is a reversed hook).
%    If you want to replace existing code under the
%    \meta{label}, first apply \cs{RemoveFromHook}.
%
%    The hook doesn't have to exist for code to be added to
%    it. However, if it is not declared later then obviously the
%    added \meta{code} will never be executed.  This
%    allows for hooks to work regardless of package loading order and
%    enables packages to add to hook of other packages without
%    worrying whether they are actually used in the current document.
%    See section~\ref{sec:querying}.
%
%    The \meta{hook} and \meta{label} can be specified using the
%    dot-syntax to denote the current package name.
%    See section~\ref{sec:default-label}.
% \end{function}
%
% \begin{function}{\RemoveFromHook}
%   \begin{syntax}
%     \cs{RemoveFromHook} \Arg{hook}\oarg{label}
%   \end{syntax}
%    Removes any code labeled by \meta{label} from the \meta{hook}.
%    If the optional
%    argument \meta{label} is not provided, if \cs{AddToHook}
%    is used in a package/class, then the current
%    package/class name is used, otherwise \hook{top-level} is used.
%
%    If the optional argument is \texttt{*}, then all code chunks are
%    removed. This is rather dangerous as it drops code from other
%    packages one may not know about!
%
%    The \meta{hook} and \meta{label} can be specified using the
%    dot-syntax to denote the current package name.
%    See section~\ref{sec:default-label}.
% \end{function}
%
% \medskip
%
% A useful application for this declaration inside the document body
% is when one wants to temporarily add code to hooks and later remove
% it again, e.g.,
%\begin{verbatim}
%   \AddToHook{env/quote/before}{\small}
%   \begin{quote}
%     A quote set in a smaller typeface
%   \end{quote}
%   ...
%   \RemoveFromHook{env/quote/before}
%   ... now back to normal for further quotes
%\end{verbatim}
% Note that you can't cancel the setting with
%\begin{verbatim}
%   \AddToHook{env/quote/before}{}
%\end{verbatim}
% because that only \enquote{adds} a further empty chunk of code to
% the hook. Adding \cs{normalsize} would work but that means the hook
% then contained \cs{small}\cs{normalsize} which means to font size
% changes for no good reason.
%
% The above is only needed if one wants to typeset several quotes in a
% smaller typeface. If the hook is only needed once then
% \cs{AddToHookNext} is simpler, because it resets itself after one use.
%
%
% \begin{function}{\AddToHookNext}
%   \begin{syntax}
%     \cs{AddToHookNext} \Arg{hook}\Arg{code}
%   \end{syntax}
%    Adds \meta{code} to the next invocation of the \meta{hook}.
%    The code is executed after the normal hook code has finished and
%    it is executed only once, i.e. it is deleted after it was used.
%
%    Using the declaration is a global operation, i.e., the code is
%    not lost, even if the declaration is used inside a group and the
%    next invocation happens after the group. If the declaration is
%    used several times before the hook is executed then all code is
%    executed in the order in which it was declared.\footnotemark
%
%    The hook doesn't have to exist for code to be added to it.  This
%    allows for hooks to work regardless of package loading order.
%    See section~\ref{sec:querying}.
%
%    The \meta{hook} can be specified using the dot-syntax to denote
%    the current package name.  See section~\ref{sec:default-label}.
% \end{function}\footnotetext{There is
%    no mechanism to reorder such code chunks (or delete them).}
%
% \subsubsection{Hook names and default labels}
% \label{sec:default-label}
%
% It is best practice to use \cs{AddToHook} in packages or classes
% \emph{without specifying a \meta{label}} because then the package
% or class name is automatically used, which is helpful if rules are
% needed, and avoids mistyping the \meta{label}.
%
% Using an explicit \meta{label} is only necessary in very specific
% situations, e.g., if you want to add several chunks of code into a
% single hook and have them placed in different parts of the hook
% (by providing some rules).
%
% The other case is when you develop a larger package with several
% sub-packages. In that case you may want to use the same
% \meta{label} throughout the sub-packages in order to avoid
% that the labels change if you internally reorganize your code.
%
% It is not enforced, but highly recommended that the hooks defined by
% a package, and the \meta{labels} used to add code to other hooks
% contain the package name to easily identify the source of the code
% chunk and to prevent clashes.  This should be the standard practice,
% so this hook management code provides a shortcut to refer to the
% current package in the name of a \meta{hook} and in a \meta{label}.
% If \meta{hook} name or \meta{label} consist just of a single dot
% (|.|), or starts with a dot followed by a slash (|./|) then the dot
% denotes the \meta{default label} (usually the current package or class
% name---see~\cs{DeclareDefaultHookLabel}).
% A \enquote{|.|} or \enquote{|./|} anywhere else in a \meta{hook} or in
% \meta{label} is treated literally and is not replaced.

% For example,
% inside the package \texttt{mypackage.sty}, the default label is
% \texttt{mypackage}, so the instructions:
% \begin{verbatim}
%   \NewHook   {./hook}
%   \AddToHook {./hook}[.]{code}     % Same as \AddToHook{./hook}{code}
%   \AddToHook {./hook}[./sub]{code}
%   \DeclareHookRule{begindocument}{.}{<}{babel}
%   \AddToHook {file/after/foo.tex}{code}
% \end{verbatim}
%    are equivalent to:
% \begin{verbatim}
%   \NewHook   {mypackage/hook}[mypackage]{code}
%   \AddToHook {mypackage/hook}[mypackage]{code}
%   \AddToHook {mypackage/hook}[mypackage/sub]{code}
%   \DeclareHookRule{begindocument}{mypackage}{<}{babel}
%   \AddToHook {file/after/foo.tex}{code}                  % unchanged
% \end{verbatim}
%
% The \meta{default label} is automatically set to the name of the
% current package or class (using \cs{@currname}).  If \cs{@currname}
% is not set (because the hook command is used outside of a package, or
% the current file wasn't loaded with \cs{usepackage} or
% \cs{documentclass}), then the \texttt{top-level} is used as the
% \meta{default label}.
%
% This syntax is available in all \meta{label} arguments and most
% \meta{hook}, both in the \LaTeXe{} interface, and the \LaTeX3
% interface described in section~\ref{sec:l3hook-interface}.
%
% Note, however, that the replacement of |.| by the \meta{default label}
% takes place when the hook command is executed, so actions that are
% somehow executed after the package ends will have the wrong
% \meta{default label} if the dot-syntax is used.  For that reason,
% this syntax is not available in \cs{UseHook} (and \cs{hook_use:n})
% because the hook is most of the time used outside of the package file
% in which it was defined. This syntax is also not available in the hook
% conditionals \cs{IfHookEmptyTF} (and \cs{hook_if_empty:nTF}) and
% \cs{IfHookExistTF} (and \cs{hook_if_exist:nTF}) because these
% conditionals are used in some performance-critical parts of the hook
% management code, and because they are usually used to refer to other
% package's hooks, so the dot-syntax doesn't make much sense.
%
% In some cases, for example in large packages, one may want to separate
% it in logical parts, but still use the main package name as
% \meta{label}, then the \meta{default label} can be set using
% \cs{DeclareDefaultHookLabel}:
%
% \begin{function}{\DeclareDefaultHookLabel}
%   \begin{syntax}
%     \cs{DeclareDefaultHookLabel} \Arg{default label}
%   \end{syntax}
%   Sets the \meta{default label} to be used in \meta{label} arguments.
%   If \cs{DeclareDefaultHookLabel} is not used in the
%   current package, \cs{@currname} is used instead.  If \cs{@currname}
%   is not set, the code is assumed to be in the main document, in which
%   case \texttt{top-level} is used.
%
%   The effect of \cs{DeclareDefaultHookLabel} holds for the current
%   file, and is reset to the previous value when the file is closed.
% \end{function}
%
%
% \subsubsection{Defining relations between hook code}
%
% The default assumption is that code added to hooks by different
% packages is independent and the order in which it is executed is
% irrelevant. While this is true in many case it is  obviously false
% in many others.
%
% Before the hook management system was introduced
% packages had to take elaborate precaution to determine of some other
% package got loaded as well (before or after) and find some ways to
% alter its behavior accordingly. In addition is was often the user's
% responsibility to load packages in the right order so that code
% added to hooks got added in the right order and some cases even
% altering the loading order wouldn't resolve the conflicts.
%
% With the new hook management system it is now possible to define
% rules (i.e., relationships) between code chunks added by different
% packages and explicitly describe in which order they should be
% processed.
%
% \begin{function}{\DeclareHookRule}
%   \begin{syntax}
%     \cs{DeclareHookRule} \Arg{hook}\Arg{label1}\Arg{relation}\Arg{label2}
%   \end{syntax}
%    Defines a relation between \meta{label1} and \meta{label2} for a
%    given \meta{hook}. If \meta{hook} is \texttt{??} this defines a
%    relation for all hooks that use the two labels, i.e., that have
%    chunks of code labeled with \meta{label1} and \meta{label2}.
%    Rules specific to a given hook take precedence over default
%    rules that use \texttt{??} as the \meta{hook}.
%
%    Currently, the supported relations are the following:
%    \begin{itemize}
%
%    \item[\texttt{before} or \texttt{\string<}]
%
%      Code for \meta{label1} comes before code for \meta{label2}.
%
%    \item[\texttt{after} or \texttt{\string>}]
%      Code for \meta{label1} comes after code for \meta{label2}.
%
%    \item[\texttt{incompatible-warning}]
%
%      Only code for either \meta{label1} or \meta{label2} can appear
%      for that hook (a way to say that two packages---or parts of
%      them---are incompatible). A warning is raised if both labels
%      appear in the same hook.
%
%    \item[\texttt{incompatible-error}]
%
%      Like \texttt{incompatible-error} but instead of a warning a
%      \LaTeX{} error is raised, and the code for both labels are
%      dropped from that hook until the conflict is resolved.
%
%    \item[\texttt{removes}]
%
%      Code for \meta{label1} overwrites code for \meta{label2}. More
%      precisely, code for \meta{label2} is dropped for that
%      hook. This can be used, for example if one package is a
%      superset in functionality of another one and therefore wants to
%      undo code in some hook and replace it with its own version.
%
%    \item[\texttt{unrelated}]
%
%       The order of code for \meta{label1} and \meta{label2} is
%      irrelevant. This rule is there to undo an incorrect rule
%      specified earlier.
%
%    \end{itemize}
%
%    The \meta{hook} and \meta{label} can be specified using the
%    dot-syntax to denote the current package name.
%    See section~\ref{sec:default-label}.
% \end{function}
%
%
% \begin{function}{\ClearHookRule}
%   \begin{syntax}
%     \cs{ClearHookRule}\Arg{hook}\Arg{label1}\Arg{label2}
%   \end{syntax}
%    Syntactic sugar for saying that \meta{label1} and \meta{label2}
%    are unrelated for the given \meta{hook}.
% \end{function}
%
%
%
% \begin{function}{\DeclareDefaultHookRule}
%   \begin{syntax}
%     \cs{DeclareDefaultHookRule}\Arg{label1}\Arg{relation}\Arg{label2}
%   \end{syntax}
%   This sets up a relation between \meta{label1} and \meta{label2}
%    for all hooks unless overwritten by a specific rule for a hook.
%    Useful for cases where one package has a specific relation to
%    some other package, e.g., is \texttt{incompatible} or always
%    needs a special ordering \texttt{before} or \texttt{after}.
%    (Technically it is just a shorthand for using \cs{DeclareHookRule}
%    with \texttt{??} as the hook name.)
%
%    Declaring default rules is only supported in the document
%    preamble.\footnotemark{}
%
%    The \meta{label} can be specified using the dot-syntax to denote
%    the current package name. See section~\ref{sec:default-label}.
% \end{function}\footnotetext{Trying to do so, e.g., via
%    \cs{DeclareHookRule} with \texttt{??}  has bad side-effects and
%    is not supported (though not explicitly caught for performance
%    reasons).}
%
%
%
% \subsubsection{Querying hooks}
% \label{sec:querying}
%
% Simpler data types, like token lists, have three possible states; they
% can:
% \begin{itemize}
%   \item exist and be empty;
%   \item exist and be non-empty; and
%   \item not exist (in which case emptiness doesn't apply);
% \end{itemize}
% Hooks are a bit more complicated: they have four possible states.
% A hook may exist or not, and either way it may or may not be empty.
% This means that even a hook that doesn't exist may be non-empty.
%
% This seemingly strange state may happen when, for example, package~$A$
% defines hook \hook{A/foo}, and package $B$ adds some code to that
% hook.  However, a document may load package $B$ before package $A$, or
% may not load package $A$ at all.  In both cases some code is added to
% hook \hook{A/foo} without that hook being defined yet, thus that
% hook is said to be non-empty, whereas it doesn't exist.  Therefore,
% querying the existence of a hook doesn't imply its emptiness, neither
% does the other way around.
%
% A hook is said to be empty when no code was added to it, either to
% its permanent code pool, or to its ``next'' token list.  The hook
% doesn't need to be declared to have code added to its code pool.
% A hook is said to exist when it was declared with \cs{NewHook} or
% some variant thereof.
%
% \begin{function}[EXP]{\IfHookEmptyTF}
%   \begin{syntax}
%     \cs{IfHookEmptyTF} \Arg{hook} \Arg{true code} \Arg{false code}
%   \end{syntax}
%   Tests if the \meta{hook} is empty (\emph{i.e.}, no code was added to
%   it using either \cs{AddToHook} or \cs{AddToHookNext}), and
%   branches to either \meta{true code} or \meta{false code} depending
%   on the result.
%
%    The \meta{hook} \emph{cannot} be specified using the dot-syntax.
%    A leading |.| is treated literally.
% \end{function}
%
% \begin{function}[EXP]{\IfHookExistTF}
%   \begin{syntax}
%     \cs{IfHookExistTF} \Arg{hook} \Arg{true code} \Arg{false code}
%   \end{syntax}
%   Tests if the \meta{hook} exists (if it was created with either
%   \cs{NewHook}, \cs{NewReversedHook}, or \cs{NewMirroredHookPair}), and
%   branches to either \meta{true code} or \meta{false code} depending
%   on the result.
%
%   The existence of a hook usually doesn't mean much from the viewpoint
%   of code that tries to add/remove code from that hook, since package
%   loading order may vary, thus the creation of hooks is asynchronous
%   to adding and removing code from it, so this test should be used
%   sparingly.
%
%    The \meta{hook} \emph{cannot} be specified using the dot-syntax.
%    A leading |.| is treated literally.
% \end{function}
%
% \fmi{Would be helpful if we provide some use cases}
%
% \subsubsection{Displaying hook code}
%
%    If one has to adjust the code execution in a hook using a hook
%    rule it is helpful to get some information about the code
%    associated with a hook, its current order and the existing rules.
%
% \begin{function}{\ShowHook}
%   \begin{syntax}
%     \cs{ShowHook} \Arg{hook}
%   \end{syntax}
%   Displays information about the \meta{hook} such as
%   \begin{itemize}
%   \item
%      the code chunks (and their labels) added to it,
%   \item
%      any rules set up to order them,
%      \fmi{currently this is missing the default rules that apply,
%      guess that needs fixing}
%   \item
%      the computed order (if already defined),
%   \item
%      any code executed on the next invocation only.
%   \end{itemize}
%
%    The \meta{hook} can be specified using the dot-syntax to denote
%    the current package name. See section~\ref{sec:default-label}.
% \end{function}
%
%
% \subsubsection{Debugging hook code}
%
% \begin{function}{\DebugHookOn,\DebugHookOff}
%   \begin{syntax}
%     \cs{DebugHookOn}
%   \end{syntax}
%    Turn the debugging of hook code on or off. This displays changes
%    made to the hook data structures. The output is rather coarse and
%      not really intended for normal use.
% \end{function}
%
%
% \subsection{L3 programming layer (\texttt{expl3}) interfaces}
% \label{sec:l3hook-interface}
%
%
% This is a quick summary of the \LaTeX3 programming interfaces for
% use with packages written in \texttt{expl3}. In contrast to the
% \LaTeXe{} interfaces they always use mandatory arguments only, e.g.,
% you always have to specify the \meta{label} for a code chunk.  We
% therefore suggest to use the declarations discussed in the previous
% section even in \texttt{expl3} packages, but the choice is yours.
%
%
% \begin{function}
%   {\hook_new:n,\hook_new_reversed:n,\hook_new_pair:nn}
%   \begin{syntax}
%     \cs{hook_new:n}\Arg{hook}
%     \cs{hook_new_pair:nn}\Arg{hook-1}\Arg{hook-2}
%   \end{syntax}
%   Creates a new \meta{hook} with normal or reverse ordering of code
%    chunks. \cs{hook_new_pair:nn} creates a pair of such hooks with
%    \Arg{hook-2} being a reversed hook.
%    If a hook name is already taken, an error is raised and the hook
%    is not created.
%
%    The \meta{hook} can be specified using the dot-syntax to denote
%    the current package name. See section~\ref{sec:default-label}.
% \end{function}
%
%
%
% \begin{function}{\hook_use:n}
%   \begin{syntax}
%     \cs{hook_use:n} \Arg{hook}
%   \end{syntax}
%    Executes the \Arg{hook} code followed (if set up) by the code for next
%    invocation only, then empties that next invocation code.
%
%    The \meta{hook} \emph{cannot} be specified using the dot-syntax.
%    A leading |.| is treated literally.
% \end{function}
%
% \begin{function}{\hook_use_once:n}
%   \begin{syntax}
%     \cs{hook_use_once:n} \Arg{hook}
%   \end{syntax}
%     Changes the \Arg{hook} status so that from now on any addition to
%     the hook code is executed immediately. Then execute any
%     \Arg{hook} code already set up.
%    \fmi{better L3 name?}
%
%    The \meta{hook} \emph{cannot} be specified using the dot-syntax.
%    A leading |.| is treated literally.
% \end{function}
%
% \begin{function}{\hook_gput_code:nnn}
%   \begin{syntax}
%     \cs{hook_gput_code:nnn} \Arg{hook} \Arg{label} \Arg{code}
%   \end{syntax}
%    Adds a chunk of \meta{code} to the \meta{hook} labeled
%    \meta{label}. If the label already exists the \meta{code} is
%    appended to the already existing code.
%
%    If code is added to an external \meta{hook} (of the kernel or
%    another package) then the convention is to use the package name
%    as the \meta{label} not some internal module name or some other
%    arbitrary string.
%
%    The \meta{hook} and \meta{label} can be specified using the
%    dot-syntax to denote the current package name.
%    See section~\ref{sec:default-label}.
% \end{function}
%
% \begin{function}
%   {\hook_gput_next_code:nn}
%   \begin{syntax}
%     \cs{hook_gput_next_code:nn} \Arg{hook} \Arg{code}
%   \end{syntax}
%    Adds a chunk of \meta{code} for use only in the next invocation of the
%    \meta{hook}. Once used it is gone.
%
%    This is simpler than \cs{hook_gput_code:nnn}, the code is simply
%    appended to the hook in the order of declaration at the very end,
%    i.e., after all standard code for the hook got executed.
%
%    Thus if one needs to undo what the standard does one has to do
%    that as part of \meta{code}.
%
%    The \meta{hook} can be specified using the dot-syntax to denote
%    the current package name. See section~\ref{sec:default-label}.
% \end{function}
%


%  \begin{function}{\hook_gremove_code:nn}
%   \begin{syntax}
%     \cs{hook_gremove_code:nn} \Arg{hook} \Arg{label}
%   \end{syntax}
%    Removes any code for \meta{hook} labeled \meta{label}.
%
%    If the code for that \meta{label} wasn't yet added to the
%    \meta{hook}, an order is set so that when some code attempts to add
%    that label, the removal order takes action and the code is not
%    added.
%
%    If the second argument is \texttt{*}, then all code chunks are
%    removed. This is rather dangerous as it drops code from other
%    packages one may not know about, so think twice before using
%    that!
%
%    The \meta{hook} and \meta{label} can be specified using the
%    dot-syntax to denote the current package name.
%    See section~\ref{sec:default-label}.
% \end{function}


%  \begin{function}{\hook_gset_rule:nnnn}
%   \begin{syntax}
%     \cs{hook_gset_rule:nnnn} \Arg{hook} \Arg{label1} \Arg{relation} \Arg{label2}
%   \end{syntax}
%    Relate \meta{label1} with \meta{label2} when used in \meta{hook}.
%    See \cs{DeclareHookRule} for the allowed \meta{relation}s.
%    If \meta{hook} is \texttt{??} a default rule is specified.
%
%    The \meta{hook} and \meta{label} can be specified using the
%    dot-syntax to denote the current package name.
%    See section~\ref{sec:default-label}.
%    The dot-syntax is parsed in both \meta{label} arguments, but it
%    usually makes sense to be used in only one of them.
% \end{function}
%
% \begin{function}[pTF]{\hook_if_empty:n}
%   \begin{syntax}
%     \cs{hook_if_empty:n} \Arg{hook} \Arg{true code} \Arg{false code}
%   \end{syntax}
%   Tests if the \meta{hook} is empty (\emph{i.e.}, no code was added to
%   it using either \cs{AddToHook} or \cs{AddToHookNext}), and
%   branches to either \meta{true code} or \meta{false code} depending
%   on the result.
%
%    The \meta{hook} \emph{cannot} be specified using the dot-syntax.
%    A leading |.| is treated literally.
% \end{function}
%
% \begin{function}[pTF]{\hook_if_exist:n}
%   \begin{syntax}
%     \cs{hook_if_exist:n} \Arg{hook} \Arg{true code} \Arg{false code}
%   \end{syntax}
%   Tests if the \meta{hook} exists (if it was created with either
%   \cs{NewHook}, \cs{NewReversedHook}, or \cs{NewMirroredHookPair}), and
%   branches to either \meta{true code} or \meta{false code} depending
%   on the result.
%
%  \fmi{what are the results for generic hooks that do not need to be declared?}
%
%   The existence of a hook usually doesn't mean much from the viewpoint
%   of code that tries to add/remove code from that hook, since package
%   loading order may vary, thus the creation of hooks is asynchronous
%   to adding and removing code from it, so this test should be used
%   sparingly.
%
%    The \meta{hook} \emph{cannot} be specified using the dot-syntax.
%    A leading |.| is treated literally.
% \end{function}
%
%
% \begin{function}{\hook_debug_on:,\hook_debug_off:}
%   \begin{syntax}
%     \cs{hook_debug_on:}
%   \end{syntax}
%    Turns the debugging of hook code on or off. This displays changes
%    to the hook data.
% \end{function}
%
%
%
% \subsection{On the order of hook code execution} \label{sec:order}
%
%    Chunks of code for a \meta{hook} under different labels are supposed
%    to be independent if there are no special rules set up that
%    define a relation between the chunks. This means that you can't
%    make assumptions about the order of execution!
%
%    Suppose you have the following declarations:
%\begin{verbatim}
%    \NewHook{myhook}
%    \AddToHook{myhook}[packageA]{\typeout{A}}
%    \AddToHook{myhook}[packageB]{\typeout{B}}
%    \AddToHook{myhook}[packageC]{\typeout{C}}
%\end{verbatim}
%    then executing the hook with \cs{UseHook} will produce the
%    typeout \texttt{A} \texttt{B} \texttt{C} in that order.  In other
%    words, the execution order is computed to be \texttt{packageA},
%    \texttt{packageB}, \texttt{packageC} which you can verify with
%    \cs{ShowHook}\texttt{\{myhook\}}:
%\begin{verbatim}
%   The hook 'myhook':
%    Code chunks:
%       packageA -> \typeout {A}
%       packageB -> \typeout {B}
%       packageC -> \typeout {C}
%    Extra code next invocation:
%       ---
%    Rules:
%       ---
%    Execution order:
%       packageA, packageB, packageC
%\end{verbatim}
%    The reason is that the code chunks are internally saved in a property list
%    and the initial order of such a property list is the order in
%    which key-value pairs got added. However, that is only true if
%    nothing other than adding happens!
%
%    Suppose, or example, you want to replace the code chunk for
%    \texttt{packageA}, e.g.,
%\begin{verbatim}
%    \RemoveFromHook{myhook}[packageA]
%    \AddToHook{myhook}[packageA]{\typeout{A alt}}
%\end{verbatim}
%    then your order becomes  \texttt{packageB},
%    \texttt{packageC}, \texttt{packageA} because the label got removed
%    from the property list and then re-added (at its end).
%
%    While that may not be too surprising,  the execution order is
%    also sometimes altered if you add a redundant rule, e.g. if you specify
%\begin{verbatim}
%    \DeclareHookRule{myhook}{packageA}{before}{packageB}
%\end{verbatim}
%    instead of the previous lines we get
%\begin{verbatim}
%   The hook 'myhook':
%    Code chunks:
%       packageA -> \typeout {A}
%       packageB -> \typeout {B}
%       packageC -> \typeout {C}
%    Extra code next invocation:
%       ---
%    Rules:
%       packageA|packageB with relation before
%    Execution order (after applying rules):
%       packageA, packageC, packageB
%\end{verbatim}
%    As you can see the code chunks are still in the same order, but
%    in the execution order for the labels \texttt{packageB} and
%    \texttt{packageC} have
%    swapped places.
%    The reason is that, with the rule there are two orders that
%    satisfy it, and the algorithm for sorting happened to pick a
%    different one compared to the case without rules (where it
%    doesn't run at all as there is nothing to resolve).
%    Incidentally, if we had instead specified the redundant rule
%\begin{verbatim}
%    \DeclareHookRule{myhook}{packageB}{before}{label-3}
%\end{verbatim}
%    the execution order would not have changed.
%
%    In summary: it is not possible to rely on the order of execution
%    unless there are rules that partially or fully define the order
%    (in which you can rely on them being fulfilled).
%
%
% \subsection{The use of \enquote{reversed} hooks} \label{sec:reversed-order}
%
%    You may have wondered why you can declare a \enquote{reversed} hook
%    with \cs{NewReversedHook} and what that does exactly.
%
%    In short: the execution order of a reversed hook (without any
%    rules!) is exactly reversed to the order you would have gotten for
%    a hook declared with \cs{NewHook}.
%
%    This is helpful if you have a pair of hooks where you expect to see
%    code added that involves grouping, e.g., starting an environment
%    in the first and closing that environment in the second hook.
%    To give a somewhat contrived example\footnote{there are simpler
%    ways to achieve the same effect.}, suppose there is a package
%    adding the following:
%\begin{verbatim}
%    \AddToHook{env/quote/before}[package-1]{\begin{itshape}}
%    \AddToHook{env/quote/after} [package-1]{\end{itshape}}
%\end{verbatim}
%    As a result, all quotes will be in italics.
%    Now suppose further that the user wants the quotes also in blue
%    and therefore adds:
%\begin{verbatim}
%    \usepackage{color}
%    \AddToHook{env/quote/before}{\begin{color}{blue}}
%    \AddToHook{env/quote/after} {\end{color}}
%\end{verbatim}
%    Now if the \hook{env/quote/after} hook would be a normal hook we
%    would get the same execution order in  both hooks, namely:
%\begin{verbatim}
%    package-1, top-level
%\end{verbatim}
%    (or vice versa) and as a result, would get:
%\begin{verbatim}
%    \begin{itshape}\begin{color}{blue} ...
%    \end{itshape}\end{color}
%\end{verbatim}
%   and an error message that \verb=\begin{color}= ended by
%    \verb=\end{itshape}=.
%    With \hook{env/quote/after} declared as a reversed hook the
%    execution order is reversed and so all environments are closed in
%    the correct sequence and \cs{ShowHook} would give us the
%    following output:
%\begin{verbatim}
%   The hook 'env/quote/after':
%    Code chunks:
%       package-1 -> \end {itshape}
%       top-level -> \end {color}
%    Extra code next invocation:
%       ---
%    Rules:
%       ---
%    Execution order (after reversal):
%       top-level, package-1
%\end{verbatim}
%
%    The reversal of the execution order happens before applying any
%    rules, so if you alter the order you will probably have to alter
%    it in both hooks, not just in one, but that depends on the use case.
%
%
%
%
% \subsection{Private \LaTeX{} kernel hooks}
%
%    There are a few places where it is absolutely essential for
%    \LaTeX{} to function correctly that code is executed in a precisely
%    defined order. Even that could have been implemented with the
%    hook management (by adding various rules to ensure the
%    appropriate ordering with respect to other code added by
%    packages). However, this makes every document unnecessary
%    slow, because there has to be sorting even through the result is
%    predetermined. Furthermore it forces package writers to
%    unnecessarily add such rules if they add further code to the hook
%    (or break \LaTeX{}).
%
%    For that reason such code is not using the hook management, but
%    instead private kernel commands directly before or after a public
%    hook with the following naming
%    convention: \cs{@kernel@before@\meta{hookname}} or
%    \cs{@kernel@after@\meta{hookname}}. For example, in
%    \cs{enddocument} you find
%\begin{verbatim}
%   \UseHook{enddocument}%
%   \@kernel@after@enddocument
%\end{verbatim}
%    which means first the user/package-accessible \hook{enddocument}
%    hook is executed and then the internal kernel hook. As their name
%    indicates these kernel commands should not be altered by third-party
%    packages, so please refrain from that in the interest of
%    stability and instead use the public hook next to it.\footnote{As
%    with everything in \TeX{} there is not enforcement of this rule,
%    and by looking at the code it is easy to find out how the kernel
%    adds to them. The main reason of this section is therefore to say
%    \enquote{please don't do that, this is unconfigurable code!}}
%
%
%
% \subsection{Legacy \LaTeXe{} interfaces}
%
%  \LaTeXe{} offered a small number of hooks together with commands to
%    add to them. They are listed here and are retained for backwards
%    compatibility.
%
%  With the new hook management several additional hooks have been added
%    to \LaTeX\ and more will follow. See the next section for what
%    is already available.
%
%
% \begin{function}{\AtBeginDocument}
%   \begin{syntax}
%     \cs{AtBeginDocument} \oarg{label} \Arg{code}
%   \end{syntax}
%   If used without the optional argument \meta{label}, it works essentially
%    like before, i.e., it is adding \meta{code} to the hook
%    \hook{begindocument} 
%    (which is executed inside \verb=\begin{document}=).
%    However, all code added this way is labeled with the label
%    \hook{top-level} if done outside of a package or class or with the
%    package/class name if called inside such a file.
%
%    This way one can add further code to the hook using
%    \cs{AddToHook} or \cs{AtBeginDocument} using a different label
%    and explicitly order the code chunks as necessary, e.g., run some
%    code before or after the \hook{top-level} code.  When using the
%    optional argument the call is equivalent to running
%    \cs{AddToHook} \texttt{\{begindocument\}} \oarg{label}
%    \Arg{code}.
%
%    For important packages with known order requirement we may over
%    time add rules to the kernel (or to those packages) so that they
%    work regardless of the loading-order in the document.
% \end{function}
%
% \begin{function}{\AtEndDocument}
%   \begin{syntax}
%     \cs{AtEndDocument} \oarg{label} \Arg{code}
%   \end{syntax}
%   Like \cs{AtBeginDocument} but for the \hook{enddocument} hook.
% \end{function}
%
%
%
% \begin{function}{\AtBeginDvi}
%   \begin{syntax}
%     \cs{AtBeginDvi} \oarg{label} \Arg{code}
%   \end{syntax}
%   This hook is discussed in conjunction with the shipout hooks.
% \end{function}
%
%
%
% \subsection{\LaTeXe{} commands and environments augmented by
%    hooks}
%
%  \emph{intro to be written}
%
% \subsubsection{Generic hooks for all environments}
%
%    Every environment \meta{env} has now four associated hooks coming
%    with it:
%    \begin{description}
%    \item[\hook{env/\meta{env}/before}]
%
%       This hook is executed as part of \cs{begin} as the very first
%       action, in particular prior to starting the environment group.
%       Its scope is therefore not restricted by the environment.
%
%    \item[\hook{env/\meta{env}/begin}]
%
%       This hook is executed as part of \cs{begin} directly in front
%       of the code specific to the environment start (e.g., the
%       second argument of \cs{newenvironment}).  Its scope is the
%       environment body.
%
%    \item[\hook{env/\meta{env}/end}]
%
%       This is executed as part of \cs{end} directly in front of the
%       code specific to the end of the environment (e.g., the third
%       argument of \cs{newenvironment}).
%
%    \item[\hook{env/\meta{env}/after}]
%
%       This is executed as part of \cs{end} after the
%       code specific to the environment end and after the environment
%       group has ended.
%       Its scope is therefore not restricted by the environment.
%
%       This hook is implemented as a reversed hook so if two packages
%       add code to \hook{env/\meta{env}/before} and to
%       \hook{env/\meta{env}/after} they can add surrounding
%       environments and the order of closing them happens in the
%       right sequence.
%
%    \end{description}
%    In contrast to other hooks these hooks do not need to be declared
%    using \cs{NewHook}.
%
%    The hooks are only executed if \cs{begin}\Arg{env} and
%    \cs{end}\Arg{env} is used. If the environment code is executed
%    via low-level calls to \cs{\meta{env}} and \cs{end\meta{env}}
%    (e.g., to avoid the environment grouping) they are not
%    available. If you want them available in code using this method,
%    you would need to add them yourself, i.e., write something like
%\begin{verbatim}
%  \UseHook{env/quote/before}\quote
%      ...
%  \endquote\UseHook{env/quote/after}
%\end{verbatim}
%    to add the outer hooks, etc.
%
%
%
% \subsubsection{Hooks provided by \cs{begin}\texttt{\{document\}}}
%
%    Until 2020 \cs{begin}\texttt{\{document\}} offered exactly one
%    hook that one had to fill using \cs{AtBeginDocument}. Experience
%    has shown that this single hook in one place was not enough and
%    as part of adding the general hook management system a number of
%    additional hooks have been added at this point. The places for
%    hooks have been chosen to provide the same support as offered by
%    external packages, such as \pkg{etoolbox} and others that
%    augmented \cs{document} to gain better control.
%
%    Supported are now the following hooks:
%    \begin{description}
%
%    \item[\hook{env/document/before}]
%
%      This is the generic environment hook executed effectively
%      before \verb=\begin{document}= starts, i.e., one can think of
%      it as a hook for code at the end of the preamble section.
%
%    \item[\hook{env/document/begin}]
%
%      This is the second generic environment hook that is executed
%      after the environment has started its group. But given that for the
%      \texttt{document} environment this group is canceled there is
%      little difference to the previous one as the two are directly
%      executed one after another (the only difference is that in this
%      hook \cs{@currenvir} is now set to \texttt{document} but
%      anybody adding to this hook would know that already).
%
%
%    \item[\hook{begindocument}]
%
%      This hook is added to by \cs{AtBeginDocument} and is executed
%      after the \texttt{.aux} file as be read in and most
%      initialization are done, so they can be altered and inspected by
%      the hook code. It is followed by a small number of further
%      initializations that shouldn't be altered and are therefore
%      coming later.
%
%    \item[\hook{begindocument/end}]
%
%      This hook is executed at the end of the \cs{document} code in
%      other words at the beginning of the document body. The only
%      command that follows it is \cs{ignorespaces}.
%
%    \end{description}
%
%
%
%
% \subsubsection{Hooks provided by \cs{end}\texttt{\{document\}}}
%
%    \LaTeXe{} always provided \cs{AtEndDocument} to add code to the
%    execution of \verb=\end{document}= just in front of the code that
%    is normally executed there. While this was a big improvement over
%    the situation in \LaTeX\,2.09 it was not flexible enough for a
%    number of use cases and so packages, such as \pkg{etoolbox},
%    \pkg{atveryend} and others patched \cs{enddocument} to add
%    additional points where code could be hooked into.
%
%    Patching using packages is always problematical as leads to
%    conflicts (code availability, ordering of patches, incompatible
%    patches, etc.).  For this reason a number of additional hooks
%    have been added to the \cs{enddocument} code to allow packages
%    to add code in various places in a controlled way without the
%    need for overwriting or patching the core code.
%
%    Supported are now the following hooks:
%    \begin{description}
%
%    \item[\hook{env/document/end}] The generic hook inside \cs{end}.
%
%    \item[\hook{enddocument}]

%      The hook associated with \cs{AtEndDocument}. It is immediately
%      called after the previous hook so there could be just
%      one.\footnote{We could make \cs{AtEndDocument} just fill the
%      \hook{env/document/end} but maybe that is a bit confusing.}
%
%      When this hook is executed there may be still unprocessed
%      material (e.g., floats on the deferlist) and the hook may add
%      further material to be typeset. After it, \cs{clearpage} is
%      called to ensure that all such material gets typeset. If there
%      is nothing waiting the \cs{clearpage} has no effect.
%
%
%    \item[\hook{enddocument/afterlastpage}]
%
%      As the name indicates this hook should not receive code that
%      generates material for further pages. It is the right place to
%      do some final housekeeping and possibly write out some
%      information to the \texttt{.aux} file (which is still open at
%      this point to receive data). It is also the correct place to
%      set up any testing code to be run when the \texttt{.aux} file
%      is re-read in the next step.
%
%
%      After this hook has been executed the \texttt{.aux} file is
%      closed for writing and then read back in to do some tests
%      (e.g., looking for missing references or duplicated labels, etc.).
%
%    \item[\hook{enddocument/afteraux}]
%
%      At this point, the \texttt{.aux} file has been reprocessed and so
%      this is a possible place for final checks and display of
%      information to the user. However, for the latter you might
%      prefer the next hook, so that your information is displayed after the
%      (possibly longish) list of files if that got requested via \cs{listfiles}.
%
%    \item[\hook{enddocument/info}]
%
%      This hook is meant to receive code that write final information
%      messages to the terminal. It follows immediately after the
%      previous hook (so both could have been combined, but then
%      packages adding further code would always need to also supply
%      an explicit rule to specify where it should go.
%
%      This hook already contains some code added by the kernel (under
%      the labels \texttt{kernel/filelist} and
%      \texttt{kernel/warnings}), namely the list of files when
%      \cs{listfiles} has been used and the warnings for duplicate
%      labels, missing references, font substitutions etc.
%
%    \item[\hook{enddocument/end}]
%
%      Finally, this hook is executed just in front of the final call
%      to \cs{@{}@end}.
%
%    \end{description}
%
%
%    There is also the hook \hook{shipout/lastpage}. This hook is
%    executed as part of the last \cs{shipout} in the document to
%    allow package to add final \cs{special}'s to that page. Where
%    this hook is executed in relation to those from the above list
%    can vary from document to document. Furthermore to determine correctly
%    which of the \cs{shipout}s is the last one, \LaTeX{} needs to be run
%    several times, so initially it might get executed on the wrong
%    page. See section~\ref{sec:shipout} for where to find the details.
%
%
% \subsubsection{Hooks provided \cs{shipout} operations}
% \label{sec:shipout}
%
%    There are several hooks and mechanisms added to \LaTeX{}'s
%    process of generating pages. These are documented in
%    \texttt{ltshipout-doc.pdf} or with code in
%    \texttt{ltshipout-code.pdf}.
%
%
% \subsubsection{Hooks provided file loading operations}
%
%    There are several hooks added to \LaTeX{}'s
%    process of loading file via its high-level interfaces such as
%    \cs{input}, \cs{include}, \cs{usepackage}, etc. These are documented in
%    \texttt{ltfilehook-doc.pdf} or with code in
%    \texttt{ltfilehook-code.pdf}.
%
%

% \StopEventually{\setlength\IndexMin{200pt}  \PrintIndex  }
%
%
% \section{The Implementation}
%    \begin{macrocode}
%<@@=hook>
%    \end{macrocode}
%
%    \begin{macrocode}
%<*2ekernel>
%    \end{macrocode}
%    
%    \begin{macrocode}
\ExplSyntaxOn
%    \end{macrocode}
%
%
%  \subsection{Debugging}
%
%  \begin{macro}{\g_@@_debug_bool}
%    Holds the current debugging state.
%    \begin{macrocode}
\bool_new:N \g_@@_debug_bool
%    \end{macrocode}
%  \end{macro}
%
%  \begin{macro}{\hook_debug_on:,\hook_debug_off:}
%  \begin{macro}{\@@_debug:n}
%  \begin{macro}{\@@_debug_gset:}
%    Turns debugging on and off by redefining \cs{@@_debug:n}.
%    \begin{macrocode}
\cs_new_eq:NN \@@_debug:n \use_none:n
\cs_new_protected:Npn \hook_debug_on:
  {
    \bool_gset_true:N \g_@@_debug_bool
    \@@_debug_gset:
  }
\cs_new_protected:Npn \hook_debug_off:
  {
    \bool_gset_false:N \g_@@_debug_bool
    \@@_debug_gset:
  }
\cs_new_protected:Npn \@@_debug_gset:
  {
    \cs_gset_protected:Npx \@@_debug:n ##1
      { \bool_if:NT \g_@@_debug_bool {##1} }
  }
%    \end{macrocode}
%  \end{macro}
%  \end{macro}
%  \end{macro}

%
%
%  \subsection{Borrowing from internals of other kernel modules}
%
%
% \begin{macro}[EXP]{\@@_str_compare:nn}
%   Private copy of \cs{__str_if_eq:nn}
%    \begin{macrocode}
\cs_new_eq:NN \@@_str_compare:nn \__str_if_eq:nn
%    \end{macrocode}
% \end{macro}
%
%  \subsection{Declarations}
%
%  \begin{macro}{\l_@@_return_tl,\l_@@_tmpa_tl,\l_@@_tmpb_tl}
%    Scratch variables used throughout the package.
%    \begin{macrocode}
\tl_new:N \l_@@_return_tl
\tl_new:N \l_@@_tmpa_tl
\tl_new:N \l_@@_tmpb_tl
%    \end{macrocode}
%  \end{macro}
%
%  \begin{macro}{\g_@@_all_seq}
%    In a few places we need a list of all hook names ever defined so
%    we keep track if them in this sequence.
%    \begin{macrocode}
\seq_new:N \g_@@_all_seq
%    \end{macrocode}
%  \end{macro}
%
% \begin{macro}{\g_@@_removal_list_prop}
%   A property list to hold delayed removals.
%    \begin{macrocode}
\tl_new:N \g_@@_removal_list_tl
%    \end{macrocode}
% \end{macro}
%
% \begin{macro}{\l_@@_cur_hook_tl}
%   Stores the name of the hook currently being sorted.
%    \begin{macrocode}
\tl_new:N \l_@@_cur_hook_tl
%    \end{macrocode}
% \end{macro}
%
% \begin{macro}{\g_@@_code_temp_prop}
%   A property list to temporarily save the original one so that it
%   isn't permanently changed during sorting.
%    \begin{macrocode}
\prop_new:N \g_@@_code_temp_prop
%    \end{macrocode}
% \end{macro}
%
% \begin{macro}{\g_@@_hook_curr_name_tl,\g_@@_name_stack_seq}
%   Default label used for hook commands, and a stack to keep track of
%   packages within packages.
%    \begin{macrocode}
\tl_new:N \g_@@_hook_curr_name_tl
\seq_new:N \g_@@_name_stack_seq
%    \end{macrocode}
% \end{macro}
%
% \begin{macro}{\@@_tmp:w}
%   Temporary macro for generic usage.
%    \begin{macrocode}
\cs_new_eq:NN \@@_tmp:w ?
%    \end{macrocode}
% \end{macro}
%
% \begin{macro}{\tl_gremove_once:Nx}
%   Some variants of \pkg{expl3} functions. \fmi{should be moved to expl3}
%    \begin{macrocode}
\cs_generate_variant:Nn \tl_gremove_once:Nn { Nx }
%    \end{macrocode}
% \end{macro}
%
% \begin{macro}{\s_@@_mark}
%   Scan mark used for delimited arguments.
%    \begin{macrocode}
\scan_new:N \s_@@_mark
%    \end{macrocode}
% \end{macro}
%
%
% \subsection{Providing new hooks}
%
%  \begin{macro}{\g_@@_..._code_prop,\g_@@_..._rules_prop,
%                \g_@@_..._code_tl,\g_@@_..._next_code_tl}
%
%    Hooks have a \meta{name} and for each hook we have to provide a number of
%    data structures. These are
%    \begin{description}
%    \item[\cs{g_@@_\meta{name}_code_prop}] A property list holding the code
%    for the hook in separate chunks. The keys are by default the
%    package names that add code to the hook, but it is possible
%    for packages to define other keys. 
%
%    \item[\cs{g_@@_\meta{name}_rules_prop}] A property listing holding
%    relation info how the code chunks should be ordered within a
%    hook. This is used for debugging only. The actual rule for a
%    \meta{hook} is stored in a separate token lists named
%    \cs[no-index]{g_@@_\meta{hook}_rule_\meta{label1}\string|\meta{label2}_tl}
%    for a pair of labels.
%
%    \item[\cs{g_@@_\meta{name}_code_tl}] The code that is actually executed
%    when the hook is called in the document is stored in this token
%    list. It is constructed from the code chunks applying the
%    information.
%
%    \item[\cs{g_@@_\meta{name}_next_code_tl}] Finally there is extra code
%    (normally empty) that is used on the next invocation of the hook
%    (and then deleted). This can be used to define some special
%    behavior for a single occasion from within the document.
%
%    \end{description}
%  \end{macro}


%
%
%  \begin{macro}{\hook_new:n}
%    The \cs{hook_new:n} declaration declare a new hook and expects
%    the hook \meta{name} as its argument, e.g.,
%    \hook{begindocument}.
%    \begin{macrocode}
\cs_new_protected:Npn \hook_new:n #1
  {
    \exp_args:Nx \@@_new:n
      { \@@_parse_label_default:nn {#1} { top-level } }
  }
\cs_new_protected:Npn \@@_new:n #1 {
%    \end{macrocode}
%    We check for one of the internal data structures and if it
%    already exists we complain.
%    \begin{macrocode}
  \hook_if_exist:nTF {#1}
     { \ErrorHookExists }
%    \end{macrocode}
%    Otherwise we add the hook name to the list of all hooks and
%    allocate the necessary data structures for the new hook.
%    \begin{macrocode}
     { \seq_gput_right:Nn \g_@@_all_seq {#1}
%    \end{macrocode}
%    This is only used by the actual code of the current hook, so
%    declare it normally:
%    \begin{macrocode}
       \tl_new:c { g_@@_#1_code_tl }
%    \end{macrocode}
%    Now ensure that the base data structure for the hook exists:
%    \begin{macrocode}
       \@@_declare:n {#1}
%    \end{macrocode}
%    The \cs{g_@@_\meta{hook}_labels_clist} holds the sorted list of
%    labels (once it got sorted). This is used only for debugging.
%    \begin{macrocode}
       \clist_new:c {g_@@_#1_labels_clist}
%    \end{macrocode}
%    Some hooks should reverse the default order of code chunks. To
%    signal this we have a token list which is empty for normal hooks
%    and contains a \verb=-= for reversed hooks.
%    \begin{macrocode}
       \tl_new:c { g_@@_#1_reversed_tl }
%    \end{macrocode}
%    The above is all in L3 convention, but we also provide an
%    interface to legacy \LaTeXe{} for use in the current kernel. This
%    is done in a separate macro.
%    \begin{macrocode}
       \@@_provide_legacy_interface:n {#1}
     }
}
%    \end{macrocode}
%  \end{macro}

%
%
% \begin{macro}{\@@_declare:n}
%   This function declares the basic data structures for a hook without
%   actually declaring the hook itself.  This is needed to allow adding
%   to undeclared hooks.  Here it is unnecessary to check whether both
%   variables exist, since both are declared at the same time (either
%   both exist, or neither).
%    \begin{macrocode}
\cs_new_protected:Npn \@@_declare:n #1
  {
    \@@_if_exist:nF {#1}
      {
        \prop_new:c { g_@@_#1_code_prop }
        \tl_new:c { g_@@_#1_next_code_tl }
        \prop_new:c { g_@@_#1_rules_prop } % only for debugging
      }
  }
%    \end{macrocode}
%  \end{macro}



%  \begin{macro}{\hook_new_reversed:n}
%
%    Declare a new hook. The default ordering of code chunks is
%    reversed, signaled by setting the token list to a minus sign.
%    \begin{macrocode}
\cs_new_protected:Npn \hook_new_reversed:n #1 {
  \hook_new:n {#1}
  \tl_gset:cn { g_@@_#1_reversed_tl } { - }
}
%    \end{macrocode}
%  \end{macro}

%  \begin{macro}{\hook_new_pair:nn}
%    A shorthand for declaring a normal and a (matching) reversed hook in one go.
%    \begin{macrocode}
\cs_new_protected:Npn \hook_new_pair:nn #1#2 {
  \hook_new:n {#1}  \hook_new_reversed:n {#2}
}
%    \end{macrocode}
%  \end{macro}


% \begin{macro}{\@@_provide_legacy_interface:n}
%    The \LaTeX{} legacy concept for hooks uses with hooks the
%    following naming scheme in the code: \cs{@...hook}.
%
%    We follow this convention and insert the hook code using this
%    naming scheme in \LaTeXe{}. At least as long as this code is in a
%    package, some such hooks are already filled with data when we move
%    them over to the new scheme. We therefore insert already existing
%    code under the label \texttt{legacy} into the hook management
%    machinery and then replace the \cs{@...hook} with its counterpart
%    which is \cs{g_@@_\#1_code_tl}.\footnote{This means one extra
%    unnecessary expansion on each invocation in the document but
%    keeps the \LaTeXe{} and the L3 coding side properly separated.}
%    \begin{macrocode}
\cs_new_protected:Npn \@@_provide_legacy_interface:n #1
  {
%    \end{macrocode}
%    If the \pkg{expl3} code is run with checking on then assigning or
%    using non L3 names such as \cs{@enddocumenthook} with \pkg{expl3}
%    functions will trigger warnings so we run this code with
%    debugging explicitly suspended.
%    \begin{macrocode}
    \debug_suspend:
    \tl_if_exist:cT { @#1hook }
%    \end{macrocode}
%    Of course if the hook exists but is still empty, there is no need
%    to add anything under \texttt{legacy} or the current package name.
%    \begin{macrocode}
      {
        \tl_if_empty:cF { @#1hook }
          {
            \@@_gput_code:nxv {#1}
              { \@@_parse_label_default:Vn \c_novalue_tl { legacy } }
              { @#1hook }
          }
      }
%    \end{macrocode}
%    We need a global definition in case the declaration is done
%    inside a group (which happens below at the end of the file).
%    This is another reason why need to suspend checking, otherwise
%    \cs{tl_gset:co} would complain about \cs{@...hook} not starting
%    with \cs{g_}.
%    \begin{macrocode}
    \tl_gset:co{@#1hook}{\cs:w g_@@_#1_code_tl\cs_end:}
    \debug_resume:
  }
%    \end{macrocode}
% \end{macro}
%
% \subsection{Parsing a label}
%
% \begin{macro}[EXP]{\@@_parse_label_default:nn,\@@_parse_label_default:Vn}
%   This macro checks if a label was given (not \cs{c_novalue_tl}), and
%   if so, tries to parse the label looking for a leading \verb|.| to
%   replace for \cs{@currname}.  Otherwise \cs{@@_currname_or_default:n}
%   is used to pick \cs{@currname} or the fallback value.
%    \begin{macrocode}
\cs_new:Npn \@@_parse_label_default:nn #1 #2
  {
    \tl_if_novalue:nTF {#1}
      { \@@_currname_or_default:n {#2} }
      { \tl_trim_spaces_apply:nN {#1} \@@_parse_dot_label:nn {#2} }
  }
\cs_generate_variant:Nn \@@_parse_label_default:nn { V }
%    \end{macrocode}
% \end{macro}
%
% \begin{macro}[EXP]{\@@_parse_dot_label:nn}
% \begin{macro}[EXP]{
%     \@@_parse_dot_label:nw,
%     \@@_parse_dot_label_cleanup:w,
%     \@@_parse_dot_label_aux:nw
%   }
%   Start by checking if the label is empty, which raises an error, and
%   uses the fallback value.  If not,
%   split the label at a \verb|./|, if any, and check if no tokens are
%   before the \verb|./|, or if the only character is a \verb|.|.
%   If these requirements are fulfilled, the leading
%   \verb|.| is replaced with \cs{@@_currname_or_default:n}.  Otherwise
%   the label is returned unchanged.
%    \begin{macrocode}
\cs_new:Npn \@@_parse_dot_label:nn #1 #2
  {
    \tl_if_empty:nTF {#1}
      {
        \msg_expandable_error:nnn { hooks } { empty-label } {#2}
        #2
      }
      {
        \str_if_eq:nnTF {#1} { . }
          { \@@_currname_or_default:n {#1} }
          { \@@_parse_dot_label:nw {#2} #1 ./ \s_@@_mark }
      }
  }
\cs_new:Npn \@@_parse_dot_label:nw #1 #2 ./ #3 \s_@@_mark
  {
    \tl_if_empty:nTF {#2}
      { \@@_parse_dot_label_aux:nw {#1} #3 \s_@@_mark }
      {
        \tl_if_empty:nTF {#3}
          {#2}
          { \@@_parse_dot_label_cleanup:w #2 ./ #3 \s_@@_mark }
      }
  }
\cs_new:Npn \@@_parse_dot_label_cleanup:w #1 ./ \s_@@_mark {#1}
\cs_new:Npn \@@_parse_dot_label_aux:nw #1 #2 ./ \s_@@_mark
  { \@@_currname_or_default:n {#1} / #2 }
%    \end{macrocode}
% \end{macro}
% \end{macro}
%
% \begin{macro}[EXP]{\@@_currname_or_default:n}
%   Uses \cs{g_@@_hook_curr_name_tl} if it is set, otherwise tries
%   \cs{@currname}.  If neither is set, uses the fallback value
%   \verb|#1| (usually \texttt{top-level}).
%    \begin{macrocode}
\cs_new:Npn \@@_currname_or_default:n #1
  {
    \tl_if_empty:NTF \g_@@_hook_curr_name_tl
      {
        \tl_if_empty:NTF \@currname
          {#1}
          { \@currname }
      }
      { \g_@@_hook_curr_name_tl }
  }
%    \end{macrocode}
% \end{macro}




% \begin{macro}{\hook_gput_code:nnn}
% \begin{macro}{\@@_gput_code:nnn,\@@_gput_code:nxv,\@@_hook_gput_code_do:nnn}
%
%    With \cs{hook_gput_code:nnn}\Arg{hook}\Arg{label}\Arg{code} a
%    chunk of \meta{code} is added to an existing \meta{hook} labeled
%    with \meta{label}.
%    \begin{macrocode}
\cs_new_protected:Npn \hook_gput_code:nnn #1 #2
  {
    \exp_args:Nxx \@@_gput_code:nnn
      { \@@_parse_label_default:nn {#1} { top-level } }
      { \@@_parse_label_default:nn {#2} { top-level } }
  }
\cs_new_protected:Npn \@@_gput_code:nnn #1 #2 #3
  {
%    \end{macrocode}
%    First we check if the hook exists.
%    \begin{macrocode}
    \@@_if_marked_removal:nnTF {#1} {#2}
      { \@@_unmark_removal:nn {#1} {#2} }
      {
%    \end{macrocode}
%    First we check if the hook exists.
%    \begin{macrocode}
        \hook_if_exist:nTF {#1}
%    \end{macrocode}
%    If so we simply add (or append) the new code to the property list
%    holding different chunks for the hook. At \verb=\begin{document}=
%    this is then sorted into a token list for fast execution.
%    \begin{macrocode}
          {
            \@@_hook_gput_code_do:nnn {#1} {#2} {#3}
%    \end{macrocode}
%    However, if there is an update within the document we need to alter
%    this execution code which is done by
%    \cs{@@_update_hook_code:n}. In the preamble this does nothing.
%    \begin{macrocode}
            \@@_update_hook_code:n {#1}
          }
%    \end{macrocode}
%    
%    \begin{macrocode}
          { \@@_try_declaring_generic_hook:nnn {#1} {#2} {#3} }
      }
  }
\cs_generate_variant:Nn \@@_gput_code:nnn { nxv }
%    \end{macrocode}
%
%   This macro will unconditionally add a chunk of code to the given hook.
%    \begin{macrocode}
\cs_new_protected:Npn \@@_hook_gput_code_do:nnn #1 #2 #3
  {
%   However, first some debugging info if debugging is enabled:
%    \begin{macrocode}
    \@@_debug:n{\iow_term:x{****~ Add~ to~
                      \hook_if_exist:nF {#1} { undeclared~ }
                      hook~ #1~ (#2)
                      \on@line\space <-~ \tl_to_str:n{#3}} }
%    \end{macrocode}
%   Then try to get the code chunk labeled \verb=#2= from the hook.
%   If there's code already there, then append \verb=#3= to that,
%   otherwise just put \verb=#3=.
%    \begin{macrocode}
    \prop_get:cnNTF { g_@@_#1_code_prop } {#2} \l_@@_return_tl
      {
        \prop_gput:cno { g_@@_#1_code_prop } {#2}
          { \l_@@_return_tl #3 }
      }
      { \prop_gput:cnn { g_@@_#1_code_prop } {#2} {#3} }
  }
%    \end{macrocode}
% \end{macro}
% \end{macro}
%
% \begin{macro}{\@@_gput_undeclared_hook:nnn}
%   Often it may happen that a package $A$ defines a hook \verb=foo=,
%   but package $B$, that adds code to that hook, is loaded before $A$.
%   In such case we need to add code to the hook before its declared.
%    \begin{macrocode}
\cs_new_protected:Npn \@@_gput_undeclared_hook:nnn #1 #2 #3
  {
    \@@_declare:n {#1}
    \@@_hook_gput_code_do:nnn {#1} {#2} {#3}
  }
%    \end{macrocode}
% \end{macro}
%
% \begin{macro}{\@@_try_declaring_generic_hook:nnn}
% \begin{macro}{\@@_try_declaring_generic_next_hook:nn}
%   These entry-level macros just pass the arguments along to the
%   common \cs{@@_try_declaring_generic_hook:nNNnn} with the right
%   functions to execute when some action is to be taken.
%
%   The wrapper \cs{@@_try_declaring_generic_hook:nnn} then defers
%   \cs{hook_gput_code:nnn} if the generic hook was declared, or to
%   \cs{@@_gput_undeclared_hook:nnn} otherwise (the hook was tested for
%   existence before, so at this point if it isn't generic, it doesn't
%   exist).
%
%   The wrapper \cs{@@_try_declaring_generic_next_hook:nn} for
%   next-execution hooks does the same: it defers the code to
%   \cs{hook_gput_next_code:nn} if the generic hook was declared, or
%   to \cs{@@_gput_next_do:nn} otherwise.
%    \begin{macrocode}
\cs_new_protected:Npn \@@_try_declaring_generic_hook:nnn #1
  {
    \@@_try_declaring_generic_hook:nNNnn {#1}
      \hook_gput_code:nnn \@@_gput_undeclared_hook:nnn
  }
\cs_new_protected:Npn \@@_try_declaring_generic_next_hook:nn #1
  {
    \@@_try_declaring_generic_hook:nNNnn {#1}
      \hook_gput_next_code:nn \@@_gput_next_do:nn
  }
%    \end{macrocode}
%
% \begin{macro}{
%     \@@_try_declaring_generic_hook:nNNnn,
%     \@@_try_declaring_generic_hook_split:nNNnn
%   }
% \begin{macro}[TF]{\@@_try_declaring_generic_hook:wn}
%   \cs{@@_try_declaring_generic_hook:nNNnn} now splits the hook name
%   at the first \texttt{/} (if any) and first checks if it is a
%   file-specific hook (they require some normalization) using
%   \cs{@@_if_file_hook:wTF}. If not then check it is one of a
%   predefined set for generic names. We also split off the second
%   component to see if we have to make a reversed hook.  In either case
%   the function returns \meta{true} for a generic hook and \meta{false}
%   in other cases.
%    \begin{macrocode}
\cs_new_protected:Npn \@@_try_declaring_generic_hook:nNNnn #1
  {
    \@@_if_file_hook:wTF #1 / / \s_@@_mark
      {
        \exp_args:Ne \@@_try_declaring_generic_hook_split:nNNnn
          { \exp_args:Ne \@@_file_hook_normalise:n {#1} }
      }
      { \@@_try_declaring_generic_hook_split:nNNnn {#1} }
  }
\cs_new_protected:Npn \@@_try_declaring_generic_hook_split:nNNnn #1 #2 #3
  {
    \@@_try_declaring_generic_hook:wnTF #1 / / / \scan_stop: {#1}
      { #2 }
      { #3 } {#1}
  }
\prg_new_protected_conditional:Npnn \@@_try_declaring_generic_hook:wn
    #1 / #2 / #3 / #4 \scan_stop: #5 { TF }
  {
    \tl_if_empty:nTF {#2}
      { \prg_return_false: }
      {
        \prop_if_in:NnTF \c_@@_generics_prop {#1}
          {
            \hook_if_exist:nF {#5} { \hook_new:n {#5} }
%    \end{macrocode}
%    After having declared the hook we check the second component (for
%    file hooks) or the third component for environment hooks) and
%    if it is on the list of components for which we should have declared
%    a reversed hook we alter the hook data structure accordingly.
%    \begin{macrocode}
            \prop_if_in:NnTF \c_@@_generics_reversed_ii_prop {#2}
              { \tl_gset:cn { g_@@_#5_reversed_tl } { - } }
              {
                \prop_if_in:NnT \c_@@_generics_reversed_iii_prop {#3}
                  { \tl_gset:cn { g_@@_#5_reversed_tl } { - } }
              }
%    \end{macrocode}
%    Now that we know that the hook is declared we can add the code to it.
%    \begin{macrocode}
            \prg_return_true:
          }
          { \prg_return_false: }
      }
  }
%    \end{macrocode}
% \end{macro}
% \end{macro}
% \end{macro}
% \end{macro}
%
% \begin{macro}[pTF]{\@@_if_file_hook:w}
%   \cs{@@_if_file_hook:wTF} checks if the argument is a valid
%   file-specific hook (not, for example, |file/before|, but
%   |file/before/foo.tex|).  If it is a file-specific hook, then it
%   executes the \meta{true} branch, otherwise \meta{false}.
%
%   A file-specific hook is \texttt{file/\meta{position}/\meta{name}}.
%   If any of these parts don't exist, it is a general file hook or not
%   a file hook at all, so the conditional evaluates to \meta{false}.
%   Otherwise, it checks that the first part is |file| and that the
%   \meta{position} is in the \cs{c_@@_generics_file_prop}.
%
%   A property list is used here to avoid having to worry with catcodes,
%   because \pkg{expl3}'s file name parsing turns all characters into
%   catcode-12 tokens, which might differ from hand-input letters.
%    \begin{macrocode}
\prg_new_conditional:Npnn \@@_if_file_hook:w
    #1 / #2 / #3 \s_@@_mark { TF }
  {
    \str_if_eq:nnTF {#1} { file }
      {
        \bool_lazy_or:nnTF
            { \tl_if_empty_p:n {#3} }
            { \str_if_eq_p:nn {#3} { / } }
          { \prg_return_false: }
          {
            \prop_if_in:NnTF \c_@@_generics_file_prop {#2}
              { \prg_return_true: }
              { \prg_return_false: }
          }
      }
      { \prg_return_false: }
  }
%    \end{macrocode}
% \end{macro}
%
% \begin{macro}[EXP]{\@@_file_hook_normalise:n}
% \begin{macro}[EXP]{\@@_strip_double_slash:n,\@@_strip_double_slash:w}
%   When a file-specific hook is found, before being declared it is
%   lightly normalized by \cs{@@_file_hook_normalise:n}.  The current
%   implementation just replaces two consecutive slashes (|//|) by a
%   single one, to cope with simple cases where the user did something
%   like \verb|\def\input@path{{./mypath/}}|, in which case a hook would
%   have to be \verb|\AddToHook{file/after/./mypath//file.tex}|.
%    \begin{macrocode}
\cs_new:Npn \@@_file_hook_normalise:n #1
  { \@@_strip_double_slash:n {#1} }
\cs_new:Npn \@@_strip_double_slash:n #1
  { \@@_strip_double_slash:w #1 // \s_@@_mark }
\cs_new:Npn \@@_strip_double_slash:w #1 // #2 \s_@@_mark
  {
    \tl_if_empty:nTF {#2}
      {#1}
      { \@@_strip_double_slash:w #1 / #2 \s_@@_mark }
  }
%    \end{macrocode}
% \end{macro}
% \end{macro}





%  \begin{macro}{\c_@@_generics_prop}
%    Clist holding the generic names. We don't provide any user
%    interface to this as this is meant to be static.
%    \begin{description}
%    \item[\texttt{env}]
%      The generic hooks used in \cs{begin} and \cs{end}.
%    \item[\texttt{file}]
%      The generic hooks used when loading a file
%    \end{description}
%    \begin{macrocode}
\prop_const_from_keyval:Nn \c_@@_generics_prop
  {env=,file=,package=,class=,include=}
%    \end{macrocode}
%  \end{macro}
%
%  \begin{macro}{\c_@@_generics_reversed_ii_prop,
%                \c_@@_generics_reversed_iii_prop,
%                \c_@@_generics_file_prop}
%    Some of the generic hooks are supposed to use reverse ordering, these are
%    the following (only the second or third sub-component is checked):
%    \begin{macrocode}
\prop_const_from_keyval:Nn \c_@@_generics_reversed_ii_prop {after=,end=}
\prop_const_from_keyval:Nn \c_@@_generics_reversed_iii_prop {after=}
\prop_const_from_keyval:Nn \c_@@_generics_file_prop {before=,after=}
%    \end{macrocode}
%  \end{macro}


%  \begin{macro}{\@@_update_hook_code:n}
%    Before \verb=\begin{document}=  this does nothing, in the body it
%    reinitializes the hook code using the altered data.
%    \begin{macrocode}
\cs_new_eq:NN \@@_update_hook_code:n \use_none:n
%    \end{macrocode}
%  \end{macro}


% \begin{macro}{\hook_gremove_code:nn}
% \begin{macro}{\@@_gremove_code:nn}
%    
%    With \cs{hook_gremove_code:nn}\Arg{hook}\Arg{label} any code
%    for \meta{hook} stored under \meta{label} is removed.
%    \begin{macrocode}
\cs_new_protected:Npn \hook_gremove_code:nn #1 #2
  {
    \exp_args:Nxx \@@_gremove_code:nn
      { \@@_parse_label_default:nn {#1} { top-level } }
      { \@@_parse_label_default:nn {#2} { top-level } }
  }
\cs_new_protected:Npn \@@_gremove_code:nn #1 #2
  {
%    \end{macrocode}
%    First check that the hook code pool exists.  \cs{hook_if_exist:nTF}
%    isn't used here because it should be possible to remove code from a
%    hook before its defined (see section~\ref{sec:querying}).
%    \begin{macrocode}
    \@@_if_exist:nTF {#1}
%    \end{macrocode}
%    Then remove the chunk and run \cs{@@_update_hook_code:n} so
%    that the execution token list reflects the change if we are after
%    \verb=\begin{document}=.
%    \begin{macrocode}
      {
        \str_if_eq:nnTF {#2} {*}
          {
            \prop_gclear:c { g_@@_#1_code_prop }
            \clist_gclear:c { g_@@_#1_labels_clist } % for debugging only
          }
          {
%    \end{macrocode}
%    Check if the label being removed exists in the code pool.  If it does,
%    just call \cs{@@_gremove_code_do:nn} to do the removal, otherwise mark it
%    to be removed.
%    \begin{macrocode}
            \prop_get:cnNTF { g_@@_#1_code_prop } {#2} \l_@@_return_tl
              { \@@_gremove_code_do:nn }
              { \@@_mark_removal:nn }
                  {#1} {#2}
          }
%    \end{macrocode}
%    Finally update the code, if the hook exists.
%    \begin{macrocode}
        \hook_if_exist:nT {#1}
          { \@@_update_hook_code:n {#1} }
      }
      { \@@_mark_removal:nn {#1} {#2} }
  }
%    \end{macrocode}
%
% \begin{macro}{\@@_gremove_code_do:nn}
%    \begin{macrocode}
\cs_new_protected:Npn \@@_gremove_code_do:nn #1 #2
  {
    \prop_gremove:cn { g_@@_#1_code_prop } {#2}
%    \end{macrocode}
%    Removing the dropped label from \verb=\g_@@_#1_labels_clist= is
%    rather tricky, because that clists holds the labels as strings
%    (i.e., not ordinary text which is what we have in \verb=#2=).
%    \begin{macrocode}
    \exp_args:Nco \clist_gremove_all:Nn
      { g_@@_#1_labels_clist } { \tl_to_str:n {#2} } % for debugging only
  }
%    \end{macrocode}
% \end{macro}
% \end{macro}
% \end{macro}


%
% \begin{macro}{\@@_mark_removal:nn}
%   Marks \meta{label} (\verb=#2=) to be removed from \meta{hook}
%   (\verb=#1=).
%    \begin{macrocode}
\cs_new_protected:Npn \@@_mark_removal:nn #1 #2
  {
    \tl_gput_right:Nx \g_@@_removal_list_tl
      { \@@_removal_tl:nn {#1} {#2} }
  }
%    \end{macrocode}
% \end{macro}
%
% \begin{macro}{\@@_unmark_removal:nn}
%   Unmarks \meta{label} (\verb=#2=) to be removed from \meta{hook}
%   (\verb=#1=).  \cs{tl_gremove_once:Nx} is used rather than
%   \cs{tl_gremove_all:Nx} so that two additions are needed to cancel
%   two marked removals, rather than only one.
%    \begin{macrocode}
\cs_new_protected:Npn \@@_unmark_removal:nn #1 #2
  {
    \tl_gremove_once:Nx \g_@@_removal_list_tl
      { \@@_removal_tl:nn {#1} {#2} }
  }
%    \end{macrocode}
% \end{macro}
%
% \begin{macro}[TF]{\@@_if_marked_removal:nn}
%   Checks if the \cs{g_@@_removal_list_tl} contains the current
%   \meta{label} (\verb=#2=) and \meta{hook} (\verb=#1=).
%    \begin{macrocode}
\prg_new_protected_conditional:Npnn \@@_if_marked_removal:nn #1 #2 { TF }
  {
    \exp_args:NNx \tl_if_in:NnTF \g_@@_removal_list_tl
      { \@@_removal_tl:nn {#1} {#2} }
      { \prg_return_true: } { \prg_return_false: }
  }
%    \end{macrocode}
% \end{macro}
%
% \begin{macro}[rEXP]{\@@_removal_tl:nn}
%   Builds a token list with \verb=#1= and \verb=#2= which can only be
%   matched by \verb=#1= and \verb=#2=.
%    \begin{macrocode}
\cs_new:Npn \@@_removal_tl:nn #1 #2
  { & \tl_to_str:n {#2} $ \tl_to_str:n {#1} $ }
%    \end{macrocode}
% \end{macro}




%
%
%
% \begin{macro}{
%     \g_@@_??_rules_prop,
%     \g_@@_??_code_prop,
%     \g_@@_??_code_tl,
%     \g_@@_??_reversed_tl,
%   }
%
%    Default rules applying to all hooks are stored in this property
%    list. Initially it simply used an empty ``label'' name (not two
%    question marks). This was a bit unfortunate, because then
%    \texttt{l3doc} complains about \verb=__= in the middle of a
%    command name when trying to typeset the documentation. However
%    using a ``normal'' name such as \texttt{default} has the
%    disadvantage of that being not really distinguishable from a real
%    hook name. I now have settled for \texttt{??} which needs some
%    gymnastics to get it into the csname, but since this is used a
%    lot things should be fast, so this is not done with \texttt{c}
%    expansion in the code later on.
%
%    \cs{g_@@_??_code_tl} isn't used, but it has to be defined to trick
%    the code into thinking that \verb=??= is actually a hook.
%    \begin{macrocode}
\prop_new:c {g_@@_??_rules_prop}
\prop_new:c {g_@@_??_code_prop}
\prop_new:c {g_@@_??_code_tl}
%    \end{macrocode}
%
%    Default rules are always given in normal ordering (never in
%    reversed ordering). If such a rule is applied to a reversed
%    hook it behaves as if the rule is reversed (e.g.,
%    \texttt{after} becomes \texttt{before})
%    because those rules are applied first and then the order is reversed.
%    \begin{macrocode}
\tl_new:c {g_@@_??_reversed_tl}
%    \end{macrocode}
%  \end{macro}



%  \begin{macro}{\@@_debug_gset_rule:nnnn}
%
%    \fmi{this needs cleanup and docu correction!}
%
%    With
%    \cs{@@_debug_gset_rule:nnnn}\Arg{hook}\Arg{label1}\Arg{relation}\Arg{label2}
%    a relation is defined between the two code labels for the given
%    \meta{hook}.  The special hook \texttt{??} stands for \emph{any}
%    hook describing a default rule.
%    \begin{macrocode}
\cs_new_protected:Npn \@@_debug_gset_rule:nnnn #1#2#3#4
  {
%    \end{macrocode}
%    If so we drop any existing rules with the two labels (in case
%    there are any).
%    \begin{macrocode}
    \prop_gremove:cn{g_@@_#1_rules_prop}{#2|#4}
    \prop_gremove:cn{g_@@_#1_rules_prop}{#4|#2}
%    \end{macrocode}
%    Then  we add the new one (normalizing the input a bit, e.g., we
%    always use \texttt{before} and not \texttt{after} and
%    instead reorder the labels):
%    \begin{macrocode}
    \str_case_e:nnF {#3}
       {
         {before} { \prop_gput:cnn {g_@@_#1_rules_prop}{#2|#4}{<} }
         {after}  { \prop_gput:cnn {g_@@_#1_rules_prop}{#4|#2}{<} }
%    \end{macrocode}
%    More special rule types \ldots
%    \begin{macrocode}
         {incompatible-error}   { \prop_gput:cnn {g_@@_#1_rules_prop}{#2|#4}{xE} }
         {incompatible-warning} { \prop_gput:cnn {g_@@_#1_rules_prop}{#2|#4}{xW} }
         {removes}      { \prop_gput:cnn {g_@@_#1_rules_prop}{#2|#4}{->} }
%    \end{macrocode}
%    Undo a setting:
%    \begin{macrocode}
         {unrelated}{ \prop_gremove:cn {g_@@_#1_rules_prop}{#2|#4} 
                      \prop_gremove:cn {g_@@_#1_rules_prop}{#4|#2} }
       }
       { \ERRORunknownrule }
  }
%    \end{macrocode}
%  \end{macro}
%
%
%  \subsection{Setting rules for hooks code}
%
%  \begin{macro}{\hook_gset_rule:nnnn}
%  \begin{macro}{\@@_gset_rule:nnnn}
%
%    \fmi{needs docu correction given new implementation}
%
%    With
%    \cs{hook_gset_rule:nnnn}\Arg{hook}\Arg{label1}\Arg{relation}\Arg{label2}
%    a relation is defined between the two code labels for the given
%    \meta{hook}.  The special hook \texttt{??} stands for \emph{any}
%    hook describing a default rule.
%    \begin{macrocode}
\cs_new_protected:Npn \hook_gset_rule:nnnn #1#2#3#4
  {
    \use:x
      {
        \@@_gset_rule:nnnn
          { \@@_parse_label_default:nn {#1} { top-level } }
          { \@@_parse_label_default:nn {#2} { top-level } }
          {#3}
          { \@@_parse_label_default:nn {#4} { top-level } }
      }
  }
%    \end{macrocode}
%    
%    \begin{macrocode}
\cs_new_protected:Npn \@@_gset_rule:nnnn #1#2#3#4
  {
%    \end{macrocode}
%    First we ensure the basic data structure of the hook exists:
%    \begin{macrocode}
    \@@_declare:n {#1}
%    \end{macrocode}
%    Then we clear any previous relationship between both labels.
%    \begin{macrocode}
    \@@_rule_gclear:nnn {#1} {#2} {#4}
%    \end{macrocode}
%    Then we call the function to handle the given rule. Throw an error if the
%    rule is invalid.
%    \begin{macrocode}
    \debug_suspend:
    \cs_if_exist_use:cTF { @@_rule_#3_gset:nnn }
      {
          {#1} {#2} {#4}
        \@@_update_hook_code:n {#1}
      }
      { \ERRORunknownrule }
    \debug_resume:
    \@@_debug_gset_rule:nnnn {#1} {#2} {#3} {#4} % for debugging
  }
%    \end{macrocode}
% \end{macro}
% \end{macro}
%
% \begin{macro}{\@@_rule_before_gset:nnn, \@@_rule_after_gset:nnn,
%               \@@_rule_<_gset:nnn, \@@_rule_>_gset:nnn}
%    Then we add the new rule.  We need to normalize the rules here to
%    allow for faster processing later.  Given a pair of labels
%    $l_A$ and $l_B$, the rule $l_A>l_B$ is the same as $l_B<l_A$\fmi{}
%    said differently.  But normalizing the
%    forms of the rule to a single representation, say, $l_B<l_A$, then
%    the time spent looking for the rules later is considerably reduced.
%
%    Here we do that normalization by using \cs[no-index]{(pdf)strcmp} to
%    lexically sort labels $l_A$ and $l_B$ to a fixed order.  This order
%    is then enforced every time these two labels are used together.
%
%    Here we use \cs{@@_label_pair:nn}~\Arg{hook}~\Arg{l_A}~\Arg{l_B}
%    to build a string \texttt{$l_B$\string|$l_A$} with a fixed order, and
%    use \cs{@@_label_ordered:nnTF} to apply the correct rule to the pair
%    of labels, depending if it was sorted or not.
%    \begin{macrocode}
\cs_new_protected:Npn \@@_rule_before_gset:nnn #1#2#3
  {
    \tl_gset:cx { g_@@_#1_rule_ \@@_label_pair:nn {#2} {#3} _tl }
      { \@@_label_ordered:nnTF {#2} {#3} { < } { > } }
  }
\cs_new_eq:cN { @@_rule_<_gset:nnn } \@@_rule_before_gset:nnn
%    \end{macrocode}
%
%    \begin{macrocode}
\cs_new_protected:Npn \@@_rule_after_gset:nnn #1#2#3
  {
    \tl_gset:cx { g_@@_#1_rule_ \@@_label_pair:nn {#3} {#2} _tl }
      { \@@_label_ordered:nnTF {#3} {#2} { < } { > } }
  }
\cs_new_eq:cN { @@_rule_>_gset:nnn } \@@_rule_after_gset:nnn
%    \end{macrocode}
%  \end{macro}
%
% \begin{macro}{\@@_rule_removes_gset:nnn}
%   This rule removes (clears, actually) the code from label |#3| if
%   label |#2| is in the hook |#1|.
%    \begin{macrocode}
\cs_new_protected:Npn \@@_rule_removes_gset:nnn #1#2#3
  {
    \tl_gset:cx { g_@@_#1_rule_ \@@_label_pair:nn {#2} {#3} _tl }
      { \@@_label_ordered:nnTF {#2} {#3} { -> } { <- } }
  }
%    \end{macrocode}
%  \end{macro}
%
% \begin{macro}{
%     \@@_rule_incompatible-error_gset:nnn,
%     \@@_rule_incompatible-warning_gset:nnn,
%   }
%   These relations make an error/warning if labels |#2| and |#3| appear
%   together in hook |#1|.
%    \begin{macrocode}
\cs_new_protected:cpn { @@_rule_incompatible-error_gset:nnn } #1#2#3
  { \tl_gset:cn { g_@@_#1_rule_ \@@_label_pair:nn {#2} {#3} _tl } { xE } }
\cs_new_protected:cpn { @@_rule_incompatible-warning_gset:nnn } #1#2#3
  { \tl_gset:cn { g_@@_#1_rule_ \@@_label_pair:nn {#2} {#3} _tl } { xW } }
%    \end{macrocode}
%  \end{macro}
%
% \begin{macro}{\@@_rule_unrelated_gset:nnn, \@@_rule_gclear:nnn}
%    Undo a setting. \cs{@@_rule_unrelated_gset:nnn} doesn't need to do anything,
%    since we use \cs{@@_rule_gclear:nnn} before setting any rule.
%    \begin{macrocode}
\cs_new_protected:Npn \@@_rule_unrelated_gset:nnn #1#2#3 { }
\cs_new_protected:Npn \@@_rule_gclear:nnn #1#2#3
  { \cs_undefine:c { g_@@_#1_rule_ \@@_label_pair:nn {#2} {#3} _tl } }
%    \end{macrocode}
%  \end{macro}
%
% \begin{macro}[EXP]{\@@_label_pair:nn}
%   Ensure that the lexically greater label comes first.
%    \begin{macrocode}
\cs_new:Npn \@@_label_pair:nn #1#2
  {
    \if_case:w \@@_str_compare:nn {#1} {#2} \exp_stop_f:
           #1 | #1 %  0
    \or:   #1 | #2 % +1
    \else: #2 | #1 % -1
    \fi:
  }
%    \end{macrocode}
%  \end{macro}
%
% \begin{macro}[pTF]{\@@_label_ordered:nn}
%   Check that labels |#1| and |#2| are in the correct order (as
%   returned by \cs{@@_label_pair:nn}) and if so return true, else
%   return false.
%    \begin{macrocode}
\prg_new_conditional:Npnn \@@_label_ordered:nn #1#2 { TF }
  {
    \if_int_compare:w \@@_str_compare:nn {#1} {#2} > 0 \exp_stop_f:
      \prg_return_true:
    \else
      \prg_return_false:
    \fi:
  }
%    \end{macrocode}
%  \end{macro}
%
% \begin{macro}[EXP]{\@@_if_label_case:nnnnn}
%   To avoid doing the string comparison twice in \cs{@@_initialize_single:NNNNn}
%   (once with \cs{str_if_eq:nn} and again with \cs{@@_label_ordered:nn}),
%   we use a three-way branching macro that will compare |#1| and |#2|
%   and expand to \cs{use_i:nnn} if they are equal, \cs{use_ii:nn} if
%   |#1| is lexically greater, and \cs{use_iii:nn} otherwise.
%    \begin{macrocode}
\cs_new:Npn \@@_if_label_case:nnnnn #1#2
   {
     \cs:w use_
       \if_case:w \@@_str_compare:nn {#1} {#2}
          i \or: ii \else: iii \fi: :nnn
     \cs_end:
   }
%    \end{macrocode}
%  \end{macro}


%  \begin{macro}{\@@_initialize_all:}
%    Initialize all known hooks (at \verb=\begin{document}=), i.e.,
%    update the fast execution token lists to hold the necessary code
%    in the right  order.
%    \begin{macrocode}
\cs_new_protected:Npn \@@_initialize_all: {
%    \end{macrocode}
%    First we change \cs{@@_update_hook_code:n} which so far was a
%    no-op to now initialize one hook. This way any later updates to
%    the hook will run that code and also update the execution token
%    list.
%    \begin{macrocode}
  \cs_gset_eq:NN \@@_update_hook_code:n \@@_initialize_hook_code:n
%    \end{macrocode}
%    Now we loop over all hooks that have been defined and update each
%    of them.
%    \begin{macrocode}
  \@@_debug:n { \prop_gclear:N \g_@@_used_prop }
  \seq_map_inline:Nn \g_@@_all_seq
      {
        \@@_update_hook_code:n {##1}
      }
%    \end{macrocode}
%    If we are debugging we show results hook by hook for all hooks
%    that have data. 
%    \begin{macrocode}
  \@@_debug:n
     { \iow_term:x{^^JAll~ initialized~ (non-empty)~ hooks:}
       \prop_map_inline:Nn \g_@@_used_prop
           { \iow_term:x{^^J~ ##1~ ->~
               \exp_not:v {g_@@_##1_code_tl}~ }
           }
     }
%    
%    \end{macrocode}
%    After all hooks are initialized we change the ``use'' to just
%    call the hook code and not initialize it (as it was done in the
%    preamble.
%    \begin{macrocode}
  \cs_gset_eq:NN \hook_use:n \@@_use_initialized:n
  \cs_gset_eq:NN \@@_preamble_hook:n \use_none:n
}
%    \end{macrocode}
%  \end{macro}



%  \begin{macro}{\@@_initialize_hook_code:n}
%    Initializing or reinitializing the fast execution hook code. In
%    the preamble this is selectively done in case a hook gets used
%    and at \verb=\begin{document}= this is done for all hooks and
%    afterwards only if the hook code changes.
%    \begin{macrocode}
\cs_new_protected:Npn \@@_initialize_hook_code:n #1 {
  \@@_debug:n{ \iow_term:x{^^JUpdate~ code~ for~ hook~
                                  '#1' \on@line :^^J} }
%    \end{macrocode}
%    This does the sorting and the updates. If there aren't any code
%    chunks for the current hook, there is no point in even starting
%    the sorting routine so we make a quick test for that and in that
%    case just update \cs{g_@@_\meta{hook}_code_tl} to hold the next
%    code. If there are code chunks we call
%    \cs{@@_initialize_single:NNNNn} and pass to it ready made csnames
%    as they are needed several times inside. This way we save a bit
%    on processing time if we do that up front.
%    \begin{macrocode}
  \hook_if_exist:nT {#1}
    {
      \prop_if_empty:cTF {g_@@_#1_code_prop}
        { \tl_gset:co {g_@@_#1_code_tl}
                      {\cs:w g_@@_#1_next_code_tl \cs_end: } }
        {
%    \end{macrocode}
%    By default the algorithm sorts the code chunks and then saves the
%    result in a token list for fast execution by adding the code one
%    after another using \cs{tl_gput_right:NV}. When we sort code for
%    a reversed hook, all we have to do is to add the code chunks in
%    the opposite order into the token list. So all we have to do
%    in preparation is to change two definitions used later on.
%    \begin{macrocode}
          \@@_if_reversed:nTF {#1}
            { \cs_set_eq:NN \@@_tl_gput:NV    \tl_gput_left:NV
              \cs_set_eq:NN \@@_clist_gput:NV \clist_gput_left:NV  }
            { \cs_set_eq:NN \@@_tl_gput:NV    \tl_gput_right:NV
              \cs_set_eq:NN \@@_clist_gput:NV \clist_gput_right:NV }
%    \end{macrocode}
%
%    When sorting, some relations (namely \verb|->| \verb|<-|) need to
%    act destructively on the code property lists to remove code that
%    shouldn't appear in the sorted hook token list.
%    \begin{macrocode}
          \prop_gset_eq:Nc \g_@@_code_temp_prop { g_@@_#1_code_prop }
          \@@_initialize_single:ccccn
            { g_@@_#1_code_prop } { g_@@_#1_code_tl }
            { g_@@_#1_next_code_tl } { g_@@_#1_labels_clist }
            {#1}
          \prop_gset_eq:cN { g_@@_#1_code_prop } \g_@@_code_temp_prop
%    \end{macrocode}
%    For debug display we want to keep track of those hooks that
%    actually got code added to them, so we record that in plist. We
%    use a plist to ensure that we record each hook name only once,
%    i.e., we are only interested in storing the keys and the value is arbitrary
%    \begin{macrocode}
          \@@_debug:n{ \exp_args:NNx \prop_gput:Nnn \g_@@_used_prop {#1}{} }
        }
    }
}
%    \end{macrocode}
%  \end{macro}


%  \begin{macro}{\g_@@_used_prop}
%    All hooks that receive code (for use in debugging display).
%    \begin{macrocode}
\prop_new:N\g_@@_used_prop
%    \end{macrocode}
%  \end{macro}


%
% \begin{macro}[EXP]{\@@_tl_csname:n,\@@_seq_csname:n}
%   It is faster to pass a single token and expand it when necessary
%   than to pass a bunch of character tokens around.
%   \fmi{note to myself: verify}
%    \begin{macrocode}
\cs_new:Npn \@@_tl_csname:n #1 { l_@@_label_#1_tl }
\cs_new:Npn \@@_seq_csname:n #1 { l_@@_label_#1_seq }
%    \end{macrocode}
% \end{macro}
%

%
%
%  \begin{macro}{\l_@@_labels_seq,\l_@@_labels_int,\l_@@_front_tl,
%      \l_@@_rear_tl,\l_@@_label_0_tl}
%
%    For the sorting I am basically implementing Knuth's algorithm for
%    topological sorting as given in TAOCP volume 1 pages 263--266.
%    For this algorithm we need a number of local variables:
%    \begin{itemize}
%    \item
%       List of labels used in the current hook to label code chunks:
%    \begin{macrocode}
\seq_new:N \l_@@_labels_seq
%    \end{macrocode}
%    \item
%      Number of labels used in the current hook. In Knuth's algorithm
%      this is called $N$:
%    \begin{macrocode}
\int_new:N \l_@@_labels_int
%    \end{macrocode}
%    \item
%      The sorted code list to be build is managed using two pointers
%      one to the front of the queue and one to the rear. We model this
%      using token list pointers. Knuth calls them $F$ and $R$:
%    \begin{macrocode}
\tl_new:N \l_@@_front_tl
\tl_new:N \l_@@_rear_tl
%    \end{macrocode}
%    \item
%      The data for the start of the queue is kept in this token list,
%      it corresponds to what Don calls \texttt{QLINK[0]} but since we
%      aren't manipulating individual words in memory it is slightly
%      differently done:
%    \begin{macrocode}
\tl_new:c { \@@_tl_csname:n { 0 } }
%    \end{macrocode}
%
%    \end{itemize}
%  \end{macro}


%  \begin{macro}{\@@_initialize_single:NNNNn,\@@_initialize_single:ccccn}
%
%    \cs{@@_initialize_single:NNNNn} implements the sorting of the code
%    chunks for a hook and saves the result in the token list for fast
%    execution (\verb=#3=). The arguments are \meta{hook-code-plist},
%    \meta{hook-code-tl}, \meta{hook-next-code-tl},
%    \meta{hook-ordered-labels-clist} and \meta{hook-name} (the latter
%    is only used for debugging---the \meta{hook-rule-plist} is accessed
%    using the \meta{hook-name}).
%
%    The additional complexity compared to Don's algorithm is that we
%    do not use simple positive integers but have arbitrary
%    alphanumeric labels. As usual Don's data structures are chosen in
%    a way that one can omit a lot of tests and I have mimicked that as
%    far as possible. The result is a restriction I do not test for at
%    the moment: a label can't be equal to the number 0!  \fmi{Needs
%    checking for, just in case}
%    \begin{macrocode}
\cs_new_protected:Npn \@@_initialize_single:NNNNn #1#2#3#4#5 {
%    \end{macrocode}
%    
%    \begin{macrocode}
  \debug_suspend:
%    \end{macrocode}
%    Step T1: Initialize the data structure \ldots
%    \begin{macrocode}
  \seq_clear:N \l_@@_labels_seq
  \int_zero:N  \l_@@_labels_int
%    \end{macrocode}
%
%    Store the name of the hook:
%    \begin{macrocode}
  \tl_set:Nn \l_@@_cur_hook_tl {#5}
%    \end{macrocode}
%    
%    We loop over the property list holding the code and record all
%    labels listed there. Only rules for those labels are of interest
%    to us. While we are at it we count them (which gives us the $N$
%    in Knuth's algorithm.  The prefix |label_| is added to the variables
%    to ensure that labels named |front|, |rear|, |labels|, or |return|
%    don't interact with our code.
%    \begin{macrocode}
  \prop_map_inline:Nn #1
     {
       \int_incr:N \l_@@_labels_int
       \seq_put_right:Nn \l_@@_labels_seq {##1}
       \tl_set:cn { \@@_tl_csname:n {##1} }{0}     % the counter k for number of
                                                   % j before k rules
       \seq_clear_new:c { \@@_seq_csname:n {##1} } % sequence of successors to k
                                                   % i.e., k before j rules (stores
                                                   % the names of the j's)
     }
%    \end{macrocode}
%    Steps T2 and T3: Sort the relevant rules into the data structure\ldots
%    
%    This loop constitutes a square matrix of the labels in |#1| in the
%    vertical and the horizontal directions.  However since the rule
%    $l_A\meta{rel}l_B$ is the same as $l_B\meta{rel}^{-1}l_A$ we can cut
%    the loop short at the diagonal of the matrix (\emph{i.e.}, when
%    both labels are equal), saving a good amount of time.  The way the
%    rules were set up (see the implementation of \cs{@@_rule_before_gset:nnn}
%    above) ensures that we have no rule in the ignored side of the
%    matrix, and all rules are seen.  The rules are applied in
%    \cs{@@_apply_label_pair:nnn}, which takes the properly-ordered pair
%    of labels as argument.
%    \begin{macrocode}
  \prop_map_inline:Nn #1
    {
      \prop_map_inline:Nn #1
        {
          \@@_if_label_case:nnnnn {##1} {####1}
            { \prop_map_break: }
            { \@@_apply_label_pair:nnn {##1} {####1} }
            { \@@_apply_label_pair:nnn {####1} {##1} }
                {#5}
        }
    }
%    \end{macrocode}
%    Take a breath and take a look at the data structures that have
%    been set up:
%    \begin{macrocode}
  \@@_debug:n { \@@_debug_label_data:N #1 }
%    \end{macrocode}
%    
%
%    Step T4:
%    \begin{macrocode}
  \tl_set:Nn \l_@@_rear_tl { 0 }
  \tl_set:cn { \@@_tl_csname:n { 0 } } { 0 } % really {l_@@_label_ \l_@@_rear_tl _tl}
  \seq_map_inline:Nn \l_@@_labels_seq
      {
        \int_compare:nNnT { \cs:w \@@_tl_csname:n {##1} \cs_end: } = 0
            {
              \tl_set:cn { \@@_tl_csname:n { \l_@@_rear_tl } }{##1}
              \tl_set:Nn \l_@@_rear_tl {##1}
            }
      }
  \tl_set_eq:Nc \l_@@_front_tl { \@@_tl_csname:n { 0 } }
%    \end{macrocode}
%    
%    \begin{macrocode}
  \tl_gclear:N #2
  \clist_gclear:N #4
%    \end{macrocode}
%
%    The whole loop combines steps T5--T7:
%    \begin{macrocode}
  \bool_while_do:nn { ! \str_if_eq_p:Vn \l_@@_front_tl { 0 } }
       {
%    \end{macrocode}
%    This part is step T5:
%    \begin{macrocode}
         \int_decr:N \l_@@_labels_int
         \prop_get:NVN #1 \l_@@_front_tl \l_@@_return_tl
         \@@_tl_gput:NV #2 \l_@@_return_tl
%    \end{macrocode}
%    
%    \begin{macrocode}
         \@@_clist_gput:NV #4 \l_@@_front_tl
         \@@_debug:n{ \iow_term:x{Handled~ code~ for~ \l_@@_front_tl} }
%    \end{macrocode}
%
%    This is step T6 except that we don't use a pointer $P$ to move
%    through the successors, but instead use \verb=##1= of the mapping
%    function.
%    \begin{macrocode}
         \seq_map_inline:cn { \@@_seq_csname:n { \l_@@_front_tl } }
             {
               \tl_set:cx { \@@_tl_csname:n {##1} }
                   { \int_eval:n { \cs:w \@@_tl_csname:n {##1} \cs_end: - 1 } }
               \int_compare:nNnT { \cs:w \@@_tl_csname:n {##1} \cs_end: } = 0
                   {
                     \tl_set:cn { \@@_tl_csname:n { \l_@@_rear_tl } } {##1}
                     \tl_set:Nn \l_@@_rear_tl            {##1}
                   }
             }
%    \end{macrocode}
%    and step T7:
%    \begin{macrocode}
         \tl_set_eq:Nc \l_@@_front_tl { \@@_tl_csname:n { \l_@@_front_tl } }
%    \end{macrocode}
%
%    This is step T8: If we haven't moved the code for all labels
%    (i.e., if \cs{l_@@_labels_int} is still greater than zero) we
%    have a loop and our partial order can't be flattened out.
%    \begin{macrocode}
       }
  \int_compare:nNnF \l_@@_labels_int = 0
      {
        \iow_term:x{====================}
        \iow_term:x{Error:~ label~ rules~ are~ incompatible:}
%    \end{macrocode}
%
%    This is not really the information one needs in the error case
%    but will do for now \ldots \fmi{fix}
%    \begin{macrocode}
        \@@_debug_label_data:N #1
        \iow_term:x{====================}
      }
%    \end{macrocode}
%    After we have added all hook code to \verb=#2= we finish it off
%    with adding extra code for a one time execution. That is stored
%    in \verb=#3= but is normally empty.
%    \begin{macrocode}
  \tl_gput_right:Nn #2 {#3}
%    \end{macrocode}
%    
%    \begin{macrocode}
  \debug_resume:
}
%    \end{macrocode}
%    
%    \begin{macrocode}
\cs_generate_variant:Nn \@@_initialize_single:NNNNn {cccc}
%    \end{macrocode}
%  \end{macro}



%  \begin{macro}{\@@_tl_gput:NV,\@@_clist_gput:NV}
%    These append either on the right (normal hook) or on the left
%    (reversed hook). This is setup up in
%    \cs{@@_initialize_hook_code:n}, elsewhere their behavior is undefined.
%    \begin{macrocode}
\cs_new:Npn \@@_tl_gput:NV     {\ERROR}
\cs_new:Npn \@@_clist_gput:NV  {\ERROR}
%    \end{macrocode}
%  \end{macro}


%
%  \begin{macro}{\@@_apply_label_pair:nnn,\@@_label_if_exist_apply:nnnF}
%
%    This is the payload of steps T2 and T3 executed in the loop described
%    above. This macro assumes |#1| and |#2| are ordered, which means that
%    any rule pertaining the pair |#1| and |#2| is
%    \cs{g_@@_\meta{hook}_rule_\#1\string|\#2_tl}, and not
%    \cs{g_@@_\meta{hook}_rule_\#2\string|\#1_tl}.  This also saves a great deal
%    of time since we only need to check the order of the labels once.
%
%    The arguments here are \meta{label1}, \meta{label2}, \meta{hook}, and
%    \meta{hook-code-plist}.  We are about to apply the next rule and
%    enter it into the data structure.  \cs{@@_apply_label_pair:nnn} will
%    just call \cs{@@_label_if_exist_apply:nnnF} for the \meta{hook}, and
%    if no rule is found, also try the \meta{hook} name \verb=??=
%    denoting a default hook rule.
%
%    \cs{@@_label_if_exist_apply:nnnF} will check if the rule exists for
%    the given hook, and if so call \cs{@@_apply_rule:nnn}.
%    \begin{macrocode}
\cs_new_protected:Npn \@@_apply_label_pair:nnn #1#2#3
  {
%    \end{macrocode}
%    Extra complication: as we use default rules and local hook specific
%    rules we first have to check if there is a local rule and if that
%    exist use it. Otherwise check if there is a default rule and use
%    that.
%    \begin{macrocode}
    \@@_label_if_exist_apply:nnnF {#1} {#2} {#3}
      {
%    \end{macrocode}
%    If there is no hook-specific rule we check for a default one and
%    use that if it exists.
%    \begin{macrocode}
        \@@_label_if_exist_apply:nnnF {#1} {#2} { ?? } { }
      }
  }
\cs_new_protected:Npn \@@_label_if_exist_apply:nnnF #1#2#3
  {
    \if_cs_exist:w g_@@_ #3 _rule_ #1 | #2 _tl \cs_end:
%    \end{macrocode}
%    What to do precisely depends on the type of rule we have
%    encountered. If it is a \texttt{before} rule it will be handled by the
%    algorithm but other types need to be managed differently. All
%    this is done in \cs{@@_apply_rule:nnnN}.
%    \begin{macrocode}
      \@@_apply_rule:nnn {#1} {#2} {#3}
      \exp_after:wN \use_none:n
    \else:
      \use:nn
    \fi:
  }
%    \end{macrocode}
%  \end{macro}




%  \begin{macro}{\@@_apply_rule:nnn}
%    This is the code executed in steps T2 and T3 while looping through
%    the matrix  This is part of step T3. We are about to apply the next
%    rule and enter it into the data structure. The arguments are
%    \meta{label1}, \meta{label2}, \meta{hook-name}, and \meta{hook-code-plist}.
%    \begin{macrocode}
\cs_new_protected:Npn \@@_apply_rule:nnn #1#2#3
  {
    \cs:w @@_apply_
      \cs:w g_@@_#3_reversed_tl \cs_end: rule_
        \cs:w g_@@_ #3 _rule_ #1 | #2 _tl \cs_end: :nnn \cs_end:
      {#1} {#2} {#3}
  }
%    \end{macrocode}
% \end{macro}
%
%  \begin{macro}{\@@_apply_rule_<:nnn,\@@_apply_rule_>:nnn}
%    The most common cases are \texttt{\string<} and \texttt{\string>} so we handle
%    that first.  They are relations $\prec$ and $\succ$ in TAOCP, and
%    they dictate sorting.
%    \begin{macrocode}
\cs_new_protected:cpn { @@_apply_rule_<:nnn } #1#2#3
  {
    \@@_debug:n { \@@_msg_pair_found:nnn {#1} {#2} {#3} }
    \tl_set:cx { \@@_tl_csname:n {#2} }
       { \int_eval:n{ \cs:w \@@_tl_csname:n {#2} \cs_end: + 1 } }
    \seq_put_right:cn{ \@@_seq_csname:n {#1} }{#2}
  }
\cs_new_protected:cpn { @@_apply_rule_>:nnn } #1#2#3
  {
    \@@_debug:n { \@@_msg_pair_found:nnn {#1} {#2} {#3} }
    \tl_set:cx { \@@_tl_csname:n {#1} }
       { \int_eval:n{ \cs:w \@@_tl_csname:n {#1} \cs_end: + 1 } }
    \seq_put_right:cn{ \@@_seq_csname:n {#2} }{#1}
  }
%    \end{macrocode}
% \end{macro}
%
% \begin{macro}{\@@_apply_rule_xE:nnn,\@@_apply_rule_xW:nnn}
%   These relations make two labels incompatible within a hook.
%   |xE| makes raises an error if the labels are found in the same
%   hook, and |xW| makes it a warning.
%    \begin{macrocode}
\cs_new_protected:cpn { @@_apply_rule_xE:nnn } #1#2#3
  {
    \@@_debug:n { \@@_msg_pair_found:nnn {#1} {#2} {#3} }
    \msg_error:nnnnnn { hooks } { labels-incompatible }
      {#1} {#2} {#3} { 1 }
    \use:c { @@_apply_rule_->:nnn } {#1} {#2} {#3}
    \use:c { @@_apply_rule_<-:nnn } {#1} {#2} {#3}
  }
\cs_new_protected:cpn { @@_apply_rule_xW:nnn } #1#2#3
  {
    \@@_debug:n { \@@_msg_pair_found:nnn {#1} {#2} {#3} }
    \msg_warning:nnnnnn { hooks } { labels-incompatible }
      {#1} {#2} {#3} { 0 }
  }
%    \end{macrocode}
% \end{macro}
%
%  \begin{macro}{\@@_apply_rule_->:nnn,\@@_apply_rule_<-:nnn}
%    If we see \texttt{\detokenize{->}} we have to drop code for label
%    \verb=#3= and carry on. We could do a little better and trop
%    everything for that label since it doesn't matter where we sort
%    in the empty code. However that would complicate the algorithm a
%    lot with little gain. So we still unnecessarily try to sort it in
%    and depending on the rules that might result in a loop that is
%    otherwise resolved. If that turns out to be a real issue, we can
%    improve the code.
%
%    Here the code is removed from \cs{l_@@_cur_hook_tl} rather than
%    \verb=#3= because the latter may be \verb=??=, and the default
%    hook doesn't store any code.  Removing from \cs{l_@@_cur_hook_tl}
%    makes default rules \verb=->= and  \verb=<-= work properly.
%    \begin{macrocode}
\cs_new_protected:cpn { @@_apply_rule_->:nnn } #1#2#3
  {
    \@@_debug:n
       {
         \@@_msg_pair_found:nnn {#1} {#2} {#3}
         \iow_term:x{--->~ Drop~ '#2'~ code~ from~
           \iow_char:N \\ g_@@_ \l_@@_cur_hook_tl _code_prop ~ because~ of~ '#1' }
       }
    \prop_gput:cnn { g_@@_ \l_@@_cur_hook_tl _code_prop } {#2} { }
  }
\cs_new_protected:cpn { @@_apply_rule_<-:nnn } #1#2#3
  {
    \@@_debug:n
       {
         \@@_msg_pair_found:nnn {#1} {#2} {#3}
         \iow_term:x{--->~ Drop~ '#1'~ code~ from~
           \iow_char:N \\ g_@@_ \l_@@_cur_hook_tl _code_prop ~ because~ of~ '#2' }
       }
    \prop_gput:cnn { g_@@_ \l_@@_cur_hook_tl _code_prop } {#1} { }
  }
%    \end{macrocode}
%  \end{macro}

% \begin{macro}{
%     \@@_apply_-rule_<:nnn,
%     \@@_apply_-rule_>:nnn,
%     \@@_apply_-rule_<-:nnn,
%     \@@_apply_-rule_->:nnn,
%     \@@_apply_-rule_x:nnn,
%   }
%   Reversed rules.
%    \begin{macrocode}
\cs_new_eq:cc { @@_apply_-rule_<:nnn  } { @@_apply_rule_>:nnn }
\cs_new_eq:cc { @@_apply_-rule_>:nnn  } { @@_apply_rule_<:nnn }
\cs_new_eq:cc { @@_apply_-rule_<-:nnn } { @@_apply_rule_<-:nnn }
\cs_new_eq:cc { @@_apply_-rule_->:nnn } { @@_apply_rule_->:nnn }
\cs_new_eq:cc { @@_apply_-rule_xE:nnn  } { @@_apply_rule_xE:nnn }
\cs_new_eq:cc { @@_apply_-rule_xW:nnn  } { @@_apply_rule_xW:nnn }
%    \end{macrocode}
% \end{macro}


% \begin{macro}{\@@_msg_pair_found:nnn}
%   A macro to avoid moving this many tokens around.
%    \begin{macrocode}
\cs_new_protected:Npn \@@_msg_pair_found:nnn #1#2#3
  {
    \iow_term:x{~ \str_if_eq:nnTF {#3} {??} {default} {~normal} ~
               rule~ \@@_label_pair:nn {#1} {#2}:~
             \use:c { g_@@_#3_rule_ \@@_label_pair:nn {#1} {#2} _tl } ~ found}
  }
%    \end{macrocode}
% \end{macro}


%  \begin{macro}{\@@_debug_label_data:N}
%    
%    \begin{macrocode}
\cs_new_protected:Npn \@@_debug_label_data:N #1 {
  \iow_term:x{Code~ labels~ for~ sorting:}
  \iow_term:x{~ \seq_use:Nnnn\l_@@_labels_seq {~and~}{,~}{~and~} }  % fix name!
  \iow_term:x{^^J Data~ structure~ for~ label~ rules:}
  \prop_map_inline:Nn #1
       {
         \iow_term:x{~ ##1~ =~ \tl_use:c{ \@@_tl_csname:n {##1} }~ ->~
           \seq_use:cnnn{ \@@_seq_csname:n {##1} }{~->~}{~->~}{~->~}
         }
       }
  \iow_term:x{}
}
%    \end{macrocode}
%  \end{macro}



%  \begin{macro}{\hook_log:n}
%    This writes out information about the hook given in its argument
%    onto the terminal and the \texttt{.log} file.
%    \begin{macrocode}
\cs_new_protected:Npn \hook_log:n #1
  {
    \exp_args:Nx \@@_log:n
      { \@@_parse_label_default:nn {#1} { top-level } }
  }
\cs_new_protected:Npn \@@_log:n #1
  {
    \iow_term:x{^^JThe~ hook~ '#1':}
%    \end{macrocode}
%    
%    \begin{macrocode}
    \hook_if_exist:nF {#1}
      { \iow_term:x {~Hook~ is~ not~ declared!} }
    \@@_if_exist:nTF {#1}
      {
        \iow_term:x{~Code~ chunks:}
        \prop_if_empty:cTF {g_@@_#1_code_prop}
          { \iow_term:x{\@spaces ---} }
          {
            \prop_map_inline:cn {g_@@_#1_code_prop}
              { \iow_term:x{\@spaces ##1~ ->~ \tl_to_str:n{##2} } }
          }
%    \end{macrocode}
%    
%    \begin{macrocode}
        \iow_term:x{~Extra~ code~ next~ invocation:}
        \iow_term:x{\@spaces
          \tl_if_empty:cTF { g_@@_#1_next_code_tl }
            {---} {->~ \str_use:c{g_@@_#1_next_code_tl} } }
%    \end{macrocode}
%
%    \fmi{This is currently only displaying the local rules, but it
%         should also show the matching global rules!}
%
%    \begin{macrocode}
        \iow_term:x{~Rules:}
        \prop_if_empty:cTF {g_@@_#1_rules_prop}
          { \iow_term:x{\@spaces ---} }
          { \prop_map_inline:cn {g_@@_#1_rules_prop}
              { \iow_term:x{\@spaces ##1~ with~ relation~ ##2} }
          }
%    \end{macrocode}
%    
%    \begin{macrocode}
        \hook_if_exist:nT {#1}
          { \iow_term:x { ~Execution~ order
               \prop_if_empty:cTF {g_@@_#1_rules_prop}
                 { \@@_if_reversed:nT {#1}
                        { ~ (after~ reversal) }
                 }
                 { ~ (after~
                   \@@_if_reversed:nT {#1} {reversal~ and~}
                   applying~ rules)
                 }
               :    
              }
            \iow_term:x { \@spaces
              \clist_if_empty:cTF{g_@@_#1_labels_clist}
                 {not~ set~ yet}
                 { \clist_use:cnnn {g_@@_#1_labels_clist}
                                   { ,~ } { ,~ } { ,~ }   }
            }
          }
      }
      { \iow_term:n { ~The~hook~is~empty. } }
    \iow_term:n { }
  }
%    \end{macrocode}
%    
%  \end{macro}



%  \subsection{Specifying code for next invocation}
%
%
%
%
%
%  \begin{macro}{\hook_gput_next_code:nn}
%    
%    \begin{macrocode}
\cs_new_protected:Npn \hook_gput_next_code:nn #1
  {
    \exp_args:Nx \@@_gput_next_code:nn
      { \@@_parse_label_default:nn {#1} { top-level } }
  }
\cs_new_protected:Npn \@@_gput_next_code:nn #1 #2
  {
    \@@_declare:n {#1}
    \hook_if_exist:nTF {#1}
      { \@@_gput_next_do:nn {#1} {#2} }
      { \@@_try_declaring_generic_next_hook:nn {#1} {#2} }
  }
\cs_new_protected:Npn \@@_gput_next_do:nn #1 #2
  {
    \tl_gput_right:cn { g_@@_#1_next_code_tl }
      { #2 \tl_gclear:c { g_@@_#1_next_code_tl } }
  }
%    \end{macrocode}
%  \end{macro}
%
%
% \subsection{Using the hook}
%
% \begin{macro}{\hook_use:n}
% \begin{macro}[EXP]{\@@_use_initialized:n}
% \begin{macro}{\@@_preamble_hook:n}
%   \cs{hook_use:n} as defined here is used in the preamble, where
%   hooks aren't initialized by default.  \cs{@@_use_initialized:n} is
%   also defined, which is the non-\tn{protected} version for use within
%   the document.  Their definition is identical, except for the
%   \cs{@@_preamble_hook:n} (which wouldn't hurt in the expandable
%   version, but it would be an unnecessary extra expansion).
%
%   \cs{@@_use_initialized:n} holds the expandable definition while in
%   the preamble. \cs{@@_preamble_hook:n} initializes the hook in the
%   preamble, and is redefined to \cs{use_none:n} at |\begin{document}|.
%
%   Both versions do the same internally:  check if the hook exist as
%   given, and if so use it as quickly as possible.  If it doesn't
%   exist, the a call to \cs{@@_use:wn} checks for file hooks.
%
%   At |\begin{document}|, all hooks are initialized, and any change in
%   them causes an update, so \cs{hook_use:n} can be made expandable.
%   This one is better not protected so that it can expand into nothing
%   if containing no code. Also important in case of generic hooks that
%   we do not generate a \cs{relax} as a side effect of checking for a
%   csname. In contrast to the \TeX{} low-level
%   \verb=\csname ...\endcsname= construct \cs{tl_if_exist:c} is
%   careful to avoid this.
%    \begin{macrocode}
\cs_new_protected:Npn \hook_use:n #1
  {
    \tl_if_exist:cTF { g_@@_#1_code_tl }
      {
        \@@_preamble_hook:n {#1}
        \cs:w g_@@_#1_code_tl \cs_end:
      }
      { \@@_use:wn #1 / \s_@@_mark {#1} }
  }
\cs_new:Npn \@@_use_initialized:n #1
  {
    \tl_if_exist:cTF { g_@@_#1_code_tl }
      { \cs:w g_@@_#1_code_tl \cs_end: }
      { \@@_use:wn #1 / \s_@@_mark {#1} }
  }
\cs_new_protected:Npn \@@_preamble_hook:n #1
  { \@@_initialize_hook_code:n {#1} }
%    \end{macrocode}
% \end{macro}
% \end{macro}
% \end{macro}
%
% \begin{macro}[EXP]{\@@_use:wn}
% \begin{macro}{\@@_try_file_hook:n,\@@_if_exist_use:n}
%   \cs{@@_use:wn} does a quick check to test if the current hook is a
%   file hook: those need a special treatment.  If it is not, the hook
%   does not exist.  If it is, then \cs{@@_try_file_hook:n} is called,
%   and checks that the current hook is a file-specific hook using
%   \cs{@@_if_file_hook:wTF}.  If it's not, then it's a generic |file/|
%   hook and is used if it exist.
%
%   If it is a file-specific hook, it passes through the same
%   normalization as during declaration, and then it is used if defined.
%   \cs{@@_if_exist_use:n} checks if the hook exist, and calls
%   \cs{@@_preamble_hook:n} if so, then uses the hook.
%    \begin{macrocode}
\cs_new:Npn \@@_use:wn #1 / #2 \s_@@_mark #3
  {
    \str_if_eq:nnTF {#1} { file }
      { \@@_try_file_hook:n {#3} }
      { } % Hook doesn't exist
  }
\cs_new_protected:Npn \@@_try_file_hook:n #1
  {
    \@@_if_file_hook:wTF #1 / / \s_@@_mark
      {
        \exp_args:Ne \@@_if_exist_use:n
          { \exp_args:Ne \@@_file_hook_normalise:n {#1} }
      }
      { \@@_if_exist_use:n {#1} } % file/ generic hook (e.g. file/before)
  }
\cs_new_protected:Npn \@@_if_exist_use:n #1
  {
    \tl_if_exist:cT { g_@@_#1_code_tl }
      {
        \@@_preamble_hook:n {#1}
        \cs:w g_@@_#1_code_tl \cs_end:
      }
  }
%    \end{macrocode}
% \end{macro}
% \end{macro}
%
%  \begin{macro}{\hook_use_once:n}
%    For hooks that can and should be used only once we have a special
%    use command that remembers the hook name in
%    \cs{g_@@_execute_immediately_clist}. This has the effect that any
%    further code added to the hook is executed immediately rather
%    than stored in the hook.
%    \begin{macrocode}
\cs_new_protected:Npn \hook_use_once:n #1
  {
    \tl_if_exist:cT { g_@@_#1_code_tl }
      {
        \clist_gput_left:Nn \g_@@_execute_immediately_clist {#1}
        \hook_use:n {#1}
      }
  }
%    \end{macrocode}
%  \end{macro}

% \subsection{Querying a hook}
%
% Simpler data types, like token lists, have three possible states; they
% can exist and be empty, exist and be non-empty, and they may not
% exist, in which case emptiness doesn't apply (though
% \cs{tl_if_empty:N} returns false in this case).
%
% Hooks are a bit more complicated: they have four possible states.
% A hook may exist or not, and either way it may or may not be empty
% (even a hook that doesn't exist may be non-empty).
%
% A hook is said to be empty when no code was added to it, either to
% its permanent code pool, or to its ``next'' token list.  The hook
% doesn't need to be declared to have code added to its code pool
% (it may happen that a package $A$ defines a hook \hook{foo}, but
% it's loaded after package $B$, which adds some code to that hook.
% In this case it is important that the code added by package $B$ is
% remembered until package $A$ is loaded).
%
% A hook is said to exist when it was declared with \cs{hook_new:n} or
% some variant thereof.
%
% \begin{macro}[pTF]{\hook_if_empty:n}
%   Test if a hook is empty (that is, no code was added to that hook).
%   A hook being empty means that \emph{both} its
%   \cs[no-index]{g_@@_\meta{hook}_code_prop} and its
%   \cs[no-index]{g_@@_\meta{hook}_next_code_tl} are empty.
%    \begin{macrocode}
\prg_new_conditional:Npnn \hook_if_empty:n #1 { p , T , F , TF }
  {
    \@@_if_exist:nTF {#1}
      {
        \bool_lazy_and:nnTF
            { \prop_if_empty_p:c { g_@@_#1_code_prop } }
            { \tl_if_empty_p:c { g_@@_#1_next_code_tl } }
          { \prg_return_true: }
          { \prg_return_false: }
      }
      { \prg_return_true: }
  }
%    \end{macrocode}
% \end{macro}

% \begin{macro}[pTF]{\hook_if_exist:n}
%   A canonical way to test if a hook exists.  A hook exists if the
%   token list that stores the sorted code for that hook,
%   \cs[no-index]{g_@@_\meta{hook}_code_tl}, exists.  The property list
%   \cs[no-index]{g_@@_\meta{hook}_code_prop} cannot be used here
%   because often it is necessary to add code to a hook without knowing
%   if such hook was already declared, or even if it will ever be
%   (for example, in case the package that defines it isn't loaded).
%    \begin{macrocode}
\prg_new_conditional:Npnn \hook_if_exist:n #1 { p , T , F , TF }
  {
    \tl_if_exist:cTF { g_@@_#1_code_tl }
      { \prg_return_true: }
      { \prg_return_false: }
  }
%    \end{macrocode}
% \end{macro}

% \begin{macro}[pTF]{\@@_if_exist:n}
%   An internal check if the hook has already been declared with
%   \cs{@@_declare:n}.  This means that the hook was already used somehow
%   (a code chunk or rule was added to it), but it still wasn't declared
%   with \cs{hook_new:n}.
%    \begin{macrocode}
\prg_new_conditional:Npnn \@@_if_exist:n #1 { p , T , F , TF }
  {
    \prop_if_exist:cTF { g_@@_#1_code_prop }
      { \prg_return_true: }
      { \prg_return_false: }
  }
%    \end{macrocode}
% \end{macro}
%
% \begin{macro}[pTF]{\@@_if_reversed:n}
%   An internal conditional that checks if a hook is reversed.
%    \begin{macrocode}
\prg_new_conditional:Npnn \@@_if_reversed:n #1 { p , T , F , TF }
  {
    \if_int_compare:w \cs:w g_@@_#1_reversed_tl \cs_end: 1 < 0 \exp_stop_f:
      \prg_return_true:
    \else:
      \prg_return_false:
    \fi:
  }
%    \end{macrocode}
% \end{macro}


%  \begin{macro}{\g_@@_execute_immediately_clist}
%    List of hooks that from no on should not longer receive code.
%    \begin{macrocode}
\clist_new:N \g_@@_execute_immediately_clist
%    \end{macrocode}
%  \end{macro}
%
%  \subsection{Messages}
%
%    \begin{macrocode}
\msg_new:nnnn { hooks } { labels-incompatible }
  {
    Labels~`#1'~and~`#2'~are~incompatible
    \str_if_eq:nnF {#3} {??} { ~in~hook~`#3' } .~
    \int_compare:nNnT {#4} = { 1 }
      { The~code~for~both~labels~will~be~dropped. }
  }
  {
    LaTeX~found~two~incompatible~labels~in~the~same~hook.~
    This~indicates~an~incompatibility~between~packages.
  }
\msg_new:nnn { hooks } { empty-label }
  { Empty~code~label~\msg_line_context:.~Using~`#1'~instead. }
%    \end{macrocode}
%
%  \subsection{\LaTeXe{} package interface commands}
%


%  \begin{macro}{\NewHook,\NewReversedHook,\NewMirroredHookPair}
%    Declaring new hooks \ldots
%    \begin{macrocode}
\NewDocumentCommand \NewHook             { m }{ \hook_new:n {#1} }
\NewDocumentCommand \NewReversedHook     { m }{ \hook_new_reversed:n {#1} }
\NewDocumentCommand \NewMirroredHookPair { mm }{ \hook_new_pair:nn {#1}{#2} }
%    \end{macrocode}
%  \end{macro}

%  \begin{macro}{\AddToHook}
%    
%    \begin{macrocode}
\NewDocumentCommand \AddToHook { m o +m }
  {
    \clist_if_in:NnTF \g_@@_execute_immediately_clist {#1}
      {#3}
      { \hook_gput_code:nnn {#1} {#2} {#3} }
  }
%    \end{macrocode}
%  \end{macro}

%  \begin{macro}{\AddToHookNext}
%    
%    \begin{macrocode}
\NewDocumentCommand \AddToHookNext { m +m }
  { \hook_gput_next_code:nn {#1} {#2} }
%    \end{macrocode}
%  \end{macro}


%  \begin{macro}{\RemoveFromHook}
%    
%    \begin{macrocode}
\NewDocumentCommand \RemoveFromHook { m o }
  { \hook_gremove_code:nn {#1} {#2} }
%    \end{macrocode}
%  \end{macro}
%
% \begin{macro}{\DeclareDefaultHookLabel}
% \begin{macro}{\@@_curr_name_push:n,\@@_curr_name_pop:}
%   The token list \cs{g_@@_hook_curr_name_tl} stores the name of the
%   current package/file to be used as label for hooks.
%   Providing a consistent interface is tricky, because packages can
%   be loaded within packages, and some packages may not use
%   \cs{DeclareDefaultHookLabel} to change the default label (in which case
%   \cs{@currname} is used, if set).
%
%   To pull that off, we keep a stack that contains the default label
%   for each level of input.  The bottom of the stack contains the
%   default label for the top-level.  Since the string \verb|top-level|
%   is hardcoded, here this item of the stack is empty.  Also, since
%   we're in an input level, add \verb|lthooks| to the stack as well.
%   This stack should never go empty, so we loop through \LaTeXe's
%   file name stack, and add empty entries to \cs{g_@@_name_stack_seq}
%   for each item in that stack.  The last item is the \verb|top-level|,
%   which also gets an empty entry.
%
%   Also check for the case we're loading \texttt{lthooks} in the
%   \LaTeXe{} kernel.  In that case, \cs{@currname} isn't \verb|lthooks|
%   and just the top-level is added to the stack as an empty entry.
%    \begin{macrocode}
\str_if_eq:VnTF \@currname { lthooks }
  {
    \seq_gpush:Nn \g_@@_name_stack_seq { lthooks }
    \cs_set_protected:Npn \@@_tmp:w #1 #2 #3
      {
        \quark_if_recursion_tail_stop:n {#1}
        \seq_gput_right:Nn \g_@@_name_stack_seq { }
        \@@_tmp:w
      }
    \exp_after:wN \@@_tmp:w
      \@currnamestack
      \q_recursion_tail \q_recursion_tail
      \q_recursion_tail \q_recursion_stop
  }
  { \seq_gpush:Nn \g_@@_name_stack_seq { } }
%    \end{macrocode}
%
%   Two commands keep track of the stack: when a file is input,
%   \cs{@@_curr_name_push:n} pushes an (empty by default) label to the
%   stack:
%    \begin{macrocode}
\cs_new_protected:Npn \@@_curr_name_push:n #1
  {
    \seq_gpush:Nn \g_@@_name_stack_seq {#1}
    \tl_gset:Nn \g_@@_hook_curr_name_tl {#1}
  }
%
%    \end{macrocode}
%   and when an input is over, the topmost item of the stack is popped,
%   since the label will not be used again, and \cs{g_@@_hook_curr_name_tl}
%   is updated to the now topmost item of the stack:
%    \begin{macrocode}
\cs_new_protected:Npn \@@_curr_name_pop:
  {
    \seq_gpop:NN \g_@@_name_stack_seq \l_@@_return_tl
    \seq_get:NNTF \g_@@_name_stack_seq \l_@@_return_tl
      { \tl_gset_eq:NN \g_@@_hook_curr_name_tl \l_@@_return_tl }
      { \ERROR_should_not_happen }
  }
%    \end{macrocode}
%
%   The token list \cs{g_@@_hook_curr_name_tl} is but a mirror of the top
%   of the stack.
%
%   Now define a wrapper that replaces the top of the stack with the
%   argument, and updates \cs{g_@@_hook_curr_name_tl} accordingly.
%    \begin{macrocode}
\NewDocumentCommand \DeclareDefaultHookLabel { m }
  {
    \seq_gpop:NN \g_@@_name_stack_seq \l_@@_return_tl
    \@@_curr_name_push:n {#1}
  }
%    \begin{macrocode}
%
%   The push and pop macros are injected in \cs{@pushfilename} and
%   \cs{@popfilename} so that they correctly keep track of the label.s
%    \begin{macrocode}
% TODO! \pho{Properly integrate in the kernel}
\tl_gput_left:Nn \@pushfilename { \@@_curr_name_push:n { } }
\tl_gput_left:Nn \@popfilename { \@@_curr_name_pop: }
% TODO! \pho{Properly integrate in the kernel}
%    \end{macrocode}
% \end{macro}
% \end{macro}
%
%
%
%
%  \begin{macro}{\UseHook}
%    Avoid the overhead of \pkg{xparse} and its protection that we
%    don't want here (since the hook should vanish without trace if empty)!
%    \begin{macrocode}
\newcommand \UseHook { \hook_use:n }
%    \end{macrocode}
%  \end{macro}
%
%  \begin{macro}{\UseOneTimeHook}
%
%    \begin{macrocode}
\cs_new_protected:Npn \UseOneTimeHook { \hook_use_once:n }
%    \end{macrocode}
%  \end{macro}



%  \begin{macro}{\ShowHook}
%    
%    \begin{macrocode}
\cs_new_protected:Npn \ShowHook { \hook_log:n }
%    \end{macrocode}
%  \end{macro}

%  \begin{macro}{\DebugHookOn,\DebugHookOff}
%    
%    \begin{macrocode}
\cs_new_protected:Npn \DebugHookOn { \hook_debug_on: }
\cs_new_protected:Npn \DebugHookOff { \hook_debug_off: }
%    \end{macrocode}
%  \end{macro}



%  \begin{macro}{\DeclareHookRule}
%    
%    \begin{macrocode}
\NewDocumentCommand \DeclareHookRule { m m m m }
{ \hook_gset_rule:nnnn {#1}{#2}{#3}{#4} }
%    \end{macrocode}
%  \end{macro}

%  \begin{macro}{\DeclareDefaultHookRule}
%    This declaration is only supported before \verb=\begin{document}=.
%    \begin{macrocode}
\NewDocumentCommand \DeclareDefaultHookRule { m m m }
                    { \hook_gset_rule:nnnn {??}{#1}{#2}{#3} }
\@onlypreamble\DeclareDefaultHookRule                    
%    \end{macrocode}
%  \end{macro}
%
%  \begin{macro}{\ClearHookRule}
%    A special setup rule that removes an existing relation.
%    Basically {@@_rule_gclear:nnn} plus fixing the property list for debugging.
%    \fmi{Need an L3 interface, or maybe it should get dropped}
%    \begin{macrocode}
\NewDocumentCommand \ClearHookRule { m m m }
{ \hook_gset_rule:nnnn {#1}{#2}{unrelated}{#3} }
%    \end{macrocode}
%  \end{macro}
%
%
% \begin{macro}{\IfHookExistTF,\IfHookEmptyTF}
%    \begin{macrocode}
\NewExpandableDocumentCommand \IfHookExistTF { m }
  { \hook_if_exist:nTF {#1} }
\NewExpandableDocumentCommand \IfHookEmptyTF { m }
  { \hook_if_empty:nTF {#1} }
%    \end{macrocode}
% \end{macro}




%  \begin{macro}{\AtBeginDocument}
%    
%    \begin{macrocode}
\renewcommand\AtBeginDocument{\AddToHook{begindocument}}
%    \end{macrocode}
%  \end{macro}

%  \begin{macro}{\AtEndDocument}
%    
%    \begin{macrocode}
\renewcommand\AtEndDocument {\AddToHook{enddocument}}
%\renewcommand\AtEndDocument {\AddToHook{env/document/end}} % alternative impl
%    \end{macrocode}
%    
%  \end{macro}



%  \subsection{Set up existing \LaTeXe{} hooks}
%
%    As we are in a package calling \cs{NewHook} would label any
%    already set up hook code under the package name, but we want it
%    under the name \hook{top-level} so we pretend that \cs{@currname}
%    is empty.
%    \begin{macrocode}
\begingroup
  \def\@currname{}
%    \end{macrocode}
%    
%    \begin{macrocode}
  \NewHook{begindocument}
  \NewHook{enddocument}
%    \end{macrocode}
%    We need to initialize the mechanism at \verb=\begin{document}=
%    but obviously before everything else, so we sneak\footnote{This
%    needs to move to \cs{document} directly.}
%    \cs{@@_initialize_all:} into the \LaTeXe{} hook name.
%
%    We can't use \cs{tl_gput_left:Nn} because that complains about
%    \cs{@begindocumenthook} not starting with \cs{g_} so we do this
%    through the backdoor.
%    \begin{macrocode}
%  \tex_global:D\tl_put_left:Nn \@begindocumenthook
%      {\@@_initialize_all:}
%    \end{macrocode}
%    There aren't many other hooks at the moment:
%    \begin{macrocode}
  \NewHook{rmfamily}
  \NewHook{sffamily}
  \NewHook{ttfamily}
  \NewHook{defaultfamily}
%    \end{macrocode}
%    Not checked what this one does and whether it should be there (or
%    is a real ``hook''.
%    \begin{macrocode}
  \NewHook{documentclass}
%    \end{macrocode}
%    
%    \begin{macrocode}
\endgroup
%    \end{macrocode}
%
%
%
% \section{Generic hooks for environments}
%
%
%    \begin{macrocode}
\let\begin\relax  % avoid redeclaration message
%    \end{macrocode}
%    
%    \begin{macrocode}
\DeclareRobustCommand\begin[1]{%
  \UseHook{env/#1/before}%
  \@ifundefined{#1}%
    {\def\reserved@a{\@latex@error{Environment~#1~undefined}\@eha}}%
    {\def\reserved@a{\def\@currenvir{#1}%
        \edef\@currenvline{\on@line}%
        \@execute@begin@hook{#1}%
        \csname #1\endcsname}}%
  \@ignorefalse
  \begingroup\@endpefalse\reserved@a}
%    \end{macrocode}
%
%    Before the \cs{document} code is executed we have to first undo
%    the \cs{endgroup} as there should be none for this environment to
%    avoid that changes on top-level unnecessarily go to \TeX's
%    savestack, and we have to initialize all hooks in the hook system.
%    So we need to test for this environment name. But once it has be
%    found all this testing is no longer needed and so we redefine
%    \cs{@execute@begin@hook} to simply use the hook
%    \begin{macrocode}
\def\@execute@begin@hook #1{%
  \expandafter\ifx\csname #1\endcsname\document
    \endgroup
    \gdef\@execute@begin@hook##1{\UseHook{env/##1/begin}}%
    \@@_initialize_all:
    \@execute@begin@hook{#1}%
%    \end{macrocode}
%    If this is an environment before \verb=\begin{document}= we just
%    run the hook.
%    \begin{macrocode}
  \else
    \UseHook{env/#1/begin}%
  \fi
}    
%    \end{macrocode}
%    
%    \begin{macrocode}
\@namedef{end~}#1{%
  \UseHook{env/#1/end}%
  \csname end#1\endcsname\@checkend{#1}%
  \expandafter\endgroup\if@endpe\@doendpe\fi
  \UseHook{env/#1/after}%
  \if@ignore\@ignorefalse\ignorespaces\fi}%
%    \end{macrocode}
%    Version that fixes tlb3722 but the change should perhaps be made in
%    \pkg{tabularx} instead.
%    \begin{macrocode}
\@namedef{end~}#1{%
\romannumeral
\IfHookEmptyTF{env/#1/end}%
  {\expandafter\z@}%
  {\z@\UseHook{env/#1/end}}%
\csname end#1\endcsname\@checkend{#1}%
\expandafter\endgroup\if@endpe\@doendpe\fi
\UseHook{env/#1/after}%
\if@ignore\@ignorefalse\ignorespaces\fi}%
%    \end{macrocode}
%    
%    
%
%    We provide 4 high-level hook interfaces directly, the others only when
%    etoolbox is loaded
%    \begin{macrocode}
\newcommand\AtBeginEnvironment[1]    {\AddToHook{env/#1/begin}}
\newcommand\AtEndEnvironment[1]      {\AddToHook{env/#1/end}}
\newcommand\BeforeBeginEnvironment[1]{\AddToHook{env/#1/before}}
\newcommand\AfterEndEnvironment[1]   {\AddToHook{env/#1/after}}
%    \end{macrocode}
%    
%    
%    
%    
%    
%    
%    
%    
%    
%    
%
% \section{Generic hooks for file loads}
%
%
%
%
%
%    
%
% \section{Hooks in \cs{begin} document}
%
%    Can't have \texttt{@{}@} notation here as this is \LaTeXe{} code
%    \ldots{} and makes for puzzling errors if the double \texttt{@}
%    signs get substituted.
%    \begin{macrocode}
%<@@=>
\ExplSyntaxOff
%    \end{macrocode}
%
%    The \hook{begindocument} hook was already set up earlier, here is now
%    the additional one (which was originally from the \pkg{etoolbox}
%    package under the name \texttt{afterpreamble}.
%    \begin{macrocode}
\NewHook{begindocument/end}
%    \end{macrocode}
%
%

%  \begin{macro}{\document}
%    
%    \begin{macrocode}
\def\document{%
%    \end{macrocode}
%    We do cancel the grouping as part of the \cs{begin} handling
%    (this is now done inside \cs{begin} instead) so that the
%    \hook{env/\meta{env}/begin} hook is not hidden inside \cs{begingroup}
%    \texttt{...} \cs{endgroup}.
%    \begin{macrocode}
%  \endgroup
%    \end{macrocode}
%    
%    \begin{macrocode}
  \@kernel@after@env@document@begin
%    \end{macrocode}
%
% Added hook to load \textsf{l3backend} code:
%    \begin{macrocode}
  \@expl@sys@load@backend@@
  \ifx\@unusedoptionlist\@empty\else
    \@latex@warning@no@line{Unused global option(s):^^J%
            \@spaces[\@unusedoptionlist]}%
  \fi
  \@colht\textheight
  \@colroom\textheight \vsize\textheight
  \columnwidth\textwidth
  \@clubpenalty\clubpenalty
  \if@twocolumn
    \advance\columnwidth -\columnsep
    \divide\columnwidth\tw@ \hsize\columnwidth \@firstcolumntrue
  \fi
  \hsize\columnwidth \linewidth\hsize
  \begingroup\@floatplacement\@dblfloatplacement
    \makeatletter\let\@writefile\@gobbletwo
    \global \let \@multiplelabels \relax
    \@input{\jobname.aux}%
  \endgroup
  \if@filesw
    \immediate\openout\@mainaux\jobname.aux
    \immediate\write\@mainaux{\relax}%
  \fi
  \process@table
  \let\glb@currsize\@empty  % Force math initialization.
  \normalsize
  \everypar{}%
  \ifx\normalsfcodes\@empty
    \ifnum\sfcode`\.=\@m
      \let\normalsfcodes\frenchspacing
    \else
      \let\normalsfcodes\nonfrenchspacing
    \fi
  \fi
  \ifx\document@default@language\m@ne
    \chardef\document@default@language\language
  \fi
  \@noskipsecfalse
  \let \@refundefined \relax
%    \end{macrocode}
%    
%    \begin{macrocode}
%  \let\AtBeginDocument\@firstofone
%  \@begindocumenthook
  \UseOneTimeHook{begindocument}%
  \@kernel@after@begindocument
%    \end{macrocode}
%    
%    \begin{macrocode}
  \ifdim\topskip<1sp\global\topskip 1sp\relax\fi
  \global\@maxdepth\maxdepth
  \global\let\@begindocumenthook\@undefined
  \ifx\@listfiles\@undefined
    \global\let\@filelist\relax
    \global\let\@addtofilelist\@gobble
  \fi
  \gdef\do##1{\global\let ##1\@notprerr}%
  \@preamblecmds
  \global\let \@nodocument \relax
  \global\let\do\noexpand
%    \end{macrocode}
%    
%    \begin{macrocode}
  \UseOneTimeHook{begindocument/end}%
  \ignorespaces}
%    \end{macrocode}
%    
%    \begin{macrocode}
\let\@kernel@after@begindocument\@empty
%    \end{macrocode}
%  \end{macro}
%
%
%  \begin{macro}{\@kernel@after@env@document@begin,\@kernel@hook@begindocument}
%    
%    \begin{macrocode}
\edef \@kernel@after@env@document@begin{%
  \let\expandafter\noexpand\csname
       g__hook_env/document/begin_code_tl\endcsname
  \noexpand\@empty}
%    \end{macrocode}
%    \begin{macrocode}
\let\@kernel@hook@begindocument\@empty
%    \end{macrocode}
%    
%  \end{macro}
%
%
% \section{Hooks in \cs{enddocument}}
%
%
%    
%    The \hook{enddocument} hook was already set up earlier, here are now
%    the additional ones:
%    \begin{macrocode}
\NewHook{enddocument/afterlastpage}
\NewHook{enddocument/afteraux}
\NewHook{enddocument/info}
\NewHook{enddocument/end}
%    \end{macrocode}



%  \begin{macro}{\enddocument}
%    
%    
%    \begin{macrocode}
\def\enddocument{%
   \UseHook{enddocument}%
   \@kernel@after@enddocument
   \@checkend{document}%
   \clearpage
   \UseHook{enddocument/afterlastpage}%
   \@kernel@after@enddocument@afterlastpage
   \begingroup
     \if@filesw
       \immediate\closeout\@mainaux
       \let\@setckpt\@gobbletwo
       \let\@newl@bel\@testdef
       \@tempswafalse
       \makeatletter \@@input\jobname.aux
     \fi
     \UseHook{enddocument/afteraux}%
%    \end{macrocode}
%    Next hook is expect to contain only code for writing info
%    messages on the terminal.
%    \begin{macrocode}
     \UseHook{enddocument/info}%
   \endgroup
   \UseHook{enddocument/end}%
   \deadcycles\z@\@@end}
%    \end{macrocode}
%    The two kernel hooks above are used by the shipout code.   
%    \begin{macrocode}
\let\@kernel@after@enddocument\@empty
\let\@kernel@after@enddocument@afterlastpage\@empty
%    \end{macrocode}
%  \end{macro}
%
%
%
%  \begin{macro}{\@enddocument@kernel@warnings}
%    
%    \begin{macrocode}
\def\@enddocument@kernel@warnings{%
   \ifdim \font@submax >\fontsubfuzz\relax
     \@font@warning{Size substitutions with differences\MessageBreak
                up to \font@submax\space have occurred.\@gobbletwo}%
   \fi
   \@defaultsubs
   \@refundefined
   \if@filesw
     \ifx \@multiplelabels \relax
       \if@tempswa
         \@latex@warning@no@line{Label(s) may have changed.
             Rerun to get cross-references right}%
       \fi
     \else
       \@multiplelabels
     \fi
   \fi
}
%    \end{macrocode}
%    
%    \begin{macrocode}
\AddToHook{enddocument/info}[kernel/filelist]{\@dofilelist}
\AddToHook{enddocument/info}[kernel/warnings]{\@enddocument@kernel@warnings}
\DeclareHookRule{enddocument/info}{kernel/filelist}{before}{kernel/warnings}
%    \end{macrocode}
%  \end{macro}
%
%
% \subsection{Adjusting at \pkg{atveryend} interfaces}
%
%    With the new hook management all of \pkg{atveryend} is taken care
%    of.
%
%    We therefore prevent the package from loading:
%    \begin{macrocode}
\expandafter\let\csname ver@atveryend.sty\endcsname\fmtversion
%    \end{macrocode}
%
%
%    Here are new definitions for its interfaces now pointing to the
%    hooks in \cs{enddocument}
%    \begin{macrocode}
\newcommand\AfterLastShipout  {\AddToHook{enddocument/afterlastpage}}
\newcommand\AtVeryEndDocument {\AddToHook{enddocument/afteraux}}
%    \end{macrocode}
%    Next one is a bit of a fake, but the result should normally be as
%    expected. If not one needs to add a rule to sort the code chunks
%    in \hook{enddocument/info}.
%    \begin{macrocode}
\newcommand\AtEndAfterFileList{\AddToHook{enddocument/info}}
%    \end{macrocode}
%    
%    \begin{macrocode}
\newcommand\AtVeryVeryEnd     {\AddToHook{enddocument/end}}
%    \end{macrocode}

%  \begin{macro}{\BeforeClearDocument}
%    This one is the only one we don't implement or rather don't have
%    a dedicated hook in the code.
%    \begin{macrocode}
\ExplSyntaxOn
\newcommand\BeforeClearDocument[1]
  { \AtEndDocument{#1}
    \@DEPRECATED{BeforeClearDocument \tl_to_str:n{#1}}
  }
\cs_new:Npn\@DEPRECATED #1
   {\iow_term:x{======~DEPRECATED~USAGE~#1~==========}}
%    \end{macrocode}
%    
%    \begin{macrocode}
\ExplSyntaxOff
%    \end{macrocode}
%  \end{macro}
%
%    \begin{macrocode}
%</2ekernel>
%    \end{macrocode}

%
% \section{A package version of the code for testing}
%

%    \begin{macrocode}
%<*package>
%    \end{macrocode}



%    \begin{macrocode}
\RequirePackage{xparse}
\ProvidesExplPackage{lthooks}{\lthooksdate}{\lthooksversion}
                    {Hook management interface for LaTeX2e}
%    \end{macrocode}
%
%
% \subsection{Core hook management code (kernel part)}
%
%    This should run in older formats so we can't use
%    \cs{IfFormatAtLeastTF} right now.
%    \begin{macrocode}
\@ifl@t@r\fmtversion{2020/10/01}
                    {}
                    {% \iffalse meta-comment
%
%% File: lthooks.dtx (C) Copyright 2020 Frank Mittelbach, LaTeX Team
%
% It may be distributed and/or modified under the conditions of the
% LaTeX Project Public License (LPPL), either version 1.3c of this
% license or (at your option) any later version.  The latest version
% of this license is in the file
%
%    https://www.latex-project.org/lppl.txt
%
%
% The development version of the bundle can be found below
%
%    https://github.com/FrankMittelbach/...
%
% for those people who are interested or want to report an issue.
%
%    \begin{macrocode}
\providecommand\lthooksversion{v0.9a}
\providecommand\lthooksdate{2020/07/15}
%    \end{macrocode}
%
%<*driver>
\RequirePackage[debug]{lthooks}

\documentclass{l3doc}

% bug fix fo l3doc.cls
\ExplSyntaxOn
\cs_set_protected:Npn \__codedoc_macro_typeset_one:nN #1#2
  {
    \vbox_set:Nn \l__codedoc_macro_box
      {
        \vbox_unpack_drop:N \l__codedoc_macro_box
        \hbox { \llap { \__codedoc_print_macroname:nN {#1} #2
            \MacroFont       % <----- without it the \ is in lmr10 if a link is made
            \      
        } }
      }
    \int_incr:N \l__codedoc_macro_int
  }
\ExplSyntaxOff

\EnableCrossrefs
\CodelineIndex
\begin{document}
  \DocInput{lthooks.dtx}
\end{document}
%</driver>
%
% \fi
%
%
% \long\def\fmi#1{\begin{quote}\itshape FMi: #1\end{quote}}
% \long\def\pho#1{\begin{quote}\itshape PhO: #1\end{quote}}
%
% \newcommand\hook[1]{\texttt{#1}}
%    
%
% \title{The \texttt{lthooks} package\thanks{This package has version
%    \lthooksversion\ dated \lthooksdate, \copyright\ \LaTeX\
%    Project.}}
%
% \author{Frank Mittelbach\thanks{Code improvements for speed and other goodies by Phelype Oleinik}}
%
% \maketitle
%
%
% \tableofcontents
%
% \section{Introduction}
%
%    Hooks are points in the code of commands or environments where it
%    is possible to add processing code into existing commands. This
%    can be done by different packages that do not know about each
%    other and to allow for hopefully safe processing it is necessary
%    to sort different chunks of code added by different packages into
%    a suitable processing order.
%
%    This is done by the packages adding chunks of code (via
%    \cs{AddToHook}) and labeling their code with some label by
%    default using the package name as a label.
%
%    At \verb=\begin{document}= all code for a hook is then sorted
%    according to some rules (given by \cs{DeclareHookRule}) for fast
%    execution without processing overhead. If the hook code is
%    modified afterwards (or the rules are changed),
%    a new version for fast processing is generated.
%
%    Some hooks are used already in the preamble of the document. If
%    that happens then the hook is prepared for execution (and sorted)
%    already at that point.
%
%
% \section{Package writer interface}
%
%    The hook management system is offered as a set of CamelCase
%    commands for traditional \LaTeXe{} packages (and for use in the
%    document preamble if needed) as well as \texttt{expl3} commands
%    for modern packages, that use the L3 programming layer of
%    \LaTeX{}. Behind the scenes, a single set of data structures is
%    accessed so that packages from both worlds can coexist and access
%    hooks in other packages.
%
%
%
% \subsection{\LaTeXe\ interfaces}
%
% \subsubsection{Declaring hooks and using them in code}
%
%    With two exceptions, hooks have to be declared before they can be
%    used. The exceptions are hooks in environments (i.e., executed at
%    \cs{begin} and \cs{end}) and hooks run when loading files,
%    e.g. before and after a package is loaded, etc. Their hook names
%    depend on the environment or the file name and so declaring them
%    beforehand is difficult.
%
%
% \begin{function}{\NewHook}
%   \begin{syntax}
%     \cs{NewHook} \Arg{hook}
%   \end{syntax}
%   Creates a new \meta{hook}.
%    If this is a hook provided as part of a package it is suggested
%    that the \meta{hook} name is always structured as follows:
%    \meta{package-name}\texttt{/}\meta{hook-name}. If necessary you
%    can further subdivide the name by adding more \texttt{/} parts.
%    If a hook name is already taken, an error is raised and the hook
%    is not created.
%
%    The \meta{hook} can be specified using the dot-syntax to denote
%    the current package name. See section~\ref{sec:default-label}.
% \end{function}
%
% \begin{function}{\NewReversedHook}
%   \begin{syntax}
%     \cs{NewReversedHook} \Arg{hook}
%   \end{syntax}
%     Like \cs{NewHook} declares a new \meta{hook}.
%     the difference is that the code chunks for this hook are in
%     reverse order by default (those added last are executed first).
%     Any rules for the hook are applied after the default ordering.
%     See sections~\ref{sec:order} and \ref{sec:reversed-order}
%    for further details.
%
%    The \meta{hook} can be specified using the dot-syntax to denote
%    the current package name. See section~\ref{sec:default-label}.
% \end{function}
%
%
% \begin{function}{\NewMirroredHookPair}
%   \begin{syntax}
%     \cs{NewMirroredHookPair} \Arg{hook-1} \Arg{hook-2}
%   \end{syntax}
%     A shorthand for
%    \cs{NewHook}\Arg{hook-1}\cs{NewReversedHook}\Arg{hook-2}.
%
%    The \meta{hooks} can be specified using the dot-syntax to denote
%    the current package name. See section~\ref{sec:default-label}.
% \end{function}
%



% \begin{function}{\UseHook}
%   \begin{syntax}
%     \cs{UseHook} \Arg{hook}
%   \end{syntax}
%    Execute the hook code inside a command or environment.\footnote{For
%     legacy hooks such as \hook{begindocument} it is also
%    possible to call \cs{@...hook}, e.g., \cs{@begindocumenthook},
%    but this syntax is discouraged.}
%
%    Before \verb=\begin{document}= the fast execution code for a hook
%    is not set up, so in order to use a hook there it is explicitly
%    initialized first. As that involves assignments using a hook at
%    those times is not 100\% the same as using it after
%    \verb=\begin{document}=.
%
%    The \meta{hook} \emph{cannot} be specified using the dot-syntax.
%    A leading |.| is treated literally.
% \end{function}
%
% \begin{function}{\UseOneTimeHook}
%   \begin{syntax}
%     \cs{UseOneTimeHook} \Arg{hook}
%   \end{syntax}
%    Some hooks are only used (and can be only used) in one place, for
%    example, those in \verb=\begin{document}= or
%    \verb=\end{document}=. Once we have passed that point adding to
%    the hook through a defined \cs{\meta{addto-cmd}} command (e.g.,
%    \cs{AddToHook} or \cs{AtBeginDocument}, etc.\@) would have no
%    effect (as would the use of such a command inside the hook code
%    itself). It is therefore customary to redefine
%    \cs{\meta{addto-cmd}} to simply  process its argument, i.e.,
%    essentially make it behave like \cs{@firstofone}.
%
%    \cs{UseOneTimeHook} does that: it records that the hook has been
%    consumed and any further attempt to add to it will result in
%    executing the code to be added immediately.
%
%    \fmi{Maybe add an error version as well?}
%
%    The \meta{hook} \emph{cannot} be specified using the dot-syntax.
%    A leading |.| is treated literally.
% \end{function}
%
%
% \subsubsection{Updating code for hooks}
%
% \begin{function}{\AddToHook}
%   \begin{syntax}
%     \cs{AddToHook} \Arg{hook}\oarg{label}\Arg{code}
%   \end{syntax}
%    Adds \meta{code} to the \meta{hook} labeled by \meta{label}. If
%    the optional argument \meta{label} is not provided, if \cs{AddToHook}
%    is used in a package/class, then the current
%    package/class name is used, otherwise \hook{top-level} is
%    used~(see section~\ref{sec:default-label}).
%
%    If there already exists code under the \meta{label} then the new
%    \meta{code} is appended to the existing one (even if this is a reversed hook).
%    If you want to replace existing code under the
%    \meta{label}, first apply \cs{RemoveFromHook}.
%
%    The hook doesn't have to exist for code to be added to
%    it. However, if it is not declared later then obviously the
%    added \meta{code} will never be executed.  This
%    allows for hooks to work regardless of package loading order and
%    enables packages to add to hook of other packages without
%    worrying whether they are actually used in the current document.
%    See section~\ref{sec:querying}.
%
%    The \meta{hook} and \meta{label} can be specified using the
%    dot-syntax to denote the current package name.
%    See section~\ref{sec:default-label}.
% \end{function}
%
% \begin{function}{\RemoveFromHook}
%   \begin{syntax}
%     \cs{RemoveFromHook} \Arg{hook}\oarg{label}
%   \end{syntax}
%    Removes any code labeled by \meta{label} from the \meta{hook}.
%    If the optional
%    argument \meta{label} is not provided, if \cs{AddToHook}
%    is used in a package/class, then the current
%    package/class name is used, otherwise \hook{top-level} is used.
%
%    If the optional argument is \texttt{*}, then all code chunks are
%    removed. This is rather dangerous as it drops code from other
%    packages one may not know about!
%
%    The \meta{hook} and \meta{label} can be specified using the
%    dot-syntax to denote the current package name.
%    See section~\ref{sec:default-label}.
% \end{function}
%
% \medskip
%
% A useful application for this declaration inside the document body
% is when one wants to temporarily add code to hooks and later remove
% it again, e.g.,
%\begin{verbatim}
%   \AddToHook{env/quote/before}{\small}
%   \begin{quote}
%     A quote set in a smaller typeface
%   \end{quote}
%   ...
%   \RemoveFromHook{env/quote/before}
%   ... now back to normal for further quotes
%\end{verbatim}
% Note that you can't cancel the setting with
%\begin{verbatim}
%   \AddToHook{env/quote/before}{}
%\end{verbatim}
% because that only \enquote{adds} a further empty chunk of code to
% the hook. Adding \cs{normalsize} would work but that means the hook
% then contained \cs{small}\cs{normalsize} which means to font size
% changes for no good reason.
%
% The above is only needed if one wants to typeset several quotes in a
% smaller typeface. If the hook is only needed once then
% \cs{AddToHookNext} is simpler, because it resets itself after one use.
%
%
% \begin{function}{\AddToHookNext}
%   \begin{syntax}
%     \cs{AddToHookNext} \Arg{hook}\Arg{code}
%   \end{syntax}
%    Adds \meta{code} to the next invocation of the \meta{hook}.
%    The code is executed after the normal hook code has finished and
%    it is executed only once, i.e. it is deleted after it was used.
%
%    Using the declaration is a global operation, i.e., the code is
%    not lost, even if the declaration is used inside a group and the
%    next invocation happens after the group. If the declaration is
%    used several times before the hook is executed then all code is
%    executed in the order in which it was declared.\footnotemark
%
%    The hook doesn't have to exist for code to be added to it.  This
%    allows for hooks to work regardless of package loading order.
%    See section~\ref{sec:querying}.
%
%    The \meta{hook} can be specified using the dot-syntax to denote
%    the current package name.  See section~\ref{sec:default-label}.
% \end{function}\footnotetext{There is
%    no mechanism to reorder such code chunks (or delete them).}
%
% \subsubsection{Hook names and default labels}
% \label{sec:default-label}
%
% It is best practice to use \cs{AddToHook} in packages or classes
% \emph{without specifying a \meta{label}} because then the package
% or class name is automatically used, which is helpful if rules are
% needed, and avoids mistyping the \meta{label}.
%
% Using an explicit \meta{label} is only necessary in very specific
% situations, e.g., if you want to add several chunks of code into a
% single hook and have them placed in different parts of the hook
% (by providing some rules).
%
% The other case is when you develop a larger package with several
% sub-packages. In that case you may want to use the same
% \meta{label} throughout the sub-packages in order to avoid
% that the labels change if you internally reorganize your code.
%
% It is not enforced, but highly recommended that the hooks defined by
% a package, and the \meta{labels} used to add code to other hooks
% contain the package name to easily identify the source of the code
% chunk and to prevent clashes.  This should be the standard practice,
% so this hook management code provides a shortcut to refer to the
% current package in the name of a \meta{hook} and in a \meta{label}.
% If \meta{hook} name or \meta{label} consist just of a single dot
% (|.|), or starts with a dot followed by a slash (|./|) then the dot
% denotes the \meta{default label} (usually the current package or class
% name---see~\cs{DeclareDefaultHookLabel}).
% A \enquote{|.|} or \enquote{|./|} anywhere else in a \meta{hook} or in
% \meta{label} is treated literally and is not replaced.

% For example,
% inside the package \texttt{mypackage.sty}, the default label is
% \texttt{mypackage}, so the instructions:
% \begin{verbatim}
%   \NewHook   {./hook}
%   \AddToHook {./hook}[.]{code}     % Same as \AddToHook{./hook}{code}
%   \AddToHook {./hook}[./sub]{code}
%   \DeclareHookRule{begindocument}{.}{<}{babel}
%   \AddToHook {file/after/foo.tex}{code}
% \end{verbatim}
%    are equivalent to:
% \begin{verbatim}
%   \NewHook   {mypackage/hook}[mypackage]{code}
%   \AddToHook {mypackage/hook}[mypackage]{code}
%   \AddToHook {mypackage/hook}[mypackage/sub]{code}
%   \DeclareHookRule{begindocument}{mypackage}{<}{babel}
%   \AddToHook {file/after/foo.tex}{code}                  % unchanged
% \end{verbatim}
%
% The \meta{default label} is automatically set to the name of the
% current package or class (using \cs{@currname}).  If \cs{@currname}
% is not set (because the hook command is used outside of a package, or
% the current file wasn't loaded with \cs{usepackage} or
% \cs{documentclass}), then the \texttt{top-level} is used as the
% \meta{default label}.
%
% This syntax is available in all \meta{label} arguments and most
% \meta{hook}, both in the \LaTeXe{} interface, and the \LaTeX3
% interface described in section~\ref{sec:l3hook-interface}.
%
% Note, however, that the replacement of |.| by the \meta{default label}
% takes place when the hook command is executed, so actions that are
% somehow executed after the package ends will have the wrong
% \meta{default label} if the dot-syntax is used.  For that reason,
% this syntax is not available in \cs{UseHook} (and \cs{hook_use:n})
% because the hook is most of the time used outside of the package file
% in which it was defined. This syntax is also not available in the hook
% conditionals \cs{IfHookEmptyTF} (and \cs{hook_if_empty:nTF}) and
% \cs{IfHookExistTF} (and \cs{hook_if_exist:nTF}) because these
% conditionals are used in some performance-critical parts of the hook
% management code, and because they are usually used to refer to other
% package's hooks, so the dot-syntax doesn't make much sense.
%
% In some cases, for example in large packages, one may want to separate
% it in logical parts, but still use the main package name as
% \meta{label}, then the \meta{default label} can be set using
% \cs{DeclareDefaultHookLabel}:
%
% \begin{function}{\DeclareDefaultHookLabel}
%   \begin{syntax}
%     \cs{DeclareDefaultHookLabel} \Arg{default label}
%   \end{syntax}
%   Sets the \meta{default label} to be used in \meta{label} arguments.
%   If \cs{DeclareDefaultHookLabel} is not used in the
%   current package, \cs{@currname} is used instead.  If \cs{@currname}
%   is not set, the code is assumed to be in the main document, in which
%   case \texttt{top-level} is used.
%
%   The effect of \cs{DeclareDefaultHookLabel} holds for the current
%   file, and is reset to the previous value when the file is closed.
% \end{function}
%
%
% \subsubsection{Defining relations between hook code}
%
% The default assumption is that code added to hooks by different
% packages is independent and the order in which it is executed is
% irrelevant. While this is true in many case it is  obviously false
% in many others.
%
% Before the hook management system was introduced
% packages had to take elaborate precaution to determine of some other
% package got loaded as well (before or after) and find some ways to
% alter its behavior accordingly. In addition is was often the user's
% responsibility to load packages in the right order so that code
% added to hooks got added in the right order and some cases even
% altering the loading order wouldn't resolve the conflicts.
%
% With the new hook management system it is now possible to define
% rules (i.e., relationships) between code chunks added by different
% packages and explicitly describe in which order they should be
% processed.
%
% \begin{function}{\DeclareHookRule}
%   \begin{syntax}
%     \cs{DeclareHookRule} \Arg{hook}\Arg{label1}\Arg{relation}\Arg{label2}
%   \end{syntax}
%    Defines a relation between \meta{label1} and \meta{label2} for a
%    given \meta{hook}. If \meta{hook} is \texttt{??} this defines a
%    relation for all hooks that use the two labels, i.e., that have
%    chunks of code labeled with \meta{label1} and \meta{label2}.
%    Rules specific to a given hook take precedence over default
%    rules that use \texttt{??} as the \meta{hook}.
%
%    Currently, the supported relations are the following:
%    \begin{itemize}
%
%    \item[\texttt{before} or \texttt{\string<}]
%
%      Code for \meta{label1} comes before code for \meta{label2}.
%
%    \item[\texttt{after} or \texttt{\string>}]
%      Code for \meta{label1} comes after code for \meta{label2}.
%
%    \item[\texttt{incompatible-warning}]
%
%      Only code for either \meta{label1} or \meta{label2} can appear
%      for that hook (a way to say that two packages---or parts of
%      them---are incompatible). A warning is raised if both labels
%      appear in the same hook.
%
%    \item[\texttt{incompatible-error}]
%
%      Like \texttt{incompatible-error} but instead of a warning a
%      \LaTeX{} error is raised, and the code for both labels are
%      dropped from that hook until the conflict is resolved.
%
%    \item[\texttt{removes}]
%
%      Code for \meta{label1} overwrites code for \meta{label2}. More
%      precisely, code for \meta{label2} is dropped for that
%      hook. This can be used, for example if one package is a
%      superset in functionality of another one and therefore wants to
%      undo code in some hook and replace it with its own version.
%
%    \item[\texttt{unrelated}]
%
%       The order of code for \meta{label1} and \meta{label2} is
%      irrelevant. This rule is there to undo an incorrect rule
%      specified earlier.
%
%    \end{itemize}
%
%    The \meta{hook} and \meta{label} can be specified using the
%    dot-syntax to denote the current package name.
%    See section~\ref{sec:default-label}.
% \end{function}
%
%
% \begin{function}{\ClearHookRule}
%   \begin{syntax}
%     \cs{ClearHookRule}\Arg{hook}\Arg{label1}\Arg{label2}
%   \end{syntax}
%    Syntactic sugar for saying that \meta{label1} and \meta{label2}
%    are unrelated for the given \meta{hook}.
% \end{function}
%
%
%
% \begin{function}{\DeclareDefaultHookRule}
%   \begin{syntax}
%     \cs{DeclareDefaultHookRule}\Arg{label1}\Arg{relation}\Arg{label2}
%   \end{syntax}
%   This sets up a relation between \meta{label1} and \meta{label2}
%    for all hooks unless overwritten by a specific rule for a hook.
%    Useful for cases where one package has a specific relation to
%    some other package, e.g., is \texttt{incompatible} or always
%    needs a special ordering \texttt{before} or \texttt{after}.
%    (Technically it is just a shorthand for using \cs{DeclareHookRule}
%    with \texttt{??} as the hook name.)
%
%    Declaring default rules is only supported in the document
%    preamble.\footnotemark{}
%
%    The \meta{label} can be specified using the dot-syntax to denote
%    the current package name. See section~\ref{sec:default-label}.
% \end{function}\footnotetext{Trying to do so, e.g., via
%    \cs{DeclareHookRule} with \texttt{??}  has bad side-effects and
%    is not supported (though not explicitly caught for performance
%    reasons).}
%
%
%
% \subsubsection{Querying hooks}
% \label{sec:querying}
%
% Simpler data types, like token lists, have three possible states; they
% can:
% \begin{itemize}
%   \item exist and be empty;
%   \item exist and be non-empty; and
%   \item not exist (in which case emptiness doesn't apply);
% \end{itemize}
% Hooks are a bit more complicated: they have four possible states.
% A hook may exist or not, and either way it may or may not be empty.
% This means that even a hook that doesn't exist may be non-empty.
%
% This seemingly strange state may happen when, for example, package~$A$
% defines hook \hook{A/foo}, and package $B$ adds some code to that
% hook.  However, a document may load package $B$ before package $A$, or
% may not load package $A$ at all.  In both cases some code is added to
% hook \hook{A/foo} without that hook being defined yet, thus that
% hook is said to be non-empty, whereas it doesn't exist.  Therefore,
% querying the existence of a hook doesn't imply its emptiness, neither
% does the other way around.
%
% A hook is said to be empty when no code was added to it, either to
% its permanent code pool, or to its ``next'' token list.  The hook
% doesn't need to be declared to have code added to its code pool.
% A hook is said to exist when it was declared with \cs{NewHook} or
% some variant thereof.
%
% \begin{function}[EXP]{\IfHookEmptyTF}
%   \begin{syntax}
%     \cs{IfHookEmptyTF} \Arg{hook} \Arg{true code} \Arg{false code}
%   \end{syntax}
%   Tests if the \meta{hook} is empty (\emph{i.e.}, no code was added to
%   it using either \cs{AddToHook} or \cs{AddToHookNext}), and
%   branches to either \meta{true code} or \meta{false code} depending
%   on the result.
%
%    The \meta{hook} \emph{cannot} be specified using the dot-syntax.
%    A leading |.| is treated literally.
% \end{function}
%
% \begin{function}[EXP]{\IfHookExistTF}
%   \begin{syntax}
%     \cs{IfHookExistTF} \Arg{hook} \Arg{true code} \Arg{false code}
%   \end{syntax}
%   Tests if the \meta{hook} exists (if it was created with either
%   \cs{NewHook}, \cs{NewReversedHook}, or \cs{NewMirroredHookPair}), and
%   branches to either \meta{true code} or \meta{false code} depending
%   on the result.
%
%   The existence of a hook usually doesn't mean much from the viewpoint
%   of code that tries to add/remove code from that hook, since package
%   loading order may vary, thus the creation of hooks is asynchronous
%   to adding and removing code from it, so this test should be used
%   sparingly.
%
%    The \meta{hook} \emph{cannot} be specified using the dot-syntax.
%    A leading |.| is treated literally.
% \end{function}
%
% \fmi{Would be helpful if we provide some use cases}
%
% \subsubsection{Displaying hook code}
%
%    If one has to adjust the code execution in a hook using a hook
%    rule it is helpful to get some information about the code
%    associated with a hook, its current order and the existing rules.
%
% \begin{function}{\ShowHook}
%   \begin{syntax}
%     \cs{ShowHook} \Arg{hook}
%   \end{syntax}
%   Displays information about the \meta{hook} such as
%   \begin{itemize}
%   \item
%      the code chunks (and their labels) added to it,
%   \item
%      any rules set up to order them,
%      \fmi{currently this is missing the default rules that apply,
%      guess that needs fixing}
%   \item
%      the computed order (if already defined),
%   \item
%      any code executed on the next invocation only.
%   \end{itemize}
%
%    The \meta{hook} can be specified using the dot-syntax to denote
%    the current package name. See section~\ref{sec:default-label}.
% \end{function}
%
%
% \subsubsection{Debugging hook code}
%
% \begin{function}{\DebugHookOn,\DebugHookOff}
%   \begin{syntax}
%     \cs{DebugHookOn}
%   \end{syntax}
%    Turn the debugging of hook code on or off. This displays changes
%    made to the hook data structures. The output is rather coarse and
%      not really intended for normal use.
% \end{function}
%
%
% \subsection{L3 programming layer (\texttt{expl3}) interfaces}
% \label{sec:l3hook-interface}
%
%
% This is a quick summary of the \LaTeX3 programming interfaces for
% use with packages written in \texttt{expl3}. In contrast to the
% \LaTeXe{} interfaces they always use mandatory arguments only, e.g.,
% you always have to specify the \meta{label} for a code chunk.  We
% therefore suggest to use the declarations discussed in the previous
% section even in \texttt{expl3} packages, but the choice is yours.
%
%
% \begin{function}
%   {\hook_new:n,\hook_new_reversed:n,\hook_new_pair:nn}
%   \begin{syntax}
%     \cs{hook_new:n}\Arg{hook}
%     \cs{hook_new_pair:nn}\Arg{hook-1}\Arg{hook-2}
%   \end{syntax}
%   Creates a new \meta{hook} with normal or reverse ordering of code
%    chunks. \cs{hook_new_pair:nn} creates a pair of such hooks with
%    \Arg{hook-2} being a reversed hook.
%    If a hook name is already taken, an error is raised and the hook
%    is not created.
%
%    The \meta{hook} can be specified using the dot-syntax to denote
%    the current package name. See section~\ref{sec:default-label}.
% \end{function}
%
%
%
% \begin{function}{\hook_use:n}
%   \begin{syntax}
%     \cs{hook_use:n} \Arg{hook}
%   \end{syntax}
%    Executes the \Arg{hook} code followed (if set up) by the code for next
%    invocation only, then empties that next invocation code.
%
%    The \meta{hook} \emph{cannot} be specified using the dot-syntax.
%    A leading |.| is treated literally.
% \end{function}
%
% \begin{function}{\hook_use_once:n}
%   \begin{syntax}
%     \cs{hook_use_once:n} \Arg{hook}
%   \end{syntax}
%     Changes the \Arg{hook} status so that from now on any addition to
%     the hook code is executed immediately. Then execute any
%     \Arg{hook} code already set up.
%    \fmi{better L3 name?}
%
%    The \meta{hook} \emph{cannot} be specified using the dot-syntax.
%    A leading |.| is treated literally.
% \end{function}
%
% \begin{function}{\hook_gput_code:nnn}
%   \begin{syntax}
%     \cs{hook_gput_code:nnn} \Arg{hook} \Arg{label} \Arg{code}
%   \end{syntax}
%    Adds a chunk of \meta{code} to the \meta{hook} labeled
%    \meta{label}. If the label already exists the \meta{code} is
%    appended to the already existing code.
%
%    If code is added to an external \meta{hook} (of the kernel or
%    another package) then the convention is to use the package name
%    as the \meta{label} not some internal module name or some other
%    arbitrary string.
%
%    The \meta{hook} and \meta{label} can be specified using the
%    dot-syntax to denote the current package name.
%    See section~\ref{sec:default-label}.
% \end{function}
%
% \begin{function}
%   {\hook_gput_next_code:nn}
%   \begin{syntax}
%     \cs{hook_gput_next_code:nn} \Arg{hook} \Arg{code}
%   \end{syntax}
%    Adds a chunk of \meta{code} for use only in the next invocation of the
%    \meta{hook}. Once used it is gone.
%
%    This is simpler than \cs{hook_gput_code:nnn}, the code is simply
%    appended to the hook in the order of declaration at the very end,
%    i.e., after all standard code for the hook got executed.
%
%    Thus if one needs to undo what the standard does one has to do
%    that as part of \meta{code}.
%
%    The \meta{hook} can be specified using the dot-syntax to denote
%    the current package name. See section~\ref{sec:default-label}.
% \end{function}
%


%  \begin{function}{\hook_gremove_code:nn}
%   \begin{syntax}
%     \cs{hook_gremove_code:nn} \Arg{hook} \Arg{label}
%   \end{syntax}
%    Removes any code for \meta{hook} labeled \meta{label}.
%
%    If the code for that \meta{label} wasn't yet added to the
%    \meta{hook}, an order is set so that when some code attempts to add
%    that label, the removal order takes action and the code is not
%    added.
%
%    If the second argument is \texttt{*}, then all code chunks are
%    removed. This is rather dangerous as it drops code from other
%    packages one may not know about, so think twice before using
%    that!
%
%    The \meta{hook} and \meta{label} can be specified using the
%    dot-syntax to denote the current package name.
%    See section~\ref{sec:default-label}.
% \end{function}


%  \begin{function}{\hook_gset_rule:nnnn}
%   \begin{syntax}
%     \cs{hook_gset_rule:nnnn} \Arg{hook} \Arg{label1} \Arg{relation} \Arg{label2}
%   \end{syntax}
%    Relate \meta{label1} with \meta{label2} when used in \meta{hook}.
%    See \cs{DeclareHookRule} for the allowed \meta{relation}s.
%    If \meta{hook} is \texttt{??} a default rule is specified.
%
%    The \meta{hook} and \meta{label} can be specified using the
%    dot-syntax to denote the current package name.
%    See section~\ref{sec:default-label}.
%    The dot-syntax is parsed in both \meta{label} arguments, but it
%    usually makes sense to be used in only one of them.
% \end{function}
%
% \begin{function}[pTF]{\hook_if_empty:n}
%   \begin{syntax}
%     \cs{hook_if_empty:n} \Arg{hook} \Arg{true code} \Arg{false code}
%   \end{syntax}
%   Tests if the \meta{hook} is empty (\emph{i.e.}, no code was added to
%   it using either \cs{AddToHook} or \cs{AddToHookNext}), and
%   branches to either \meta{true code} or \meta{false code} depending
%   on the result.
%
%    The \meta{hook} \emph{cannot} be specified using the dot-syntax.
%    A leading |.| is treated literally.
% \end{function}
%
% \begin{function}[pTF]{\hook_if_exist:n}
%   \begin{syntax}
%     \cs{hook_if_exist:n} \Arg{hook} \Arg{true code} \Arg{false code}
%   \end{syntax}
%   Tests if the \meta{hook} exists (if it was created with either
%   \cs{NewHook}, \cs{NewReversedHook}, or \cs{NewMirroredHookPair}), and
%   branches to either \meta{true code} or \meta{false code} depending
%   on the result.
%
%  \fmi{what are the results for generic hooks that do not need to be declared?}
%
%   The existence of a hook usually doesn't mean much from the viewpoint
%   of code that tries to add/remove code from that hook, since package
%   loading order may vary, thus the creation of hooks is asynchronous
%   to adding and removing code from it, so this test should be used
%   sparingly.
%
%    The \meta{hook} \emph{cannot} be specified using the dot-syntax.
%    A leading |.| is treated literally.
% \end{function}
%
%
% \begin{function}{\hook_debug_on:,\hook_debug_off:}
%   \begin{syntax}
%     \cs{hook_debug_on:}
%   \end{syntax}
%    Turns the debugging of hook code on or off. This displays changes
%    to the hook data.
% \end{function}
%
%
%
% \subsection{On the order of hook code execution} \label{sec:order}
%
%    Chunks of code for a \meta{hook} under different labels are supposed
%    to be independent if there are no special rules set up that
%    define a relation between the chunks. This means that you can't
%    make assumptions about the order of execution!
%
%    Suppose you have the following declarations:
%\begin{verbatim}
%    \NewHook{myhook}
%    \AddToHook{myhook}[packageA]{\typeout{A}}
%    \AddToHook{myhook}[packageB]{\typeout{B}}
%    \AddToHook{myhook}[packageC]{\typeout{C}}
%\end{verbatim}
%    then executing the hook with \cs{UseHook} will produce the
%    typeout \texttt{A} \texttt{B} \texttt{C} in that order.  In other
%    words, the execution order is computed to be \texttt{packageA},
%    \texttt{packageB}, \texttt{packageC} which you can verify with
%    \cs{ShowHook}\texttt{\{myhook\}}:
%\begin{verbatim}
%   The hook 'myhook':
%    Code chunks:
%       packageA -> \typeout {A}
%       packageB -> \typeout {B}
%       packageC -> \typeout {C}
%    Extra code next invocation:
%       ---
%    Rules:
%       ---
%    Execution order:
%       packageA, packageB, packageC
%\end{verbatim}
%    The reason is that the code chunks are internally saved in a property list
%    and the initial order of such a property list is the order in
%    which key-value pairs got added. However, that is only true if
%    nothing other than adding happens!
%
%    Suppose, or example, you want to replace the code chunk for
%    \texttt{packageA}, e.g.,
%\begin{verbatim}
%    \RemoveFromHook{myhook}[packageA]
%    \AddToHook{myhook}[packageA]{\typeout{A alt}}
%\end{verbatim}
%    then your order becomes  \texttt{packageB},
%    \texttt{packageC}, \texttt{packageA} because the label got removed
%    from the property list and then re-added (at its end).
%
%    While that may not be too surprising,  the execution order is
%    also sometimes altered if you add a redundant rule, e.g. if you specify
%\begin{verbatim}
%    \DeclareHookRule{myhook}{packageA}{before}{packageB}
%\end{verbatim}
%    instead of the previous lines we get
%\begin{verbatim}
%   The hook 'myhook':
%    Code chunks:
%       packageA -> \typeout {A}
%       packageB -> \typeout {B}
%       packageC -> \typeout {C}
%    Extra code next invocation:
%       ---
%    Rules:
%       packageA|packageB with relation before
%    Execution order (after applying rules):
%       packageA, packageC, packageB
%\end{verbatim}
%    As you can see the code chunks are still in the same order, but
%    in the execution order for the labels \texttt{packageB} and
%    \texttt{packageC} have
%    swapped places.
%    The reason is that, with the rule there are two orders that
%    satisfy it, and the algorithm for sorting happened to pick a
%    different one compared to the case without rules (where it
%    doesn't run at all as there is nothing to resolve).
%    Incidentally, if we had instead specified the redundant rule
%\begin{verbatim}
%    \DeclareHookRule{myhook}{packageB}{before}{label-3}
%\end{verbatim}
%    the execution order would not have changed.
%
%    In summary: it is not possible to rely on the order of execution
%    unless there are rules that partially or fully define the order
%    (in which you can rely on them being fulfilled).
%
%
% \subsection{The use of \enquote{reversed} hooks} \label{sec:reversed-order}
%
%    You may have wondered why you can declare a \enquote{reversed} hook
%    with \cs{NewReversedHook} and what that does exactly.
%
%    In short: the execution order of a reversed hook (without any
%    rules!) is exactly reversed to the order you would have gotten for
%    a hook declared with \cs{NewHook}.
%
%    This is helpful if you have a pair of hooks where you expect to see
%    code added that involves grouping, e.g., starting an environment
%    in the first and closing that environment in the second hook.
%    To give a somewhat contrived example\footnote{there are simpler
%    ways to achieve the same effect.}, suppose there is a package
%    adding the following:
%\begin{verbatim}
%    \AddToHook{env/quote/before}[package-1]{\begin{itshape}}
%    \AddToHook{env/quote/after} [package-1]{\end{itshape}}
%\end{verbatim}
%    As a result, all quotes will be in italics.
%    Now suppose further that the user wants the quotes also in blue
%    and therefore adds:
%\begin{verbatim}
%    \usepackage{color}
%    \AddToHook{env/quote/before}{\begin{color}{blue}}
%    \AddToHook{env/quote/after} {\end{color}}
%\end{verbatim}
%    Now if the \hook{env/quote/after} hook would be a normal hook we
%    would get the same execution order in  both hooks, namely:
%\begin{verbatim}
%    package-1, top-level
%\end{verbatim}
%    (or vice versa) and as a result, would get:
%\begin{verbatim}
%    \begin{itshape}\begin{color}{blue} ...
%    \end{itshape}\end{color}
%\end{verbatim}
%   and an error message that \verb=\begin{color}= ended by
%    \verb=\end{itshape}=.
%    With \hook{env/quote/after} declared as a reversed hook the
%    execution order is reversed and so all environments are closed in
%    the correct sequence and \cs{ShowHook} would give us the
%    following output:
%\begin{verbatim}
%   The hook 'env/quote/after':
%    Code chunks:
%       package-1 -> \end {itshape}
%       top-level -> \end {color}
%    Extra code next invocation:
%       ---
%    Rules:
%       ---
%    Execution order (after reversal):
%       top-level, package-1
%\end{verbatim}
%
%    The reversal of the execution order happens before applying any
%    rules, so if you alter the order you will probably have to alter
%    it in both hooks, not just in one, but that depends on the use case.
%
%
%
%
% \subsection{Private \LaTeX{} kernel hooks}
%
%    There are a few places where it is absolutely essential for
%    \LaTeX{} to function correctly that code is executed in a precisely
%    defined order. Even that could have been implemented with the
%    hook management (by adding various rules to ensure the
%    appropriate ordering with respect to other code added by
%    packages). However, this makes every document unnecessary
%    slow, because there has to be sorting even through the result is
%    predetermined. Furthermore it forces package writers to
%    unnecessarily add such rules if they add further code to the hook
%    (or break \LaTeX{}).
%
%    For that reason such code is not using the hook management, but
%    instead private kernel commands directly before or after a public
%    hook with the following naming
%    convention: \cs{@kernel@before@\meta{hookname}} or
%    \cs{@kernel@after@\meta{hookname}}. For example, in
%    \cs{enddocument} you find
%\begin{verbatim}
%   \UseHook{enddocument}%
%   \@kernel@after@enddocument
%\end{verbatim}
%    which means first the user/package-accessible \hook{enddocument}
%    hook is executed and then the internal kernel hook. As their name
%    indicates these kernel commands should not be altered by third-party
%    packages, so please refrain from that in the interest of
%    stability and instead use the public hook next to it.\footnote{As
%    with everything in \TeX{} there is not enforcement of this rule,
%    and by looking at the code it is easy to find out how the kernel
%    adds to them. The main reason of this section is therefore to say
%    \enquote{please don't do that, this is unconfigurable code!}}
%
%
%
% \subsection{Legacy \LaTeXe{} interfaces}
%
%  \LaTeXe{} offered a small number of hooks together with commands to
%    add to them. They are listed here and are retained for backwards
%    compatibility.
%
%  With the new hook management several additional hooks have been added
%    to \LaTeX\ and more will follow. See the next section for what
%    is already available.
%
%
% \begin{function}{\AtBeginDocument}
%   \begin{syntax}
%     \cs{AtBeginDocument} \oarg{label} \Arg{code}
%   \end{syntax}
%   If used without the optional argument \meta{label}, it works essentially
%    like before, i.e., it is adding \meta{code} to the hook
%    \hook{begindocument} 
%    (which is executed inside \verb=\begin{document}=).
%    However, all code added this way is labeled with the label
%    \hook{top-level} if done outside of a package or class or with the
%    package/class name if called inside such a file.
%
%    This way one can add further code to the hook using
%    \cs{AddToHook} or \cs{AtBeginDocument} using a different label
%    and explicitly order the code chunks as necessary, e.g., run some
%    code before or after the \hook{top-level} code.  When using the
%    optional argument the call is equivalent to running
%    \cs{AddToHook} \texttt{\{begindocument\}} \oarg{label}
%    \Arg{code}.
%
%    For important packages with known order requirement we may over
%    time add rules to the kernel (or to those packages) so that they
%    work regardless of the loading-order in the document.
% \end{function}
%
% \begin{function}{\AtEndDocument}
%   \begin{syntax}
%     \cs{AtEndDocument} \oarg{label} \Arg{code}
%   \end{syntax}
%   Like \cs{AtBeginDocument} but for the \hook{enddocument} hook.
% \end{function}
%
%
%
% \begin{function}{\AtBeginDvi}
%   \begin{syntax}
%     \cs{AtBeginDvi} \oarg{label} \Arg{code}
%   \end{syntax}
%   This hook is discussed in conjunction with the shipout hooks.
% \end{function}
%
%
%
% \subsection{\LaTeXe{} commands and environments augmented by
%    hooks}
%
%  \emph{intro to be written}
%
% \subsubsection{Generic hooks for all environments}
%
%    Every environment \meta{env} has now four associated hooks coming
%    with it:
%    \begin{description}
%    \item[\hook{env/\meta{env}/before}]
%
%       This hook is executed as part of \cs{begin} as the very first
%       action, in particular prior to starting the environment group.
%       Its scope is therefore not restricted by the environment.
%
%    \item[\hook{env/\meta{env}/begin}]
%
%       This hook is executed as part of \cs{begin} directly in front
%       of the code specific to the environment start (e.g., the
%       second argument of \cs{newenvironment}).  Its scope is the
%       environment body.
%
%    \item[\hook{env/\meta{env}/end}]
%
%       This is executed as part of \cs{end} directly in front of the
%       code specific to the end of the environment (e.g., the third
%       argument of \cs{newenvironment}).
%
%    \item[\hook{env/\meta{env}/after}]
%
%       This is executed as part of \cs{end} after the
%       code specific to the environment end and after the environment
%       group has ended.
%       Its scope is therefore not restricted by the environment.
%
%       This hook is implemented as a reversed hook so if two packages
%       add code to \hook{env/\meta{env}/before} and to
%       \hook{env/\meta{env}/after} they can add surrounding
%       environments and the order of closing them happens in the
%       right sequence.
%
%    \end{description}
%    In contrast to other hooks these hooks do not need to be declared
%    using \cs{NewHook}.
%
%    The hooks are only executed if \cs{begin}\Arg{env} and
%    \cs{end}\Arg{env} is used. If the environment code is executed
%    via low-level calls to \cs{\meta{env}} and \cs{end\meta{env}}
%    (e.g., to avoid the environment grouping) they are not
%    available. If you want them available in code using this method,
%    you would need to add them yourself, i.e., write something like
%\begin{verbatim}
%  \UseHook{env/quote/before}\quote
%      ...
%  \endquote\UseHook{env/quote/after}
%\end{verbatim}
%    to add the outer hooks, etc.
%
%
%
% \subsubsection{Hooks provided by \cs{begin}\texttt{\{document\}}}
%
%    Until 2020 \cs{begin}\texttt{\{document\}} offered exactly one
%    hook that one had to fill using \cs{AtBeginDocument}. Experience
%    has shown that this single hook in one place was not enough and
%    as part of adding the general hook management system a number of
%    additional hooks have been added at this point. The places for
%    hooks have been chosen to provide the same support as offered by
%    external packages, such as \pkg{etoolbox} and others that
%    augmented \cs{document} to gain better control.
%
%    Supported are now the following hooks:
%    \begin{description}
%
%    \item[\hook{env/document/before}]
%
%      This is the generic environment hook executed effectively
%      before \verb=\begin{document}= starts, i.e., one can think of
%      it as a hook for code at the end of the preamble section.
%
%    \item[\hook{env/document/begin}]
%
%      This is the second generic environment hook that is executed
%      after the environment has started its group. But given that for the
%      \texttt{document} environment this group is canceled there is
%      little difference to the previous one as the two are directly
%      executed one after another (the only difference is that in this
%      hook \cs{@currenvir} is now set to \texttt{document} but
%      anybody adding to this hook would know that already).
%
%
%    \item[\hook{begindocument}]
%
%      This hook is added to by \cs{AtBeginDocument} and is executed
%      after the \texttt{.aux} file as be read in and most
%      initialization are done, so they can be altered and inspected by
%      the hook code. It is followed by a small number of further
%      initializations that shouldn't be altered and are therefore
%      coming later.
%
%    \item[\hook{begindocument/end}]
%
%      This hook is executed at the end of the \cs{document} code in
%      other words at the beginning of the document body. The only
%      command that follows it is \cs{ignorespaces}.
%
%    \end{description}
%
%
%
%
% \subsubsection{Hooks provided by \cs{end}\texttt{\{document\}}}
%
%    \LaTeXe{} always provided \cs{AtEndDocument} to add code to the
%    execution of \verb=\end{document}= just in front of the code that
%    is normally executed there. While this was a big improvement over
%    the situation in \LaTeX\,2.09 it was not flexible enough for a
%    number of use cases and so packages, such as \pkg{etoolbox},
%    \pkg{atveryend} and others patched \cs{enddocument} to add
%    additional points where code could be hooked into.
%
%    Patching using packages is always problematical as leads to
%    conflicts (code availability, ordering of patches, incompatible
%    patches, etc.).  For this reason a number of additional hooks
%    have been added to the \cs{enddocument} code to allow packages
%    to add code in various places in a controlled way without the
%    need for overwriting or patching the core code.
%
%    Supported are now the following hooks:
%    \begin{description}
%
%    \item[\hook{env/document/end}] The generic hook inside \cs{end}.
%
%    \item[\hook{enddocument}]

%      The hook associated with \cs{AtEndDocument}. It is immediately
%      called after the previous hook so there could be just
%      one.\footnote{We could make \cs{AtEndDocument} just fill the
%      \hook{env/document/end} but maybe that is a bit confusing.}
%
%      When this hook is executed there may be still unprocessed
%      material (e.g., floats on the deferlist) and the hook may add
%      further material to be typeset. After it, \cs{clearpage} is
%      called to ensure that all such material gets typeset. If there
%      is nothing waiting the \cs{clearpage} has no effect.
%
%
%    \item[\hook{enddocument/afterlastpage}]
%
%      As the name indicates this hook should not receive code that
%      generates material for further pages. It is the right place to
%      do some final housekeeping and possibly write out some
%      information to the \texttt{.aux} file (which is still open at
%      this point to receive data). It is also the correct place to
%      set up any testing code to be run when the \texttt{.aux} file
%      is re-read in the next step.
%
%
%      After this hook has been executed the \texttt{.aux} file is
%      closed for writing and then read back in to do some tests
%      (e.g., looking for missing references or duplicated labels, etc.).
%
%    \item[\hook{enddocument/afteraux}]
%
%      At this point, the \texttt{.aux} file has been reprocessed and so
%      this is a possible place for final checks and display of
%      information to the user. However, for the latter you might
%      prefer the next hook, so that your information is displayed after the
%      (possibly longish) list of files if that got requested via \cs{listfiles}.
%
%    \item[\hook{enddocument/info}]
%
%      This hook is meant to receive code that write final information
%      messages to the terminal. It follows immediately after the
%      previous hook (so both could have been combined, but then
%      packages adding further code would always need to also supply
%      an explicit rule to specify where it should go.
%
%      This hook already contains some code added by the kernel (under
%      the labels \texttt{kernel/filelist} and
%      \texttt{kernel/warnings}), namely the list of files when
%      \cs{listfiles} has been used and the warnings for duplicate
%      labels, missing references, font substitutions etc.
%
%    \item[\hook{enddocument/end}]
%
%      Finally, this hook is executed just in front of the final call
%      to \cs{@{}@end}.
%
%    \end{description}
%
%
%    There is also the hook \hook{shipout/lastpage}. This hook is
%    executed as part of the last \cs{shipout} in the document to
%    allow package to add final \cs{special}'s to that page. Where
%    this hook is executed in relation to those from the above list
%    can vary from document to document. Furthermore to determine correctly
%    which of the \cs{shipout}s is the last one, \LaTeX{} needs to be run
%    several times, so initially it might get executed on the wrong
%    page. See section~\ref{sec:shipout} for where to find the details.
%
%
% \subsubsection{Hooks provided \cs{shipout} operations}
% \label{sec:shipout}
%
%    There are several hooks and mechanisms added to \LaTeX{}'s
%    process of generating pages. These are documented in
%    \texttt{ltshipout-doc.pdf} or with code in
%    \texttt{ltshipout-code.pdf}.
%
%
% \subsubsection{Hooks provided file loading operations}
%
%    There are several hooks added to \LaTeX{}'s
%    process of loading file via its high-level interfaces such as
%    \cs{input}, \cs{include}, \cs{usepackage}, etc. These are documented in
%    \texttt{ltfilehook-doc.pdf} or with code in
%    \texttt{ltfilehook-code.pdf}.
%
%

% \StopEventually{\setlength\IndexMin{200pt}  \PrintIndex  }
%
%
% \section{The Implementation}
%    \begin{macrocode}
%<@@=hook>
%    \end{macrocode}
%
%    \begin{macrocode}
%<*2ekernel>
%    \end{macrocode}
%    
%    \begin{macrocode}
\ExplSyntaxOn
%    \end{macrocode}
%
%
%  \subsection{Debugging}
%
%  \begin{macro}{\g_@@_debug_bool}
%    Holds the current debugging state.
%    \begin{macrocode}
\bool_new:N \g_@@_debug_bool
%    \end{macrocode}
%  \end{macro}
%
%  \begin{macro}{\hook_debug_on:,\hook_debug_off:}
%  \begin{macro}{\@@_debug:n}
%  \begin{macro}{\@@_debug_gset:}
%    Turns debugging on and off by redefining \cs{@@_debug:n}.
%    \begin{macrocode}
\cs_new_eq:NN \@@_debug:n \use_none:n
\cs_new_protected:Npn \hook_debug_on:
  {
    \bool_gset_true:N \g_@@_debug_bool
    \@@_debug_gset:
  }
\cs_new_protected:Npn \hook_debug_off:
  {
    \bool_gset_false:N \g_@@_debug_bool
    \@@_debug_gset:
  }
\cs_new_protected:Npn \@@_debug_gset:
  {
    \cs_gset_protected:Npx \@@_debug:n ##1
      { \bool_if:NT \g_@@_debug_bool {##1} }
  }
%    \end{macrocode}
%  \end{macro}
%  \end{macro}
%  \end{macro}

%
%
%  \subsection{Borrowing from internals of other kernel modules}
%
%
% \begin{macro}[EXP]{\@@_str_compare:nn}
%   Private copy of \cs{__str_if_eq:nn}
%    \begin{macrocode}
\cs_new_eq:NN \@@_str_compare:nn \__str_if_eq:nn
%    \end{macrocode}
% \end{macro}
%
%  \subsection{Declarations}
%
%  \begin{macro}{\l_@@_return_tl,\l_@@_tmpa_tl,\l_@@_tmpb_tl}
%    Scratch variables used throughout the package.
%    \begin{macrocode}
\tl_new:N \l_@@_return_tl
\tl_new:N \l_@@_tmpa_tl
\tl_new:N \l_@@_tmpb_tl
%    \end{macrocode}
%  \end{macro}
%
%  \begin{macro}{\g_@@_all_seq}
%    In a few places we need a list of all hook names ever defined so
%    we keep track if them in this sequence.
%    \begin{macrocode}
\seq_new:N \g_@@_all_seq
%    \end{macrocode}
%  \end{macro}
%
% \begin{macro}{\g_@@_removal_list_prop}
%   A property list to hold delayed removals.
%    \begin{macrocode}
\tl_new:N \g_@@_removal_list_tl
%    \end{macrocode}
% \end{macro}
%
% \begin{macro}{\l_@@_cur_hook_tl}
%   Stores the name of the hook currently being sorted.
%    \begin{macrocode}
\tl_new:N \l_@@_cur_hook_tl
%    \end{macrocode}
% \end{macro}
%
% \begin{macro}{\g_@@_code_temp_prop}
%   A property list to temporarily save the original one so that it
%   isn't permanently changed during sorting.
%    \begin{macrocode}
\prop_new:N \g_@@_code_temp_prop
%    \end{macrocode}
% \end{macro}
%
% \begin{macro}{\g_@@_hook_curr_name_tl,\g_@@_name_stack_seq}
%   Default label used for hook commands, and a stack to keep track of
%   packages within packages.
%    \begin{macrocode}
\tl_new:N \g_@@_hook_curr_name_tl
\seq_new:N \g_@@_name_stack_seq
%    \end{macrocode}
% \end{macro}
%
% \begin{macro}{\@@_tmp:w}
%   Temporary macro for generic usage.
%    \begin{macrocode}
\cs_new_eq:NN \@@_tmp:w ?
%    \end{macrocode}
% \end{macro}
%
% \begin{macro}{\tl_gremove_once:Nx}
%   Some variants of \pkg{expl3} functions. \fmi{should be moved to expl3}
%    \begin{macrocode}
\cs_generate_variant:Nn \tl_gremove_once:Nn { Nx }
%    \end{macrocode}
% \end{macro}
%
% \begin{macro}{\s_@@_mark}
%   Scan mark used for delimited arguments.
%    \begin{macrocode}
\scan_new:N \s_@@_mark
%    \end{macrocode}
% \end{macro}
%
%
% \subsection{Providing new hooks}
%
%  \begin{macro}{\g_@@_..._code_prop,\g_@@_..._rules_prop,
%                \g_@@_..._code_tl,\g_@@_..._next_code_tl}
%
%    Hooks have a \meta{name} and for each hook we have to provide a number of
%    data structures. These are
%    \begin{description}
%    \item[\cs{g_@@_\meta{name}_code_prop}] A property list holding the code
%    for the hook in separate chunks. The keys are by default the
%    package names that add code to the hook, but it is possible
%    for packages to define other keys. 
%
%    \item[\cs{g_@@_\meta{name}_rules_prop}] A property listing holding
%    relation info how the code chunks should be ordered within a
%    hook. This is used for debugging only. The actual rule for a
%    \meta{hook} is stored in a separate token lists named
%    \cs[no-index]{g_@@_\meta{hook}_rule_\meta{label1}\string|\meta{label2}_tl}
%    for a pair of labels.
%
%    \item[\cs{g_@@_\meta{name}_code_tl}] The code that is actually executed
%    when the hook is called in the document is stored in this token
%    list. It is constructed from the code chunks applying the
%    information.
%
%    \item[\cs{g_@@_\meta{name}_next_code_tl}] Finally there is extra code
%    (normally empty) that is used on the next invocation of the hook
%    (and then deleted). This can be used to define some special
%    behavior for a single occasion from within the document.
%
%    \end{description}
%  \end{macro}


%
%
%  \begin{macro}{\hook_new:n}
%    The \cs{hook_new:n} declaration declare a new hook and expects
%    the hook \meta{name} as its argument, e.g.,
%    \hook{begindocument}.
%    \begin{macrocode}
\cs_new_protected:Npn \hook_new:n #1
  {
    \exp_args:Nx \@@_new:n
      { \@@_parse_label_default:nn {#1} { top-level } }
  }
\cs_new_protected:Npn \@@_new:n #1 {
%    \end{macrocode}
%    We check for one of the internal data structures and if it
%    already exists we complain.
%    \begin{macrocode}
  \hook_if_exist:nTF {#1}
     { \ErrorHookExists }
%    \end{macrocode}
%    Otherwise we add the hook name to the list of all hooks and
%    allocate the necessary data structures for the new hook.
%    \begin{macrocode}
     { \seq_gput_right:Nn \g_@@_all_seq {#1}
%    \end{macrocode}
%    This is only used by the actual code of the current hook, so
%    declare it normally:
%    \begin{macrocode}
       \tl_new:c { g_@@_#1_code_tl }
%    \end{macrocode}
%    Now ensure that the base data structure for the hook exists:
%    \begin{macrocode}
       \@@_declare:n {#1}
%    \end{macrocode}
%    The \cs{g_@@_\meta{hook}_labels_clist} holds the sorted list of
%    labels (once it got sorted). This is used only for debugging.
%    \begin{macrocode}
       \clist_new:c {g_@@_#1_labels_clist}
%    \end{macrocode}
%    Some hooks should reverse the default order of code chunks. To
%    signal this we have a token list which is empty for normal hooks
%    and contains a \verb=-= for reversed hooks.
%    \begin{macrocode}
       \tl_new:c { g_@@_#1_reversed_tl }
%    \end{macrocode}
%    The above is all in L3 convention, but we also provide an
%    interface to legacy \LaTeXe{} for use in the current kernel. This
%    is done in a separate macro.
%    \begin{macrocode}
       \@@_provide_legacy_interface:n {#1}
     }
}
%    \end{macrocode}
%  \end{macro}

%
%
% \begin{macro}{\@@_declare:n}
%   This function declares the basic data structures for a hook without
%   actually declaring the hook itself.  This is needed to allow adding
%   to undeclared hooks.  Here it is unnecessary to check whether both
%   variables exist, since both are declared at the same time (either
%   both exist, or neither).
%    \begin{macrocode}
\cs_new_protected:Npn \@@_declare:n #1
  {
    \@@_if_exist:nF {#1}
      {
        \prop_new:c { g_@@_#1_code_prop }
        \tl_new:c { g_@@_#1_next_code_tl }
        \prop_new:c { g_@@_#1_rules_prop } % only for debugging
      }
  }
%    \end{macrocode}
%  \end{macro}



%  \begin{macro}{\hook_new_reversed:n}
%
%    Declare a new hook. The default ordering of code chunks is
%    reversed, signaled by setting the token list to a minus sign.
%    \begin{macrocode}
\cs_new_protected:Npn \hook_new_reversed:n #1 {
  \hook_new:n {#1}
  \tl_gset:cn { g_@@_#1_reversed_tl } { - }
}
%    \end{macrocode}
%  \end{macro}

%  \begin{macro}{\hook_new_pair:nn}
%    A shorthand for declaring a normal and a (matching) reversed hook in one go.
%    \begin{macrocode}
\cs_new_protected:Npn \hook_new_pair:nn #1#2 {
  \hook_new:n {#1}  \hook_new_reversed:n {#2}
}
%    \end{macrocode}
%  \end{macro}


% \begin{macro}{\@@_provide_legacy_interface:n}
%    The \LaTeX{} legacy concept for hooks uses with hooks the
%    following naming scheme in the code: \cs{@...hook}.
%
%    We follow this convention and insert the hook code using this
%    naming scheme in \LaTeXe{}. At least as long as this code is in a
%    package, some such hooks are already filled with data when we move
%    them over to the new scheme. We therefore insert already existing
%    code under the label \texttt{legacy} into the hook management
%    machinery and then replace the \cs{@...hook} with its counterpart
%    which is \cs{g_@@_\#1_code_tl}.\footnote{This means one extra
%    unnecessary expansion on each invocation in the document but
%    keeps the \LaTeXe{} and the L3 coding side properly separated.}
%    \begin{macrocode}
\cs_new_protected:Npn \@@_provide_legacy_interface:n #1
  {
%    \end{macrocode}
%    If the \pkg{expl3} code is run with checking on then assigning or
%    using non L3 names such as \cs{@enddocumenthook} with \pkg{expl3}
%    functions will trigger warnings so we run this code with
%    debugging explicitly suspended.
%    \begin{macrocode}
    \debug_suspend:
    \tl_if_exist:cT { @#1hook }
%    \end{macrocode}
%    Of course if the hook exists but is still empty, there is no need
%    to add anything under \texttt{legacy} or the current package name.
%    \begin{macrocode}
      {
        \tl_if_empty:cF { @#1hook }
          {
            \@@_gput_code:nxv {#1}
              { \@@_parse_label_default:Vn \c_novalue_tl { legacy } }
              { @#1hook }
          }
      }
%    \end{macrocode}
%    We need a global definition in case the declaration is done
%    inside a group (which happens below at the end of the file).
%    This is another reason why need to suspend checking, otherwise
%    \cs{tl_gset:co} would complain about \cs{@...hook} not starting
%    with \cs{g_}.
%    \begin{macrocode}
    \tl_gset:co{@#1hook}{\cs:w g_@@_#1_code_tl\cs_end:}
    \debug_resume:
  }
%    \end{macrocode}
% \end{macro}
%
% \subsection{Parsing a label}
%
% \begin{macro}[EXP]{\@@_parse_label_default:nn,\@@_parse_label_default:Vn}
%   This macro checks if a label was given (not \cs{c_novalue_tl}), and
%   if so, tries to parse the label looking for a leading \verb|.| to
%   replace for \cs{@currname}.  Otherwise \cs{@@_currname_or_default:n}
%   is used to pick \cs{@currname} or the fallback value.
%    \begin{macrocode}
\cs_new:Npn \@@_parse_label_default:nn #1 #2
  {
    \tl_if_novalue:nTF {#1}
      { \@@_currname_or_default:n {#2} }
      { \tl_trim_spaces_apply:nN {#1} \@@_parse_dot_label:nn {#2} }
  }
\cs_generate_variant:Nn \@@_parse_label_default:nn { V }
%    \end{macrocode}
% \end{macro}
%
% \begin{macro}[EXP]{\@@_parse_dot_label:nn}
% \begin{macro}[EXP]{
%     \@@_parse_dot_label:nw,
%     \@@_parse_dot_label_cleanup:w,
%     \@@_parse_dot_label_aux:nw
%   }
%   Start by checking if the label is empty, which raises an error, and
%   uses the fallback value.  If not,
%   split the label at a \verb|./|, if any, and check if no tokens are
%   before the \verb|./|, or if the only character is a \verb|.|.
%   If these requirements are fulfilled, the leading
%   \verb|.| is replaced with \cs{@@_currname_or_default:n}.  Otherwise
%   the label is returned unchanged.
%    \begin{macrocode}
\cs_new:Npn \@@_parse_dot_label:nn #1 #2
  {
    \tl_if_empty:nTF {#1}
      {
        \msg_expandable_error:nnn { hooks } { empty-label } {#2}
        #2
      }
      {
        \str_if_eq:nnTF {#1} { . }
          { \@@_currname_or_default:n {#1} }
          { \@@_parse_dot_label:nw {#2} #1 ./ \s_@@_mark }
      }
  }
\cs_new:Npn \@@_parse_dot_label:nw #1 #2 ./ #3 \s_@@_mark
  {
    \tl_if_empty:nTF {#2}
      { \@@_parse_dot_label_aux:nw {#1} #3 \s_@@_mark }
      {
        \tl_if_empty:nTF {#3}
          {#2}
          { \@@_parse_dot_label_cleanup:w #2 ./ #3 \s_@@_mark }
      }
  }
\cs_new:Npn \@@_parse_dot_label_cleanup:w #1 ./ \s_@@_mark {#1}
\cs_new:Npn \@@_parse_dot_label_aux:nw #1 #2 ./ \s_@@_mark
  { \@@_currname_or_default:n {#1} / #2 }
%    \end{macrocode}
% \end{macro}
% \end{macro}
%
% \begin{macro}[EXP]{\@@_currname_or_default:n}
%   Uses \cs{g_@@_hook_curr_name_tl} if it is set, otherwise tries
%   \cs{@currname}.  If neither is set, uses the fallback value
%   \verb|#1| (usually \texttt{top-level}).
%    \begin{macrocode}
\cs_new:Npn \@@_currname_or_default:n #1
  {
    \tl_if_empty:NTF \g_@@_hook_curr_name_tl
      {
        \tl_if_empty:NTF \@currname
          {#1}
          { \@currname }
      }
      { \g_@@_hook_curr_name_tl }
  }
%    \end{macrocode}
% \end{macro}




% \begin{macro}{\hook_gput_code:nnn}
% \begin{macro}{\@@_gput_code:nnn,\@@_gput_code:nxv,\@@_hook_gput_code_do:nnn}
%
%    With \cs{hook_gput_code:nnn}\Arg{hook}\Arg{label}\Arg{code} a
%    chunk of \meta{code} is added to an existing \meta{hook} labeled
%    with \meta{label}.
%    \begin{macrocode}
\cs_new_protected:Npn \hook_gput_code:nnn #1 #2
  {
    \exp_args:Nxx \@@_gput_code:nnn
      { \@@_parse_label_default:nn {#1} { top-level } }
      { \@@_parse_label_default:nn {#2} { top-level } }
  }
\cs_new_protected:Npn \@@_gput_code:nnn #1 #2 #3
  {
%    \end{macrocode}
%    First we check if the hook exists.
%    \begin{macrocode}
    \@@_if_marked_removal:nnTF {#1} {#2}
      { \@@_unmark_removal:nn {#1} {#2} }
      {
%    \end{macrocode}
%    First we check if the hook exists.
%    \begin{macrocode}
        \hook_if_exist:nTF {#1}
%    \end{macrocode}
%    If so we simply add (or append) the new code to the property list
%    holding different chunks for the hook. At \verb=\begin{document}=
%    this is then sorted into a token list for fast execution.
%    \begin{macrocode}
          {
            \@@_hook_gput_code_do:nnn {#1} {#2} {#3}
%    \end{macrocode}
%    However, if there is an update within the document we need to alter
%    this execution code which is done by
%    \cs{@@_update_hook_code:n}. In the preamble this does nothing.
%    \begin{macrocode}
            \@@_update_hook_code:n {#1}
          }
%    \end{macrocode}
%    
%    \begin{macrocode}
          { \@@_try_declaring_generic_hook:nnn {#1} {#2} {#3} }
      }
  }
\cs_generate_variant:Nn \@@_gput_code:nnn { nxv }
%    \end{macrocode}
%
%   This macro will unconditionally add a chunk of code to the given hook.
%    \begin{macrocode}
\cs_new_protected:Npn \@@_hook_gput_code_do:nnn #1 #2 #3
  {
%   However, first some debugging info if debugging is enabled:
%    \begin{macrocode}
    \@@_debug:n{\iow_term:x{****~ Add~ to~
                      \hook_if_exist:nF {#1} { undeclared~ }
                      hook~ #1~ (#2)
                      \on@line\space <-~ \tl_to_str:n{#3}} }
%    \end{macrocode}
%   Then try to get the code chunk labeled \verb=#2= from the hook.
%   If there's code already there, then append \verb=#3= to that,
%   otherwise just put \verb=#3=.
%    \begin{macrocode}
    \prop_get:cnNTF { g_@@_#1_code_prop } {#2} \l_@@_return_tl
      {
        \prop_gput:cno { g_@@_#1_code_prop } {#2}
          { \l_@@_return_tl #3 }
      }
      { \prop_gput:cnn { g_@@_#1_code_prop } {#2} {#3} }
  }
%    \end{macrocode}
% \end{macro}
% \end{macro}
%
% \begin{macro}{\@@_gput_undeclared_hook:nnn}
%   Often it may happen that a package $A$ defines a hook \verb=foo=,
%   but package $B$, that adds code to that hook, is loaded before $A$.
%   In such case we need to add code to the hook before its declared.
%    \begin{macrocode}
\cs_new_protected:Npn \@@_gput_undeclared_hook:nnn #1 #2 #3
  {
    \@@_declare:n {#1}
    \@@_hook_gput_code_do:nnn {#1} {#2} {#3}
  }
%    \end{macrocode}
% \end{macro}
%
% \begin{macro}{\@@_try_declaring_generic_hook:nnn}
% \begin{macro}{\@@_try_declaring_generic_next_hook:nn}
%   These entry-level macros just pass the arguments along to the
%   common \cs{@@_try_declaring_generic_hook:nNNnn} with the right
%   functions to execute when some action is to be taken.
%
%   The wrapper \cs{@@_try_declaring_generic_hook:nnn} then defers
%   \cs{hook_gput_code:nnn} if the generic hook was declared, or to
%   \cs{@@_gput_undeclared_hook:nnn} otherwise (the hook was tested for
%   existence before, so at this point if it isn't generic, it doesn't
%   exist).
%
%   The wrapper \cs{@@_try_declaring_generic_next_hook:nn} for
%   next-execution hooks does the same: it defers the code to
%   \cs{hook_gput_next_code:nn} if the generic hook was declared, or
%   to \cs{@@_gput_next_do:nn} otherwise.
%    \begin{macrocode}
\cs_new_protected:Npn \@@_try_declaring_generic_hook:nnn #1
  {
    \@@_try_declaring_generic_hook:nNNnn {#1}
      \hook_gput_code:nnn \@@_gput_undeclared_hook:nnn
  }
\cs_new_protected:Npn \@@_try_declaring_generic_next_hook:nn #1
  {
    \@@_try_declaring_generic_hook:nNNnn {#1}
      \hook_gput_next_code:nn \@@_gput_next_do:nn
  }
%    \end{macrocode}
%
% \begin{macro}{
%     \@@_try_declaring_generic_hook:nNNnn,
%     \@@_try_declaring_generic_hook_split:nNNnn
%   }
% \begin{macro}[TF]{\@@_try_declaring_generic_hook:wn}
%   \cs{@@_try_declaring_generic_hook:nNNnn} now splits the hook name
%   at the first \texttt{/} (if any) and first checks if it is a
%   file-specific hook (they require some normalization) using
%   \cs{@@_if_file_hook:wTF}. If not then check it is one of a
%   predefined set for generic names. We also split off the second
%   component to see if we have to make a reversed hook.  In either case
%   the function returns \meta{true} for a generic hook and \meta{false}
%   in other cases.
%    \begin{macrocode}
\cs_new_protected:Npn \@@_try_declaring_generic_hook:nNNnn #1
  {
    \@@_if_file_hook:wTF #1 / / \s_@@_mark
      {
        \exp_args:Ne \@@_try_declaring_generic_hook_split:nNNnn
          { \exp_args:Ne \@@_file_hook_normalise:n {#1} }
      }
      { \@@_try_declaring_generic_hook_split:nNNnn {#1} }
  }
\cs_new_protected:Npn \@@_try_declaring_generic_hook_split:nNNnn #1 #2 #3
  {
    \@@_try_declaring_generic_hook:wnTF #1 / / / \scan_stop: {#1}
      { #2 }
      { #3 } {#1}
  }
\prg_new_protected_conditional:Npnn \@@_try_declaring_generic_hook:wn
    #1 / #2 / #3 / #4 \scan_stop: #5 { TF }
  {
    \tl_if_empty:nTF {#2}
      { \prg_return_false: }
      {
        \prop_if_in:NnTF \c_@@_generics_prop {#1}
          {
            \hook_if_exist:nF {#5} { \hook_new:n {#5} }
%    \end{macrocode}
%    After having declared the hook we check the second component (for
%    file hooks) or the third component for environment hooks) and
%    if it is on the list of components for which we should have declared
%    a reversed hook we alter the hook data structure accordingly.
%    \begin{macrocode}
            \prop_if_in:NnTF \c_@@_generics_reversed_ii_prop {#2}
              { \tl_gset:cn { g_@@_#5_reversed_tl } { - } }
              {
                \prop_if_in:NnT \c_@@_generics_reversed_iii_prop {#3}
                  { \tl_gset:cn { g_@@_#5_reversed_tl } { - } }
              }
%    \end{macrocode}
%    Now that we know that the hook is declared we can add the code to it.
%    \begin{macrocode}
            \prg_return_true:
          }
          { \prg_return_false: }
      }
  }
%    \end{macrocode}
% \end{macro}
% \end{macro}
% \end{macro}
% \end{macro}
%
% \begin{macro}[pTF]{\@@_if_file_hook:w}
%   \cs{@@_if_file_hook:wTF} checks if the argument is a valid
%   file-specific hook (not, for example, |file/before|, but
%   |file/before/foo.tex|).  If it is a file-specific hook, then it
%   executes the \meta{true} branch, otherwise \meta{false}.
%
%   A file-specific hook is \texttt{file/\meta{position}/\meta{name}}.
%   If any of these parts don't exist, it is a general file hook or not
%   a file hook at all, so the conditional evaluates to \meta{false}.
%   Otherwise, it checks that the first part is |file| and that the
%   \meta{position} is in the \cs{c_@@_generics_file_prop}.
%
%   A property list is used here to avoid having to worry with catcodes,
%   because \pkg{expl3}'s file name parsing turns all characters into
%   catcode-12 tokens, which might differ from hand-input letters.
%    \begin{macrocode}
\prg_new_conditional:Npnn \@@_if_file_hook:w
    #1 / #2 / #3 \s_@@_mark { TF }
  {
    \str_if_eq:nnTF {#1} { file }
      {
        \bool_lazy_or:nnTF
            { \tl_if_empty_p:n {#3} }
            { \str_if_eq_p:nn {#3} { / } }
          { \prg_return_false: }
          {
            \prop_if_in:NnTF \c_@@_generics_file_prop {#2}
              { \prg_return_true: }
              { \prg_return_false: }
          }
      }
      { \prg_return_false: }
  }
%    \end{macrocode}
% \end{macro}
%
% \begin{macro}[EXP]{\@@_file_hook_normalise:n}
% \begin{macro}[EXP]{\@@_strip_double_slash:n,\@@_strip_double_slash:w}
%   When a file-specific hook is found, before being declared it is
%   lightly normalized by \cs{@@_file_hook_normalise:n}.  The current
%   implementation just replaces two consecutive slashes (|//|) by a
%   single one, to cope with simple cases where the user did something
%   like \verb|\def\input@path{{./mypath/}}|, in which case a hook would
%   have to be \verb|\AddToHook{file/after/./mypath//file.tex}|.
%    \begin{macrocode}
\cs_new:Npn \@@_file_hook_normalise:n #1
  { \@@_strip_double_slash:n {#1} }
\cs_new:Npn \@@_strip_double_slash:n #1
  { \@@_strip_double_slash:w #1 // \s_@@_mark }
\cs_new:Npn \@@_strip_double_slash:w #1 // #2 \s_@@_mark
  {
    \tl_if_empty:nTF {#2}
      {#1}
      { \@@_strip_double_slash:w #1 / #2 \s_@@_mark }
  }
%    \end{macrocode}
% \end{macro}
% \end{macro}





%  \begin{macro}{\c_@@_generics_prop}
%    Clist holding the generic names. We don't provide any user
%    interface to this as this is meant to be static.
%    \begin{description}
%    \item[\texttt{env}]
%      The generic hooks used in \cs{begin} and \cs{end}.
%    \item[\texttt{file}]
%      The generic hooks used when loading a file
%    \end{description}
%    \begin{macrocode}
\prop_const_from_keyval:Nn \c_@@_generics_prop
  {env=,file=,package=,class=,include=}
%    \end{macrocode}
%  \end{macro}
%
%  \begin{macro}{\c_@@_generics_reversed_ii_prop,
%                \c_@@_generics_reversed_iii_prop,
%                \c_@@_generics_file_prop}
%    Some of the generic hooks are supposed to use reverse ordering, these are
%    the following (only the second or third sub-component is checked):
%    \begin{macrocode}
\prop_const_from_keyval:Nn \c_@@_generics_reversed_ii_prop {after=,end=}
\prop_const_from_keyval:Nn \c_@@_generics_reversed_iii_prop {after=}
\prop_const_from_keyval:Nn \c_@@_generics_file_prop {before=,after=}
%    \end{macrocode}
%  \end{macro}


%  \begin{macro}{\@@_update_hook_code:n}
%    Before \verb=\begin{document}=  this does nothing, in the body it
%    reinitializes the hook code using the altered data.
%    \begin{macrocode}
\cs_new_eq:NN \@@_update_hook_code:n \use_none:n
%    \end{macrocode}
%  \end{macro}


% \begin{macro}{\hook_gremove_code:nn}
% \begin{macro}{\@@_gremove_code:nn}
%    
%    With \cs{hook_gremove_code:nn}\Arg{hook}\Arg{label} any code
%    for \meta{hook} stored under \meta{label} is removed.
%    \begin{macrocode}
\cs_new_protected:Npn \hook_gremove_code:nn #1 #2
  {
    \exp_args:Nxx \@@_gremove_code:nn
      { \@@_parse_label_default:nn {#1} { top-level } }
      { \@@_parse_label_default:nn {#2} { top-level } }
  }
\cs_new_protected:Npn \@@_gremove_code:nn #1 #2
  {
%    \end{macrocode}
%    First check that the hook code pool exists.  \cs{hook_if_exist:nTF}
%    isn't used here because it should be possible to remove code from a
%    hook before its defined (see section~\ref{sec:querying}).
%    \begin{macrocode}
    \@@_if_exist:nTF {#1}
%    \end{macrocode}
%    Then remove the chunk and run \cs{@@_update_hook_code:n} so
%    that the execution token list reflects the change if we are after
%    \verb=\begin{document}=.
%    \begin{macrocode}
      {
        \str_if_eq:nnTF {#2} {*}
          {
            \prop_gclear:c { g_@@_#1_code_prop }
            \clist_gclear:c { g_@@_#1_labels_clist } % for debugging only
          }
          {
%    \end{macrocode}
%    Check if the label being removed exists in the code pool.  If it does,
%    just call \cs{@@_gremove_code_do:nn} to do the removal, otherwise mark it
%    to be removed.
%    \begin{macrocode}
            \prop_get:cnNTF { g_@@_#1_code_prop } {#2} \l_@@_return_tl
              { \@@_gremove_code_do:nn }
              { \@@_mark_removal:nn }
                  {#1} {#2}
          }
%    \end{macrocode}
%    Finally update the code, if the hook exists.
%    \begin{macrocode}
        \hook_if_exist:nT {#1}
          { \@@_update_hook_code:n {#1} }
      }
      { \@@_mark_removal:nn {#1} {#2} }
  }
%    \end{macrocode}
%
% \begin{macro}{\@@_gremove_code_do:nn}
%    \begin{macrocode}
\cs_new_protected:Npn \@@_gremove_code_do:nn #1 #2
  {
    \prop_gremove:cn { g_@@_#1_code_prop } {#2}
%    \end{macrocode}
%    Removing the dropped label from \verb=\g_@@_#1_labels_clist= is
%    rather tricky, because that clists holds the labels as strings
%    (i.e., not ordinary text which is what we have in \verb=#2=).
%    \begin{macrocode}
    \exp_args:Nco \clist_gremove_all:Nn
      { g_@@_#1_labels_clist } { \tl_to_str:n {#2} } % for debugging only
  }
%    \end{macrocode}
% \end{macro}
% \end{macro}
% \end{macro}


%
% \begin{macro}{\@@_mark_removal:nn}
%   Marks \meta{label} (\verb=#2=) to be removed from \meta{hook}
%   (\verb=#1=).
%    \begin{macrocode}
\cs_new_protected:Npn \@@_mark_removal:nn #1 #2
  {
    \tl_gput_right:Nx \g_@@_removal_list_tl
      { \@@_removal_tl:nn {#1} {#2} }
  }
%    \end{macrocode}
% \end{macro}
%
% \begin{macro}{\@@_unmark_removal:nn}
%   Unmarks \meta{label} (\verb=#2=) to be removed from \meta{hook}
%   (\verb=#1=).  \cs{tl_gremove_once:Nx} is used rather than
%   \cs{tl_gremove_all:Nx} so that two additions are needed to cancel
%   two marked removals, rather than only one.
%    \begin{macrocode}
\cs_new_protected:Npn \@@_unmark_removal:nn #1 #2
  {
    \tl_gremove_once:Nx \g_@@_removal_list_tl
      { \@@_removal_tl:nn {#1} {#2} }
  }
%    \end{macrocode}
% \end{macro}
%
% \begin{macro}[TF]{\@@_if_marked_removal:nn}
%   Checks if the \cs{g_@@_removal_list_tl} contains the current
%   \meta{label} (\verb=#2=) and \meta{hook} (\verb=#1=).
%    \begin{macrocode}
\prg_new_protected_conditional:Npnn \@@_if_marked_removal:nn #1 #2 { TF }
  {
    \exp_args:NNx \tl_if_in:NnTF \g_@@_removal_list_tl
      { \@@_removal_tl:nn {#1} {#2} }
      { \prg_return_true: } { \prg_return_false: }
  }
%    \end{macrocode}
% \end{macro}
%
% \begin{macro}[rEXP]{\@@_removal_tl:nn}
%   Builds a token list with \verb=#1= and \verb=#2= which can only be
%   matched by \verb=#1= and \verb=#2=.
%    \begin{macrocode}
\cs_new:Npn \@@_removal_tl:nn #1 #2
  { & \tl_to_str:n {#2} $ \tl_to_str:n {#1} $ }
%    \end{macrocode}
% \end{macro}




%
%
%
% \begin{macro}{
%     \g_@@_??_rules_prop,
%     \g_@@_??_code_prop,
%     \g_@@_??_code_tl,
%     \g_@@_??_reversed_tl,
%   }
%
%    Default rules applying to all hooks are stored in this property
%    list. Initially it simply used an empty ``label'' name (not two
%    question marks). This was a bit unfortunate, because then
%    \texttt{l3doc} complains about \verb=__= in the middle of a
%    command name when trying to typeset the documentation. However
%    using a ``normal'' name such as \texttt{default} has the
%    disadvantage of that being not really distinguishable from a real
%    hook name. I now have settled for \texttt{??} which needs some
%    gymnastics to get it into the csname, but since this is used a
%    lot things should be fast, so this is not done with \texttt{c}
%    expansion in the code later on.
%
%    \cs{g_@@_??_code_tl} isn't used, but it has to be defined to trick
%    the code into thinking that \verb=??= is actually a hook.
%    \begin{macrocode}
\prop_new:c {g_@@_??_rules_prop}
\prop_new:c {g_@@_??_code_prop}
\prop_new:c {g_@@_??_code_tl}
%    \end{macrocode}
%
%    Default rules are always given in normal ordering (never in
%    reversed ordering). If such a rule is applied to a reversed
%    hook it behaves as if the rule is reversed (e.g.,
%    \texttt{after} becomes \texttt{before})
%    because those rules are applied first and then the order is reversed.
%    \begin{macrocode}
\tl_new:c {g_@@_??_reversed_tl}
%    \end{macrocode}
%  \end{macro}



%  \begin{macro}{\@@_debug_gset_rule:nnnn}
%
%    \fmi{this needs cleanup and docu correction!}
%
%    With
%    \cs{@@_debug_gset_rule:nnnn}\Arg{hook}\Arg{label1}\Arg{relation}\Arg{label2}
%    a relation is defined between the two code labels for the given
%    \meta{hook}.  The special hook \texttt{??} stands for \emph{any}
%    hook describing a default rule.
%    \begin{macrocode}
\cs_new_protected:Npn \@@_debug_gset_rule:nnnn #1#2#3#4
  {
%    \end{macrocode}
%    If so we drop any existing rules with the two labels (in case
%    there are any).
%    \begin{macrocode}
    \prop_gremove:cn{g_@@_#1_rules_prop}{#2|#4}
    \prop_gremove:cn{g_@@_#1_rules_prop}{#4|#2}
%    \end{macrocode}
%    Then  we add the new one (normalizing the input a bit, e.g., we
%    always use \texttt{before} and not \texttt{after} and
%    instead reorder the labels):
%    \begin{macrocode}
    \str_case_e:nnF {#3}
       {
         {before} { \prop_gput:cnn {g_@@_#1_rules_prop}{#2|#4}{<} }
         {after}  { \prop_gput:cnn {g_@@_#1_rules_prop}{#4|#2}{<} }
%    \end{macrocode}
%    More special rule types \ldots
%    \begin{macrocode}
         {incompatible-error}   { \prop_gput:cnn {g_@@_#1_rules_prop}{#2|#4}{xE} }
         {incompatible-warning} { \prop_gput:cnn {g_@@_#1_rules_prop}{#2|#4}{xW} }
         {removes}      { \prop_gput:cnn {g_@@_#1_rules_prop}{#2|#4}{->} }
%    \end{macrocode}
%    Undo a setting:
%    \begin{macrocode}
         {unrelated}{ \prop_gremove:cn {g_@@_#1_rules_prop}{#2|#4} 
                      \prop_gremove:cn {g_@@_#1_rules_prop}{#4|#2} }
       }
       { \ERRORunknownrule }
  }
%    \end{macrocode}
%  \end{macro}
%
%
%  \subsection{Setting rules for hooks code}
%
%  \begin{macro}{\hook_gset_rule:nnnn}
%  \begin{macro}{\@@_gset_rule:nnnn}
%
%    \fmi{needs docu correction given new implementation}
%
%    With
%    \cs{hook_gset_rule:nnnn}\Arg{hook}\Arg{label1}\Arg{relation}\Arg{label2}
%    a relation is defined between the two code labels for the given
%    \meta{hook}.  The special hook \texttt{??} stands for \emph{any}
%    hook describing a default rule.
%    \begin{macrocode}
\cs_new_protected:Npn \hook_gset_rule:nnnn #1#2#3#4
  {
    \use:x
      {
        \@@_gset_rule:nnnn
          { \@@_parse_label_default:nn {#1} { top-level } }
          { \@@_parse_label_default:nn {#2} { top-level } }
          {#3}
          { \@@_parse_label_default:nn {#4} { top-level } }
      }
  }
%    \end{macrocode}
%    
%    \begin{macrocode}
\cs_new_protected:Npn \@@_gset_rule:nnnn #1#2#3#4
  {
%    \end{macrocode}
%    First we ensure the basic data structure of the hook exists:
%    \begin{macrocode}
    \@@_declare:n {#1}
%    \end{macrocode}
%    Then we clear any previous relationship between both labels.
%    \begin{macrocode}
    \@@_rule_gclear:nnn {#1} {#2} {#4}
%    \end{macrocode}
%    Then we call the function to handle the given rule. Throw an error if the
%    rule is invalid.
%    \begin{macrocode}
    \debug_suspend:
    \cs_if_exist_use:cTF { @@_rule_#3_gset:nnn }
      {
          {#1} {#2} {#4}
        \@@_update_hook_code:n {#1}
      }
      { \ERRORunknownrule }
    \debug_resume:
    \@@_debug_gset_rule:nnnn {#1} {#2} {#3} {#4} % for debugging
  }
%    \end{macrocode}
% \end{macro}
% \end{macro}
%
% \begin{macro}{\@@_rule_before_gset:nnn, \@@_rule_after_gset:nnn,
%               \@@_rule_<_gset:nnn, \@@_rule_>_gset:nnn}
%    Then we add the new rule.  We need to normalize the rules here to
%    allow for faster processing later.  Given a pair of labels
%    $l_A$ and $l_B$, the rule $l_A>l_B$ is the same as $l_B<l_A$\fmi{}
%    said differently.  But normalizing the
%    forms of the rule to a single representation, say, $l_B<l_A$, then
%    the time spent looking for the rules later is considerably reduced.
%
%    Here we do that normalization by using \cs[no-index]{(pdf)strcmp} to
%    lexically sort labels $l_A$ and $l_B$ to a fixed order.  This order
%    is then enforced every time these two labels are used together.
%
%    Here we use \cs{@@_label_pair:nn}~\Arg{hook}~\Arg{l_A}~\Arg{l_B}
%    to build a string \texttt{$l_B$\string|$l_A$} with a fixed order, and
%    use \cs{@@_label_ordered:nnTF} to apply the correct rule to the pair
%    of labels, depending if it was sorted or not.
%    \begin{macrocode}
\cs_new_protected:Npn \@@_rule_before_gset:nnn #1#2#3
  {
    \tl_gset:cx { g_@@_#1_rule_ \@@_label_pair:nn {#2} {#3} _tl }
      { \@@_label_ordered:nnTF {#2} {#3} { < } { > } }
  }
\cs_new_eq:cN { @@_rule_<_gset:nnn } \@@_rule_before_gset:nnn
%    \end{macrocode}
%
%    \begin{macrocode}
\cs_new_protected:Npn \@@_rule_after_gset:nnn #1#2#3
  {
    \tl_gset:cx { g_@@_#1_rule_ \@@_label_pair:nn {#3} {#2} _tl }
      { \@@_label_ordered:nnTF {#3} {#2} { < } { > } }
  }
\cs_new_eq:cN { @@_rule_>_gset:nnn } \@@_rule_after_gset:nnn
%    \end{macrocode}
%  \end{macro}
%
% \begin{macro}{\@@_rule_removes_gset:nnn}
%   This rule removes (clears, actually) the code from label |#3| if
%   label |#2| is in the hook |#1|.
%    \begin{macrocode}
\cs_new_protected:Npn \@@_rule_removes_gset:nnn #1#2#3
  {
    \tl_gset:cx { g_@@_#1_rule_ \@@_label_pair:nn {#2} {#3} _tl }
      { \@@_label_ordered:nnTF {#2} {#3} { -> } { <- } }
  }
%    \end{macrocode}
%  \end{macro}
%
% \begin{macro}{
%     \@@_rule_incompatible-error_gset:nnn,
%     \@@_rule_incompatible-warning_gset:nnn,
%   }
%   These relations make an error/warning if labels |#2| and |#3| appear
%   together in hook |#1|.
%    \begin{macrocode}
\cs_new_protected:cpn { @@_rule_incompatible-error_gset:nnn } #1#2#3
  { \tl_gset:cn { g_@@_#1_rule_ \@@_label_pair:nn {#2} {#3} _tl } { xE } }
\cs_new_protected:cpn { @@_rule_incompatible-warning_gset:nnn } #1#2#3
  { \tl_gset:cn { g_@@_#1_rule_ \@@_label_pair:nn {#2} {#3} _tl } { xW } }
%    \end{macrocode}
%  \end{macro}
%
% \begin{macro}{\@@_rule_unrelated_gset:nnn, \@@_rule_gclear:nnn}
%    Undo a setting. \cs{@@_rule_unrelated_gset:nnn} doesn't need to do anything,
%    since we use \cs{@@_rule_gclear:nnn} before setting any rule.
%    \begin{macrocode}
\cs_new_protected:Npn \@@_rule_unrelated_gset:nnn #1#2#3 { }
\cs_new_protected:Npn \@@_rule_gclear:nnn #1#2#3
  { \cs_undefine:c { g_@@_#1_rule_ \@@_label_pair:nn {#2} {#3} _tl } }
%    \end{macrocode}
%  \end{macro}
%
% \begin{macro}[EXP]{\@@_label_pair:nn}
%   Ensure that the lexically greater label comes first.
%    \begin{macrocode}
\cs_new:Npn \@@_label_pair:nn #1#2
  {
    \if_case:w \@@_str_compare:nn {#1} {#2} \exp_stop_f:
           #1 | #1 %  0
    \or:   #1 | #2 % +1
    \else: #2 | #1 % -1
    \fi:
  }
%    \end{macrocode}
%  \end{macro}
%
% \begin{macro}[pTF]{\@@_label_ordered:nn}
%   Check that labels |#1| and |#2| are in the correct order (as
%   returned by \cs{@@_label_pair:nn}) and if so return true, else
%   return false.
%    \begin{macrocode}
\prg_new_conditional:Npnn \@@_label_ordered:nn #1#2 { TF }
  {
    \if_int_compare:w \@@_str_compare:nn {#1} {#2} > 0 \exp_stop_f:
      \prg_return_true:
    \else
      \prg_return_false:
    \fi:
  }
%    \end{macrocode}
%  \end{macro}
%
% \begin{macro}[EXP]{\@@_if_label_case:nnnnn}
%   To avoid doing the string comparison twice in \cs{@@_initialize_single:NNNNn}
%   (once with \cs{str_if_eq:nn} and again with \cs{@@_label_ordered:nn}),
%   we use a three-way branching macro that will compare |#1| and |#2|
%   and expand to \cs{use_i:nnn} if they are equal, \cs{use_ii:nn} if
%   |#1| is lexically greater, and \cs{use_iii:nn} otherwise.
%    \begin{macrocode}
\cs_new:Npn \@@_if_label_case:nnnnn #1#2
   {
     \cs:w use_
       \if_case:w \@@_str_compare:nn {#1} {#2}
          i \or: ii \else: iii \fi: :nnn
     \cs_end:
   }
%    \end{macrocode}
%  \end{macro}


%  \begin{macro}{\@@_initialize_all:}
%    Initialize all known hooks (at \verb=\begin{document}=), i.e.,
%    update the fast execution token lists to hold the necessary code
%    in the right  order.
%    \begin{macrocode}
\cs_new_protected:Npn \@@_initialize_all: {
%    \end{macrocode}
%    First we change \cs{@@_update_hook_code:n} which so far was a
%    no-op to now initialize one hook. This way any later updates to
%    the hook will run that code and also update the execution token
%    list.
%    \begin{macrocode}
  \cs_gset_eq:NN \@@_update_hook_code:n \@@_initialize_hook_code:n
%    \end{macrocode}
%    Now we loop over all hooks that have been defined and update each
%    of them.
%    \begin{macrocode}
  \@@_debug:n { \prop_gclear:N \g_@@_used_prop }
  \seq_map_inline:Nn \g_@@_all_seq
      {
        \@@_update_hook_code:n {##1}
      }
%    \end{macrocode}
%    If we are debugging we show results hook by hook for all hooks
%    that have data. 
%    \begin{macrocode}
  \@@_debug:n
     { \iow_term:x{^^JAll~ initialized~ (non-empty)~ hooks:}
       \prop_map_inline:Nn \g_@@_used_prop
           { \iow_term:x{^^J~ ##1~ ->~
               \exp_not:v {g_@@_##1_code_tl}~ }
           }
     }
%    
%    \end{macrocode}
%    After all hooks are initialized we change the ``use'' to just
%    call the hook code and not initialize it (as it was done in the
%    preamble.
%    \begin{macrocode}
  \cs_gset_eq:NN \hook_use:n \@@_use_initialized:n
  \cs_gset_eq:NN \@@_preamble_hook:n \use_none:n
}
%    \end{macrocode}
%  \end{macro}



%  \begin{macro}{\@@_initialize_hook_code:n}
%    Initializing or reinitializing the fast execution hook code. In
%    the preamble this is selectively done in case a hook gets used
%    and at \verb=\begin{document}= this is done for all hooks and
%    afterwards only if the hook code changes.
%    \begin{macrocode}
\cs_new_protected:Npn \@@_initialize_hook_code:n #1 {
  \@@_debug:n{ \iow_term:x{^^JUpdate~ code~ for~ hook~
                                  '#1' \on@line :^^J} }
%    \end{macrocode}
%    This does the sorting and the updates. If there aren't any code
%    chunks for the current hook, there is no point in even starting
%    the sorting routine so we make a quick test for that and in that
%    case just update \cs{g_@@_\meta{hook}_code_tl} to hold the next
%    code. If there are code chunks we call
%    \cs{@@_initialize_single:NNNNn} and pass to it ready made csnames
%    as they are needed several times inside. This way we save a bit
%    on processing time if we do that up front.
%    \begin{macrocode}
  \hook_if_exist:nT {#1}
    {
      \prop_if_empty:cTF {g_@@_#1_code_prop}
        { \tl_gset:co {g_@@_#1_code_tl}
                      {\cs:w g_@@_#1_next_code_tl \cs_end: } }
        {
%    \end{macrocode}
%    By default the algorithm sorts the code chunks and then saves the
%    result in a token list for fast execution by adding the code one
%    after another using \cs{tl_gput_right:NV}. When we sort code for
%    a reversed hook, all we have to do is to add the code chunks in
%    the opposite order into the token list. So all we have to do
%    in preparation is to change two definitions used later on.
%    \begin{macrocode}
          \@@_if_reversed:nTF {#1}
            { \cs_set_eq:NN \@@_tl_gput:NV    \tl_gput_left:NV
              \cs_set_eq:NN \@@_clist_gput:NV \clist_gput_left:NV  }
            { \cs_set_eq:NN \@@_tl_gput:NV    \tl_gput_right:NV
              \cs_set_eq:NN \@@_clist_gput:NV \clist_gput_right:NV }
%    \end{macrocode}
%
%    When sorting, some relations (namely \verb|->| \verb|<-|) need to
%    act destructively on the code property lists to remove code that
%    shouldn't appear in the sorted hook token list.
%    \begin{macrocode}
          \prop_gset_eq:Nc \g_@@_code_temp_prop { g_@@_#1_code_prop }
          \@@_initialize_single:ccccn
            { g_@@_#1_code_prop } { g_@@_#1_code_tl }
            { g_@@_#1_next_code_tl } { g_@@_#1_labels_clist }
            {#1}
          \prop_gset_eq:cN { g_@@_#1_code_prop } \g_@@_code_temp_prop
%    \end{macrocode}
%    For debug display we want to keep track of those hooks that
%    actually got code added to them, so we record that in plist. We
%    use a plist to ensure that we record each hook name only once,
%    i.e., we are only interested in storing the keys and the value is arbitrary
%    \begin{macrocode}
          \@@_debug:n{ \exp_args:NNx \prop_gput:Nnn \g_@@_used_prop {#1}{} }
        }
    }
}
%    \end{macrocode}
%  \end{macro}


%  \begin{macro}{\g_@@_used_prop}
%    All hooks that receive code (for use in debugging display).
%    \begin{macrocode}
\prop_new:N\g_@@_used_prop
%    \end{macrocode}
%  \end{macro}


%
% \begin{macro}[EXP]{\@@_tl_csname:n,\@@_seq_csname:n}
%   It is faster to pass a single token and expand it when necessary
%   than to pass a bunch of character tokens around.
%   \fmi{note to myself: verify}
%    \begin{macrocode}
\cs_new:Npn \@@_tl_csname:n #1 { l_@@_label_#1_tl }
\cs_new:Npn \@@_seq_csname:n #1 { l_@@_label_#1_seq }
%    \end{macrocode}
% \end{macro}
%

%
%
%  \begin{macro}{\l_@@_labels_seq,\l_@@_labels_int,\l_@@_front_tl,
%      \l_@@_rear_tl,\l_@@_label_0_tl}
%
%    For the sorting I am basically implementing Knuth's algorithm for
%    topological sorting as given in TAOCP volume 1 pages 263--266.
%    For this algorithm we need a number of local variables:
%    \begin{itemize}
%    \item
%       List of labels used in the current hook to label code chunks:
%    \begin{macrocode}
\seq_new:N \l_@@_labels_seq
%    \end{macrocode}
%    \item
%      Number of labels used in the current hook. In Knuth's algorithm
%      this is called $N$:
%    \begin{macrocode}
\int_new:N \l_@@_labels_int
%    \end{macrocode}
%    \item
%      The sorted code list to be build is managed using two pointers
%      one to the front of the queue and one to the rear. We model this
%      using token list pointers. Knuth calls them $F$ and $R$:
%    \begin{macrocode}
\tl_new:N \l_@@_front_tl
\tl_new:N \l_@@_rear_tl
%    \end{macrocode}
%    \item
%      The data for the start of the queue is kept in this token list,
%      it corresponds to what Don calls \texttt{QLINK[0]} but since we
%      aren't manipulating individual words in memory it is slightly
%      differently done:
%    \begin{macrocode}
\tl_new:c { \@@_tl_csname:n { 0 } }
%    \end{macrocode}
%
%    \end{itemize}
%  \end{macro}


%  \begin{macro}{\@@_initialize_single:NNNNn,\@@_initialize_single:ccccn}
%
%    \cs{@@_initialize_single:NNNNn} implements the sorting of the code
%    chunks for a hook and saves the result in the token list for fast
%    execution (\verb=#3=). The arguments are \meta{hook-code-plist},
%    \meta{hook-code-tl}, \meta{hook-next-code-tl},
%    \meta{hook-ordered-labels-clist} and \meta{hook-name} (the latter
%    is only used for debugging---the \meta{hook-rule-plist} is accessed
%    using the \meta{hook-name}).
%
%    The additional complexity compared to Don's algorithm is that we
%    do not use simple positive integers but have arbitrary
%    alphanumeric labels. As usual Don's data structures are chosen in
%    a way that one can omit a lot of tests and I have mimicked that as
%    far as possible. The result is a restriction I do not test for at
%    the moment: a label can't be equal to the number 0!  \fmi{Needs
%    checking for, just in case}
%    \begin{macrocode}
\cs_new_protected:Npn \@@_initialize_single:NNNNn #1#2#3#4#5 {
%    \end{macrocode}
%    
%    \begin{macrocode}
  \debug_suspend:
%    \end{macrocode}
%    Step T1: Initialize the data structure \ldots
%    \begin{macrocode}
  \seq_clear:N \l_@@_labels_seq
  \int_zero:N  \l_@@_labels_int
%    \end{macrocode}
%
%    Store the name of the hook:
%    \begin{macrocode}
  \tl_set:Nn \l_@@_cur_hook_tl {#5}
%    \end{macrocode}
%    
%    We loop over the property list holding the code and record all
%    labels listed there. Only rules for those labels are of interest
%    to us. While we are at it we count them (which gives us the $N$
%    in Knuth's algorithm.  The prefix |label_| is added to the variables
%    to ensure that labels named |front|, |rear|, |labels|, or |return|
%    don't interact with our code.
%    \begin{macrocode}
  \prop_map_inline:Nn #1
     {
       \int_incr:N \l_@@_labels_int
       \seq_put_right:Nn \l_@@_labels_seq {##1}
       \tl_set:cn { \@@_tl_csname:n {##1} }{0}     % the counter k for number of
                                                   % j before k rules
       \seq_clear_new:c { \@@_seq_csname:n {##1} } % sequence of successors to k
                                                   % i.e., k before j rules (stores
                                                   % the names of the j's)
     }
%    \end{macrocode}
%    Steps T2 and T3: Sort the relevant rules into the data structure\ldots
%    
%    This loop constitutes a square matrix of the labels in |#1| in the
%    vertical and the horizontal directions.  However since the rule
%    $l_A\meta{rel}l_B$ is the same as $l_B\meta{rel}^{-1}l_A$ we can cut
%    the loop short at the diagonal of the matrix (\emph{i.e.}, when
%    both labels are equal), saving a good amount of time.  The way the
%    rules were set up (see the implementation of \cs{@@_rule_before_gset:nnn}
%    above) ensures that we have no rule in the ignored side of the
%    matrix, and all rules are seen.  The rules are applied in
%    \cs{@@_apply_label_pair:nnn}, which takes the properly-ordered pair
%    of labels as argument.
%    \begin{macrocode}
  \prop_map_inline:Nn #1
    {
      \prop_map_inline:Nn #1
        {
          \@@_if_label_case:nnnnn {##1} {####1}
            { \prop_map_break: }
            { \@@_apply_label_pair:nnn {##1} {####1} }
            { \@@_apply_label_pair:nnn {####1} {##1} }
                {#5}
        }
    }
%    \end{macrocode}
%    Take a breath and take a look at the data structures that have
%    been set up:
%    \begin{macrocode}
  \@@_debug:n { \@@_debug_label_data:N #1 }
%    \end{macrocode}
%    
%
%    Step T4:
%    \begin{macrocode}
  \tl_set:Nn \l_@@_rear_tl { 0 }
  \tl_set:cn { \@@_tl_csname:n { 0 } } { 0 } % really {l_@@_label_ \l_@@_rear_tl _tl}
  \seq_map_inline:Nn \l_@@_labels_seq
      {
        \int_compare:nNnT { \cs:w \@@_tl_csname:n {##1} \cs_end: } = 0
            {
              \tl_set:cn { \@@_tl_csname:n { \l_@@_rear_tl } }{##1}
              \tl_set:Nn \l_@@_rear_tl {##1}
            }
      }
  \tl_set_eq:Nc \l_@@_front_tl { \@@_tl_csname:n { 0 } }
%    \end{macrocode}
%    
%    \begin{macrocode}
  \tl_gclear:N #2
  \clist_gclear:N #4
%    \end{macrocode}
%
%    The whole loop combines steps T5--T7:
%    \begin{macrocode}
  \bool_while_do:nn { ! \str_if_eq_p:Vn \l_@@_front_tl { 0 } }
       {
%    \end{macrocode}
%    This part is step T5:
%    \begin{macrocode}
         \int_decr:N \l_@@_labels_int
         \prop_get:NVN #1 \l_@@_front_tl \l_@@_return_tl
         \@@_tl_gput:NV #2 \l_@@_return_tl
%    \end{macrocode}
%    
%    \begin{macrocode}
         \@@_clist_gput:NV #4 \l_@@_front_tl
         \@@_debug:n{ \iow_term:x{Handled~ code~ for~ \l_@@_front_tl} }
%    \end{macrocode}
%
%    This is step T6 except that we don't use a pointer $P$ to move
%    through the successors, but instead use \verb=##1= of the mapping
%    function.
%    \begin{macrocode}
         \seq_map_inline:cn { \@@_seq_csname:n { \l_@@_front_tl } }
             {
               \tl_set:cx { \@@_tl_csname:n {##1} }
                   { \int_eval:n { \cs:w \@@_tl_csname:n {##1} \cs_end: - 1 } }
               \int_compare:nNnT { \cs:w \@@_tl_csname:n {##1} \cs_end: } = 0
                   {
                     \tl_set:cn { \@@_tl_csname:n { \l_@@_rear_tl } } {##1}
                     \tl_set:Nn \l_@@_rear_tl            {##1}
                   }
             }
%    \end{macrocode}
%    and step T7:
%    \begin{macrocode}
         \tl_set_eq:Nc \l_@@_front_tl { \@@_tl_csname:n { \l_@@_front_tl } }
%    \end{macrocode}
%
%    This is step T8: If we haven't moved the code for all labels
%    (i.e., if \cs{l_@@_labels_int} is still greater than zero) we
%    have a loop and our partial order can't be flattened out.
%    \begin{macrocode}
       }
  \int_compare:nNnF \l_@@_labels_int = 0
      {
        \iow_term:x{====================}
        \iow_term:x{Error:~ label~ rules~ are~ incompatible:}
%    \end{macrocode}
%
%    This is not really the information one needs in the error case
%    but will do for now \ldots \fmi{fix}
%    \begin{macrocode}
        \@@_debug_label_data:N #1
        \iow_term:x{====================}
      }
%    \end{macrocode}
%    After we have added all hook code to \verb=#2= we finish it off
%    with adding extra code for a one time execution. That is stored
%    in \verb=#3= but is normally empty.
%    \begin{macrocode}
  \tl_gput_right:Nn #2 {#3}
%    \end{macrocode}
%    
%    \begin{macrocode}
  \debug_resume:
}
%    \end{macrocode}
%    
%    \begin{macrocode}
\cs_generate_variant:Nn \@@_initialize_single:NNNNn {cccc}
%    \end{macrocode}
%  \end{macro}



%  \begin{macro}{\@@_tl_gput:NV,\@@_clist_gput:NV}
%    These append either on the right (normal hook) or on the left
%    (reversed hook). This is setup up in
%    \cs{@@_initialize_hook_code:n}, elsewhere their behavior is undefined.
%    \begin{macrocode}
\cs_new:Npn \@@_tl_gput:NV     {\ERROR}
\cs_new:Npn \@@_clist_gput:NV  {\ERROR}
%    \end{macrocode}
%  \end{macro}


%
%  \begin{macro}{\@@_apply_label_pair:nnn,\@@_label_if_exist_apply:nnnF}
%
%    This is the payload of steps T2 and T3 executed in the loop described
%    above. This macro assumes |#1| and |#2| are ordered, which means that
%    any rule pertaining the pair |#1| and |#2| is
%    \cs{g_@@_\meta{hook}_rule_\#1\string|\#2_tl}, and not
%    \cs{g_@@_\meta{hook}_rule_\#2\string|\#1_tl}.  This also saves a great deal
%    of time since we only need to check the order of the labels once.
%
%    The arguments here are \meta{label1}, \meta{label2}, \meta{hook}, and
%    \meta{hook-code-plist}.  We are about to apply the next rule and
%    enter it into the data structure.  \cs{@@_apply_label_pair:nnn} will
%    just call \cs{@@_label_if_exist_apply:nnnF} for the \meta{hook}, and
%    if no rule is found, also try the \meta{hook} name \verb=??=
%    denoting a default hook rule.
%
%    \cs{@@_label_if_exist_apply:nnnF} will check if the rule exists for
%    the given hook, and if so call \cs{@@_apply_rule:nnn}.
%    \begin{macrocode}
\cs_new_protected:Npn \@@_apply_label_pair:nnn #1#2#3
  {
%    \end{macrocode}
%    Extra complication: as we use default rules and local hook specific
%    rules we first have to check if there is a local rule and if that
%    exist use it. Otherwise check if there is a default rule and use
%    that.
%    \begin{macrocode}
    \@@_label_if_exist_apply:nnnF {#1} {#2} {#3}
      {
%    \end{macrocode}
%    If there is no hook-specific rule we check for a default one and
%    use that if it exists.
%    \begin{macrocode}
        \@@_label_if_exist_apply:nnnF {#1} {#2} { ?? } { }
      }
  }
\cs_new_protected:Npn \@@_label_if_exist_apply:nnnF #1#2#3
  {
    \if_cs_exist:w g_@@_ #3 _rule_ #1 | #2 _tl \cs_end:
%    \end{macrocode}
%    What to do precisely depends on the type of rule we have
%    encountered. If it is a \texttt{before} rule it will be handled by the
%    algorithm but other types need to be managed differently. All
%    this is done in \cs{@@_apply_rule:nnnN}.
%    \begin{macrocode}
      \@@_apply_rule:nnn {#1} {#2} {#3}
      \exp_after:wN \use_none:n
    \else:
      \use:nn
    \fi:
  }
%    \end{macrocode}
%  \end{macro}




%  \begin{macro}{\@@_apply_rule:nnn}
%    This is the code executed in steps T2 and T3 while looping through
%    the matrix  This is part of step T3. We are about to apply the next
%    rule and enter it into the data structure. The arguments are
%    \meta{label1}, \meta{label2}, \meta{hook-name}, and \meta{hook-code-plist}.
%    \begin{macrocode}
\cs_new_protected:Npn \@@_apply_rule:nnn #1#2#3
  {
    \cs:w @@_apply_
      \cs:w g_@@_#3_reversed_tl \cs_end: rule_
        \cs:w g_@@_ #3 _rule_ #1 | #2 _tl \cs_end: :nnn \cs_end:
      {#1} {#2} {#3}
  }
%    \end{macrocode}
% \end{macro}
%
%  \begin{macro}{\@@_apply_rule_<:nnn,\@@_apply_rule_>:nnn}
%    The most common cases are \texttt{\string<} and \texttt{\string>} so we handle
%    that first.  They are relations $\prec$ and $\succ$ in TAOCP, and
%    they dictate sorting.
%    \begin{macrocode}
\cs_new_protected:cpn { @@_apply_rule_<:nnn } #1#2#3
  {
    \@@_debug:n { \@@_msg_pair_found:nnn {#1} {#2} {#3} }
    \tl_set:cx { \@@_tl_csname:n {#2} }
       { \int_eval:n{ \cs:w \@@_tl_csname:n {#2} \cs_end: + 1 } }
    \seq_put_right:cn{ \@@_seq_csname:n {#1} }{#2}
  }
\cs_new_protected:cpn { @@_apply_rule_>:nnn } #1#2#3
  {
    \@@_debug:n { \@@_msg_pair_found:nnn {#1} {#2} {#3} }
    \tl_set:cx { \@@_tl_csname:n {#1} }
       { \int_eval:n{ \cs:w \@@_tl_csname:n {#1} \cs_end: + 1 } }
    \seq_put_right:cn{ \@@_seq_csname:n {#2} }{#1}
  }
%    \end{macrocode}
% \end{macro}
%
% \begin{macro}{\@@_apply_rule_xE:nnn,\@@_apply_rule_xW:nnn}
%   These relations make two labels incompatible within a hook.
%   |xE| makes raises an error if the labels are found in the same
%   hook, and |xW| makes it a warning.
%    \begin{macrocode}
\cs_new_protected:cpn { @@_apply_rule_xE:nnn } #1#2#3
  {
    \@@_debug:n { \@@_msg_pair_found:nnn {#1} {#2} {#3} }
    \msg_error:nnnnnn { hooks } { labels-incompatible }
      {#1} {#2} {#3} { 1 }
    \use:c { @@_apply_rule_->:nnn } {#1} {#2} {#3}
    \use:c { @@_apply_rule_<-:nnn } {#1} {#2} {#3}
  }
\cs_new_protected:cpn { @@_apply_rule_xW:nnn } #1#2#3
  {
    \@@_debug:n { \@@_msg_pair_found:nnn {#1} {#2} {#3} }
    \msg_warning:nnnnnn { hooks } { labels-incompatible }
      {#1} {#2} {#3} { 0 }
  }
%    \end{macrocode}
% \end{macro}
%
%  \begin{macro}{\@@_apply_rule_->:nnn,\@@_apply_rule_<-:nnn}
%    If we see \texttt{\detokenize{->}} we have to drop code for label
%    \verb=#3= and carry on. We could do a little better and trop
%    everything for that label since it doesn't matter where we sort
%    in the empty code. However that would complicate the algorithm a
%    lot with little gain. So we still unnecessarily try to sort it in
%    and depending on the rules that might result in a loop that is
%    otherwise resolved. If that turns out to be a real issue, we can
%    improve the code.
%
%    Here the code is removed from \cs{l_@@_cur_hook_tl} rather than
%    \verb=#3= because the latter may be \verb=??=, and the default
%    hook doesn't store any code.  Removing from \cs{l_@@_cur_hook_tl}
%    makes default rules \verb=->= and  \verb=<-= work properly.
%    \begin{macrocode}
\cs_new_protected:cpn { @@_apply_rule_->:nnn } #1#2#3
  {
    \@@_debug:n
       {
         \@@_msg_pair_found:nnn {#1} {#2} {#3}
         \iow_term:x{--->~ Drop~ '#2'~ code~ from~
           \iow_char:N \\ g_@@_ \l_@@_cur_hook_tl _code_prop ~ because~ of~ '#1' }
       }
    \prop_gput:cnn { g_@@_ \l_@@_cur_hook_tl _code_prop } {#2} { }
  }
\cs_new_protected:cpn { @@_apply_rule_<-:nnn } #1#2#3
  {
    \@@_debug:n
       {
         \@@_msg_pair_found:nnn {#1} {#2} {#3}
         \iow_term:x{--->~ Drop~ '#1'~ code~ from~
           \iow_char:N \\ g_@@_ \l_@@_cur_hook_tl _code_prop ~ because~ of~ '#2' }
       }
    \prop_gput:cnn { g_@@_ \l_@@_cur_hook_tl _code_prop } {#1} { }
  }
%    \end{macrocode}
%  \end{macro}

% \begin{macro}{
%     \@@_apply_-rule_<:nnn,
%     \@@_apply_-rule_>:nnn,
%     \@@_apply_-rule_<-:nnn,
%     \@@_apply_-rule_->:nnn,
%     \@@_apply_-rule_x:nnn,
%   }
%   Reversed rules.
%    \begin{macrocode}
\cs_new_eq:cc { @@_apply_-rule_<:nnn  } { @@_apply_rule_>:nnn }
\cs_new_eq:cc { @@_apply_-rule_>:nnn  } { @@_apply_rule_<:nnn }
\cs_new_eq:cc { @@_apply_-rule_<-:nnn } { @@_apply_rule_<-:nnn }
\cs_new_eq:cc { @@_apply_-rule_->:nnn } { @@_apply_rule_->:nnn }
\cs_new_eq:cc { @@_apply_-rule_xE:nnn  } { @@_apply_rule_xE:nnn }
\cs_new_eq:cc { @@_apply_-rule_xW:nnn  } { @@_apply_rule_xW:nnn }
%    \end{macrocode}
% \end{macro}


% \begin{macro}{\@@_msg_pair_found:nnn}
%   A macro to avoid moving this many tokens around.
%    \begin{macrocode}
\cs_new_protected:Npn \@@_msg_pair_found:nnn #1#2#3
  {
    \iow_term:x{~ \str_if_eq:nnTF {#3} {??} {default} {~normal} ~
               rule~ \@@_label_pair:nn {#1} {#2}:~
             \use:c { g_@@_#3_rule_ \@@_label_pair:nn {#1} {#2} _tl } ~ found}
  }
%    \end{macrocode}
% \end{macro}


%  \begin{macro}{\@@_debug_label_data:N}
%    
%    \begin{macrocode}
\cs_new_protected:Npn \@@_debug_label_data:N #1 {
  \iow_term:x{Code~ labels~ for~ sorting:}
  \iow_term:x{~ \seq_use:Nnnn\l_@@_labels_seq {~and~}{,~}{~and~} }  % fix name!
  \iow_term:x{^^J Data~ structure~ for~ label~ rules:}
  \prop_map_inline:Nn #1
       {
         \iow_term:x{~ ##1~ =~ \tl_use:c{ \@@_tl_csname:n {##1} }~ ->~
           \seq_use:cnnn{ \@@_seq_csname:n {##1} }{~->~}{~->~}{~->~}
         }
       }
  \iow_term:x{}
}
%    \end{macrocode}
%  \end{macro}



%  \begin{macro}{\hook_log:n}
%    This writes out information about the hook given in its argument
%    onto the terminal and the \texttt{.log} file.
%    \begin{macrocode}
\cs_new_protected:Npn \hook_log:n #1
  {
    \exp_args:Nx \@@_log:n
      { \@@_parse_label_default:nn {#1} { top-level } }
  }
\cs_new_protected:Npn \@@_log:n #1
  {
    \iow_term:x{^^JThe~ hook~ '#1':}
%    \end{macrocode}
%    
%    \begin{macrocode}
    \hook_if_exist:nF {#1}
      { \iow_term:x {~Hook~ is~ not~ declared!} }
    \@@_if_exist:nTF {#1}
      {
        \iow_term:x{~Code~ chunks:}
        \prop_if_empty:cTF {g_@@_#1_code_prop}
          { \iow_term:x{\@spaces ---} }
          {
            \prop_map_inline:cn {g_@@_#1_code_prop}
              { \iow_term:x{\@spaces ##1~ ->~ \tl_to_str:n{##2} } }
          }
%    \end{macrocode}
%    
%    \begin{macrocode}
        \iow_term:x{~Extra~ code~ next~ invocation:}
        \iow_term:x{\@spaces
          \tl_if_empty:cTF { g_@@_#1_next_code_tl }
            {---} {->~ \str_use:c{g_@@_#1_next_code_tl} } }
%    \end{macrocode}
%
%    \fmi{This is currently only displaying the local rules, but it
%         should also show the matching global rules!}
%
%    \begin{macrocode}
        \iow_term:x{~Rules:}
        \prop_if_empty:cTF {g_@@_#1_rules_prop}
          { \iow_term:x{\@spaces ---} }
          { \prop_map_inline:cn {g_@@_#1_rules_prop}
              { \iow_term:x{\@spaces ##1~ with~ relation~ ##2} }
          }
%    \end{macrocode}
%    
%    \begin{macrocode}
        \hook_if_exist:nT {#1}
          { \iow_term:x { ~Execution~ order
               \prop_if_empty:cTF {g_@@_#1_rules_prop}
                 { \@@_if_reversed:nT {#1}
                        { ~ (after~ reversal) }
                 }
                 { ~ (after~
                   \@@_if_reversed:nT {#1} {reversal~ and~}
                   applying~ rules)
                 }
               :    
              }
            \iow_term:x { \@spaces
              \clist_if_empty:cTF{g_@@_#1_labels_clist}
                 {not~ set~ yet}
                 { \clist_use:cnnn {g_@@_#1_labels_clist}
                                   { ,~ } { ,~ } { ,~ }   }
            }
          }
      }
      { \iow_term:n { ~The~hook~is~empty. } }
    \iow_term:n { }
  }
%    \end{macrocode}
%    
%  \end{macro}



%  \subsection{Specifying code for next invocation}
%
%
%
%
%
%  \begin{macro}{\hook_gput_next_code:nn}
%    
%    \begin{macrocode}
\cs_new_protected:Npn \hook_gput_next_code:nn #1
  {
    \exp_args:Nx \@@_gput_next_code:nn
      { \@@_parse_label_default:nn {#1} { top-level } }
  }
\cs_new_protected:Npn \@@_gput_next_code:nn #1 #2
  {
    \@@_declare:n {#1}
    \hook_if_exist:nTF {#1}
      { \@@_gput_next_do:nn {#1} {#2} }
      { \@@_try_declaring_generic_next_hook:nn {#1} {#2} }
  }
\cs_new_protected:Npn \@@_gput_next_do:nn #1 #2
  {
    \tl_gput_right:cn { g_@@_#1_next_code_tl }
      { #2 \tl_gclear:c { g_@@_#1_next_code_tl } }
  }
%    \end{macrocode}
%  \end{macro}
%
%
% \subsection{Using the hook}
%
% \begin{macro}{\hook_use:n}
% \begin{macro}[EXP]{\@@_use_initialized:n}
% \begin{macro}{\@@_preamble_hook:n}
%   \cs{hook_use:n} as defined here is used in the preamble, where
%   hooks aren't initialized by default.  \cs{@@_use_initialized:n} is
%   also defined, which is the non-\tn{protected} version for use within
%   the document.  Their definition is identical, except for the
%   \cs{@@_preamble_hook:n} (which wouldn't hurt in the expandable
%   version, but it would be an unnecessary extra expansion).
%
%   \cs{@@_use_initialized:n} holds the expandable definition while in
%   the preamble. \cs{@@_preamble_hook:n} initializes the hook in the
%   preamble, and is redefined to \cs{use_none:n} at |\begin{document}|.
%
%   Both versions do the same internally:  check if the hook exist as
%   given, and if so use it as quickly as possible.  If it doesn't
%   exist, the a call to \cs{@@_use:wn} checks for file hooks.
%
%   At |\begin{document}|, all hooks are initialized, and any change in
%   them causes an update, so \cs{hook_use:n} can be made expandable.
%   This one is better not protected so that it can expand into nothing
%   if containing no code. Also important in case of generic hooks that
%   we do not generate a \cs{relax} as a side effect of checking for a
%   csname. In contrast to the \TeX{} low-level
%   \verb=\csname ...\endcsname= construct \cs{tl_if_exist:c} is
%   careful to avoid this.
%    \begin{macrocode}
\cs_new_protected:Npn \hook_use:n #1
  {
    \tl_if_exist:cTF { g_@@_#1_code_tl }
      {
        \@@_preamble_hook:n {#1}
        \cs:w g_@@_#1_code_tl \cs_end:
      }
      { \@@_use:wn #1 / \s_@@_mark {#1} }
  }
\cs_new:Npn \@@_use_initialized:n #1
  {
    \tl_if_exist:cTF { g_@@_#1_code_tl }
      { \cs:w g_@@_#1_code_tl \cs_end: }
      { \@@_use:wn #1 / \s_@@_mark {#1} }
  }
\cs_new_protected:Npn \@@_preamble_hook:n #1
  { \@@_initialize_hook_code:n {#1} }
%    \end{macrocode}
% \end{macro}
% \end{macro}
% \end{macro}
%
% \begin{macro}[EXP]{\@@_use:wn}
% \begin{macro}{\@@_try_file_hook:n,\@@_if_exist_use:n}
%   \cs{@@_use:wn} does a quick check to test if the current hook is a
%   file hook: those need a special treatment.  If it is not, the hook
%   does not exist.  If it is, then \cs{@@_try_file_hook:n} is called,
%   and checks that the current hook is a file-specific hook using
%   \cs{@@_if_file_hook:wTF}.  If it's not, then it's a generic |file/|
%   hook and is used if it exist.
%
%   If it is a file-specific hook, it passes through the same
%   normalization as during declaration, and then it is used if defined.
%   \cs{@@_if_exist_use:n} checks if the hook exist, and calls
%   \cs{@@_preamble_hook:n} if so, then uses the hook.
%    \begin{macrocode}
\cs_new:Npn \@@_use:wn #1 / #2 \s_@@_mark #3
  {
    \str_if_eq:nnTF {#1} { file }
      { \@@_try_file_hook:n {#3} }
      { } % Hook doesn't exist
  }
\cs_new_protected:Npn \@@_try_file_hook:n #1
  {
    \@@_if_file_hook:wTF #1 / / \s_@@_mark
      {
        \exp_args:Ne \@@_if_exist_use:n
          { \exp_args:Ne \@@_file_hook_normalise:n {#1} }
      }
      { \@@_if_exist_use:n {#1} } % file/ generic hook (e.g. file/before)
  }
\cs_new_protected:Npn \@@_if_exist_use:n #1
  {
    \tl_if_exist:cT { g_@@_#1_code_tl }
      {
        \@@_preamble_hook:n {#1}
        \cs:w g_@@_#1_code_tl \cs_end:
      }
  }
%    \end{macrocode}
% \end{macro}
% \end{macro}
%
%  \begin{macro}{\hook_use_once:n}
%    For hooks that can and should be used only once we have a special
%    use command that remembers the hook name in
%    \cs{g_@@_execute_immediately_clist}. This has the effect that any
%    further code added to the hook is executed immediately rather
%    than stored in the hook.
%    \begin{macrocode}
\cs_new_protected:Npn \hook_use_once:n #1
  {
    \tl_if_exist:cT { g_@@_#1_code_tl }
      {
        \clist_gput_left:Nn \g_@@_execute_immediately_clist {#1}
        \hook_use:n {#1}
      }
  }
%    \end{macrocode}
%  \end{macro}

% \subsection{Querying a hook}
%
% Simpler data types, like token lists, have three possible states; they
% can exist and be empty, exist and be non-empty, and they may not
% exist, in which case emptiness doesn't apply (though
% \cs{tl_if_empty:N} returns false in this case).
%
% Hooks are a bit more complicated: they have four possible states.
% A hook may exist or not, and either way it may or may not be empty
% (even a hook that doesn't exist may be non-empty).
%
% A hook is said to be empty when no code was added to it, either to
% its permanent code pool, or to its ``next'' token list.  The hook
% doesn't need to be declared to have code added to its code pool
% (it may happen that a package $A$ defines a hook \hook{foo}, but
% it's loaded after package $B$, which adds some code to that hook.
% In this case it is important that the code added by package $B$ is
% remembered until package $A$ is loaded).
%
% A hook is said to exist when it was declared with \cs{hook_new:n} or
% some variant thereof.
%
% \begin{macro}[pTF]{\hook_if_empty:n}
%   Test if a hook is empty (that is, no code was added to that hook).
%   A hook being empty means that \emph{both} its
%   \cs[no-index]{g_@@_\meta{hook}_code_prop} and its
%   \cs[no-index]{g_@@_\meta{hook}_next_code_tl} are empty.
%    \begin{macrocode}
\prg_new_conditional:Npnn \hook_if_empty:n #1 { p , T , F , TF }
  {
    \@@_if_exist:nTF {#1}
      {
        \bool_lazy_and:nnTF
            { \prop_if_empty_p:c { g_@@_#1_code_prop } }
            { \tl_if_empty_p:c { g_@@_#1_next_code_tl } }
          { \prg_return_true: }
          { \prg_return_false: }
      }
      { \prg_return_true: }
  }
%    \end{macrocode}
% \end{macro}

% \begin{macro}[pTF]{\hook_if_exist:n}
%   A canonical way to test if a hook exists.  A hook exists if the
%   token list that stores the sorted code for that hook,
%   \cs[no-index]{g_@@_\meta{hook}_code_tl}, exists.  The property list
%   \cs[no-index]{g_@@_\meta{hook}_code_prop} cannot be used here
%   because often it is necessary to add code to a hook without knowing
%   if such hook was already declared, or even if it will ever be
%   (for example, in case the package that defines it isn't loaded).
%    \begin{macrocode}
\prg_new_conditional:Npnn \hook_if_exist:n #1 { p , T , F , TF }
  {
    \tl_if_exist:cTF { g_@@_#1_code_tl }
      { \prg_return_true: }
      { \prg_return_false: }
  }
%    \end{macrocode}
% \end{macro}

% \begin{macro}[pTF]{\@@_if_exist:n}
%   An internal check if the hook has already been declared with
%   \cs{@@_declare:n}.  This means that the hook was already used somehow
%   (a code chunk or rule was added to it), but it still wasn't declared
%   with \cs{hook_new:n}.
%    \begin{macrocode}
\prg_new_conditional:Npnn \@@_if_exist:n #1 { p , T , F , TF }
  {
    \prop_if_exist:cTF { g_@@_#1_code_prop }
      { \prg_return_true: }
      { \prg_return_false: }
  }
%    \end{macrocode}
% \end{macro}
%
% \begin{macro}[pTF]{\@@_if_reversed:n}
%   An internal conditional that checks if a hook is reversed.
%    \begin{macrocode}
\prg_new_conditional:Npnn \@@_if_reversed:n #1 { p , T , F , TF }
  {
    \if_int_compare:w \cs:w g_@@_#1_reversed_tl \cs_end: 1 < 0 \exp_stop_f:
      \prg_return_true:
    \else:
      \prg_return_false:
    \fi:
  }
%    \end{macrocode}
% \end{macro}


%  \begin{macro}{\g_@@_execute_immediately_clist}
%    List of hooks that from no on should not longer receive code.
%    \begin{macrocode}
\clist_new:N \g_@@_execute_immediately_clist
%    \end{macrocode}
%  \end{macro}
%
%  \subsection{Messages}
%
%    \begin{macrocode}
\msg_new:nnnn { hooks } { labels-incompatible }
  {
    Labels~`#1'~and~`#2'~are~incompatible
    \str_if_eq:nnF {#3} {??} { ~in~hook~`#3' } .~
    \int_compare:nNnT {#4} = { 1 }
      { The~code~for~both~labels~will~be~dropped. }
  }
  {
    LaTeX~found~two~incompatible~labels~in~the~same~hook.~
    This~indicates~an~incompatibility~between~packages.
  }
\msg_new:nnn { hooks } { empty-label }
  { Empty~code~label~\msg_line_context:.~Using~`#1'~instead. }
%    \end{macrocode}
%
%  \subsection{\LaTeXe{} package interface commands}
%


%  \begin{macro}{\NewHook,\NewReversedHook,\NewMirroredHookPair}
%    Declaring new hooks \ldots
%    \begin{macrocode}
\NewDocumentCommand \NewHook             { m }{ \hook_new:n {#1} }
\NewDocumentCommand \NewReversedHook     { m }{ \hook_new_reversed:n {#1} }
\NewDocumentCommand \NewMirroredHookPair { mm }{ \hook_new_pair:nn {#1}{#2} }
%    \end{macrocode}
%  \end{macro}

%  \begin{macro}{\AddToHook}
%    
%    \begin{macrocode}
\NewDocumentCommand \AddToHook { m o +m }
  {
    \clist_if_in:NnTF \g_@@_execute_immediately_clist {#1}
      {#3}
      { \hook_gput_code:nnn {#1} {#2} {#3} }
  }
%    \end{macrocode}
%  \end{macro}

%  \begin{macro}{\AddToHookNext}
%    
%    \begin{macrocode}
\NewDocumentCommand \AddToHookNext { m +m }
  { \hook_gput_next_code:nn {#1} {#2} }
%    \end{macrocode}
%  \end{macro}


%  \begin{macro}{\RemoveFromHook}
%    
%    \begin{macrocode}
\NewDocumentCommand \RemoveFromHook { m o }
  { \hook_gremove_code:nn {#1} {#2} }
%    \end{macrocode}
%  \end{macro}
%
% \begin{macro}{\DeclareDefaultHookLabel}
% \begin{macro}{\@@_curr_name_push:n,\@@_curr_name_pop:}
%   The token list \cs{g_@@_hook_curr_name_tl} stores the name of the
%   current package/file to be used as label for hooks.
%   Providing a consistent interface is tricky, because packages can
%   be loaded within packages, and some packages may not use
%   \cs{DeclareDefaultHookLabel} to change the default label (in which case
%   \cs{@currname} is used, if set).
%
%   To pull that off, we keep a stack that contains the default label
%   for each level of input.  The bottom of the stack contains the
%   default label for the top-level.  Since the string \verb|top-level|
%   is hardcoded, here this item of the stack is empty.  Also, since
%   we're in an input level, add \verb|lthooks| to the stack as well.
%   This stack should never go empty, so we loop through \LaTeXe's
%   file name stack, and add empty entries to \cs{g_@@_name_stack_seq}
%   for each item in that stack.  The last item is the \verb|top-level|,
%   which also gets an empty entry.
%
%   Also check for the case we're loading \texttt{lthooks} in the
%   \LaTeXe{} kernel.  In that case, \cs{@currname} isn't \verb|lthooks|
%   and just the top-level is added to the stack as an empty entry.
%    \begin{macrocode}
\str_if_eq:VnTF \@currname { lthooks }
  {
    \seq_gpush:Nn \g_@@_name_stack_seq { lthooks }
    \cs_set_protected:Npn \@@_tmp:w #1 #2 #3
      {
        \quark_if_recursion_tail_stop:n {#1}
        \seq_gput_right:Nn \g_@@_name_stack_seq { }
        \@@_tmp:w
      }
    \exp_after:wN \@@_tmp:w
      \@currnamestack
      \q_recursion_tail \q_recursion_tail
      \q_recursion_tail \q_recursion_stop
  }
  { \seq_gpush:Nn \g_@@_name_stack_seq { } }
%    \end{macrocode}
%
%   Two commands keep track of the stack: when a file is input,
%   \cs{@@_curr_name_push:n} pushes an (empty by default) label to the
%   stack:
%    \begin{macrocode}
\cs_new_protected:Npn \@@_curr_name_push:n #1
  {
    \seq_gpush:Nn \g_@@_name_stack_seq {#1}
    \tl_gset:Nn \g_@@_hook_curr_name_tl {#1}
  }
%
%    \end{macrocode}
%   and when an input is over, the topmost item of the stack is popped,
%   since the label will not be used again, and \cs{g_@@_hook_curr_name_tl}
%   is updated to the now topmost item of the stack:
%    \begin{macrocode}
\cs_new_protected:Npn \@@_curr_name_pop:
  {
    \seq_gpop:NN \g_@@_name_stack_seq \l_@@_return_tl
    \seq_get:NNTF \g_@@_name_stack_seq \l_@@_return_tl
      { \tl_gset_eq:NN \g_@@_hook_curr_name_tl \l_@@_return_tl }
      { \ERROR_should_not_happen }
  }
%    \end{macrocode}
%
%   The token list \cs{g_@@_hook_curr_name_tl} is but a mirror of the top
%   of the stack.
%
%   Now define a wrapper that replaces the top of the stack with the
%   argument, and updates \cs{g_@@_hook_curr_name_tl} accordingly.
%    \begin{macrocode}
\NewDocumentCommand \DeclareDefaultHookLabel { m }
  {
    \seq_gpop:NN \g_@@_name_stack_seq \l_@@_return_tl
    \@@_curr_name_push:n {#1}
  }
%    \begin{macrocode}
%
%   The push and pop macros are injected in \cs{@pushfilename} and
%   \cs{@popfilename} so that they correctly keep track of the label.s
%    \begin{macrocode}
% TODO! \pho{Properly integrate in the kernel}
\tl_gput_left:Nn \@pushfilename { \@@_curr_name_push:n { } }
\tl_gput_left:Nn \@popfilename { \@@_curr_name_pop: }
% TODO! \pho{Properly integrate in the kernel}
%    \end{macrocode}
% \end{macro}
% \end{macro}
%
%
%
%
%  \begin{macro}{\UseHook}
%    Avoid the overhead of \pkg{xparse} and its protection that we
%    don't want here (since the hook should vanish without trace if empty)!
%    \begin{macrocode}
\newcommand \UseHook { \hook_use:n }
%    \end{macrocode}
%  \end{macro}
%
%  \begin{macro}{\UseOneTimeHook}
%
%    \begin{macrocode}
\cs_new_protected:Npn \UseOneTimeHook { \hook_use_once:n }
%    \end{macrocode}
%  \end{macro}



%  \begin{macro}{\ShowHook}
%    
%    \begin{macrocode}
\cs_new_protected:Npn \ShowHook { \hook_log:n }
%    \end{macrocode}
%  \end{macro}

%  \begin{macro}{\DebugHookOn,\DebugHookOff}
%    
%    \begin{macrocode}
\cs_new_protected:Npn \DebugHookOn { \hook_debug_on: }
\cs_new_protected:Npn \DebugHookOff { \hook_debug_off: }
%    \end{macrocode}
%  \end{macro}



%  \begin{macro}{\DeclareHookRule}
%    
%    \begin{macrocode}
\NewDocumentCommand \DeclareHookRule { m m m m }
{ \hook_gset_rule:nnnn {#1}{#2}{#3}{#4} }
%    \end{macrocode}
%  \end{macro}

%  \begin{macro}{\DeclareDefaultHookRule}
%    This declaration is only supported before \verb=\begin{document}=.
%    \begin{macrocode}
\NewDocumentCommand \DeclareDefaultHookRule { m m m }
                    { \hook_gset_rule:nnnn {??}{#1}{#2}{#3} }
\@onlypreamble\DeclareDefaultHookRule                    
%    \end{macrocode}
%  \end{macro}
%
%  \begin{macro}{\ClearHookRule}
%    A special setup rule that removes an existing relation.
%    Basically {@@_rule_gclear:nnn} plus fixing the property list for debugging.
%    \fmi{Need an L3 interface, or maybe it should get dropped}
%    \begin{macrocode}
\NewDocumentCommand \ClearHookRule { m m m }
{ \hook_gset_rule:nnnn {#1}{#2}{unrelated}{#3} }
%    \end{macrocode}
%  \end{macro}
%
%
% \begin{macro}{\IfHookExistTF,\IfHookEmptyTF}
%    \begin{macrocode}
\NewExpandableDocumentCommand \IfHookExistTF { m }
  { \hook_if_exist:nTF {#1} }
\NewExpandableDocumentCommand \IfHookEmptyTF { m }
  { \hook_if_empty:nTF {#1} }
%    \end{macrocode}
% \end{macro}




%  \begin{macro}{\AtBeginDocument}
%    
%    \begin{macrocode}
\renewcommand\AtBeginDocument{\AddToHook{begindocument}}
%    \end{macrocode}
%  \end{macro}

%  \begin{macro}{\AtEndDocument}
%    
%    \begin{macrocode}
\renewcommand\AtEndDocument {\AddToHook{enddocument}}
%\renewcommand\AtEndDocument {\AddToHook{env/document/end}} % alternative impl
%    \end{macrocode}
%    
%  \end{macro}



%  \subsection{Set up existing \LaTeXe{} hooks}
%
%    As we are in a package calling \cs{NewHook} would label any
%    already set up hook code under the package name, but we want it
%    under the name \hook{top-level} so we pretend that \cs{@currname}
%    is empty.
%    \begin{macrocode}
\begingroup
  \def\@currname{}
%    \end{macrocode}
%    
%    \begin{macrocode}
  \NewHook{begindocument}
  \NewHook{enddocument}
%    \end{macrocode}
%    We need to initialize the mechanism at \verb=\begin{document}=
%    but obviously before everything else, so we sneak\footnote{This
%    needs to move to \cs{document} directly.}
%    \cs{@@_initialize_all:} into the \LaTeXe{} hook name.
%
%    We can't use \cs{tl_gput_left:Nn} because that complains about
%    \cs{@begindocumenthook} not starting with \cs{g_} so we do this
%    through the backdoor.
%    \begin{macrocode}
%  \tex_global:D\tl_put_left:Nn \@begindocumenthook
%      {\@@_initialize_all:}
%    \end{macrocode}
%    There aren't many other hooks at the moment:
%    \begin{macrocode}
  \NewHook{rmfamily}
  \NewHook{sffamily}
  \NewHook{ttfamily}
  \NewHook{defaultfamily}
%    \end{macrocode}
%    Not checked what this one does and whether it should be there (or
%    is a real ``hook''.
%    \begin{macrocode}
  \NewHook{documentclass}
%    \end{macrocode}
%    
%    \begin{macrocode}
\endgroup
%    \end{macrocode}
%
%
%
% \section{Generic hooks for environments}
%
%
%    \begin{macrocode}
\let\begin\relax  % avoid redeclaration message
%    \end{macrocode}
%    
%    \begin{macrocode}
\DeclareRobustCommand\begin[1]{%
  \UseHook{env/#1/before}%
  \@ifundefined{#1}%
    {\def\reserved@a{\@latex@error{Environment~#1~undefined}\@eha}}%
    {\def\reserved@a{\def\@currenvir{#1}%
        \edef\@currenvline{\on@line}%
        \@execute@begin@hook{#1}%
        \csname #1\endcsname}}%
  \@ignorefalse
  \begingroup\@endpefalse\reserved@a}
%    \end{macrocode}
%
%    Before the \cs{document} code is executed we have to first undo
%    the \cs{endgroup} as there should be none for this environment to
%    avoid that changes on top-level unnecessarily go to \TeX's
%    savestack, and we have to initialize all hooks in the hook system.
%    So we need to test for this environment name. But once it has be
%    found all this testing is no longer needed and so we redefine
%    \cs{@execute@begin@hook} to simply use the hook
%    \begin{macrocode}
\def\@execute@begin@hook #1{%
  \expandafter\ifx\csname #1\endcsname\document
    \endgroup
    \gdef\@execute@begin@hook##1{\UseHook{env/##1/begin}}%
    \@@_initialize_all:
    \@execute@begin@hook{#1}%
%    \end{macrocode}
%    If this is an environment before \verb=\begin{document}= we just
%    run the hook.
%    \begin{macrocode}
  \else
    \UseHook{env/#1/begin}%
  \fi
}    
%    \end{macrocode}
%    
%    \begin{macrocode}
\@namedef{end~}#1{%
  \UseHook{env/#1/end}%
  \csname end#1\endcsname\@checkend{#1}%
  \expandafter\endgroup\if@endpe\@doendpe\fi
  \UseHook{env/#1/after}%
  \if@ignore\@ignorefalse\ignorespaces\fi}%
%    \end{macrocode}
%    Version that fixes tlb3722 but the change should perhaps be made in
%    \pkg{tabularx} instead.
%    \begin{macrocode}
\@namedef{end~}#1{%
\romannumeral
\IfHookEmptyTF{env/#1/end}%
  {\expandafter\z@}%
  {\z@\UseHook{env/#1/end}}%
\csname end#1\endcsname\@checkend{#1}%
\expandafter\endgroup\if@endpe\@doendpe\fi
\UseHook{env/#1/after}%
\if@ignore\@ignorefalse\ignorespaces\fi}%
%    \end{macrocode}
%    
%    
%
%    We provide 4 high-level hook interfaces directly, the others only when
%    etoolbox is loaded
%    \begin{macrocode}
\newcommand\AtBeginEnvironment[1]    {\AddToHook{env/#1/begin}}
\newcommand\AtEndEnvironment[1]      {\AddToHook{env/#1/end}}
\newcommand\BeforeBeginEnvironment[1]{\AddToHook{env/#1/before}}
\newcommand\AfterEndEnvironment[1]   {\AddToHook{env/#1/after}}
%    \end{macrocode}
%    
%    
%    
%    
%    
%    
%    
%    
%    
%    
%
% \section{Generic hooks for file loads}
%
%
%
%
%
%    
%
% \section{Hooks in \cs{begin} document}
%
%    Can't have \texttt{@{}@} notation here as this is \LaTeXe{} code
%    \ldots{} and makes for puzzling errors if the double \texttt{@}
%    signs get substituted.
%    \begin{macrocode}
%<@@=>
\ExplSyntaxOff
%    \end{macrocode}
%
%    The \hook{begindocument} hook was already set up earlier, here is now
%    the additional one (which was originally from the \pkg{etoolbox}
%    package under the name \texttt{afterpreamble}.
%    \begin{macrocode}
\NewHook{begindocument/end}
%    \end{macrocode}
%
%

%  \begin{macro}{\document}
%    
%    \begin{macrocode}
\def\document{%
%    \end{macrocode}
%    We do cancel the grouping as part of the \cs{begin} handling
%    (this is now done inside \cs{begin} instead) so that the
%    \hook{env/\meta{env}/begin} hook is not hidden inside \cs{begingroup}
%    \texttt{...} \cs{endgroup}.
%    \begin{macrocode}
%  \endgroup
%    \end{macrocode}
%    
%    \begin{macrocode}
  \@kernel@after@env@document@begin
%    \end{macrocode}
%
% Added hook to load \textsf{l3backend} code:
%    \begin{macrocode}
  \@expl@sys@load@backend@@
  \ifx\@unusedoptionlist\@empty\else
    \@latex@warning@no@line{Unused global option(s):^^J%
            \@spaces[\@unusedoptionlist]}%
  \fi
  \@colht\textheight
  \@colroom\textheight \vsize\textheight
  \columnwidth\textwidth
  \@clubpenalty\clubpenalty
  \if@twocolumn
    \advance\columnwidth -\columnsep
    \divide\columnwidth\tw@ \hsize\columnwidth \@firstcolumntrue
  \fi
  \hsize\columnwidth \linewidth\hsize
  \begingroup\@floatplacement\@dblfloatplacement
    \makeatletter\let\@writefile\@gobbletwo
    \global \let \@multiplelabels \relax
    \@input{\jobname.aux}%
  \endgroup
  \if@filesw
    \immediate\openout\@mainaux\jobname.aux
    \immediate\write\@mainaux{\relax}%
  \fi
  \process@table
  \let\glb@currsize\@empty  % Force math initialization.
  \normalsize
  \everypar{}%
  \ifx\normalsfcodes\@empty
    \ifnum\sfcode`\.=\@m
      \let\normalsfcodes\frenchspacing
    \else
      \let\normalsfcodes\nonfrenchspacing
    \fi
  \fi
  \ifx\document@default@language\m@ne
    \chardef\document@default@language\language
  \fi
  \@noskipsecfalse
  \let \@refundefined \relax
%    \end{macrocode}
%    
%    \begin{macrocode}
%  \let\AtBeginDocument\@firstofone
%  \@begindocumenthook
  \UseOneTimeHook{begindocument}%
  \@kernel@after@begindocument
%    \end{macrocode}
%    
%    \begin{macrocode}
  \ifdim\topskip<1sp\global\topskip 1sp\relax\fi
  \global\@maxdepth\maxdepth
  \global\let\@begindocumenthook\@undefined
  \ifx\@listfiles\@undefined
    \global\let\@filelist\relax
    \global\let\@addtofilelist\@gobble
  \fi
  \gdef\do##1{\global\let ##1\@notprerr}%
  \@preamblecmds
  \global\let \@nodocument \relax
  \global\let\do\noexpand
%    \end{macrocode}
%    
%    \begin{macrocode}
  \UseOneTimeHook{begindocument/end}%
  \ignorespaces}
%    \end{macrocode}
%    
%    \begin{macrocode}
\let\@kernel@after@begindocument\@empty
%    \end{macrocode}
%  \end{macro}
%
%
%  \begin{macro}{\@kernel@after@env@document@begin,\@kernel@hook@begindocument}
%    
%    \begin{macrocode}
\edef \@kernel@after@env@document@begin{%
  \let\expandafter\noexpand\csname
       g__hook_env/document/begin_code_tl\endcsname
  \noexpand\@empty}
%    \end{macrocode}
%    \begin{macrocode}
\let\@kernel@hook@begindocument\@empty
%    \end{macrocode}
%    
%  \end{macro}
%
%
% \section{Hooks in \cs{enddocument}}
%
%
%    
%    The \hook{enddocument} hook was already set up earlier, here are now
%    the additional ones:
%    \begin{macrocode}
\NewHook{enddocument/afterlastpage}
\NewHook{enddocument/afteraux}
\NewHook{enddocument/info}
\NewHook{enddocument/end}
%    \end{macrocode}



%  \begin{macro}{\enddocument}
%    
%    
%    \begin{macrocode}
\def\enddocument{%
   \UseHook{enddocument}%
   \@kernel@after@enddocument
   \@checkend{document}%
   \clearpage
   \UseHook{enddocument/afterlastpage}%
   \@kernel@after@enddocument@afterlastpage
   \begingroup
     \if@filesw
       \immediate\closeout\@mainaux
       \let\@setckpt\@gobbletwo
       \let\@newl@bel\@testdef
       \@tempswafalse
       \makeatletter \@@input\jobname.aux
     \fi
     \UseHook{enddocument/afteraux}%
%    \end{macrocode}
%    Next hook is expect to contain only code for writing info
%    messages on the terminal.
%    \begin{macrocode}
     \UseHook{enddocument/info}%
   \endgroup
   \UseHook{enddocument/end}%
   \deadcycles\z@\@@end}
%    \end{macrocode}
%    The two kernel hooks above are used by the shipout code.   
%    \begin{macrocode}
\let\@kernel@after@enddocument\@empty
\let\@kernel@after@enddocument@afterlastpage\@empty
%    \end{macrocode}
%  \end{macro}
%
%
%
%  \begin{macro}{\@enddocument@kernel@warnings}
%    
%    \begin{macrocode}
\def\@enddocument@kernel@warnings{%
   \ifdim \font@submax >\fontsubfuzz\relax
     \@font@warning{Size substitutions with differences\MessageBreak
                up to \font@submax\space have occurred.\@gobbletwo}%
   \fi
   \@defaultsubs
   \@refundefined
   \if@filesw
     \ifx \@multiplelabels \relax
       \if@tempswa
         \@latex@warning@no@line{Label(s) may have changed.
             Rerun to get cross-references right}%
       \fi
     \else
       \@multiplelabels
     \fi
   \fi
}
%    \end{macrocode}
%    
%    \begin{macrocode}
\AddToHook{enddocument/info}[kernel/filelist]{\@dofilelist}
\AddToHook{enddocument/info}[kernel/warnings]{\@enddocument@kernel@warnings}
\DeclareHookRule{enddocument/info}{kernel/filelist}{before}{kernel/warnings}
%    \end{macrocode}
%  \end{macro}
%
%
% \subsection{Adjusting at \pkg{atveryend} interfaces}
%
%    With the new hook management all of \pkg{atveryend} is taken care
%    of.
%
%    We therefore prevent the package from loading:
%    \begin{macrocode}
\expandafter\let\csname ver@atveryend.sty\endcsname\fmtversion
%    \end{macrocode}
%
%
%    Here are new definitions for its interfaces now pointing to the
%    hooks in \cs{enddocument}
%    \begin{macrocode}
\newcommand\AfterLastShipout  {\AddToHook{enddocument/afterlastpage}}
\newcommand\AtVeryEndDocument {\AddToHook{enddocument/afteraux}}
%    \end{macrocode}
%    Next one is a bit of a fake, but the result should normally be as
%    expected. If not one needs to add a rule to sort the code chunks
%    in \hook{enddocument/info}.
%    \begin{macrocode}
\newcommand\AtEndAfterFileList{\AddToHook{enddocument/info}}
%    \end{macrocode}
%    
%    \begin{macrocode}
\newcommand\AtVeryVeryEnd     {\AddToHook{enddocument/end}}
%    \end{macrocode}

%  \begin{macro}{\BeforeClearDocument}
%    This one is the only one we don't implement or rather don't have
%    a dedicated hook in the code.
%    \begin{macrocode}
\ExplSyntaxOn
\newcommand\BeforeClearDocument[1]
  { \AtEndDocument{#1}
    \@DEPRECATED{BeforeClearDocument \tl_to_str:n{#1}}
  }
\cs_new:Npn\@DEPRECATED #1
   {\iow_term:x{======~DEPRECATED~USAGE~#1~==========}}
%    \end{macrocode}
%    
%    \begin{macrocode}
\ExplSyntaxOff
%    \end{macrocode}
%  \end{macro}
%
%    \begin{macrocode}
%</2ekernel>
%    \end{macrocode}

%
% \section{A package version of the code for testing}
%

%    \begin{macrocode}
%<*package>
%    \end{macrocode}



%    \begin{macrocode}
\RequirePackage{xparse}
\ProvidesExplPackage{lthooks}{\lthooksdate}{\lthooksversion}
                    {Hook management interface for LaTeX2e}
%    \end{macrocode}
%
%
% \subsection{Core hook management code (kernel part)}
%
%    This should run in older formats so we can't use
%    \cs{IfFormatAtLeastTF} right now.
%    \begin{macrocode}
\@ifl@t@r\fmtversion{2020/10/01}
                    {}
                    {% \iffalse meta-comment
%
%% File: lthooks.dtx (C) Copyright 2020 Frank Mittelbach, LaTeX Team
%
% It may be distributed and/or modified under the conditions of the
% LaTeX Project Public License (LPPL), either version 1.3c of this
% license or (at your option) any later version.  The latest version
% of this license is in the file
%
%    https://www.latex-project.org/lppl.txt
%
%
% The development version of the bundle can be found below
%
%    https://github.com/FrankMittelbach/...
%
% for those people who are interested or want to report an issue.
%
%    \begin{macrocode}
\providecommand\lthooksversion{v0.9a}
\providecommand\lthooksdate{2020/07/15}
%    \end{macrocode}
%
%<*driver>
\RequirePackage[debug]{lthooks}

\documentclass{l3doc}

% bug fix fo l3doc.cls
\ExplSyntaxOn
\cs_set_protected:Npn \__codedoc_macro_typeset_one:nN #1#2
  {
    \vbox_set:Nn \l__codedoc_macro_box
      {
        \vbox_unpack_drop:N \l__codedoc_macro_box
        \hbox { \llap { \__codedoc_print_macroname:nN {#1} #2
            \MacroFont       % <----- without it the \ is in lmr10 if a link is made
            \      
        } }
      }
    \int_incr:N \l__codedoc_macro_int
  }
\ExplSyntaxOff

\EnableCrossrefs
\CodelineIndex
\begin{document}
  \DocInput{lthooks.dtx}
\end{document}
%</driver>
%
% \fi
%
%
% \long\def\fmi#1{\begin{quote}\itshape FMi: #1\end{quote}}
% \long\def\pho#1{\begin{quote}\itshape PhO: #1\end{quote}}
%
% \newcommand\hook[1]{\texttt{#1}}
%    
%
% \title{The \texttt{lthooks} package\thanks{This package has version
%    \lthooksversion\ dated \lthooksdate, \copyright\ \LaTeX\
%    Project.}}
%
% \author{Frank Mittelbach\thanks{Code improvements for speed and other goodies by Phelype Oleinik}}
%
% \maketitle
%
%
% \tableofcontents
%
% \section{Introduction}
%
%    Hooks are points in the code of commands or environments where it
%    is possible to add processing code into existing commands. This
%    can be done by different packages that do not know about each
%    other and to allow for hopefully safe processing it is necessary
%    to sort different chunks of code added by different packages into
%    a suitable processing order.
%
%    This is done by the packages adding chunks of code (via
%    \cs{AddToHook}) and labeling their code with some label by
%    default using the package name as a label.
%
%    At \verb=\begin{document}= all code for a hook is then sorted
%    according to some rules (given by \cs{DeclareHookRule}) for fast
%    execution without processing overhead. If the hook code is
%    modified afterwards (or the rules are changed),
%    a new version for fast processing is generated.
%
%    Some hooks are used already in the preamble of the document. If
%    that happens then the hook is prepared for execution (and sorted)
%    already at that point.
%
%
% \section{Package writer interface}
%
%    The hook management system is offered as a set of CamelCase
%    commands for traditional \LaTeXe{} packages (and for use in the
%    document preamble if needed) as well as \texttt{expl3} commands
%    for modern packages, that use the L3 programming layer of
%    \LaTeX{}. Behind the scenes, a single set of data structures is
%    accessed so that packages from both worlds can coexist and access
%    hooks in other packages.
%
%
%
% \subsection{\LaTeXe\ interfaces}
%
% \subsubsection{Declaring hooks and using them in code}
%
%    With two exceptions, hooks have to be declared before they can be
%    used. The exceptions are hooks in environments (i.e., executed at
%    \cs{begin} and \cs{end}) and hooks run when loading files,
%    e.g. before and after a package is loaded, etc. Their hook names
%    depend on the environment or the file name and so declaring them
%    beforehand is difficult.
%
%
% \begin{function}{\NewHook}
%   \begin{syntax}
%     \cs{NewHook} \Arg{hook}
%   \end{syntax}
%   Creates a new \meta{hook}.
%    If this is a hook provided as part of a package it is suggested
%    that the \meta{hook} name is always structured as follows:
%    \meta{package-name}\texttt{/}\meta{hook-name}. If necessary you
%    can further subdivide the name by adding more \texttt{/} parts.
%    If a hook name is already taken, an error is raised and the hook
%    is not created.
%
%    The \meta{hook} can be specified using the dot-syntax to denote
%    the current package name. See section~\ref{sec:default-label}.
% \end{function}
%
% \begin{function}{\NewReversedHook}
%   \begin{syntax}
%     \cs{NewReversedHook} \Arg{hook}
%   \end{syntax}
%     Like \cs{NewHook} declares a new \meta{hook}.
%     the difference is that the code chunks for this hook are in
%     reverse order by default (those added last are executed first).
%     Any rules for the hook are applied after the default ordering.
%     See sections~\ref{sec:order} and \ref{sec:reversed-order}
%    for further details.
%
%    The \meta{hook} can be specified using the dot-syntax to denote
%    the current package name. See section~\ref{sec:default-label}.
% \end{function}
%
%
% \begin{function}{\NewMirroredHookPair}
%   \begin{syntax}
%     \cs{NewMirroredHookPair} \Arg{hook-1} \Arg{hook-2}
%   \end{syntax}
%     A shorthand for
%    \cs{NewHook}\Arg{hook-1}\cs{NewReversedHook}\Arg{hook-2}.
%
%    The \meta{hooks} can be specified using the dot-syntax to denote
%    the current package name. See section~\ref{sec:default-label}.
% \end{function}
%



% \begin{function}{\UseHook}
%   \begin{syntax}
%     \cs{UseHook} \Arg{hook}
%   \end{syntax}
%    Execute the hook code inside a command or environment.\footnote{For
%     legacy hooks such as \hook{begindocument} it is also
%    possible to call \cs{@...hook}, e.g., \cs{@begindocumenthook},
%    but this syntax is discouraged.}
%
%    Before \verb=\begin{document}= the fast execution code for a hook
%    is not set up, so in order to use a hook there it is explicitly
%    initialized first. As that involves assignments using a hook at
%    those times is not 100\% the same as using it after
%    \verb=\begin{document}=.
%
%    The \meta{hook} \emph{cannot} be specified using the dot-syntax.
%    A leading |.| is treated literally.
% \end{function}
%
% \begin{function}{\UseOneTimeHook}
%   \begin{syntax}
%     \cs{UseOneTimeHook} \Arg{hook}
%   \end{syntax}
%    Some hooks are only used (and can be only used) in one place, for
%    example, those in \verb=\begin{document}= or
%    \verb=\end{document}=. Once we have passed that point adding to
%    the hook through a defined \cs{\meta{addto-cmd}} command (e.g.,
%    \cs{AddToHook} or \cs{AtBeginDocument}, etc.\@) would have no
%    effect (as would the use of such a command inside the hook code
%    itself). It is therefore customary to redefine
%    \cs{\meta{addto-cmd}} to simply  process its argument, i.e.,
%    essentially make it behave like \cs{@firstofone}.
%
%    \cs{UseOneTimeHook} does that: it records that the hook has been
%    consumed and any further attempt to add to it will result in
%    executing the code to be added immediately.
%
%    \fmi{Maybe add an error version as well?}
%
%    The \meta{hook} \emph{cannot} be specified using the dot-syntax.
%    A leading |.| is treated literally.
% \end{function}
%
%
% \subsubsection{Updating code for hooks}
%
% \begin{function}{\AddToHook}
%   \begin{syntax}
%     \cs{AddToHook} \Arg{hook}\oarg{label}\Arg{code}
%   \end{syntax}
%    Adds \meta{code} to the \meta{hook} labeled by \meta{label}. If
%    the optional argument \meta{label} is not provided, if \cs{AddToHook}
%    is used in a package/class, then the current
%    package/class name is used, otherwise \hook{top-level} is
%    used~(see section~\ref{sec:default-label}).
%
%    If there already exists code under the \meta{label} then the new
%    \meta{code} is appended to the existing one (even if this is a reversed hook).
%    If you want to replace existing code under the
%    \meta{label}, first apply \cs{RemoveFromHook}.
%
%    The hook doesn't have to exist for code to be added to
%    it. However, if it is not declared later then obviously the
%    added \meta{code} will never be executed.  This
%    allows for hooks to work regardless of package loading order and
%    enables packages to add to hook of other packages without
%    worrying whether they are actually used in the current document.
%    See section~\ref{sec:querying}.
%
%    The \meta{hook} and \meta{label} can be specified using the
%    dot-syntax to denote the current package name.
%    See section~\ref{sec:default-label}.
% \end{function}
%
% \begin{function}{\RemoveFromHook}
%   \begin{syntax}
%     \cs{RemoveFromHook} \Arg{hook}\oarg{label}
%   \end{syntax}
%    Removes any code labeled by \meta{label} from the \meta{hook}.
%    If the optional
%    argument \meta{label} is not provided, if \cs{AddToHook}
%    is used in a package/class, then the current
%    package/class name is used, otherwise \hook{top-level} is used.
%
%    If the optional argument is \texttt{*}, then all code chunks are
%    removed. This is rather dangerous as it drops code from other
%    packages one may not know about!
%
%    The \meta{hook} and \meta{label} can be specified using the
%    dot-syntax to denote the current package name.
%    See section~\ref{sec:default-label}.
% \end{function}
%
% \medskip
%
% A useful application for this declaration inside the document body
% is when one wants to temporarily add code to hooks and later remove
% it again, e.g.,
%\begin{verbatim}
%   \AddToHook{env/quote/before}{\small}
%   \begin{quote}
%     A quote set in a smaller typeface
%   \end{quote}
%   ...
%   \RemoveFromHook{env/quote/before}
%   ... now back to normal for further quotes
%\end{verbatim}
% Note that you can't cancel the setting with
%\begin{verbatim}
%   \AddToHook{env/quote/before}{}
%\end{verbatim}
% because that only \enquote{adds} a further empty chunk of code to
% the hook. Adding \cs{normalsize} would work but that means the hook
% then contained \cs{small}\cs{normalsize} which means to font size
% changes for no good reason.
%
% The above is only needed if one wants to typeset several quotes in a
% smaller typeface. If the hook is only needed once then
% \cs{AddToHookNext} is simpler, because it resets itself after one use.
%
%
% \begin{function}{\AddToHookNext}
%   \begin{syntax}
%     \cs{AddToHookNext} \Arg{hook}\Arg{code}
%   \end{syntax}
%    Adds \meta{code} to the next invocation of the \meta{hook}.
%    The code is executed after the normal hook code has finished and
%    it is executed only once, i.e. it is deleted after it was used.
%
%    Using the declaration is a global operation, i.e., the code is
%    not lost, even if the declaration is used inside a group and the
%    next invocation happens after the group. If the declaration is
%    used several times before the hook is executed then all code is
%    executed in the order in which it was declared.\footnotemark
%
%    The hook doesn't have to exist for code to be added to it.  This
%    allows for hooks to work regardless of package loading order.
%    See section~\ref{sec:querying}.
%
%    The \meta{hook} can be specified using the dot-syntax to denote
%    the current package name.  See section~\ref{sec:default-label}.
% \end{function}\footnotetext{There is
%    no mechanism to reorder such code chunks (or delete them).}
%
% \subsubsection{Hook names and default labels}
% \label{sec:default-label}
%
% It is best practice to use \cs{AddToHook} in packages or classes
% \emph{without specifying a \meta{label}} because then the package
% or class name is automatically used, which is helpful if rules are
% needed, and avoids mistyping the \meta{label}.
%
% Using an explicit \meta{label} is only necessary in very specific
% situations, e.g., if you want to add several chunks of code into a
% single hook and have them placed in different parts of the hook
% (by providing some rules).
%
% The other case is when you develop a larger package with several
% sub-packages. In that case you may want to use the same
% \meta{label} throughout the sub-packages in order to avoid
% that the labels change if you internally reorganize your code.
%
% It is not enforced, but highly recommended that the hooks defined by
% a package, and the \meta{labels} used to add code to other hooks
% contain the package name to easily identify the source of the code
% chunk and to prevent clashes.  This should be the standard practice,
% so this hook management code provides a shortcut to refer to the
% current package in the name of a \meta{hook} and in a \meta{label}.
% If \meta{hook} name or \meta{label} consist just of a single dot
% (|.|), or starts with a dot followed by a slash (|./|) then the dot
% denotes the \meta{default label} (usually the current package or class
% name---see~\cs{DeclareDefaultHookLabel}).
% A \enquote{|.|} or \enquote{|./|} anywhere else in a \meta{hook} or in
% \meta{label} is treated literally and is not replaced.

% For example,
% inside the package \texttt{mypackage.sty}, the default label is
% \texttt{mypackage}, so the instructions:
% \begin{verbatim}
%   \NewHook   {./hook}
%   \AddToHook {./hook}[.]{code}     % Same as \AddToHook{./hook}{code}
%   \AddToHook {./hook}[./sub]{code}
%   \DeclareHookRule{begindocument}{.}{<}{babel}
%   \AddToHook {file/after/foo.tex}{code}
% \end{verbatim}
%    are equivalent to:
% \begin{verbatim}
%   \NewHook   {mypackage/hook}[mypackage]{code}
%   \AddToHook {mypackage/hook}[mypackage]{code}
%   \AddToHook {mypackage/hook}[mypackage/sub]{code}
%   \DeclareHookRule{begindocument}{mypackage}{<}{babel}
%   \AddToHook {file/after/foo.tex}{code}                  % unchanged
% \end{verbatim}
%
% The \meta{default label} is automatically set to the name of the
% current package or class (using \cs{@currname}).  If \cs{@currname}
% is not set (because the hook command is used outside of a package, or
% the current file wasn't loaded with \cs{usepackage} or
% \cs{documentclass}), then the \texttt{top-level} is used as the
% \meta{default label}.
%
% This syntax is available in all \meta{label} arguments and most
% \meta{hook}, both in the \LaTeXe{} interface, and the \LaTeX3
% interface described in section~\ref{sec:l3hook-interface}.
%
% Note, however, that the replacement of |.| by the \meta{default label}
% takes place when the hook command is executed, so actions that are
% somehow executed after the package ends will have the wrong
% \meta{default label} if the dot-syntax is used.  For that reason,
% this syntax is not available in \cs{UseHook} (and \cs{hook_use:n})
% because the hook is most of the time used outside of the package file
% in which it was defined. This syntax is also not available in the hook
% conditionals \cs{IfHookEmptyTF} (and \cs{hook_if_empty:nTF}) and
% \cs{IfHookExistTF} (and \cs{hook_if_exist:nTF}) because these
% conditionals are used in some performance-critical parts of the hook
% management code, and because they are usually used to refer to other
% package's hooks, so the dot-syntax doesn't make much sense.
%
% In some cases, for example in large packages, one may want to separate
% it in logical parts, but still use the main package name as
% \meta{label}, then the \meta{default label} can be set using
% \cs{DeclareDefaultHookLabel}:
%
% \begin{function}{\DeclareDefaultHookLabel}
%   \begin{syntax}
%     \cs{DeclareDefaultHookLabel} \Arg{default label}
%   \end{syntax}
%   Sets the \meta{default label} to be used in \meta{label} arguments.
%   If \cs{DeclareDefaultHookLabel} is not used in the
%   current package, \cs{@currname} is used instead.  If \cs{@currname}
%   is not set, the code is assumed to be in the main document, in which
%   case \texttt{top-level} is used.
%
%   The effect of \cs{DeclareDefaultHookLabel} holds for the current
%   file, and is reset to the previous value when the file is closed.
% \end{function}
%
%
% \subsubsection{Defining relations between hook code}
%
% The default assumption is that code added to hooks by different
% packages is independent and the order in which it is executed is
% irrelevant. While this is true in many case it is  obviously false
% in many others.
%
% Before the hook management system was introduced
% packages had to take elaborate precaution to determine of some other
% package got loaded as well (before or after) and find some ways to
% alter its behavior accordingly. In addition is was often the user's
% responsibility to load packages in the right order so that code
% added to hooks got added in the right order and some cases even
% altering the loading order wouldn't resolve the conflicts.
%
% With the new hook management system it is now possible to define
% rules (i.e., relationships) between code chunks added by different
% packages and explicitly describe in which order they should be
% processed.
%
% \begin{function}{\DeclareHookRule}
%   \begin{syntax}
%     \cs{DeclareHookRule} \Arg{hook}\Arg{label1}\Arg{relation}\Arg{label2}
%   \end{syntax}
%    Defines a relation between \meta{label1} and \meta{label2} for a
%    given \meta{hook}. If \meta{hook} is \texttt{??} this defines a
%    relation for all hooks that use the two labels, i.e., that have
%    chunks of code labeled with \meta{label1} and \meta{label2}.
%    Rules specific to a given hook take precedence over default
%    rules that use \texttt{??} as the \meta{hook}.
%
%    Currently, the supported relations are the following:
%    \begin{itemize}
%
%    \item[\texttt{before} or \texttt{\string<}]
%
%      Code for \meta{label1} comes before code for \meta{label2}.
%
%    \item[\texttt{after} or \texttt{\string>}]
%      Code for \meta{label1} comes after code for \meta{label2}.
%
%    \item[\texttt{incompatible-warning}]
%
%      Only code for either \meta{label1} or \meta{label2} can appear
%      for that hook (a way to say that two packages---or parts of
%      them---are incompatible). A warning is raised if both labels
%      appear in the same hook.
%
%    \item[\texttt{incompatible-error}]
%
%      Like \texttt{incompatible-error} but instead of a warning a
%      \LaTeX{} error is raised, and the code for both labels are
%      dropped from that hook until the conflict is resolved.
%
%    \item[\texttt{removes}]
%
%      Code for \meta{label1} overwrites code for \meta{label2}. More
%      precisely, code for \meta{label2} is dropped for that
%      hook. This can be used, for example if one package is a
%      superset in functionality of another one and therefore wants to
%      undo code in some hook and replace it with its own version.
%
%    \item[\texttt{unrelated}]
%
%       The order of code for \meta{label1} and \meta{label2} is
%      irrelevant. This rule is there to undo an incorrect rule
%      specified earlier.
%
%    \end{itemize}
%
%    The \meta{hook} and \meta{label} can be specified using the
%    dot-syntax to denote the current package name.
%    See section~\ref{sec:default-label}.
% \end{function}
%
%
% \begin{function}{\ClearHookRule}
%   \begin{syntax}
%     \cs{ClearHookRule}\Arg{hook}\Arg{label1}\Arg{label2}
%   \end{syntax}
%    Syntactic sugar for saying that \meta{label1} and \meta{label2}
%    are unrelated for the given \meta{hook}.
% \end{function}
%
%
%
% \begin{function}{\DeclareDefaultHookRule}
%   \begin{syntax}
%     \cs{DeclareDefaultHookRule}\Arg{label1}\Arg{relation}\Arg{label2}
%   \end{syntax}
%   This sets up a relation between \meta{label1} and \meta{label2}
%    for all hooks unless overwritten by a specific rule for a hook.
%    Useful for cases where one package has a specific relation to
%    some other package, e.g., is \texttt{incompatible} or always
%    needs a special ordering \texttt{before} or \texttt{after}.
%    (Technically it is just a shorthand for using \cs{DeclareHookRule}
%    with \texttt{??} as the hook name.)
%
%    Declaring default rules is only supported in the document
%    preamble.\footnotemark{}
%
%    The \meta{label} can be specified using the dot-syntax to denote
%    the current package name. See section~\ref{sec:default-label}.
% \end{function}\footnotetext{Trying to do so, e.g., via
%    \cs{DeclareHookRule} with \texttt{??}  has bad side-effects and
%    is not supported (though not explicitly caught for performance
%    reasons).}
%
%
%
% \subsubsection{Querying hooks}
% \label{sec:querying}
%
% Simpler data types, like token lists, have three possible states; they
% can:
% \begin{itemize}
%   \item exist and be empty;
%   \item exist and be non-empty; and
%   \item not exist (in which case emptiness doesn't apply);
% \end{itemize}
% Hooks are a bit more complicated: they have four possible states.
% A hook may exist or not, and either way it may or may not be empty.
% This means that even a hook that doesn't exist may be non-empty.
%
% This seemingly strange state may happen when, for example, package~$A$
% defines hook \hook{A/foo}, and package $B$ adds some code to that
% hook.  However, a document may load package $B$ before package $A$, or
% may not load package $A$ at all.  In both cases some code is added to
% hook \hook{A/foo} without that hook being defined yet, thus that
% hook is said to be non-empty, whereas it doesn't exist.  Therefore,
% querying the existence of a hook doesn't imply its emptiness, neither
% does the other way around.
%
% A hook is said to be empty when no code was added to it, either to
% its permanent code pool, or to its ``next'' token list.  The hook
% doesn't need to be declared to have code added to its code pool.
% A hook is said to exist when it was declared with \cs{NewHook} or
% some variant thereof.
%
% \begin{function}[EXP]{\IfHookEmptyTF}
%   \begin{syntax}
%     \cs{IfHookEmptyTF} \Arg{hook} \Arg{true code} \Arg{false code}
%   \end{syntax}
%   Tests if the \meta{hook} is empty (\emph{i.e.}, no code was added to
%   it using either \cs{AddToHook} or \cs{AddToHookNext}), and
%   branches to either \meta{true code} or \meta{false code} depending
%   on the result.
%
%    The \meta{hook} \emph{cannot} be specified using the dot-syntax.
%    A leading |.| is treated literally.
% \end{function}
%
% \begin{function}[EXP]{\IfHookExistTF}
%   \begin{syntax}
%     \cs{IfHookExistTF} \Arg{hook} \Arg{true code} \Arg{false code}
%   \end{syntax}
%   Tests if the \meta{hook} exists (if it was created with either
%   \cs{NewHook}, \cs{NewReversedHook}, or \cs{NewMirroredHookPair}), and
%   branches to either \meta{true code} or \meta{false code} depending
%   on the result.
%
%   The existence of a hook usually doesn't mean much from the viewpoint
%   of code that tries to add/remove code from that hook, since package
%   loading order may vary, thus the creation of hooks is asynchronous
%   to adding and removing code from it, so this test should be used
%   sparingly.
%
%    The \meta{hook} \emph{cannot} be specified using the dot-syntax.
%    A leading |.| is treated literally.
% \end{function}
%
% \fmi{Would be helpful if we provide some use cases}
%
% \subsubsection{Displaying hook code}
%
%    If one has to adjust the code execution in a hook using a hook
%    rule it is helpful to get some information about the code
%    associated with a hook, its current order and the existing rules.
%
% \begin{function}{\ShowHook}
%   \begin{syntax}
%     \cs{ShowHook} \Arg{hook}
%   \end{syntax}
%   Displays information about the \meta{hook} such as
%   \begin{itemize}
%   \item
%      the code chunks (and their labels) added to it,
%   \item
%      any rules set up to order them,
%      \fmi{currently this is missing the default rules that apply,
%      guess that needs fixing}
%   \item
%      the computed order (if already defined),
%   \item
%      any code executed on the next invocation only.
%   \end{itemize}
%
%    The \meta{hook} can be specified using the dot-syntax to denote
%    the current package name. See section~\ref{sec:default-label}.
% \end{function}
%
%
% \subsubsection{Debugging hook code}
%
% \begin{function}{\DebugHookOn,\DebugHookOff}
%   \begin{syntax}
%     \cs{DebugHookOn}
%   \end{syntax}
%    Turn the debugging of hook code on or off. This displays changes
%    made to the hook data structures. The output is rather coarse and
%      not really intended for normal use.
% \end{function}
%
%
% \subsection{L3 programming layer (\texttt{expl3}) interfaces}
% \label{sec:l3hook-interface}
%
%
% This is a quick summary of the \LaTeX3 programming interfaces for
% use with packages written in \texttt{expl3}. In contrast to the
% \LaTeXe{} interfaces they always use mandatory arguments only, e.g.,
% you always have to specify the \meta{label} for a code chunk.  We
% therefore suggest to use the declarations discussed in the previous
% section even in \texttt{expl3} packages, but the choice is yours.
%
%
% \begin{function}
%   {\hook_new:n,\hook_new_reversed:n,\hook_new_pair:nn}
%   \begin{syntax}
%     \cs{hook_new:n}\Arg{hook}
%     \cs{hook_new_pair:nn}\Arg{hook-1}\Arg{hook-2}
%   \end{syntax}
%   Creates a new \meta{hook} with normal or reverse ordering of code
%    chunks. \cs{hook_new_pair:nn} creates a pair of such hooks with
%    \Arg{hook-2} being a reversed hook.
%    If a hook name is already taken, an error is raised and the hook
%    is not created.
%
%    The \meta{hook} can be specified using the dot-syntax to denote
%    the current package name. See section~\ref{sec:default-label}.
% \end{function}
%
%
%
% \begin{function}{\hook_use:n}
%   \begin{syntax}
%     \cs{hook_use:n} \Arg{hook}
%   \end{syntax}
%    Executes the \Arg{hook} code followed (if set up) by the code for next
%    invocation only, then empties that next invocation code.
%
%    The \meta{hook} \emph{cannot} be specified using the dot-syntax.
%    A leading |.| is treated literally.
% \end{function}
%
% \begin{function}{\hook_use_once:n}
%   \begin{syntax}
%     \cs{hook_use_once:n} \Arg{hook}
%   \end{syntax}
%     Changes the \Arg{hook} status so that from now on any addition to
%     the hook code is executed immediately. Then execute any
%     \Arg{hook} code already set up.
%    \fmi{better L3 name?}
%
%    The \meta{hook} \emph{cannot} be specified using the dot-syntax.
%    A leading |.| is treated literally.
% \end{function}
%
% \begin{function}{\hook_gput_code:nnn}
%   \begin{syntax}
%     \cs{hook_gput_code:nnn} \Arg{hook} \Arg{label} \Arg{code}
%   \end{syntax}
%    Adds a chunk of \meta{code} to the \meta{hook} labeled
%    \meta{label}. If the label already exists the \meta{code} is
%    appended to the already existing code.
%
%    If code is added to an external \meta{hook} (of the kernel or
%    another package) then the convention is to use the package name
%    as the \meta{label} not some internal module name or some other
%    arbitrary string.
%
%    The \meta{hook} and \meta{label} can be specified using the
%    dot-syntax to denote the current package name.
%    See section~\ref{sec:default-label}.
% \end{function}
%
% \begin{function}
%   {\hook_gput_next_code:nn}
%   \begin{syntax}
%     \cs{hook_gput_next_code:nn} \Arg{hook} \Arg{code}
%   \end{syntax}
%    Adds a chunk of \meta{code} for use only in the next invocation of the
%    \meta{hook}. Once used it is gone.
%
%    This is simpler than \cs{hook_gput_code:nnn}, the code is simply
%    appended to the hook in the order of declaration at the very end,
%    i.e., after all standard code for the hook got executed.
%
%    Thus if one needs to undo what the standard does one has to do
%    that as part of \meta{code}.
%
%    The \meta{hook} can be specified using the dot-syntax to denote
%    the current package name. See section~\ref{sec:default-label}.
% \end{function}
%


%  \begin{function}{\hook_gremove_code:nn}
%   \begin{syntax}
%     \cs{hook_gremove_code:nn} \Arg{hook} \Arg{label}
%   \end{syntax}
%    Removes any code for \meta{hook} labeled \meta{label}.
%
%    If the code for that \meta{label} wasn't yet added to the
%    \meta{hook}, an order is set so that when some code attempts to add
%    that label, the removal order takes action and the code is not
%    added.
%
%    If the second argument is \texttt{*}, then all code chunks are
%    removed. This is rather dangerous as it drops code from other
%    packages one may not know about, so think twice before using
%    that!
%
%    The \meta{hook} and \meta{label} can be specified using the
%    dot-syntax to denote the current package name.
%    See section~\ref{sec:default-label}.
% \end{function}


%  \begin{function}{\hook_gset_rule:nnnn}
%   \begin{syntax}
%     \cs{hook_gset_rule:nnnn} \Arg{hook} \Arg{label1} \Arg{relation} \Arg{label2}
%   \end{syntax}
%    Relate \meta{label1} with \meta{label2} when used in \meta{hook}.
%    See \cs{DeclareHookRule} for the allowed \meta{relation}s.
%    If \meta{hook} is \texttt{??} a default rule is specified.
%
%    The \meta{hook} and \meta{label} can be specified using the
%    dot-syntax to denote the current package name.
%    See section~\ref{sec:default-label}.
%    The dot-syntax is parsed in both \meta{label} arguments, but it
%    usually makes sense to be used in only one of them.
% \end{function}
%
% \begin{function}[pTF]{\hook_if_empty:n}
%   \begin{syntax}
%     \cs{hook_if_empty:n} \Arg{hook} \Arg{true code} \Arg{false code}
%   \end{syntax}
%   Tests if the \meta{hook} is empty (\emph{i.e.}, no code was added to
%   it using either \cs{AddToHook} or \cs{AddToHookNext}), and
%   branches to either \meta{true code} or \meta{false code} depending
%   on the result.
%
%    The \meta{hook} \emph{cannot} be specified using the dot-syntax.
%    A leading |.| is treated literally.
% \end{function}
%
% \begin{function}[pTF]{\hook_if_exist:n}
%   \begin{syntax}
%     \cs{hook_if_exist:n} \Arg{hook} \Arg{true code} \Arg{false code}
%   \end{syntax}
%   Tests if the \meta{hook} exists (if it was created with either
%   \cs{NewHook}, \cs{NewReversedHook}, or \cs{NewMirroredHookPair}), and
%   branches to either \meta{true code} or \meta{false code} depending
%   on the result.
%
%  \fmi{what are the results for generic hooks that do not need to be declared?}
%
%   The existence of a hook usually doesn't mean much from the viewpoint
%   of code that tries to add/remove code from that hook, since package
%   loading order may vary, thus the creation of hooks is asynchronous
%   to adding and removing code from it, so this test should be used
%   sparingly.
%
%    The \meta{hook} \emph{cannot} be specified using the dot-syntax.
%    A leading |.| is treated literally.
% \end{function}
%
%
% \begin{function}{\hook_debug_on:,\hook_debug_off:}
%   \begin{syntax}
%     \cs{hook_debug_on:}
%   \end{syntax}
%    Turns the debugging of hook code on or off. This displays changes
%    to the hook data.
% \end{function}
%
%
%
% \subsection{On the order of hook code execution} \label{sec:order}
%
%    Chunks of code for a \meta{hook} under different labels are supposed
%    to be independent if there are no special rules set up that
%    define a relation between the chunks. This means that you can't
%    make assumptions about the order of execution!
%
%    Suppose you have the following declarations:
%\begin{verbatim}
%    \NewHook{myhook}
%    \AddToHook{myhook}[packageA]{\typeout{A}}
%    \AddToHook{myhook}[packageB]{\typeout{B}}
%    \AddToHook{myhook}[packageC]{\typeout{C}}
%\end{verbatim}
%    then executing the hook with \cs{UseHook} will produce the
%    typeout \texttt{A} \texttt{B} \texttt{C} in that order.  In other
%    words, the execution order is computed to be \texttt{packageA},
%    \texttt{packageB}, \texttt{packageC} which you can verify with
%    \cs{ShowHook}\texttt{\{myhook\}}:
%\begin{verbatim}
%   The hook 'myhook':
%    Code chunks:
%       packageA -> \typeout {A}
%       packageB -> \typeout {B}
%       packageC -> \typeout {C}
%    Extra code next invocation:
%       ---
%    Rules:
%       ---
%    Execution order:
%       packageA, packageB, packageC
%\end{verbatim}
%    The reason is that the code chunks are internally saved in a property list
%    and the initial order of such a property list is the order in
%    which key-value pairs got added. However, that is only true if
%    nothing other than adding happens!
%
%    Suppose, or example, you want to replace the code chunk for
%    \texttt{packageA}, e.g.,
%\begin{verbatim}
%    \RemoveFromHook{myhook}[packageA]
%    \AddToHook{myhook}[packageA]{\typeout{A alt}}
%\end{verbatim}
%    then your order becomes  \texttt{packageB},
%    \texttt{packageC}, \texttt{packageA} because the label got removed
%    from the property list and then re-added (at its end).
%
%    While that may not be too surprising,  the execution order is
%    also sometimes altered if you add a redundant rule, e.g. if you specify
%\begin{verbatim}
%    \DeclareHookRule{myhook}{packageA}{before}{packageB}
%\end{verbatim}
%    instead of the previous lines we get
%\begin{verbatim}
%   The hook 'myhook':
%    Code chunks:
%       packageA -> \typeout {A}
%       packageB -> \typeout {B}
%       packageC -> \typeout {C}
%    Extra code next invocation:
%       ---
%    Rules:
%       packageA|packageB with relation before
%    Execution order (after applying rules):
%       packageA, packageC, packageB
%\end{verbatim}
%    As you can see the code chunks are still in the same order, but
%    in the execution order for the labels \texttt{packageB} and
%    \texttt{packageC} have
%    swapped places.
%    The reason is that, with the rule there are two orders that
%    satisfy it, and the algorithm for sorting happened to pick a
%    different one compared to the case without rules (where it
%    doesn't run at all as there is nothing to resolve).
%    Incidentally, if we had instead specified the redundant rule
%\begin{verbatim}
%    \DeclareHookRule{myhook}{packageB}{before}{label-3}
%\end{verbatim}
%    the execution order would not have changed.
%
%    In summary: it is not possible to rely on the order of execution
%    unless there are rules that partially or fully define the order
%    (in which you can rely on them being fulfilled).
%
%
% \subsection{The use of \enquote{reversed} hooks} \label{sec:reversed-order}
%
%    You may have wondered why you can declare a \enquote{reversed} hook
%    with \cs{NewReversedHook} and what that does exactly.
%
%    In short: the execution order of a reversed hook (without any
%    rules!) is exactly reversed to the order you would have gotten for
%    a hook declared with \cs{NewHook}.
%
%    This is helpful if you have a pair of hooks where you expect to see
%    code added that involves grouping, e.g., starting an environment
%    in the first and closing that environment in the second hook.
%    To give a somewhat contrived example\footnote{there are simpler
%    ways to achieve the same effect.}, suppose there is a package
%    adding the following:
%\begin{verbatim}
%    \AddToHook{env/quote/before}[package-1]{\begin{itshape}}
%    \AddToHook{env/quote/after} [package-1]{\end{itshape}}
%\end{verbatim}
%    As a result, all quotes will be in italics.
%    Now suppose further that the user wants the quotes also in blue
%    and therefore adds:
%\begin{verbatim}
%    \usepackage{color}
%    \AddToHook{env/quote/before}{\begin{color}{blue}}
%    \AddToHook{env/quote/after} {\end{color}}
%\end{verbatim}
%    Now if the \hook{env/quote/after} hook would be a normal hook we
%    would get the same execution order in  both hooks, namely:
%\begin{verbatim}
%    package-1, top-level
%\end{verbatim}
%    (or vice versa) and as a result, would get:
%\begin{verbatim}
%    \begin{itshape}\begin{color}{blue} ...
%    \end{itshape}\end{color}
%\end{verbatim}
%   and an error message that \verb=\begin{color}= ended by
%    \verb=\end{itshape}=.
%    With \hook{env/quote/after} declared as a reversed hook the
%    execution order is reversed and so all environments are closed in
%    the correct sequence and \cs{ShowHook} would give us the
%    following output:
%\begin{verbatim}
%   The hook 'env/quote/after':
%    Code chunks:
%       package-1 -> \end {itshape}
%       top-level -> \end {color}
%    Extra code next invocation:
%       ---
%    Rules:
%       ---
%    Execution order (after reversal):
%       top-level, package-1
%\end{verbatim}
%
%    The reversal of the execution order happens before applying any
%    rules, so if you alter the order you will probably have to alter
%    it in both hooks, not just in one, but that depends on the use case.
%
%
%
%
% \subsection{Private \LaTeX{} kernel hooks}
%
%    There are a few places where it is absolutely essential for
%    \LaTeX{} to function correctly that code is executed in a precisely
%    defined order. Even that could have been implemented with the
%    hook management (by adding various rules to ensure the
%    appropriate ordering with respect to other code added by
%    packages). However, this makes every document unnecessary
%    slow, because there has to be sorting even through the result is
%    predetermined. Furthermore it forces package writers to
%    unnecessarily add such rules if they add further code to the hook
%    (or break \LaTeX{}).
%
%    For that reason such code is not using the hook management, but
%    instead private kernel commands directly before or after a public
%    hook with the following naming
%    convention: \cs{@kernel@before@\meta{hookname}} or
%    \cs{@kernel@after@\meta{hookname}}. For example, in
%    \cs{enddocument} you find
%\begin{verbatim}
%   \UseHook{enddocument}%
%   \@kernel@after@enddocument
%\end{verbatim}
%    which means first the user/package-accessible \hook{enddocument}
%    hook is executed and then the internal kernel hook. As their name
%    indicates these kernel commands should not be altered by third-party
%    packages, so please refrain from that in the interest of
%    stability and instead use the public hook next to it.\footnote{As
%    with everything in \TeX{} there is not enforcement of this rule,
%    and by looking at the code it is easy to find out how the kernel
%    adds to them. The main reason of this section is therefore to say
%    \enquote{please don't do that, this is unconfigurable code!}}
%
%
%
% \subsection{Legacy \LaTeXe{} interfaces}
%
%  \LaTeXe{} offered a small number of hooks together with commands to
%    add to them. They are listed here and are retained for backwards
%    compatibility.
%
%  With the new hook management several additional hooks have been added
%    to \LaTeX\ and more will follow. See the next section for what
%    is already available.
%
%
% \begin{function}{\AtBeginDocument}
%   \begin{syntax}
%     \cs{AtBeginDocument} \oarg{label} \Arg{code}
%   \end{syntax}
%   If used without the optional argument \meta{label}, it works essentially
%    like before, i.e., it is adding \meta{code} to the hook
%    \hook{begindocument} 
%    (which is executed inside \verb=\begin{document}=).
%    However, all code added this way is labeled with the label
%    \hook{top-level} if done outside of a package or class or with the
%    package/class name if called inside such a file.
%
%    This way one can add further code to the hook using
%    \cs{AddToHook} or \cs{AtBeginDocument} using a different label
%    and explicitly order the code chunks as necessary, e.g., run some
%    code before or after the \hook{top-level} code.  When using the
%    optional argument the call is equivalent to running
%    \cs{AddToHook} \texttt{\{begindocument\}} \oarg{label}
%    \Arg{code}.
%
%    For important packages with known order requirement we may over
%    time add rules to the kernel (or to those packages) so that they
%    work regardless of the loading-order in the document.
% \end{function}
%
% \begin{function}{\AtEndDocument}
%   \begin{syntax}
%     \cs{AtEndDocument} \oarg{label} \Arg{code}
%   \end{syntax}
%   Like \cs{AtBeginDocument} but for the \hook{enddocument} hook.
% \end{function}
%
%
%
% \begin{function}{\AtBeginDvi}
%   \begin{syntax}
%     \cs{AtBeginDvi} \oarg{label} \Arg{code}
%   \end{syntax}
%   This hook is discussed in conjunction with the shipout hooks.
% \end{function}
%
%
%
% \subsection{\LaTeXe{} commands and environments augmented by
%    hooks}
%
%  \emph{intro to be written}
%
% \subsubsection{Generic hooks for all environments}
%
%    Every environment \meta{env} has now four associated hooks coming
%    with it:
%    \begin{description}
%    \item[\hook{env/\meta{env}/before}]
%
%       This hook is executed as part of \cs{begin} as the very first
%       action, in particular prior to starting the environment group.
%       Its scope is therefore not restricted by the environment.
%
%    \item[\hook{env/\meta{env}/begin}]
%
%       This hook is executed as part of \cs{begin} directly in front
%       of the code specific to the environment start (e.g., the
%       second argument of \cs{newenvironment}).  Its scope is the
%       environment body.
%
%    \item[\hook{env/\meta{env}/end}]
%
%       This is executed as part of \cs{end} directly in front of the
%       code specific to the end of the environment (e.g., the third
%       argument of \cs{newenvironment}).
%
%    \item[\hook{env/\meta{env}/after}]
%
%       This is executed as part of \cs{end} after the
%       code specific to the environment end and after the environment
%       group has ended.
%       Its scope is therefore not restricted by the environment.
%
%       This hook is implemented as a reversed hook so if two packages
%       add code to \hook{env/\meta{env}/before} and to
%       \hook{env/\meta{env}/after} they can add surrounding
%       environments and the order of closing them happens in the
%       right sequence.
%
%    \end{description}
%    In contrast to other hooks these hooks do not need to be declared
%    using \cs{NewHook}.
%
%    The hooks are only executed if \cs{begin}\Arg{env} and
%    \cs{end}\Arg{env} is used. If the environment code is executed
%    via low-level calls to \cs{\meta{env}} and \cs{end\meta{env}}
%    (e.g., to avoid the environment grouping) they are not
%    available. If you want them available in code using this method,
%    you would need to add them yourself, i.e., write something like
%\begin{verbatim}
%  \UseHook{env/quote/before}\quote
%      ...
%  \endquote\UseHook{env/quote/after}
%\end{verbatim}
%    to add the outer hooks, etc.
%
%
%
% \subsubsection{Hooks provided by \cs{begin}\texttt{\{document\}}}
%
%    Until 2020 \cs{begin}\texttt{\{document\}} offered exactly one
%    hook that one had to fill using \cs{AtBeginDocument}. Experience
%    has shown that this single hook in one place was not enough and
%    as part of adding the general hook management system a number of
%    additional hooks have been added at this point. The places for
%    hooks have been chosen to provide the same support as offered by
%    external packages, such as \pkg{etoolbox} and others that
%    augmented \cs{document} to gain better control.
%
%    Supported are now the following hooks:
%    \begin{description}
%
%    \item[\hook{env/document/before}]
%
%      This is the generic environment hook executed effectively
%      before \verb=\begin{document}= starts, i.e., one can think of
%      it as a hook for code at the end of the preamble section.
%
%    \item[\hook{env/document/begin}]
%
%      This is the second generic environment hook that is executed
%      after the environment has started its group. But given that for the
%      \texttt{document} environment this group is canceled there is
%      little difference to the previous one as the two are directly
%      executed one after another (the only difference is that in this
%      hook \cs{@currenvir} is now set to \texttt{document} but
%      anybody adding to this hook would know that already).
%
%
%    \item[\hook{begindocument}]
%
%      This hook is added to by \cs{AtBeginDocument} and is executed
%      after the \texttt{.aux} file as be read in and most
%      initialization are done, so they can be altered and inspected by
%      the hook code. It is followed by a small number of further
%      initializations that shouldn't be altered and are therefore
%      coming later.
%
%    \item[\hook{begindocument/end}]
%
%      This hook is executed at the end of the \cs{document} code in
%      other words at the beginning of the document body. The only
%      command that follows it is \cs{ignorespaces}.
%
%    \end{description}
%
%
%
%
% \subsubsection{Hooks provided by \cs{end}\texttt{\{document\}}}
%
%    \LaTeXe{} always provided \cs{AtEndDocument} to add code to the
%    execution of \verb=\end{document}= just in front of the code that
%    is normally executed there. While this was a big improvement over
%    the situation in \LaTeX\,2.09 it was not flexible enough for a
%    number of use cases and so packages, such as \pkg{etoolbox},
%    \pkg{atveryend} and others patched \cs{enddocument} to add
%    additional points where code could be hooked into.
%
%    Patching using packages is always problematical as leads to
%    conflicts (code availability, ordering of patches, incompatible
%    patches, etc.).  For this reason a number of additional hooks
%    have been added to the \cs{enddocument} code to allow packages
%    to add code in various places in a controlled way without the
%    need for overwriting or patching the core code.
%
%    Supported are now the following hooks:
%    \begin{description}
%
%    \item[\hook{env/document/end}] The generic hook inside \cs{end}.
%
%    \item[\hook{enddocument}]

%      The hook associated with \cs{AtEndDocument}. It is immediately
%      called after the previous hook so there could be just
%      one.\footnote{We could make \cs{AtEndDocument} just fill the
%      \hook{env/document/end} but maybe that is a bit confusing.}
%
%      When this hook is executed there may be still unprocessed
%      material (e.g., floats on the deferlist) and the hook may add
%      further material to be typeset. After it, \cs{clearpage} is
%      called to ensure that all such material gets typeset. If there
%      is nothing waiting the \cs{clearpage} has no effect.
%
%
%    \item[\hook{enddocument/afterlastpage}]
%
%      As the name indicates this hook should not receive code that
%      generates material for further pages. It is the right place to
%      do some final housekeeping and possibly write out some
%      information to the \texttt{.aux} file (which is still open at
%      this point to receive data). It is also the correct place to
%      set up any testing code to be run when the \texttt{.aux} file
%      is re-read in the next step.
%
%
%      After this hook has been executed the \texttt{.aux} file is
%      closed for writing and then read back in to do some tests
%      (e.g., looking for missing references or duplicated labels, etc.).
%
%    \item[\hook{enddocument/afteraux}]
%
%      At this point, the \texttt{.aux} file has been reprocessed and so
%      this is a possible place for final checks and display of
%      information to the user. However, for the latter you might
%      prefer the next hook, so that your information is displayed after the
%      (possibly longish) list of files if that got requested via \cs{listfiles}.
%
%    \item[\hook{enddocument/info}]
%
%      This hook is meant to receive code that write final information
%      messages to the terminal. It follows immediately after the
%      previous hook (so both could have been combined, but then
%      packages adding further code would always need to also supply
%      an explicit rule to specify where it should go.
%
%      This hook already contains some code added by the kernel (under
%      the labels \texttt{kernel/filelist} and
%      \texttt{kernel/warnings}), namely the list of files when
%      \cs{listfiles} has been used and the warnings for duplicate
%      labels, missing references, font substitutions etc.
%
%    \item[\hook{enddocument/end}]
%
%      Finally, this hook is executed just in front of the final call
%      to \cs{@{}@end}.
%
%    \end{description}
%
%
%    There is also the hook \hook{shipout/lastpage}. This hook is
%    executed as part of the last \cs{shipout} in the document to
%    allow package to add final \cs{special}'s to that page. Where
%    this hook is executed in relation to those from the above list
%    can vary from document to document. Furthermore to determine correctly
%    which of the \cs{shipout}s is the last one, \LaTeX{} needs to be run
%    several times, so initially it might get executed on the wrong
%    page. See section~\ref{sec:shipout} for where to find the details.
%
%
% \subsubsection{Hooks provided \cs{shipout} operations}
% \label{sec:shipout}
%
%    There are several hooks and mechanisms added to \LaTeX{}'s
%    process of generating pages. These are documented in
%    \texttt{ltshipout-doc.pdf} or with code in
%    \texttt{ltshipout-code.pdf}.
%
%
% \subsubsection{Hooks provided file loading operations}
%
%    There are several hooks added to \LaTeX{}'s
%    process of loading file via its high-level interfaces such as
%    \cs{input}, \cs{include}, \cs{usepackage}, etc. These are documented in
%    \texttt{ltfilehook-doc.pdf} or with code in
%    \texttt{ltfilehook-code.pdf}.
%
%

% \StopEventually{\setlength\IndexMin{200pt}  \PrintIndex  }
%
%
% \section{The Implementation}
%    \begin{macrocode}
%<@@=hook>
%    \end{macrocode}
%
%    \begin{macrocode}
%<*2ekernel>
%    \end{macrocode}
%    
%    \begin{macrocode}
\ExplSyntaxOn
%    \end{macrocode}
%
%
%  \subsection{Debugging}
%
%  \begin{macro}{\g_@@_debug_bool}
%    Holds the current debugging state.
%    \begin{macrocode}
\bool_new:N \g_@@_debug_bool
%    \end{macrocode}
%  \end{macro}
%
%  \begin{macro}{\hook_debug_on:,\hook_debug_off:}
%  \begin{macro}{\@@_debug:n}
%  \begin{macro}{\@@_debug_gset:}
%    Turns debugging on and off by redefining \cs{@@_debug:n}.
%    \begin{macrocode}
\cs_new_eq:NN \@@_debug:n \use_none:n
\cs_new_protected:Npn \hook_debug_on:
  {
    \bool_gset_true:N \g_@@_debug_bool
    \@@_debug_gset:
  }
\cs_new_protected:Npn \hook_debug_off:
  {
    \bool_gset_false:N \g_@@_debug_bool
    \@@_debug_gset:
  }
\cs_new_protected:Npn \@@_debug_gset:
  {
    \cs_gset_protected:Npx \@@_debug:n ##1
      { \bool_if:NT \g_@@_debug_bool {##1} }
  }
%    \end{macrocode}
%  \end{macro}
%  \end{macro}
%  \end{macro}

%
%
%  \subsection{Borrowing from internals of other kernel modules}
%
%
% \begin{macro}[EXP]{\@@_str_compare:nn}
%   Private copy of \cs{__str_if_eq:nn}
%    \begin{macrocode}
\cs_new_eq:NN \@@_str_compare:nn \__str_if_eq:nn
%    \end{macrocode}
% \end{macro}
%
%  \subsection{Declarations}
%
%  \begin{macro}{\l_@@_return_tl,\l_@@_tmpa_tl,\l_@@_tmpb_tl}
%    Scratch variables used throughout the package.
%    \begin{macrocode}
\tl_new:N \l_@@_return_tl
\tl_new:N \l_@@_tmpa_tl
\tl_new:N \l_@@_tmpb_tl
%    \end{macrocode}
%  \end{macro}
%
%  \begin{macro}{\g_@@_all_seq}
%    In a few places we need a list of all hook names ever defined so
%    we keep track if them in this sequence.
%    \begin{macrocode}
\seq_new:N \g_@@_all_seq
%    \end{macrocode}
%  \end{macro}
%
% \begin{macro}{\g_@@_removal_list_prop}
%   A property list to hold delayed removals.
%    \begin{macrocode}
\tl_new:N \g_@@_removal_list_tl
%    \end{macrocode}
% \end{macro}
%
% \begin{macro}{\l_@@_cur_hook_tl}
%   Stores the name of the hook currently being sorted.
%    \begin{macrocode}
\tl_new:N \l_@@_cur_hook_tl
%    \end{macrocode}
% \end{macro}
%
% \begin{macro}{\g_@@_code_temp_prop}
%   A property list to temporarily save the original one so that it
%   isn't permanently changed during sorting.
%    \begin{macrocode}
\prop_new:N \g_@@_code_temp_prop
%    \end{macrocode}
% \end{macro}
%
% \begin{macro}{\g_@@_hook_curr_name_tl,\g_@@_name_stack_seq}
%   Default label used for hook commands, and a stack to keep track of
%   packages within packages.
%    \begin{macrocode}
\tl_new:N \g_@@_hook_curr_name_tl
\seq_new:N \g_@@_name_stack_seq
%    \end{macrocode}
% \end{macro}
%
% \begin{macro}{\@@_tmp:w}
%   Temporary macro for generic usage.
%    \begin{macrocode}
\cs_new_eq:NN \@@_tmp:w ?
%    \end{macrocode}
% \end{macro}
%
% \begin{macro}{\tl_gremove_once:Nx}
%   Some variants of \pkg{expl3} functions. \fmi{should be moved to expl3}
%    \begin{macrocode}
\cs_generate_variant:Nn \tl_gremove_once:Nn { Nx }
%    \end{macrocode}
% \end{macro}
%
% \begin{macro}{\s_@@_mark}
%   Scan mark used for delimited arguments.
%    \begin{macrocode}
\scan_new:N \s_@@_mark
%    \end{macrocode}
% \end{macro}
%
%
% \subsection{Providing new hooks}
%
%  \begin{macro}{\g_@@_..._code_prop,\g_@@_..._rules_prop,
%                \g_@@_..._code_tl,\g_@@_..._next_code_tl}
%
%    Hooks have a \meta{name} and for each hook we have to provide a number of
%    data structures. These are
%    \begin{description}
%    \item[\cs{g_@@_\meta{name}_code_prop}] A property list holding the code
%    for the hook in separate chunks. The keys are by default the
%    package names that add code to the hook, but it is possible
%    for packages to define other keys. 
%
%    \item[\cs{g_@@_\meta{name}_rules_prop}] A property listing holding
%    relation info how the code chunks should be ordered within a
%    hook. This is used for debugging only. The actual rule for a
%    \meta{hook} is stored in a separate token lists named
%    \cs[no-index]{g_@@_\meta{hook}_rule_\meta{label1}\string|\meta{label2}_tl}
%    for a pair of labels.
%
%    \item[\cs{g_@@_\meta{name}_code_tl}] The code that is actually executed
%    when the hook is called in the document is stored in this token
%    list. It is constructed from the code chunks applying the
%    information.
%
%    \item[\cs{g_@@_\meta{name}_next_code_tl}] Finally there is extra code
%    (normally empty) that is used on the next invocation of the hook
%    (and then deleted). This can be used to define some special
%    behavior for a single occasion from within the document.
%
%    \end{description}
%  \end{macro}


%
%
%  \begin{macro}{\hook_new:n}
%    The \cs{hook_new:n} declaration declare a new hook and expects
%    the hook \meta{name} as its argument, e.g.,
%    \hook{begindocument}.
%    \begin{macrocode}
\cs_new_protected:Npn \hook_new:n #1
  {
    \exp_args:Nx \@@_new:n
      { \@@_parse_label_default:nn {#1} { top-level } }
  }
\cs_new_protected:Npn \@@_new:n #1 {
%    \end{macrocode}
%    We check for one of the internal data structures and if it
%    already exists we complain.
%    \begin{macrocode}
  \hook_if_exist:nTF {#1}
     { \ErrorHookExists }
%    \end{macrocode}
%    Otherwise we add the hook name to the list of all hooks and
%    allocate the necessary data structures for the new hook.
%    \begin{macrocode}
     { \seq_gput_right:Nn \g_@@_all_seq {#1}
%    \end{macrocode}
%    This is only used by the actual code of the current hook, so
%    declare it normally:
%    \begin{macrocode}
       \tl_new:c { g_@@_#1_code_tl }
%    \end{macrocode}
%    Now ensure that the base data structure for the hook exists:
%    \begin{macrocode}
       \@@_declare:n {#1}
%    \end{macrocode}
%    The \cs{g_@@_\meta{hook}_labels_clist} holds the sorted list of
%    labels (once it got sorted). This is used only for debugging.
%    \begin{macrocode}
       \clist_new:c {g_@@_#1_labels_clist}
%    \end{macrocode}
%    Some hooks should reverse the default order of code chunks. To
%    signal this we have a token list which is empty for normal hooks
%    and contains a \verb=-= for reversed hooks.
%    \begin{macrocode}
       \tl_new:c { g_@@_#1_reversed_tl }
%    \end{macrocode}
%    The above is all in L3 convention, but we also provide an
%    interface to legacy \LaTeXe{} for use in the current kernel. This
%    is done in a separate macro.
%    \begin{macrocode}
       \@@_provide_legacy_interface:n {#1}
     }
}
%    \end{macrocode}
%  \end{macro}

%
%
% \begin{macro}{\@@_declare:n}
%   This function declares the basic data structures for a hook without
%   actually declaring the hook itself.  This is needed to allow adding
%   to undeclared hooks.  Here it is unnecessary to check whether both
%   variables exist, since both are declared at the same time (either
%   both exist, or neither).
%    \begin{macrocode}
\cs_new_protected:Npn \@@_declare:n #1
  {
    \@@_if_exist:nF {#1}
      {
        \prop_new:c { g_@@_#1_code_prop }
        \tl_new:c { g_@@_#1_next_code_tl }
        \prop_new:c { g_@@_#1_rules_prop } % only for debugging
      }
  }
%    \end{macrocode}
%  \end{macro}



%  \begin{macro}{\hook_new_reversed:n}
%
%    Declare a new hook. The default ordering of code chunks is
%    reversed, signaled by setting the token list to a minus sign.
%    \begin{macrocode}
\cs_new_protected:Npn \hook_new_reversed:n #1 {
  \hook_new:n {#1}
  \tl_gset:cn { g_@@_#1_reversed_tl } { - }
}
%    \end{macrocode}
%  \end{macro}

%  \begin{macro}{\hook_new_pair:nn}
%    A shorthand for declaring a normal and a (matching) reversed hook in one go.
%    \begin{macrocode}
\cs_new_protected:Npn \hook_new_pair:nn #1#2 {
  \hook_new:n {#1}  \hook_new_reversed:n {#2}
}
%    \end{macrocode}
%  \end{macro}


% \begin{macro}{\@@_provide_legacy_interface:n}
%    The \LaTeX{} legacy concept for hooks uses with hooks the
%    following naming scheme in the code: \cs{@...hook}.
%
%    We follow this convention and insert the hook code using this
%    naming scheme in \LaTeXe{}. At least as long as this code is in a
%    package, some such hooks are already filled with data when we move
%    them over to the new scheme. We therefore insert already existing
%    code under the label \texttt{legacy} into the hook management
%    machinery and then replace the \cs{@...hook} with its counterpart
%    which is \cs{g_@@_\#1_code_tl}.\footnote{This means one extra
%    unnecessary expansion on each invocation in the document but
%    keeps the \LaTeXe{} and the L3 coding side properly separated.}
%    \begin{macrocode}
\cs_new_protected:Npn \@@_provide_legacy_interface:n #1
  {
%    \end{macrocode}
%    If the \pkg{expl3} code is run with checking on then assigning or
%    using non L3 names such as \cs{@enddocumenthook} with \pkg{expl3}
%    functions will trigger warnings so we run this code with
%    debugging explicitly suspended.
%    \begin{macrocode}
    \debug_suspend:
    \tl_if_exist:cT { @#1hook }
%    \end{macrocode}
%    Of course if the hook exists but is still empty, there is no need
%    to add anything under \texttt{legacy} or the current package name.
%    \begin{macrocode}
      {
        \tl_if_empty:cF { @#1hook }
          {
            \@@_gput_code:nxv {#1}
              { \@@_parse_label_default:Vn \c_novalue_tl { legacy } }
              { @#1hook }
          }
      }
%    \end{macrocode}
%    We need a global definition in case the declaration is done
%    inside a group (which happens below at the end of the file).
%    This is another reason why need to suspend checking, otherwise
%    \cs{tl_gset:co} would complain about \cs{@...hook} not starting
%    with \cs{g_}.
%    \begin{macrocode}
    \tl_gset:co{@#1hook}{\cs:w g_@@_#1_code_tl\cs_end:}
    \debug_resume:
  }
%    \end{macrocode}
% \end{macro}
%
% \subsection{Parsing a label}
%
% \begin{macro}[EXP]{\@@_parse_label_default:nn,\@@_parse_label_default:Vn}
%   This macro checks if a label was given (not \cs{c_novalue_tl}), and
%   if so, tries to parse the label looking for a leading \verb|.| to
%   replace for \cs{@currname}.  Otherwise \cs{@@_currname_or_default:n}
%   is used to pick \cs{@currname} or the fallback value.
%    \begin{macrocode}
\cs_new:Npn \@@_parse_label_default:nn #1 #2
  {
    \tl_if_novalue:nTF {#1}
      { \@@_currname_or_default:n {#2} }
      { \tl_trim_spaces_apply:nN {#1} \@@_parse_dot_label:nn {#2} }
  }
\cs_generate_variant:Nn \@@_parse_label_default:nn { V }
%    \end{macrocode}
% \end{macro}
%
% \begin{macro}[EXP]{\@@_parse_dot_label:nn}
% \begin{macro}[EXP]{
%     \@@_parse_dot_label:nw,
%     \@@_parse_dot_label_cleanup:w,
%     \@@_parse_dot_label_aux:nw
%   }
%   Start by checking if the label is empty, which raises an error, and
%   uses the fallback value.  If not,
%   split the label at a \verb|./|, if any, and check if no tokens are
%   before the \verb|./|, or if the only character is a \verb|.|.
%   If these requirements are fulfilled, the leading
%   \verb|.| is replaced with \cs{@@_currname_or_default:n}.  Otherwise
%   the label is returned unchanged.
%    \begin{macrocode}
\cs_new:Npn \@@_parse_dot_label:nn #1 #2
  {
    \tl_if_empty:nTF {#1}
      {
        \msg_expandable_error:nnn { hooks } { empty-label } {#2}
        #2
      }
      {
        \str_if_eq:nnTF {#1} { . }
          { \@@_currname_or_default:n {#1} }
          { \@@_parse_dot_label:nw {#2} #1 ./ \s_@@_mark }
      }
  }
\cs_new:Npn \@@_parse_dot_label:nw #1 #2 ./ #3 \s_@@_mark
  {
    \tl_if_empty:nTF {#2}
      { \@@_parse_dot_label_aux:nw {#1} #3 \s_@@_mark }
      {
        \tl_if_empty:nTF {#3}
          {#2}
          { \@@_parse_dot_label_cleanup:w #2 ./ #3 \s_@@_mark }
      }
  }
\cs_new:Npn \@@_parse_dot_label_cleanup:w #1 ./ \s_@@_mark {#1}
\cs_new:Npn \@@_parse_dot_label_aux:nw #1 #2 ./ \s_@@_mark
  { \@@_currname_or_default:n {#1} / #2 }
%    \end{macrocode}
% \end{macro}
% \end{macro}
%
% \begin{macro}[EXP]{\@@_currname_or_default:n}
%   Uses \cs{g_@@_hook_curr_name_tl} if it is set, otherwise tries
%   \cs{@currname}.  If neither is set, uses the fallback value
%   \verb|#1| (usually \texttt{top-level}).
%    \begin{macrocode}
\cs_new:Npn \@@_currname_or_default:n #1
  {
    \tl_if_empty:NTF \g_@@_hook_curr_name_tl
      {
        \tl_if_empty:NTF \@currname
          {#1}
          { \@currname }
      }
      { \g_@@_hook_curr_name_tl }
  }
%    \end{macrocode}
% \end{macro}




% \begin{macro}{\hook_gput_code:nnn}
% \begin{macro}{\@@_gput_code:nnn,\@@_gput_code:nxv,\@@_hook_gput_code_do:nnn}
%
%    With \cs{hook_gput_code:nnn}\Arg{hook}\Arg{label}\Arg{code} a
%    chunk of \meta{code} is added to an existing \meta{hook} labeled
%    with \meta{label}.
%    \begin{macrocode}
\cs_new_protected:Npn \hook_gput_code:nnn #1 #2
  {
    \exp_args:Nxx \@@_gput_code:nnn
      { \@@_parse_label_default:nn {#1} { top-level } }
      { \@@_parse_label_default:nn {#2} { top-level } }
  }
\cs_new_protected:Npn \@@_gput_code:nnn #1 #2 #3
  {
%    \end{macrocode}
%    First we check if the hook exists.
%    \begin{macrocode}
    \@@_if_marked_removal:nnTF {#1} {#2}
      { \@@_unmark_removal:nn {#1} {#2} }
      {
%    \end{macrocode}
%    First we check if the hook exists.
%    \begin{macrocode}
        \hook_if_exist:nTF {#1}
%    \end{macrocode}
%    If so we simply add (or append) the new code to the property list
%    holding different chunks for the hook. At \verb=\begin{document}=
%    this is then sorted into a token list for fast execution.
%    \begin{macrocode}
          {
            \@@_hook_gput_code_do:nnn {#1} {#2} {#3}
%    \end{macrocode}
%    However, if there is an update within the document we need to alter
%    this execution code which is done by
%    \cs{@@_update_hook_code:n}. In the preamble this does nothing.
%    \begin{macrocode}
            \@@_update_hook_code:n {#1}
          }
%    \end{macrocode}
%    
%    \begin{macrocode}
          { \@@_try_declaring_generic_hook:nnn {#1} {#2} {#3} }
      }
  }
\cs_generate_variant:Nn \@@_gput_code:nnn { nxv }
%    \end{macrocode}
%
%   This macro will unconditionally add a chunk of code to the given hook.
%    \begin{macrocode}
\cs_new_protected:Npn \@@_hook_gput_code_do:nnn #1 #2 #3
  {
%   However, first some debugging info if debugging is enabled:
%    \begin{macrocode}
    \@@_debug:n{\iow_term:x{****~ Add~ to~
                      \hook_if_exist:nF {#1} { undeclared~ }
                      hook~ #1~ (#2)
                      \on@line\space <-~ \tl_to_str:n{#3}} }
%    \end{macrocode}
%   Then try to get the code chunk labeled \verb=#2= from the hook.
%   If there's code already there, then append \verb=#3= to that,
%   otherwise just put \verb=#3=.
%    \begin{macrocode}
    \prop_get:cnNTF { g_@@_#1_code_prop } {#2} \l_@@_return_tl
      {
        \prop_gput:cno { g_@@_#1_code_prop } {#2}
          { \l_@@_return_tl #3 }
      }
      { \prop_gput:cnn { g_@@_#1_code_prop } {#2} {#3} }
  }
%    \end{macrocode}
% \end{macro}
% \end{macro}
%
% \begin{macro}{\@@_gput_undeclared_hook:nnn}
%   Often it may happen that a package $A$ defines a hook \verb=foo=,
%   but package $B$, that adds code to that hook, is loaded before $A$.
%   In such case we need to add code to the hook before its declared.
%    \begin{macrocode}
\cs_new_protected:Npn \@@_gput_undeclared_hook:nnn #1 #2 #3
  {
    \@@_declare:n {#1}
    \@@_hook_gput_code_do:nnn {#1} {#2} {#3}
  }
%    \end{macrocode}
% \end{macro}
%
% \begin{macro}{\@@_try_declaring_generic_hook:nnn}
% \begin{macro}{\@@_try_declaring_generic_next_hook:nn}
%   These entry-level macros just pass the arguments along to the
%   common \cs{@@_try_declaring_generic_hook:nNNnn} with the right
%   functions to execute when some action is to be taken.
%
%   The wrapper \cs{@@_try_declaring_generic_hook:nnn} then defers
%   \cs{hook_gput_code:nnn} if the generic hook was declared, or to
%   \cs{@@_gput_undeclared_hook:nnn} otherwise (the hook was tested for
%   existence before, so at this point if it isn't generic, it doesn't
%   exist).
%
%   The wrapper \cs{@@_try_declaring_generic_next_hook:nn} for
%   next-execution hooks does the same: it defers the code to
%   \cs{hook_gput_next_code:nn} if the generic hook was declared, or
%   to \cs{@@_gput_next_do:nn} otherwise.
%    \begin{macrocode}
\cs_new_protected:Npn \@@_try_declaring_generic_hook:nnn #1
  {
    \@@_try_declaring_generic_hook:nNNnn {#1}
      \hook_gput_code:nnn \@@_gput_undeclared_hook:nnn
  }
\cs_new_protected:Npn \@@_try_declaring_generic_next_hook:nn #1
  {
    \@@_try_declaring_generic_hook:nNNnn {#1}
      \hook_gput_next_code:nn \@@_gput_next_do:nn
  }
%    \end{macrocode}
%
% \begin{macro}{
%     \@@_try_declaring_generic_hook:nNNnn,
%     \@@_try_declaring_generic_hook_split:nNNnn
%   }
% \begin{macro}[TF]{\@@_try_declaring_generic_hook:wn}
%   \cs{@@_try_declaring_generic_hook:nNNnn} now splits the hook name
%   at the first \texttt{/} (if any) and first checks if it is a
%   file-specific hook (they require some normalization) using
%   \cs{@@_if_file_hook:wTF}. If not then check it is one of a
%   predefined set for generic names. We also split off the second
%   component to see if we have to make a reversed hook.  In either case
%   the function returns \meta{true} for a generic hook and \meta{false}
%   in other cases.
%    \begin{macrocode}
\cs_new_protected:Npn \@@_try_declaring_generic_hook:nNNnn #1
  {
    \@@_if_file_hook:wTF #1 / / \s_@@_mark
      {
        \exp_args:Ne \@@_try_declaring_generic_hook_split:nNNnn
          { \exp_args:Ne \@@_file_hook_normalise:n {#1} }
      }
      { \@@_try_declaring_generic_hook_split:nNNnn {#1} }
  }
\cs_new_protected:Npn \@@_try_declaring_generic_hook_split:nNNnn #1 #2 #3
  {
    \@@_try_declaring_generic_hook:wnTF #1 / / / \scan_stop: {#1}
      { #2 }
      { #3 } {#1}
  }
\prg_new_protected_conditional:Npnn \@@_try_declaring_generic_hook:wn
    #1 / #2 / #3 / #4 \scan_stop: #5 { TF }
  {
    \tl_if_empty:nTF {#2}
      { \prg_return_false: }
      {
        \prop_if_in:NnTF \c_@@_generics_prop {#1}
          {
            \hook_if_exist:nF {#5} { \hook_new:n {#5} }
%    \end{macrocode}
%    After having declared the hook we check the second component (for
%    file hooks) or the third component for environment hooks) and
%    if it is on the list of components for which we should have declared
%    a reversed hook we alter the hook data structure accordingly.
%    \begin{macrocode}
            \prop_if_in:NnTF \c_@@_generics_reversed_ii_prop {#2}
              { \tl_gset:cn { g_@@_#5_reversed_tl } { - } }
              {
                \prop_if_in:NnT \c_@@_generics_reversed_iii_prop {#3}
                  { \tl_gset:cn { g_@@_#5_reversed_tl } { - } }
              }
%    \end{macrocode}
%    Now that we know that the hook is declared we can add the code to it.
%    \begin{macrocode}
            \prg_return_true:
          }
          { \prg_return_false: }
      }
  }
%    \end{macrocode}
% \end{macro}
% \end{macro}
% \end{macro}
% \end{macro}
%
% \begin{macro}[pTF]{\@@_if_file_hook:w}
%   \cs{@@_if_file_hook:wTF} checks if the argument is a valid
%   file-specific hook (not, for example, |file/before|, but
%   |file/before/foo.tex|).  If it is a file-specific hook, then it
%   executes the \meta{true} branch, otherwise \meta{false}.
%
%   A file-specific hook is \texttt{file/\meta{position}/\meta{name}}.
%   If any of these parts don't exist, it is a general file hook or not
%   a file hook at all, so the conditional evaluates to \meta{false}.
%   Otherwise, it checks that the first part is |file| and that the
%   \meta{position} is in the \cs{c_@@_generics_file_prop}.
%
%   A property list is used here to avoid having to worry with catcodes,
%   because \pkg{expl3}'s file name parsing turns all characters into
%   catcode-12 tokens, which might differ from hand-input letters.
%    \begin{macrocode}
\prg_new_conditional:Npnn \@@_if_file_hook:w
    #1 / #2 / #3 \s_@@_mark { TF }
  {
    \str_if_eq:nnTF {#1} { file }
      {
        \bool_lazy_or:nnTF
            { \tl_if_empty_p:n {#3} }
            { \str_if_eq_p:nn {#3} { / } }
          { \prg_return_false: }
          {
            \prop_if_in:NnTF \c_@@_generics_file_prop {#2}
              { \prg_return_true: }
              { \prg_return_false: }
          }
      }
      { \prg_return_false: }
  }
%    \end{macrocode}
% \end{macro}
%
% \begin{macro}[EXP]{\@@_file_hook_normalise:n}
% \begin{macro}[EXP]{\@@_strip_double_slash:n,\@@_strip_double_slash:w}
%   When a file-specific hook is found, before being declared it is
%   lightly normalized by \cs{@@_file_hook_normalise:n}.  The current
%   implementation just replaces two consecutive slashes (|//|) by a
%   single one, to cope with simple cases where the user did something
%   like \verb|\def\input@path{{./mypath/}}|, in which case a hook would
%   have to be \verb|\AddToHook{file/after/./mypath//file.tex}|.
%    \begin{macrocode}
\cs_new:Npn \@@_file_hook_normalise:n #1
  { \@@_strip_double_slash:n {#1} }
\cs_new:Npn \@@_strip_double_slash:n #1
  { \@@_strip_double_slash:w #1 // \s_@@_mark }
\cs_new:Npn \@@_strip_double_slash:w #1 // #2 \s_@@_mark
  {
    \tl_if_empty:nTF {#2}
      {#1}
      { \@@_strip_double_slash:w #1 / #2 \s_@@_mark }
  }
%    \end{macrocode}
% \end{macro}
% \end{macro}





%  \begin{macro}{\c_@@_generics_prop}
%    Clist holding the generic names. We don't provide any user
%    interface to this as this is meant to be static.
%    \begin{description}
%    \item[\texttt{env}]
%      The generic hooks used in \cs{begin} and \cs{end}.
%    \item[\texttt{file}]
%      The generic hooks used when loading a file
%    \end{description}
%    \begin{macrocode}
\prop_const_from_keyval:Nn \c_@@_generics_prop
  {env=,file=,package=,class=,include=}
%    \end{macrocode}
%  \end{macro}
%
%  \begin{macro}{\c_@@_generics_reversed_ii_prop,
%                \c_@@_generics_reversed_iii_prop,
%                \c_@@_generics_file_prop}
%    Some of the generic hooks are supposed to use reverse ordering, these are
%    the following (only the second or third sub-component is checked):
%    \begin{macrocode}
\prop_const_from_keyval:Nn \c_@@_generics_reversed_ii_prop {after=,end=}
\prop_const_from_keyval:Nn \c_@@_generics_reversed_iii_prop {after=}
\prop_const_from_keyval:Nn \c_@@_generics_file_prop {before=,after=}
%    \end{macrocode}
%  \end{macro}


%  \begin{macro}{\@@_update_hook_code:n}
%    Before \verb=\begin{document}=  this does nothing, in the body it
%    reinitializes the hook code using the altered data.
%    \begin{macrocode}
\cs_new_eq:NN \@@_update_hook_code:n \use_none:n
%    \end{macrocode}
%  \end{macro}


% \begin{macro}{\hook_gremove_code:nn}
% \begin{macro}{\@@_gremove_code:nn}
%    
%    With \cs{hook_gremove_code:nn}\Arg{hook}\Arg{label} any code
%    for \meta{hook} stored under \meta{label} is removed.
%    \begin{macrocode}
\cs_new_protected:Npn \hook_gremove_code:nn #1 #2
  {
    \exp_args:Nxx \@@_gremove_code:nn
      { \@@_parse_label_default:nn {#1} { top-level } }
      { \@@_parse_label_default:nn {#2} { top-level } }
  }
\cs_new_protected:Npn \@@_gremove_code:nn #1 #2
  {
%    \end{macrocode}
%    First check that the hook code pool exists.  \cs{hook_if_exist:nTF}
%    isn't used here because it should be possible to remove code from a
%    hook before its defined (see section~\ref{sec:querying}).
%    \begin{macrocode}
    \@@_if_exist:nTF {#1}
%    \end{macrocode}
%    Then remove the chunk and run \cs{@@_update_hook_code:n} so
%    that the execution token list reflects the change if we are after
%    \verb=\begin{document}=.
%    \begin{macrocode}
      {
        \str_if_eq:nnTF {#2} {*}
          {
            \prop_gclear:c { g_@@_#1_code_prop }
            \clist_gclear:c { g_@@_#1_labels_clist } % for debugging only
          }
          {
%    \end{macrocode}
%    Check if the label being removed exists in the code pool.  If it does,
%    just call \cs{@@_gremove_code_do:nn} to do the removal, otherwise mark it
%    to be removed.
%    \begin{macrocode}
            \prop_get:cnNTF { g_@@_#1_code_prop } {#2} \l_@@_return_tl
              { \@@_gremove_code_do:nn }
              { \@@_mark_removal:nn }
                  {#1} {#2}
          }
%    \end{macrocode}
%    Finally update the code, if the hook exists.
%    \begin{macrocode}
        \hook_if_exist:nT {#1}
          { \@@_update_hook_code:n {#1} }
      }
      { \@@_mark_removal:nn {#1} {#2} }
  }
%    \end{macrocode}
%
% \begin{macro}{\@@_gremove_code_do:nn}
%    \begin{macrocode}
\cs_new_protected:Npn \@@_gremove_code_do:nn #1 #2
  {
    \prop_gremove:cn { g_@@_#1_code_prop } {#2}
%    \end{macrocode}
%    Removing the dropped label from \verb=\g_@@_#1_labels_clist= is
%    rather tricky, because that clists holds the labels as strings
%    (i.e., not ordinary text which is what we have in \verb=#2=).
%    \begin{macrocode}
    \exp_args:Nco \clist_gremove_all:Nn
      { g_@@_#1_labels_clist } { \tl_to_str:n {#2} } % for debugging only
  }
%    \end{macrocode}
% \end{macro}
% \end{macro}
% \end{macro}


%
% \begin{macro}{\@@_mark_removal:nn}
%   Marks \meta{label} (\verb=#2=) to be removed from \meta{hook}
%   (\verb=#1=).
%    \begin{macrocode}
\cs_new_protected:Npn \@@_mark_removal:nn #1 #2
  {
    \tl_gput_right:Nx \g_@@_removal_list_tl
      { \@@_removal_tl:nn {#1} {#2} }
  }
%    \end{macrocode}
% \end{macro}
%
% \begin{macro}{\@@_unmark_removal:nn}
%   Unmarks \meta{label} (\verb=#2=) to be removed from \meta{hook}
%   (\verb=#1=).  \cs{tl_gremove_once:Nx} is used rather than
%   \cs{tl_gremove_all:Nx} so that two additions are needed to cancel
%   two marked removals, rather than only one.
%    \begin{macrocode}
\cs_new_protected:Npn \@@_unmark_removal:nn #1 #2
  {
    \tl_gremove_once:Nx \g_@@_removal_list_tl
      { \@@_removal_tl:nn {#1} {#2} }
  }
%    \end{macrocode}
% \end{macro}
%
% \begin{macro}[TF]{\@@_if_marked_removal:nn}
%   Checks if the \cs{g_@@_removal_list_tl} contains the current
%   \meta{label} (\verb=#2=) and \meta{hook} (\verb=#1=).
%    \begin{macrocode}
\prg_new_protected_conditional:Npnn \@@_if_marked_removal:nn #1 #2 { TF }
  {
    \exp_args:NNx \tl_if_in:NnTF \g_@@_removal_list_tl
      { \@@_removal_tl:nn {#1} {#2} }
      { \prg_return_true: } { \prg_return_false: }
  }
%    \end{macrocode}
% \end{macro}
%
% \begin{macro}[rEXP]{\@@_removal_tl:nn}
%   Builds a token list with \verb=#1= and \verb=#2= which can only be
%   matched by \verb=#1= and \verb=#2=.
%    \begin{macrocode}
\cs_new:Npn \@@_removal_tl:nn #1 #2
  { & \tl_to_str:n {#2} $ \tl_to_str:n {#1} $ }
%    \end{macrocode}
% \end{macro}




%
%
%
% \begin{macro}{
%     \g_@@_??_rules_prop,
%     \g_@@_??_code_prop,
%     \g_@@_??_code_tl,
%     \g_@@_??_reversed_tl,
%   }
%
%    Default rules applying to all hooks are stored in this property
%    list. Initially it simply used an empty ``label'' name (not two
%    question marks). This was a bit unfortunate, because then
%    \texttt{l3doc} complains about \verb=__= in the middle of a
%    command name when trying to typeset the documentation. However
%    using a ``normal'' name such as \texttt{default} has the
%    disadvantage of that being not really distinguishable from a real
%    hook name. I now have settled for \texttt{??} which needs some
%    gymnastics to get it into the csname, but since this is used a
%    lot things should be fast, so this is not done with \texttt{c}
%    expansion in the code later on.
%
%    \cs{g_@@_??_code_tl} isn't used, but it has to be defined to trick
%    the code into thinking that \verb=??= is actually a hook.
%    \begin{macrocode}
\prop_new:c {g_@@_??_rules_prop}
\prop_new:c {g_@@_??_code_prop}
\prop_new:c {g_@@_??_code_tl}
%    \end{macrocode}
%
%    Default rules are always given in normal ordering (never in
%    reversed ordering). If such a rule is applied to a reversed
%    hook it behaves as if the rule is reversed (e.g.,
%    \texttt{after} becomes \texttt{before})
%    because those rules are applied first and then the order is reversed.
%    \begin{macrocode}
\tl_new:c {g_@@_??_reversed_tl}
%    \end{macrocode}
%  \end{macro}



%  \begin{macro}{\@@_debug_gset_rule:nnnn}
%
%    \fmi{this needs cleanup and docu correction!}
%
%    With
%    \cs{@@_debug_gset_rule:nnnn}\Arg{hook}\Arg{label1}\Arg{relation}\Arg{label2}
%    a relation is defined between the two code labels for the given
%    \meta{hook}.  The special hook \texttt{??} stands for \emph{any}
%    hook describing a default rule.
%    \begin{macrocode}
\cs_new_protected:Npn \@@_debug_gset_rule:nnnn #1#2#3#4
  {
%    \end{macrocode}
%    If so we drop any existing rules with the two labels (in case
%    there are any).
%    \begin{macrocode}
    \prop_gremove:cn{g_@@_#1_rules_prop}{#2|#4}
    \prop_gremove:cn{g_@@_#1_rules_prop}{#4|#2}
%    \end{macrocode}
%    Then  we add the new one (normalizing the input a bit, e.g., we
%    always use \texttt{before} and not \texttt{after} and
%    instead reorder the labels):
%    \begin{macrocode}
    \str_case_e:nnF {#3}
       {
         {before} { \prop_gput:cnn {g_@@_#1_rules_prop}{#2|#4}{<} }
         {after}  { \prop_gput:cnn {g_@@_#1_rules_prop}{#4|#2}{<} }
%    \end{macrocode}
%    More special rule types \ldots
%    \begin{macrocode}
         {incompatible-error}   { \prop_gput:cnn {g_@@_#1_rules_prop}{#2|#4}{xE} }
         {incompatible-warning} { \prop_gput:cnn {g_@@_#1_rules_prop}{#2|#4}{xW} }
         {removes}      { \prop_gput:cnn {g_@@_#1_rules_prop}{#2|#4}{->} }
%    \end{macrocode}
%    Undo a setting:
%    \begin{macrocode}
         {unrelated}{ \prop_gremove:cn {g_@@_#1_rules_prop}{#2|#4} 
                      \prop_gremove:cn {g_@@_#1_rules_prop}{#4|#2} }
       }
       { \ERRORunknownrule }
  }
%    \end{macrocode}
%  \end{macro}
%
%
%  \subsection{Setting rules for hooks code}
%
%  \begin{macro}{\hook_gset_rule:nnnn}
%  \begin{macro}{\@@_gset_rule:nnnn}
%
%    \fmi{needs docu correction given new implementation}
%
%    With
%    \cs{hook_gset_rule:nnnn}\Arg{hook}\Arg{label1}\Arg{relation}\Arg{label2}
%    a relation is defined between the two code labels for the given
%    \meta{hook}.  The special hook \texttt{??} stands for \emph{any}
%    hook describing a default rule.
%    \begin{macrocode}
\cs_new_protected:Npn \hook_gset_rule:nnnn #1#2#3#4
  {
    \use:x
      {
        \@@_gset_rule:nnnn
          { \@@_parse_label_default:nn {#1} { top-level } }
          { \@@_parse_label_default:nn {#2} { top-level } }
          {#3}
          { \@@_parse_label_default:nn {#4} { top-level } }
      }
  }
%    \end{macrocode}
%    
%    \begin{macrocode}
\cs_new_protected:Npn \@@_gset_rule:nnnn #1#2#3#4
  {
%    \end{macrocode}
%    First we ensure the basic data structure of the hook exists:
%    \begin{macrocode}
    \@@_declare:n {#1}
%    \end{macrocode}
%    Then we clear any previous relationship between both labels.
%    \begin{macrocode}
    \@@_rule_gclear:nnn {#1} {#2} {#4}
%    \end{macrocode}
%    Then we call the function to handle the given rule. Throw an error if the
%    rule is invalid.
%    \begin{macrocode}
    \debug_suspend:
    \cs_if_exist_use:cTF { @@_rule_#3_gset:nnn }
      {
          {#1} {#2} {#4}
        \@@_update_hook_code:n {#1}
      }
      { \ERRORunknownrule }
    \debug_resume:
    \@@_debug_gset_rule:nnnn {#1} {#2} {#3} {#4} % for debugging
  }
%    \end{macrocode}
% \end{macro}
% \end{macro}
%
% \begin{macro}{\@@_rule_before_gset:nnn, \@@_rule_after_gset:nnn,
%               \@@_rule_<_gset:nnn, \@@_rule_>_gset:nnn}
%    Then we add the new rule.  We need to normalize the rules here to
%    allow for faster processing later.  Given a pair of labels
%    $l_A$ and $l_B$, the rule $l_A>l_B$ is the same as $l_B<l_A$\fmi{}
%    said differently.  But normalizing the
%    forms of the rule to a single representation, say, $l_B<l_A$, then
%    the time spent looking for the rules later is considerably reduced.
%
%    Here we do that normalization by using \cs[no-index]{(pdf)strcmp} to
%    lexically sort labels $l_A$ and $l_B$ to a fixed order.  This order
%    is then enforced every time these two labels are used together.
%
%    Here we use \cs{@@_label_pair:nn}~\Arg{hook}~\Arg{l_A}~\Arg{l_B}
%    to build a string \texttt{$l_B$\string|$l_A$} with a fixed order, and
%    use \cs{@@_label_ordered:nnTF} to apply the correct rule to the pair
%    of labels, depending if it was sorted or not.
%    \begin{macrocode}
\cs_new_protected:Npn \@@_rule_before_gset:nnn #1#2#3
  {
    \tl_gset:cx { g_@@_#1_rule_ \@@_label_pair:nn {#2} {#3} _tl }
      { \@@_label_ordered:nnTF {#2} {#3} { < } { > } }
  }
\cs_new_eq:cN { @@_rule_<_gset:nnn } \@@_rule_before_gset:nnn
%    \end{macrocode}
%
%    \begin{macrocode}
\cs_new_protected:Npn \@@_rule_after_gset:nnn #1#2#3
  {
    \tl_gset:cx { g_@@_#1_rule_ \@@_label_pair:nn {#3} {#2} _tl }
      { \@@_label_ordered:nnTF {#3} {#2} { < } { > } }
  }
\cs_new_eq:cN { @@_rule_>_gset:nnn } \@@_rule_after_gset:nnn
%    \end{macrocode}
%  \end{macro}
%
% \begin{macro}{\@@_rule_removes_gset:nnn}
%   This rule removes (clears, actually) the code from label |#3| if
%   label |#2| is in the hook |#1|.
%    \begin{macrocode}
\cs_new_protected:Npn \@@_rule_removes_gset:nnn #1#2#3
  {
    \tl_gset:cx { g_@@_#1_rule_ \@@_label_pair:nn {#2} {#3} _tl }
      { \@@_label_ordered:nnTF {#2} {#3} { -> } { <- } }
  }
%    \end{macrocode}
%  \end{macro}
%
% \begin{macro}{
%     \@@_rule_incompatible-error_gset:nnn,
%     \@@_rule_incompatible-warning_gset:nnn,
%   }
%   These relations make an error/warning if labels |#2| and |#3| appear
%   together in hook |#1|.
%    \begin{macrocode}
\cs_new_protected:cpn { @@_rule_incompatible-error_gset:nnn } #1#2#3
  { \tl_gset:cn { g_@@_#1_rule_ \@@_label_pair:nn {#2} {#3} _tl } { xE } }
\cs_new_protected:cpn { @@_rule_incompatible-warning_gset:nnn } #1#2#3
  { \tl_gset:cn { g_@@_#1_rule_ \@@_label_pair:nn {#2} {#3} _tl } { xW } }
%    \end{macrocode}
%  \end{macro}
%
% \begin{macro}{\@@_rule_unrelated_gset:nnn, \@@_rule_gclear:nnn}
%    Undo a setting. \cs{@@_rule_unrelated_gset:nnn} doesn't need to do anything,
%    since we use \cs{@@_rule_gclear:nnn} before setting any rule.
%    \begin{macrocode}
\cs_new_protected:Npn \@@_rule_unrelated_gset:nnn #1#2#3 { }
\cs_new_protected:Npn \@@_rule_gclear:nnn #1#2#3
  { \cs_undefine:c { g_@@_#1_rule_ \@@_label_pair:nn {#2} {#3} _tl } }
%    \end{macrocode}
%  \end{macro}
%
% \begin{macro}[EXP]{\@@_label_pair:nn}
%   Ensure that the lexically greater label comes first.
%    \begin{macrocode}
\cs_new:Npn \@@_label_pair:nn #1#2
  {
    \if_case:w \@@_str_compare:nn {#1} {#2} \exp_stop_f:
           #1 | #1 %  0
    \or:   #1 | #2 % +1
    \else: #2 | #1 % -1
    \fi:
  }
%    \end{macrocode}
%  \end{macro}
%
% \begin{macro}[pTF]{\@@_label_ordered:nn}
%   Check that labels |#1| and |#2| are in the correct order (as
%   returned by \cs{@@_label_pair:nn}) and if so return true, else
%   return false.
%    \begin{macrocode}
\prg_new_conditional:Npnn \@@_label_ordered:nn #1#2 { TF }
  {
    \if_int_compare:w \@@_str_compare:nn {#1} {#2} > 0 \exp_stop_f:
      \prg_return_true:
    \else
      \prg_return_false:
    \fi:
  }
%    \end{macrocode}
%  \end{macro}
%
% \begin{macro}[EXP]{\@@_if_label_case:nnnnn}
%   To avoid doing the string comparison twice in \cs{@@_initialize_single:NNNNn}
%   (once with \cs{str_if_eq:nn} and again with \cs{@@_label_ordered:nn}),
%   we use a three-way branching macro that will compare |#1| and |#2|
%   and expand to \cs{use_i:nnn} if they are equal, \cs{use_ii:nn} if
%   |#1| is lexically greater, and \cs{use_iii:nn} otherwise.
%    \begin{macrocode}
\cs_new:Npn \@@_if_label_case:nnnnn #1#2
   {
     \cs:w use_
       \if_case:w \@@_str_compare:nn {#1} {#2}
          i \or: ii \else: iii \fi: :nnn
     \cs_end:
   }
%    \end{macrocode}
%  \end{macro}


%  \begin{macro}{\@@_initialize_all:}
%    Initialize all known hooks (at \verb=\begin{document}=), i.e.,
%    update the fast execution token lists to hold the necessary code
%    in the right  order.
%    \begin{macrocode}
\cs_new_protected:Npn \@@_initialize_all: {
%    \end{macrocode}
%    First we change \cs{@@_update_hook_code:n} which so far was a
%    no-op to now initialize one hook. This way any later updates to
%    the hook will run that code and also update the execution token
%    list.
%    \begin{macrocode}
  \cs_gset_eq:NN \@@_update_hook_code:n \@@_initialize_hook_code:n
%    \end{macrocode}
%    Now we loop over all hooks that have been defined and update each
%    of them.
%    \begin{macrocode}
  \@@_debug:n { \prop_gclear:N \g_@@_used_prop }
  \seq_map_inline:Nn \g_@@_all_seq
      {
        \@@_update_hook_code:n {##1}
      }
%    \end{macrocode}
%    If we are debugging we show results hook by hook for all hooks
%    that have data. 
%    \begin{macrocode}
  \@@_debug:n
     { \iow_term:x{^^JAll~ initialized~ (non-empty)~ hooks:}
       \prop_map_inline:Nn \g_@@_used_prop
           { \iow_term:x{^^J~ ##1~ ->~
               \exp_not:v {g_@@_##1_code_tl}~ }
           }
     }
%    
%    \end{macrocode}
%    After all hooks are initialized we change the ``use'' to just
%    call the hook code and not initialize it (as it was done in the
%    preamble.
%    \begin{macrocode}
  \cs_gset_eq:NN \hook_use:n \@@_use_initialized:n
  \cs_gset_eq:NN \@@_preamble_hook:n \use_none:n
}
%    \end{macrocode}
%  \end{macro}



%  \begin{macro}{\@@_initialize_hook_code:n}
%    Initializing or reinitializing the fast execution hook code. In
%    the preamble this is selectively done in case a hook gets used
%    and at \verb=\begin{document}= this is done for all hooks and
%    afterwards only if the hook code changes.
%    \begin{macrocode}
\cs_new_protected:Npn \@@_initialize_hook_code:n #1 {
  \@@_debug:n{ \iow_term:x{^^JUpdate~ code~ for~ hook~
                                  '#1' \on@line :^^J} }
%    \end{macrocode}
%    This does the sorting and the updates. If there aren't any code
%    chunks for the current hook, there is no point in even starting
%    the sorting routine so we make a quick test for that and in that
%    case just update \cs{g_@@_\meta{hook}_code_tl} to hold the next
%    code. If there are code chunks we call
%    \cs{@@_initialize_single:NNNNn} and pass to it ready made csnames
%    as they are needed several times inside. This way we save a bit
%    on processing time if we do that up front.
%    \begin{macrocode}
  \hook_if_exist:nT {#1}
    {
      \prop_if_empty:cTF {g_@@_#1_code_prop}
        { \tl_gset:co {g_@@_#1_code_tl}
                      {\cs:w g_@@_#1_next_code_tl \cs_end: } }
        {
%    \end{macrocode}
%    By default the algorithm sorts the code chunks and then saves the
%    result in a token list for fast execution by adding the code one
%    after another using \cs{tl_gput_right:NV}. When we sort code for
%    a reversed hook, all we have to do is to add the code chunks in
%    the opposite order into the token list. So all we have to do
%    in preparation is to change two definitions used later on.
%    \begin{macrocode}
          \@@_if_reversed:nTF {#1}
            { \cs_set_eq:NN \@@_tl_gput:NV    \tl_gput_left:NV
              \cs_set_eq:NN \@@_clist_gput:NV \clist_gput_left:NV  }
            { \cs_set_eq:NN \@@_tl_gput:NV    \tl_gput_right:NV
              \cs_set_eq:NN \@@_clist_gput:NV \clist_gput_right:NV }
%    \end{macrocode}
%
%    When sorting, some relations (namely \verb|->| \verb|<-|) need to
%    act destructively on the code property lists to remove code that
%    shouldn't appear in the sorted hook token list.
%    \begin{macrocode}
          \prop_gset_eq:Nc \g_@@_code_temp_prop { g_@@_#1_code_prop }
          \@@_initialize_single:ccccn
            { g_@@_#1_code_prop } { g_@@_#1_code_tl }
            { g_@@_#1_next_code_tl } { g_@@_#1_labels_clist }
            {#1}
          \prop_gset_eq:cN { g_@@_#1_code_prop } \g_@@_code_temp_prop
%    \end{macrocode}
%    For debug display we want to keep track of those hooks that
%    actually got code added to them, so we record that in plist. We
%    use a plist to ensure that we record each hook name only once,
%    i.e., we are only interested in storing the keys and the value is arbitrary
%    \begin{macrocode}
          \@@_debug:n{ \exp_args:NNx \prop_gput:Nnn \g_@@_used_prop {#1}{} }
        }
    }
}
%    \end{macrocode}
%  \end{macro}


%  \begin{macro}{\g_@@_used_prop}
%    All hooks that receive code (for use in debugging display).
%    \begin{macrocode}
\prop_new:N\g_@@_used_prop
%    \end{macrocode}
%  \end{macro}


%
% \begin{macro}[EXP]{\@@_tl_csname:n,\@@_seq_csname:n}
%   It is faster to pass a single token and expand it when necessary
%   than to pass a bunch of character tokens around.
%   \fmi{note to myself: verify}
%    \begin{macrocode}
\cs_new:Npn \@@_tl_csname:n #1 { l_@@_label_#1_tl }
\cs_new:Npn \@@_seq_csname:n #1 { l_@@_label_#1_seq }
%    \end{macrocode}
% \end{macro}
%

%
%
%  \begin{macro}{\l_@@_labels_seq,\l_@@_labels_int,\l_@@_front_tl,
%      \l_@@_rear_tl,\l_@@_label_0_tl}
%
%    For the sorting I am basically implementing Knuth's algorithm for
%    topological sorting as given in TAOCP volume 1 pages 263--266.
%    For this algorithm we need a number of local variables:
%    \begin{itemize}
%    \item
%       List of labels used in the current hook to label code chunks:
%    \begin{macrocode}
\seq_new:N \l_@@_labels_seq
%    \end{macrocode}
%    \item
%      Number of labels used in the current hook. In Knuth's algorithm
%      this is called $N$:
%    \begin{macrocode}
\int_new:N \l_@@_labels_int
%    \end{macrocode}
%    \item
%      The sorted code list to be build is managed using two pointers
%      one to the front of the queue and one to the rear. We model this
%      using token list pointers. Knuth calls them $F$ and $R$:
%    \begin{macrocode}
\tl_new:N \l_@@_front_tl
\tl_new:N \l_@@_rear_tl
%    \end{macrocode}
%    \item
%      The data for the start of the queue is kept in this token list,
%      it corresponds to what Don calls \texttt{QLINK[0]} but since we
%      aren't manipulating individual words in memory it is slightly
%      differently done:
%    \begin{macrocode}
\tl_new:c { \@@_tl_csname:n { 0 } }
%    \end{macrocode}
%
%    \end{itemize}
%  \end{macro}


%  \begin{macro}{\@@_initialize_single:NNNNn,\@@_initialize_single:ccccn}
%
%    \cs{@@_initialize_single:NNNNn} implements the sorting of the code
%    chunks for a hook and saves the result in the token list for fast
%    execution (\verb=#3=). The arguments are \meta{hook-code-plist},
%    \meta{hook-code-tl}, \meta{hook-next-code-tl},
%    \meta{hook-ordered-labels-clist} and \meta{hook-name} (the latter
%    is only used for debugging---the \meta{hook-rule-plist} is accessed
%    using the \meta{hook-name}).
%
%    The additional complexity compared to Don's algorithm is that we
%    do not use simple positive integers but have arbitrary
%    alphanumeric labels. As usual Don's data structures are chosen in
%    a way that one can omit a lot of tests and I have mimicked that as
%    far as possible. The result is a restriction I do not test for at
%    the moment: a label can't be equal to the number 0!  \fmi{Needs
%    checking for, just in case}
%    \begin{macrocode}
\cs_new_protected:Npn \@@_initialize_single:NNNNn #1#2#3#4#5 {
%    \end{macrocode}
%    
%    \begin{macrocode}
  \debug_suspend:
%    \end{macrocode}
%    Step T1: Initialize the data structure \ldots
%    \begin{macrocode}
  \seq_clear:N \l_@@_labels_seq
  \int_zero:N  \l_@@_labels_int
%    \end{macrocode}
%
%    Store the name of the hook:
%    \begin{macrocode}
  \tl_set:Nn \l_@@_cur_hook_tl {#5}
%    \end{macrocode}
%    
%    We loop over the property list holding the code and record all
%    labels listed there. Only rules for those labels are of interest
%    to us. While we are at it we count them (which gives us the $N$
%    in Knuth's algorithm.  The prefix |label_| is added to the variables
%    to ensure that labels named |front|, |rear|, |labels|, or |return|
%    don't interact with our code.
%    \begin{macrocode}
  \prop_map_inline:Nn #1
     {
       \int_incr:N \l_@@_labels_int
       \seq_put_right:Nn \l_@@_labels_seq {##1}
       \tl_set:cn { \@@_tl_csname:n {##1} }{0}     % the counter k for number of
                                                   % j before k rules
       \seq_clear_new:c { \@@_seq_csname:n {##1} } % sequence of successors to k
                                                   % i.e., k before j rules (stores
                                                   % the names of the j's)
     }
%    \end{macrocode}
%    Steps T2 and T3: Sort the relevant rules into the data structure\ldots
%    
%    This loop constitutes a square matrix of the labels in |#1| in the
%    vertical and the horizontal directions.  However since the rule
%    $l_A\meta{rel}l_B$ is the same as $l_B\meta{rel}^{-1}l_A$ we can cut
%    the loop short at the diagonal of the matrix (\emph{i.e.}, when
%    both labels are equal), saving a good amount of time.  The way the
%    rules were set up (see the implementation of \cs{@@_rule_before_gset:nnn}
%    above) ensures that we have no rule in the ignored side of the
%    matrix, and all rules are seen.  The rules are applied in
%    \cs{@@_apply_label_pair:nnn}, which takes the properly-ordered pair
%    of labels as argument.
%    \begin{macrocode}
  \prop_map_inline:Nn #1
    {
      \prop_map_inline:Nn #1
        {
          \@@_if_label_case:nnnnn {##1} {####1}
            { \prop_map_break: }
            { \@@_apply_label_pair:nnn {##1} {####1} }
            { \@@_apply_label_pair:nnn {####1} {##1} }
                {#5}
        }
    }
%    \end{macrocode}
%    Take a breath and take a look at the data structures that have
%    been set up:
%    \begin{macrocode}
  \@@_debug:n { \@@_debug_label_data:N #1 }
%    \end{macrocode}
%    
%
%    Step T4:
%    \begin{macrocode}
  \tl_set:Nn \l_@@_rear_tl { 0 }
  \tl_set:cn { \@@_tl_csname:n { 0 } } { 0 } % really {l_@@_label_ \l_@@_rear_tl _tl}
  \seq_map_inline:Nn \l_@@_labels_seq
      {
        \int_compare:nNnT { \cs:w \@@_tl_csname:n {##1} \cs_end: } = 0
            {
              \tl_set:cn { \@@_tl_csname:n { \l_@@_rear_tl } }{##1}
              \tl_set:Nn \l_@@_rear_tl {##1}
            }
      }
  \tl_set_eq:Nc \l_@@_front_tl { \@@_tl_csname:n { 0 } }
%    \end{macrocode}
%    
%    \begin{macrocode}
  \tl_gclear:N #2
  \clist_gclear:N #4
%    \end{macrocode}
%
%    The whole loop combines steps T5--T7:
%    \begin{macrocode}
  \bool_while_do:nn { ! \str_if_eq_p:Vn \l_@@_front_tl { 0 } }
       {
%    \end{macrocode}
%    This part is step T5:
%    \begin{macrocode}
         \int_decr:N \l_@@_labels_int
         \prop_get:NVN #1 \l_@@_front_tl \l_@@_return_tl
         \@@_tl_gput:NV #2 \l_@@_return_tl
%    \end{macrocode}
%    
%    \begin{macrocode}
         \@@_clist_gput:NV #4 \l_@@_front_tl
         \@@_debug:n{ \iow_term:x{Handled~ code~ for~ \l_@@_front_tl} }
%    \end{macrocode}
%
%    This is step T6 except that we don't use a pointer $P$ to move
%    through the successors, but instead use \verb=##1= of the mapping
%    function.
%    \begin{macrocode}
         \seq_map_inline:cn { \@@_seq_csname:n { \l_@@_front_tl } }
             {
               \tl_set:cx { \@@_tl_csname:n {##1} }
                   { \int_eval:n { \cs:w \@@_tl_csname:n {##1} \cs_end: - 1 } }
               \int_compare:nNnT { \cs:w \@@_tl_csname:n {##1} \cs_end: } = 0
                   {
                     \tl_set:cn { \@@_tl_csname:n { \l_@@_rear_tl } } {##1}
                     \tl_set:Nn \l_@@_rear_tl            {##1}
                   }
             }
%    \end{macrocode}
%    and step T7:
%    \begin{macrocode}
         \tl_set_eq:Nc \l_@@_front_tl { \@@_tl_csname:n { \l_@@_front_tl } }
%    \end{macrocode}
%
%    This is step T8: If we haven't moved the code for all labels
%    (i.e., if \cs{l_@@_labels_int} is still greater than zero) we
%    have a loop and our partial order can't be flattened out.
%    \begin{macrocode}
       }
  \int_compare:nNnF \l_@@_labels_int = 0
      {
        \iow_term:x{====================}
        \iow_term:x{Error:~ label~ rules~ are~ incompatible:}
%    \end{macrocode}
%
%    This is not really the information one needs in the error case
%    but will do for now \ldots \fmi{fix}
%    \begin{macrocode}
        \@@_debug_label_data:N #1
        \iow_term:x{====================}
      }
%    \end{macrocode}
%    After we have added all hook code to \verb=#2= we finish it off
%    with adding extra code for a one time execution. That is stored
%    in \verb=#3= but is normally empty.
%    \begin{macrocode}
  \tl_gput_right:Nn #2 {#3}
%    \end{macrocode}
%    
%    \begin{macrocode}
  \debug_resume:
}
%    \end{macrocode}
%    
%    \begin{macrocode}
\cs_generate_variant:Nn \@@_initialize_single:NNNNn {cccc}
%    \end{macrocode}
%  \end{macro}



%  \begin{macro}{\@@_tl_gput:NV,\@@_clist_gput:NV}
%    These append either on the right (normal hook) or on the left
%    (reversed hook). This is setup up in
%    \cs{@@_initialize_hook_code:n}, elsewhere their behavior is undefined.
%    \begin{macrocode}
\cs_new:Npn \@@_tl_gput:NV     {\ERROR}
\cs_new:Npn \@@_clist_gput:NV  {\ERROR}
%    \end{macrocode}
%  \end{macro}


%
%  \begin{macro}{\@@_apply_label_pair:nnn,\@@_label_if_exist_apply:nnnF}
%
%    This is the payload of steps T2 and T3 executed in the loop described
%    above. This macro assumes |#1| and |#2| are ordered, which means that
%    any rule pertaining the pair |#1| and |#2| is
%    \cs{g_@@_\meta{hook}_rule_\#1\string|\#2_tl}, and not
%    \cs{g_@@_\meta{hook}_rule_\#2\string|\#1_tl}.  This also saves a great deal
%    of time since we only need to check the order of the labels once.
%
%    The arguments here are \meta{label1}, \meta{label2}, \meta{hook}, and
%    \meta{hook-code-plist}.  We are about to apply the next rule and
%    enter it into the data structure.  \cs{@@_apply_label_pair:nnn} will
%    just call \cs{@@_label_if_exist_apply:nnnF} for the \meta{hook}, and
%    if no rule is found, also try the \meta{hook} name \verb=??=
%    denoting a default hook rule.
%
%    \cs{@@_label_if_exist_apply:nnnF} will check if the rule exists for
%    the given hook, and if so call \cs{@@_apply_rule:nnn}.
%    \begin{macrocode}
\cs_new_protected:Npn \@@_apply_label_pair:nnn #1#2#3
  {
%    \end{macrocode}
%    Extra complication: as we use default rules and local hook specific
%    rules we first have to check if there is a local rule and if that
%    exist use it. Otherwise check if there is a default rule and use
%    that.
%    \begin{macrocode}
    \@@_label_if_exist_apply:nnnF {#1} {#2} {#3}
      {
%    \end{macrocode}
%    If there is no hook-specific rule we check for a default one and
%    use that if it exists.
%    \begin{macrocode}
        \@@_label_if_exist_apply:nnnF {#1} {#2} { ?? } { }
      }
  }
\cs_new_protected:Npn \@@_label_if_exist_apply:nnnF #1#2#3
  {
    \if_cs_exist:w g_@@_ #3 _rule_ #1 | #2 _tl \cs_end:
%    \end{macrocode}
%    What to do precisely depends on the type of rule we have
%    encountered. If it is a \texttt{before} rule it will be handled by the
%    algorithm but other types need to be managed differently. All
%    this is done in \cs{@@_apply_rule:nnnN}.
%    \begin{macrocode}
      \@@_apply_rule:nnn {#1} {#2} {#3}
      \exp_after:wN \use_none:n
    \else:
      \use:nn
    \fi:
  }
%    \end{macrocode}
%  \end{macro}




%  \begin{macro}{\@@_apply_rule:nnn}
%    This is the code executed in steps T2 and T3 while looping through
%    the matrix  This is part of step T3. We are about to apply the next
%    rule and enter it into the data structure. The arguments are
%    \meta{label1}, \meta{label2}, \meta{hook-name}, and \meta{hook-code-plist}.
%    \begin{macrocode}
\cs_new_protected:Npn \@@_apply_rule:nnn #1#2#3
  {
    \cs:w @@_apply_
      \cs:w g_@@_#3_reversed_tl \cs_end: rule_
        \cs:w g_@@_ #3 _rule_ #1 | #2 _tl \cs_end: :nnn \cs_end:
      {#1} {#2} {#3}
  }
%    \end{macrocode}
% \end{macro}
%
%  \begin{macro}{\@@_apply_rule_<:nnn,\@@_apply_rule_>:nnn}
%    The most common cases are \texttt{\string<} and \texttt{\string>} so we handle
%    that first.  They are relations $\prec$ and $\succ$ in TAOCP, and
%    they dictate sorting.
%    \begin{macrocode}
\cs_new_protected:cpn { @@_apply_rule_<:nnn } #1#2#3
  {
    \@@_debug:n { \@@_msg_pair_found:nnn {#1} {#2} {#3} }
    \tl_set:cx { \@@_tl_csname:n {#2} }
       { \int_eval:n{ \cs:w \@@_tl_csname:n {#2} \cs_end: + 1 } }
    \seq_put_right:cn{ \@@_seq_csname:n {#1} }{#2}
  }
\cs_new_protected:cpn { @@_apply_rule_>:nnn } #1#2#3
  {
    \@@_debug:n { \@@_msg_pair_found:nnn {#1} {#2} {#3} }
    \tl_set:cx { \@@_tl_csname:n {#1} }
       { \int_eval:n{ \cs:w \@@_tl_csname:n {#1} \cs_end: + 1 } }
    \seq_put_right:cn{ \@@_seq_csname:n {#2} }{#1}
  }
%    \end{macrocode}
% \end{macro}
%
% \begin{macro}{\@@_apply_rule_xE:nnn,\@@_apply_rule_xW:nnn}
%   These relations make two labels incompatible within a hook.
%   |xE| makes raises an error if the labels are found in the same
%   hook, and |xW| makes it a warning.
%    \begin{macrocode}
\cs_new_protected:cpn { @@_apply_rule_xE:nnn } #1#2#3
  {
    \@@_debug:n { \@@_msg_pair_found:nnn {#1} {#2} {#3} }
    \msg_error:nnnnnn { hooks } { labels-incompatible }
      {#1} {#2} {#3} { 1 }
    \use:c { @@_apply_rule_->:nnn } {#1} {#2} {#3}
    \use:c { @@_apply_rule_<-:nnn } {#1} {#2} {#3}
  }
\cs_new_protected:cpn { @@_apply_rule_xW:nnn } #1#2#3
  {
    \@@_debug:n { \@@_msg_pair_found:nnn {#1} {#2} {#3} }
    \msg_warning:nnnnnn { hooks } { labels-incompatible }
      {#1} {#2} {#3} { 0 }
  }
%    \end{macrocode}
% \end{macro}
%
%  \begin{macro}{\@@_apply_rule_->:nnn,\@@_apply_rule_<-:nnn}
%    If we see \texttt{\detokenize{->}} we have to drop code for label
%    \verb=#3= and carry on. We could do a little better and trop
%    everything for that label since it doesn't matter where we sort
%    in the empty code. However that would complicate the algorithm a
%    lot with little gain. So we still unnecessarily try to sort it in
%    and depending on the rules that might result in a loop that is
%    otherwise resolved. If that turns out to be a real issue, we can
%    improve the code.
%
%    Here the code is removed from \cs{l_@@_cur_hook_tl} rather than
%    \verb=#3= because the latter may be \verb=??=, and the default
%    hook doesn't store any code.  Removing from \cs{l_@@_cur_hook_tl}
%    makes default rules \verb=->= and  \verb=<-= work properly.
%    \begin{macrocode}
\cs_new_protected:cpn { @@_apply_rule_->:nnn } #1#2#3
  {
    \@@_debug:n
       {
         \@@_msg_pair_found:nnn {#1} {#2} {#3}
         \iow_term:x{--->~ Drop~ '#2'~ code~ from~
           \iow_char:N \\ g_@@_ \l_@@_cur_hook_tl _code_prop ~ because~ of~ '#1' }
       }
    \prop_gput:cnn { g_@@_ \l_@@_cur_hook_tl _code_prop } {#2} { }
  }
\cs_new_protected:cpn { @@_apply_rule_<-:nnn } #1#2#3
  {
    \@@_debug:n
       {
         \@@_msg_pair_found:nnn {#1} {#2} {#3}
         \iow_term:x{--->~ Drop~ '#1'~ code~ from~
           \iow_char:N \\ g_@@_ \l_@@_cur_hook_tl _code_prop ~ because~ of~ '#2' }
       }
    \prop_gput:cnn { g_@@_ \l_@@_cur_hook_tl _code_prop } {#1} { }
  }
%    \end{macrocode}
%  \end{macro}

% \begin{macro}{
%     \@@_apply_-rule_<:nnn,
%     \@@_apply_-rule_>:nnn,
%     \@@_apply_-rule_<-:nnn,
%     \@@_apply_-rule_->:nnn,
%     \@@_apply_-rule_x:nnn,
%   }
%   Reversed rules.
%    \begin{macrocode}
\cs_new_eq:cc { @@_apply_-rule_<:nnn  } { @@_apply_rule_>:nnn }
\cs_new_eq:cc { @@_apply_-rule_>:nnn  } { @@_apply_rule_<:nnn }
\cs_new_eq:cc { @@_apply_-rule_<-:nnn } { @@_apply_rule_<-:nnn }
\cs_new_eq:cc { @@_apply_-rule_->:nnn } { @@_apply_rule_->:nnn }
\cs_new_eq:cc { @@_apply_-rule_xE:nnn  } { @@_apply_rule_xE:nnn }
\cs_new_eq:cc { @@_apply_-rule_xW:nnn  } { @@_apply_rule_xW:nnn }
%    \end{macrocode}
% \end{macro}


% \begin{macro}{\@@_msg_pair_found:nnn}
%   A macro to avoid moving this many tokens around.
%    \begin{macrocode}
\cs_new_protected:Npn \@@_msg_pair_found:nnn #1#2#3
  {
    \iow_term:x{~ \str_if_eq:nnTF {#3} {??} {default} {~normal} ~
               rule~ \@@_label_pair:nn {#1} {#2}:~
             \use:c { g_@@_#3_rule_ \@@_label_pair:nn {#1} {#2} _tl } ~ found}
  }
%    \end{macrocode}
% \end{macro}


%  \begin{macro}{\@@_debug_label_data:N}
%    
%    \begin{macrocode}
\cs_new_protected:Npn \@@_debug_label_data:N #1 {
  \iow_term:x{Code~ labels~ for~ sorting:}
  \iow_term:x{~ \seq_use:Nnnn\l_@@_labels_seq {~and~}{,~}{~and~} }  % fix name!
  \iow_term:x{^^J Data~ structure~ for~ label~ rules:}
  \prop_map_inline:Nn #1
       {
         \iow_term:x{~ ##1~ =~ \tl_use:c{ \@@_tl_csname:n {##1} }~ ->~
           \seq_use:cnnn{ \@@_seq_csname:n {##1} }{~->~}{~->~}{~->~}
         }
       }
  \iow_term:x{}
}
%    \end{macrocode}
%  \end{macro}



%  \begin{macro}{\hook_log:n}
%    This writes out information about the hook given in its argument
%    onto the terminal and the \texttt{.log} file.
%    \begin{macrocode}
\cs_new_protected:Npn \hook_log:n #1
  {
    \exp_args:Nx \@@_log:n
      { \@@_parse_label_default:nn {#1} { top-level } }
  }
\cs_new_protected:Npn \@@_log:n #1
  {
    \iow_term:x{^^JThe~ hook~ '#1':}
%    \end{macrocode}
%    
%    \begin{macrocode}
    \hook_if_exist:nF {#1}
      { \iow_term:x {~Hook~ is~ not~ declared!} }
    \@@_if_exist:nTF {#1}
      {
        \iow_term:x{~Code~ chunks:}
        \prop_if_empty:cTF {g_@@_#1_code_prop}
          { \iow_term:x{\@spaces ---} }
          {
            \prop_map_inline:cn {g_@@_#1_code_prop}
              { \iow_term:x{\@spaces ##1~ ->~ \tl_to_str:n{##2} } }
          }
%    \end{macrocode}
%    
%    \begin{macrocode}
        \iow_term:x{~Extra~ code~ next~ invocation:}
        \iow_term:x{\@spaces
          \tl_if_empty:cTF { g_@@_#1_next_code_tl }
            {---} {->~ \str_use:c{g_@@_#1_next_code_tl} } }
%    \end{macrocode}
%
%    \fmi{This is currently only displaying the local rules, but it
%         should also show the matching global rules!}
%
%    \begin{macrocode}
        \iow_term:x{~Rules:}
        \prop_if_empty:cTF {g_@@_#1_rules_prop}
          { \iow_term:x{\@spaces ---} }
          { \prop_map_inline:cn {g_@@_#1_rules_prop}
              { \iow_term:x{\@spaces ##1~ with~ relation~ ##2} }
          }
%    \end{macrocode}
%    
%    \begin{macrocode}
        \hook_if_exist:nT {#1}
          { \iow_term:x { ~Execution~ order
               \prop_if_empty:cTF {g_@@_#1_rules_prop}
                 { \@@_if_reversed:nT {#1}
                        { ~ (after~ reversal) }
                 }
                 { ~ (after~
                   \@@_if_reversed:nT {#1} {reversal~ and~}
                   applying~ rules)
                 }
               :    
              }
            \iow_term:x { \@spaces
              \clist_if_empty:cTF{g_@@_#1_labels_clist}
                 {not~ set~ yet}
                 { \clist_use:cnnn {g_@@_#1_labels_clist}
                                   { ,~ } { ,~ } { ,~ }   }
            }
          }
      }
      { \iow_term:n { ~The~hook~is~empty. } }
    \iow_term:n { }
  }
%    \end{macrocode}
%    
%  \end{macro}



%  \subsection{Specifying code for next invocation}
%
%
%
%
%
%  \begin{macro}{\hook_gput_next_code:nn}
%    
%    \begin{macrocode}
\cs_new_protected:Npn \hook_gput_next_code:nn #1
  {
    \exp_args:Nx \@@_gput_next_code:nn
      { \@@_parse_label_default:nn {#1} { top-level } }
  }
\cs_new_protected:Npn \@@_gput_next_code:nn #1 #2
  {
    \@@_declare:n {#1}
    \hook_if_exist:nTF {#1}
      { \@@_gput_next_do:nn {#1} {#2} }
      { \@@_try_declaring_generic_next_hook:nn {#1} {#2} }
  }
\cs_new_protected:Npn \@@_gput_next_do:nn #1 #2
  {
    \tl_gput_right:cn { g_@@_#1_next_code_tl }
      { #2 \tl_gclear:c { g_@@_#1_next_code_tl } }
  }
%    \end{macrocode}
%  \end{macro}
%
%
% \subsection{Using the hook}
%
% \begin{macro}{\hook_use:n}
% \begin{macro}[EXP]{\@@_use_initialized:n}
% \begin{macro}{\@@_preamble_hook:n}
%   \cs{hook_use:n} as defined here is used in the preamble, where
%   hooks aren't initialized by default.  \cs{@@_use_initialized:n} is
%   also defined, which is the non-\tn{protected} version for use within
%   the document.  Their definition is identical, except for the
%   \cs{@@_preamble_hook:n} (which wouldn't hurt in the expandable
%   version, but it would be an unnecessary extra expansion).
%
%   \cs{@@_use_initialized:n} holds the expandable definition while in
%   the preamble. \cs{@@_preamble_hook:n} initializes the hook in the
%   preamble, and is redefined to \cs{use_none:n} at |\begin{document}|.
%
%   Both versions do the same internally:  check if the hook exist as
%   given, and if so use it as quickly as possible.  If it doesn't
%   exist, the a call to \cs{@@_use:wn} checks for file hooks.
%
%   At |\begin{document}|, all hooks are initialized, and any change in
%   them causes an update, so \cs{hook_use:n} can be made expandable.
%   This one is better not protected so that it can expand into nothing
%   if containing no code. Also important in case of generic hooks that
%   we do not generate a \cs{relax} as a side effect of checking for a
%   csname. In contrast to the \TeX{} low-level
%   \verb=\csname ...\endcsname= construct \cs{tl_if_exist:c} is
%   careful to avoid this.
%    \begin{macrocode}
\cs_new_protected:Npn \hook_use:n #1
  {
    \tl_if_exist:cTF { g_@@_#1_code_tl }
      {
        \@@_preamble_hook:n {#1}
        \cs:w g_@@_#1_code_tl \cs_end:
      }
      { \@@_use:wn #1 / \s_@@_mark {#1} }
  }
\cs_new:Npn \@@_use_initialized:n #1
  {
    \tl_if_exist:cTF { g_@@_#1_code_tl }
      { \cs:w g_@@_#1_code_tl \cs_end: }
      { \@@_use:wn #1 / \s_@@_mark {#1} }
  }
\cs_new_protected:Npn \@@_preamble_hook:n #1
  { \@@_initialize_hook_code:n {#1} }
%    \end{macrocode}
% \end{macro}
% \end{macro}
% \end{macro}
%
% \begin{macro}[EXP]{\@@_use:wn}
% \begin{macro}{\@@_try_file_hook:n,\@@_if_exist_use:n}
%   \cs{@@_use:wn} does a quick check to test if the current hook is a
%   file hook: those need a special treatment.  If it is not, the hook
%   does not exist.  If it is, then \cs{@@_try_file_hook:n} is called,
%   and checks that the current hook is a file-specific hook using
%   \cs{@@_if_file_hook:wTF}.  If it's not, then it's a generic |file/|
%   hook and is used if it exist.
%
%   If it is a file-specific hook, it passes through the same
%   normalization as during declaration, and then it is used if defined.
%   \cs{@@_if_exist_use:n} checks if the hook exist, and calls
%   \cs{@@_preamble_hook:n} if so, then uses the hook.
%    \begin{macrocode}
\cs_new:Npn \@@_use:wn #1 / #2 \s_@@_mark #3
  {
    \str_if_eq:nnTF {#1} { file }
      { \@@_try_file_hook:n {#3} }
      { } % Hook doesn't exist
  }
\cs_new_protected:Npn \@@_try_file_hook:n #1
  {
    \@@_if_file_hook:wTF #1 / / \s_@@_mark
      {
        \exp_args:Ne \@@_if_exist_use:n
          { \exp_args:Ne \@@_file_hook_normalise:n {#1} }
      }
      { \@@_if_exist_use:n {#1} } % file/ generic hook (e.g. file/before)
  }
\cs_new_protected:Npn \@@_if_exist_use:n #1
  {
    \tl_if_exist:cT { g_@@_#1_code_tl }
      {
        \@@_preamble_hook:n {#1}
        \cs:w g_@@_#1_code_tl \cs_end:
      }
  }
%    \end{macrocode}
% \end{macro}
% \end{macro}
%
%  \begin{macro}{\hook_use_once:n}
%    For hooks that can and should be used only once we have a special
%    use command that remembers the hook name in
%    \cs{g_@@_execute_immediately_clist}. This has the effect that any
%    further code added to the hook is executed immediately rather
%    than stored in the hook.
%    \begin{macrocode}
\cs_new_protected:Npn \hook_use_once:n #1
  {
    \tl_if_exist:cT { g_@@_#1_code_tl }
      {
        \clist_gput_left:Nn \g_@@_execute_immediately_clist {#1}
        \hook_use:n {#1}
      }
  }
%    \end{macrocode}
%  \end{macro}

% \subsection{Querying a hook}
%
% Simpler data types, like token lists, have three possible states; they
% can exist and be empty, exist and be non-empty, and they may not
% exist, in which case emptiness doesn't apply (though
% \cs{tl_if_empty:N} returns false in this case).
%
% Hooks are a bit more complicated: they have four possible states.
% A hook may exist or not, and either way it may or may not be empty
% (even a hook that doesn't exist may be non-empty).
%
% A hook is said to be empty when no code was added to it, either to
% its permanent code pool, or to its ``next'' token list.  The hook
% doesn't need to be declared to have code added to its code pool
% (it may happen that a package $A$ defines a hook \hook{foo}, but
% it's loaded after package $B$, which adds some code to that hook.
% In this case it is important that the code added by package $B$ is
% remembered until package $A$ is loaded).
%
% A hook is said to exist when it was declared with \cs{hook_new:n} or
% some variant thereof.
%
% \begin{macro}[pTF]{\hook_if_empty:n}
%   Test if a hook is empty (that is, no code was added to that hook).
%   A hook being empty means that \emph{both} its
%   \cs[no-index]{g_@@_\meta{hook}_code_prop} and its
%   \cs[no-index]{g_@@_\meta{hook}_next_code_tl} are empty.
%    \begin{macrocode}
\prg_new_conditional:Npnn \hook_if_empty:n #1 { p , T , F , TF }
  {
    \@@_if_exist:nTF {#1}
      {
        \bool_lazy_and:nnTF
            { \prop_if_empty_p:c { g_@@_#1_code_prop } }
            { \tl_if_empty_p:c { g_@@_#1_next_code_tl } }
          { \prg_return_true: }
          { \prg_return_false: }
      }
      { \prg_return_true: }
  }
%    \end{macrocode}
% \end{macro}

% \begin{macro}[pTF]{\hook_if_exist:n}
%   A canonical way to test if a hook exists.  A hook exists if the
%   token list that stores the sorted code for that hook,
%   \cs[no-index]{g_@@_\meta{hook}_code_tl}, exists.  The property list
%   \cs[no-index]{g_@@_\meta{hook}_code_prop} cannot be used here
%   because often it is necessary to add code to a hook without knowing
%   if such hook was already declared, or even if it will ever be
%   (for example, in case the package that defines it isn't loaded).
%    \begin{macrocode}
\prg_new_conditional:Npnn \hook_if_exist:n #1 { p , T , F , TF }
  {
    \tl_if_exist:cTF { g_@@_#1_code_tl }
      { \prg_return_true: }
      { \prg_return_false: }
  }
%    \end{macrocode}
% \end{macro}

% \begin{macro}[pTF]{\@@_if_exist:n}
%   An internal check if the hook has already been declared with
%   \cs{@@_declare:n}.  This means that the hook was already used somehow
%   (a code chunk or rule was added to it), but it still wasn't declared
%   with \cs{hook_new:n}.
%    \begin{macrocode}
\prg_new_conditional:Npnn \@@_if_exist:n #1 { p , T , F , TF }
  {
    \prop_if_exist:cTF { g_@@_#1_code_prop }
      { \prg_return_true: }
      { \prg_return_false: }
  }
%    \end{macrocode}
% \end{macro}
%
% \begin{macro}[pTF]{\@@_if_reversed:n}
%   An internal conditional that checks if a hook is reversed.
%    \begin{macrocode}
\prg_new_conditional:Npnn \@@_if_reversed:n #1 { p , T , F , TF }
  {
    \if_int_compare:w \cs:w g_@@_#1_reversed_tl \cs_end: 1 < 0 \exp_stop_f:
      \prg_return_true:
    \else:
      \prg_return_false:
    \fi:
  }
%    \end{macrocode}
% \end{macro}


%  \begin{macro}{\g_@@_execute_immediately_clist}
%    List of hooks that from no on should not longer receive code.
%    \begin{macrocode}
\clist_new:N \g_@@_execute_immediately_clist
%    \end{macrocode}
%  \end{macro}
%
%  \subsection{Messages}
%
%    \begin{macrocode}
\msg_new:nnnn { hooks } { labels-incompatible }
  {
    Labels~`#1'~and~`#2'~are~incompatible
    \str_if_eq:nnF {#3} {??} { ~in~hook~`#3' } .~
    \int_compare:nNnT {#4} = { 1 }
      { The~code~for~both~labels~will~be~dropped. }
  }
  {
    LaTeX~found~two~incompatible~labels~in~the~same~hook.~
    This~indicates~an~incompatibility~between~packages.
  }
\msg_new:nnn { hooks } { empty-label }
  { Empty~code~label~\msg_line_context:.~Using~`#1'~instead. }
%    \end{macrocode}
%
%  \subsection{\LaTeXe{} package interface commands}
%


%  \begin{macro}{\NewHook,\NewReversedHook,\NewMirroredHookPair}
%    Declaring new hooks \ldots
%    \begin{macrocode}
\NewDocumentCommand \NewHook             { m }{ \hook_new:n {#1} }
\NewDocumentCommand \NewReversedHook     { m }{ \hook_new_reversed:n {#1} }
\NewDocumentCommand \NewMirroredHookPair { mm }{ \hook_new_pair:nn {#1}{#2} }
%    \end{macrocode}
%  \end{macro}

%  \begin{macro}{\AddToHook}
%    
%    \begin{macrocode}
\NewDocumentCommand \AddToHook { m o +m }
  {
    \clist_if_in:NnTF \g_@@_execute_immediately_clist {#1}
      {#3}
      { \hook_gput_code:nnn {#1} {#2} {#3} }
  }
%    \end{macrocode}
%  \end{macro}

%  \begin{macro}{\AddToHookNext}
%    
%    \begin{macrocode}
\NewDocumentCommand \AddToHookNext { m +m }
  { \hook_gput_next_code:nn {#1} {#2} }
%    \end{macrocode}
%  \end{macro}


%  \begin{macro}{\RemoveFromHook}
%    
%    \begin{macrocode}
\NewDocumentCommand \RemoveFromHook { m o }
  { \hook_gremove_code:nn {#1} {#2} }
%    \end{macrocode}
%  \end{macro}
%
% \begin{macro}{\DeclareDefaultHookLabel}
% \begin{macro}{\@@_curr_name_push:n,\@@_curr_name_pop:}
%   The token list \cs{g_@@_hook_curr_name_tl} stores the name of the
%   current package/file to be used as label for hooks.
%   Providing a consistent interface is tricky, because packages can
%   be loaded within packages, and some packages may not use
%   \cs{DeclareDefaultHookLabel} to change the default label (in which case
%   \cs{@currname} is used, if set).
%
%   To pull that off, we keep a stack that contains the default label
%   for each level of input.  The bottom of the stack contains the
%   default label for the top-level.  Since the string \verb|top-level|
%   is hardcoded, here this item of the stack is empty.  Also, since
%   we're in an input level, add \verb|lthooks| to the stack as well.
%   This stack should never go empty, so we loop through \LaTeXe's
%   file name stack, and add empty entries to \cs{g_@@_name_stack_seq}
%   for each item in that stack.  The last item is the \verb|top-level|,
%   which also gets an empty entry.
%
%   Also check for the case we're loading \texttt{lthooks} in the
%   \LaTeXe{} kernel.  In that case, \cs{@currname} isn't \verb|lthooks|
%   and just the top-level is added to the stack as an empty entry.
%    \begin{macrocode}
\str_if_eq:VnTF \@currname { lthooks }
  {
    \seq_gpush:Nn \g_@@_name_stack_seq { lthooks }
    \cs_set_protected:Npn \@@_tmp:w #1 #2 #3
      {
        \quark_if_recursion_tail_stop:n {#1}
        \seq_gput_right:Nn \g_@@_name_stack_seq { }
        \@@_tmp:w
      }
    \exp_after:wN \@@_tmp:w
      \@currnamestack
      \q_recursion_tail \q_recursion_tail
      \q_recursion_tail \q_recursion_stop
  }
  { \seq_gpush:Nn \g_@@_name_stack_seq { } }
%    \end{macrocode}
%
%   Two commands keep track of the stack: when a file is input,
%   \cs{@@_curr_name_push:n} pushes an (empty by default) label to the
%   stack:
%    \begin{macrocode}
\cs_new_protected:Npn \@@_curr_name_push:n #1
  {
    \seq_gpush:Nn \g_@@_name_stack_seq {#1}
    \tl_gset:Nn \g_@@_hook_curr_name_tl {#1}
  }
%
%    \end{macrocode}
%   and when an input is over, the topmost item of the stack is popped,
%   since the label will not be used again, and \cs{g_@@_hook_curr_name_tl}
%   is updated to the now topmost item of the stack:
%    \begin{macrocode}
\cs_new_protected:Npn \@@_curr_name_pop:
  {
    \seq_gpop:NN \g_@@_name_stack_seq \l_@@_return_tl
    \seq_get:NNTF \g_@@_name_stack_seq \l_@@_return_tl
      { \tl_gset_eq:NN \g_@@_hook_curr_name_tl \l_@@_return_tl }
      { \ERROR_should_not_happen }
  }
%    \end{macrocode}
%
%   The token list \cs{g_@@_hook_curr_name_tl} is but a mirror of the top
%   of the stack.
%
%   Now define a wrapper that replaces the top of the stack with the
%   argument, and updates \cs{g_@@_hook_curr_name_tl} accordingly.
%    \begin{macrocode}
\NewDocumentCommand \DeclareDefaultHookLabel { m }
  {
    \seq_gpop:NN \g_@@_name_stack_seq \l_@@_return_tl
    \@@_curr_name_push:n {#1}
  }
%    \begin{macrocode}
%
%   The push and pop macros are injected in \cs{@pushfilename} and
%   \cs{@popfilename} so that they correctly keep track of the label.s
%    \begin{macrocode}
% TODO! \pho{Properly integrate in the kernel}
\tl_gput_left:Nn \@pushfilename { \@@_curr_name_push:n { } }
\tl_gput_left:Nn \@popfilename { \@@_curr_name_pop: }
% TODO! \pho{Properly integrate in the kernel}
%    \end{macrocode}
% \end{macro}
% \end{macro}
%
%
%
%
%  \begin{macro}{\UseHook}
%    Avoid the overhead of \pkg{xparse} and its protection that we
%    don't want here (since the hook should vanish without trace if empty)!
%    \begin{macrocode}
\newcommand \UseHook { \hook_use:n }
%    \end{macrocode}
%  \end{macro}
%
%  \begin{macro}{\UseOneTimeHook}
%
%    \begin{macrocode}
\cs_new_protected:Npn \UseOneTimeHook { \hook_use_once:n }
%    \end{macrocode}
%  \end{macro}



%  \begin{macro}{\ShowHook}
%    
%    \begin{macrocode}
\cs_new_protected:Npn \ShowHook { \hook_log:n }
%    \end{macrocode}
%  \end{macro}

%  \begin{macro}{\DebugHookOn,\DebugHookOff}
%    
%    \begin{macrocode}
\cs_new_protected:Npn \DebugHookOn { \hook_debug_on: }
\cs_new_protected:Npn \DebugHookOff { \hook_debug_off: }
%    \end{macrocode}
%  \end{macro}



%  \begin{macro}{\DeclareHookRule}
%    
%    \begin{macrocode}
\NewDocumentCommand \DeclareHookRule { m m m m }
{ \hook_gset_rule:nnnn {#1}{#2}{#3}{#4} }
%    \end{macrocode}
%  \end{macro}

%  \begin{macro}{\DeclareDefaultHookRule}
%    This declaration is only supported before \verb=\begin{document}=.
%    \begin{macrocode}
\NewDocumentCommand \DeclareDefaultHookRule { m m m }
                    { \hook_gset_rule:nnnn {??}{#1}{#2}{#3} }
\@onlypreamble\DeclareDefaultHookRule                    
%    \end{macrocode}
%  \end{macro}
%
%  \begin{macro}{\ClearHookRule}
%    A special setup rule that removes an existing relation.
%    Basically {@@_rule_gclear:nnn} plus fixing the property list for debugging.
%    \fmi{Need an L3 interface, or maybe it should get dropped}
%    \begin{macrocode}
\NewDocumentCommand \ClearHookRule { m m m }
{ \hook_gset_rule:nnnn {#1}{#2}{unrelated}{#3} }
%    \end{macrocode}
%  \end{macro}
%
%
% \begin{macro}{\IfHookExistTF,\IfHookEmptyTF}
%    \begin{macrocode}
\NewExpandableDocumentCommand \IfHookExistTF { m }
  { \hook_if_exist:nTF {#1} }
\NewExpandableDocumentCommand \IfHookEmptyTF { m }
  { \hook_if_empty:nTF {#1} }
%    \end{macrocode}
% \end{macro}




%  \begin{macro}{\AtBeginDocument}
%    
%    \begin{macrocode}
\renewcommand\AtBeginDocument{\AddToHook{begindocument}}
%    \end{macrocode}
%  \end{macro}

%  \begin{macro}{\AtEndDocument}
%    
%    \begin{macrocode}
\renewcommand\AtEndDocument {\AddToHook{enddocument}}
%\renewcommand\AtEndDocument {\AddToHook{env/document/end}} % alternative impl
%    \end{macrocode}
%    
%  \end{macro}



%  \subsection{Set up existing \LaTeXe{} hooks}
%
%    As we are in a package calling \cs{NewHook} would label any
%    already set up hook code under the package name, but we want it
%    under the name \hook{top-level} so we pretend that \cs{@currname}
%    is empty.
%    \begin{macrocode}
\begingroup
  \def\@currname{}
%    \end{macrocode}
%    
%    \begin{macrocode}
  \NewHook{begindocument}
  \NewHook{enddocument}
%    \end{macrocode}
%    We need to initialize the mechanism at \verb=\begin{document}=
%    but obviously before everything else, so we sneak\footnote{This
%    needs to move to \cs{document} directly.}
%    \cs{@@_initialize_all:} into the \LaTeXe{} hook name.
%
%    We can't use \cs{tl_gput_left:Nn} because that complains about
%    \cs{@begindocumenthook} not starting with \cs{g_} so we do this
%    through the backdoor.
%    \begin{macrocode}
%  \tex_global:D\tl_put_left:Nn \@begindocumenthook
%      {\@@_initialize_all:}
%    \end{macrocode}
%    There aren't many other hooks at the moment:
%    \begin{macrocode}
  \NewHook{rmfamily}
  \NewHook{sffamily}
  \NewHook{ttfamily}
  \NewHook{defaultfamily}
%    \end{macrocode}
%    Not checked what this one does and whether it should be there (or
%    is a real ``hook''.
%    \begin{macrocode}
  \NewHook{documentclass}
%    \end{macrocode}
%    
%    \begin{macrocode}
\endgroup
%    \end{macrocode}
%
%
%
% \section{Generic hooks for environments}
%
%
%    \begin{macrocode}
\let\begin\relax  % avoid redeclaration message
%    \end{macrocode}
%    
%    \begin{macrocode}
\DeclareRobustCommand\begin[1]{%
  \UseHook{env/#1/before}%
  \@ifundefined{#1}%
    {\def\reserved@a{\@latex@error{Environment~#1~undefined}\@eha}}%
    {\def\reserved@a{\def\@currenvir{#1}%
        \edef\@currenvline{\on@line}%
        \@execute@begin@hook{#1}%
        \csname #1\endcsname}}%
  \@ignorefalse
  \begingroup\@endpefalse\reserved@a}
%    \end{macrocode}
%
%    Before the \cs{document} code is executed we have to first undo
%    the \cs{endgroup} as there should be none for this environment to
%    avoid that changes on top-level unnecessarily go to \TeX's
%    savestack, and we have to initialize all hooks in the hook system.
%    So we need to test for this environment name. But once it has be
%    found all this testing is no longer needed and so we redefine
%    \cs{@execute@begin@hook} to simply use the hook
%    \begin{macrocode}
\def\@execute@begin@hook #1{%
  \expandafter\ifx\csname #1\endcsname\document
    \endgroup
    \gdef\@execute@begin@hook##1{\UseHook{env/##1/begin}}%
    \@@_initialize_all:
    \@execute@begin@hook{#1}%
%    \end{macrocode}
%    If this is an environment before \verb=\begin{document}= we just
%    run the hook.
%    \begin{macrocode}
  \else
    \UseHook{env/#1/begin}%
  \fi
}    
%    \end{macrocode}
%    
%    \begin{macrocode}
\@namedef{end~}#1{%
  \UseHook{env/#1/end}%
  \csname end#1\endcsname\@checkend{#1}%
  \expandafter\endgroup\if@endpe\@doendpe\fi
  \UseHook{env/#1/after}%
  \if@ignore\@ignorefalse\ignorespaces\fi}%
%    \end{macrocode}
%    Version that fixes tlb3722 but the change should perhaps be made in
%    \pkg{tabularx} instead.
%    \begin{macrocode}
\@namedef{end~}#1{%
\romannumeral
\IfHookEmptyTF{env/#1/end}%
  {\expandafter\z@}%
  {\z@\UseHook{env/#1/end}}%
\csname end#1\endcsname\@checkend{#1}%
\expandafter\endgroup\if@endpe\@doendpe\fi
\UseHook{env/#1/after}%
\if@ignore\@ignorefalse\ignorespaces\fi}%
%    \end{macrocode}
%    
%    
%
%    We provide 4 high-level hook interfaces directly, the others only when
%    etoolbox is loaded
%    \begin{macrocode}
\newcommand\AtBeginEnvironment[1]    {\AddToHook{env/#1/begin}}
\newcommand\AtEndEnvironment[1]      {\AddToHook{env/#1/end}}
\newcommand\BeforeBeginEnvironment[1]{\AddToHook{env/#1/before}}
\newcommand\AfterEndEnvironment[1]   {\AddToHook{env/#1/after}}
%    \end{macrocode}
%    
%    
%    
%    
%    
%    
%    
%    
%    
%    
%
% \section{Generic hooks for file loads}
%
%
%
%
%
%    
%
% \section{Hooks in \cs{begin} document}
%
%    Can't have \texttt{@{}@} notation here as this is \LaTeXe{} code
%    \ldots{} and makes for puzzling errors if the double \texttt{@}
%    signs get substituted.
%    \begin{macrocode}
%<@@=>
\ExplSyntaxOff
%    \end{macrocode}
%
%    The \hook{begindocument} hook was already set up earlier, here is now
%    the additional one (which was originally from the \pkg{etoolbox}
%    package under the name \texttt{afterpreamble}.
%    \begin{macrocode}
\NewHook{begindocument/end}
%    \end{macrocode}
%
%

%  \begin{macro}{\document}
%    
%    \begin{macrocode}
\def\document{%
%    \end{macrocode}
%    We do cancel the grouping as part of the \cs{begin} handling
%    (this is now done inside \cs{begin} instead) so that the
%    \hook{env/\meta{env}/begin} hook is not hidden inside \cs{begingroup}
%    \texttt{...} \cs{endgroup}.
%    \begin{macrocode}
%  \endgroup
%    \end{macrocode}
%    
%    \begin{macrocode}
  \@kernel@after@env@document@begin
%    \end{macrocode}
%
% Added hook to load \textsf{l3backend} code:
%    \begin{macrocode}
  \@expl@sys@load@backend@@
  \ifx\@unusedoptionlist\@empty\else
    \@latex@warning@no@line{Unused global option(s):^^J%
            \@spaces[\@unusedoptionlist]}%
  \fi
  \@colht\textheight
  \@colroom\textheight \vsize\textheight
  \columnwidth\textwidth
  \@clubpenalty\clubpenalty
  \if@twocolumn
    \advance\columnwidth -\columnsep
    \divide\columnwidth\tw@ \hsize\columnwidth \@firstcolumntrue
  \fi
  \hsize\columnwidth \linewidth\hsize
  \begingroup\@floatplacement\@dblfloatplacement
    \makeatletter\let\@writefile\@gobbletwo
    \global \let \@multiplelabels \relax
    \@input{\jobname.aux}%
  \endgroup
  \if@filesw
    \immediate\openout\@mainaux\jobname.aux
    \immediate\write\@mainaux{\relax}%
  \fi
  \process@table
  \let\glb@currsize\@empty  % Force math initialization.
  \normalsize
  \everypar{}%
  \ifx\normalsfcodes\@empty
    \ifnum\sfcode`\.=\@m
      \let\normalsfcodes\frenchspacing
    \else
      \let\normalsfcodes\nonfrenchspacing
    \fi
  \fi
  \ifx\document@default@language\m@ne
    \chardef\document@default@language\language
  \fi
  \@noskipsecfalse
  \let \@refundefined \relax
%    \end{macrocode}
%    
%    \begin{macrocode}
%  \let\AtBeginDocument\@firstofone
%  \@begindocumenthook
  \UseOneTimeHook{begindocument}%
  \@kernel@after@begindocument
%    \end{macrocode}
%    
%    \begin{macrocode}
  \ifdim\topskip<1sp\global\topskip 1sp\relax\fi
  \global\@maxdepth\maxdepth
  \global\let\@begindocumenthook\@undefined
  \ifx\@listfiles\@undefined
    \global\let\@filelist\relax
    \global\let\@addtofilelist\@gobble
  \fi
  \gdef\do##1{\global\let ##1\@notprerr}%
  \@preamblecmds
  \global\let \@nodocument \relax
  \global\let\do\noexpand
%    \end{macrocode}
%    
%    \begin{macrocode}
  \UseOneTimeHook{begindocument/end}%
  \ignorespaces}
%    \end{macrocode}
%    
%    \begin{macrocode}
\let\@kernel@after@begindocument\@empty
%    \end{macrocode}
%  \end{macro}
%
%
%  \begin{macro}{\@kernel@after@env@document@begin,\@kernel@hook@begindocument}
%    
%    \begin{macrocode}
\edef \@kernel@after@env@document@begin{%
  \let\expandafter\noexpand\csname
       g__hook_env/document/begin_code_tl\endcsname
  \noexpand\@empty}
%    \end{macrocode}
%    \begin{macrocode}
\let\@kernel@hook@begindocument\@empty
%    \end{macrocode}
%    
%  \end{macro}
%
%
% \section{Hooks in \cs{enddocument}}
%
%
%    
%    The \hook{enddocument} hook was already set up earlier, here are now
%    the additional ones:
%    \begin{macrocode}
\NewHook{enddocument/afterlastpage}
\NewHook{enddocument/afteraux}
\NewHook{enddocument/info}
\NewHook{enddocument/end}
%    \end{macrocode}



%  \begin{macro}{\enddocument}
%    
%    
%    \begin{macrocode}
\def\enddocument{%
   \UseHook{enddocument}%
   \@kernel@after@enddocument
   \@checkend{document}%
   \clearpage
   \UseHook{enddocument/afterlastpage}%
   \@kernel@after@enddocument@afterlastpage
   \begingroup
     \if@filesw
       \immediate\closeout\@mainaux
       \let\@setckpt\@gobbletwo
       \let\@newl@bel\@testdef
       \@tempswafalse
       \makeatletter \@@input\jobname.aux
     \fi
     \UseHook{enddocument/afteraux}%
%    \end{macrocode}
%    Next hook is expect to contain only code for writing info
%    messages on the terminal.
%    \begin{macrocode}
     \UseHook{enddocument/info}%
   \endgroup
   \UseHook{enddocument/end}%
   \deadcycles\z@\@@end}
%    \end{macrocode}
%    The two kernel hooks above are used by the shipout code.   
%    \begin{macrocode}
\let\@kernel@after@enddocument\@empty
\let\@kernel@after@enddocument@afterlastpage\@empty
%    \end{macrocode}
%  \end{macro}
%
%
%
%  \begin{macro}{\@enddocument@kernel@warnings}
%    
%    \begin{macrocode}
\def\@enddocument@kernel@warnings{%
   \ifdim \font@submax >\fontsubfuzz\relax
     \@font@warning{Size substitutions with differences\MessageBreak
                up to \font@submax\space have occurred.\@gobbletwo}%
   \fi
   \@defaultsubs
   \@refundefined
   \if@filesw
     \ifx \@multiplelabels \relax
       \if@tempswa
         \@latex@warning@no@line{Label(s) may have changed.
             Rerun to get cross-references right}%
       \fi
     \else
       \@multiplelabels
     \fi
   \fi
}
%    \end{macrocode}
%    
%    \begin{macrocode}
\AddToHook{enddocument/info}[kernel/filelist]{\@dofilelist}
\AddToHook{enddocument/info}[kernel/warnings]{\@enddocument@kernel@warnings}
\DeclareHookRule{enddocument/info}{kernel/filelist}{before}{kernel/warnings}
%    \end{macrocode}
%  \end{macro}
%
%
% \subsection{Adjusting at \pkg{atveryend} interfaces}
%
%    With the new hook management all of \pkg{atveryend} is taken care
%    of.
%
%    We therefore prevent the package from loading:
%    \begin{macrocode}
\expandafter\let\csname ver@atveryend.sty\endcsname\fmtversion
%    \end{macrocode}
%
%
%    Here are new definitions for its interfaces now pointing to the
%    hooks in \cs{enddocument}
%    \begin{macrocode}
\newcommand\AfterLastShipout  {\AddToHook{enddocument/afterlastpage}}
\newcommand\AtVeryEndDocument {\AddToHook{enddocument/afteraux}}
%    \end{macrocode}
%    Next one is a bit of a fake, but the result should normally be as
%    expected. If not one needs to add a rule to sort the code chunks
%    in \hook{enddocument/info}.
%    \begin{macrocode}
\newcommand\AtEndAfterFileList{\AddToHook{enddocument/info}}
%    \end{macrocode}
%    
%    \begin{macrocode}
\newcommand\AtVeryVeryEnd     {\AddToHook{enddocument/end}}
%    \end{macrocode}

%  \begin{macro}{\BeforeClearDocument}
%    This one is the only one we don't implement or rather don't have
%    a dedicated hook in the code.
%    \begin{macrocode}
\ExplSyntaxOn
\newcommand\BeforeClearDocument[1]
  { \AtEndDocument{#1}
    \@DEPRECATED{BeforeClearDocument \tl_to_str:n{#1}}
  }
\cs_new:Npn\@DEPRECATED #1
   {\iow_term:x{======~DEPRECATED~USAGE~#1~==========}}
%    \end{macrocode}
%    
%    \begin{macrocode}
\ExplSyntaxOff
%    \end{macrocode}
%  \end{macro}
%
%    \begin{macrocode}
%</2ekernel>
%    \end{macrocode}

%
% \section{A package version of the code for testing}
%

%    \begin{macrocode}
%<*package>
%    \end{macrocode}



%    \begin{macrocode}
\RequirePackage{xparse}
\ProvidesExplPackage{lthooks}{\lthooksdate}{\lthooksversion}
                    {Hook management interface for LaTeX2e}
%    \end{macrocode}
%
%
% \subsection{Core hook management code (kernel part)}
%
%    This should run in older formats so we can't use
%    \cs{IfFormatAtLeastTF} right now.
%    \begin{macrocode}
\@ifl@t@r\fmtversion{2020/10/01}
                    {}
                    {\input{lthooks.ltx}
                     \input{ltshipout.ltx}
                     \input{ltfilehook.ltx}
                    }
%    \end{macrocode}
%
%  \subsection{Package options}
%
%    For now we offer a simple debug option which turns on a lot of
%    strange \cs{typeout} messages, nothing fancy.
%    \begin{macrocode}
\ExplSyntaxOn
%    \end{macrocode}
%    
%    \begin{macrocode}
\hook_debug_off:
\DeclareOption { debug } { \hook_debug_on:
                           \shipout_debug_on: }
%    \end{macrocode}
%
%
%    For now we offer a simple debug option which turns on a lot of
%    strange \cs{typeout} messages, nothing fancy.
%    \begin{macrocode}
\shipout_debug_off:
\DeclareOption { debug-shipout } { \shipout_debug_on: }
%    \end{macrocode}
%    
%    \begin{macrocode}
\ProcessOptions
%    \end{macrocode}
%
%
%  \subsection{Temporarily patching package until changed}
%
%
%
%    \pkg{filehook} support until that package is patched:
%    \begin{macrocode}
\RequirePackage{filehook-ltx}
%    \end{macrocode}
%
%
%
%    \begin{macrocode}
%</package>
%    \end{macrocode}
%
%
%
% \Finale
%



                     % \iffalse meta-comment
%
%% File: ltshipout.dtx (C) Copyright 2020 Frank Mittelbach, LaTeX Team
%
% It may be distributed and/or modified under the conditions of the
% LaTeX Project Public License (LPPL), either version 1.3c of this
% license or (at your option) any later version.  The latest version
% of this license is in the file
%
%    https://www.latex-project.org/lppl.txt
%
%
% The development version of the bundle can be found below
%
%    https://github.com/FrankMittelbach/...
%
% for those people who are interested or want to report an issue.
%
%    \begin{macrocode}
\providecommand\ltshipoutversion{v0.9a}
\providecommand\ltshipoutdate{2020/07/15}
%    \end{macrocode}
%
%<*driver>

\documentclass{l3doc}

% bug fix fo l3doc.cls
\ExplSyntaxOn
\cs_set_protected:Npn \__codedoc_macro_typeset_one:nN #1#2
  {
    \vbox_set:Nn \l__codedoc_macro_box
      {
        \vbox_unpack_drop:N \l__codedoc_macro_box
        \hbox { \llap { \__codedoc_print_macroname:nN {#1} #2
            \MacroFont       % <----- without it the \ is in lmr10 if a link is made
            \      
        } }
      }
    \int_incr:N \l__codedoc_macro_int
  }
\ExplSyntaxOff

\EnableCrossrefs
\CodelineIndex
\begin{document}
  \DocInput{ltshipout.dtx}
\end{document}
%</driver>
%
% \fi
%
%
% \long\def\fmi#1{\begin{quote}\itshape Todo: #1\end{quote}}
%
% \newcommand\hook[1]{\texttt{#1}}
%
%
% \title{The \texttt{ltshipout} package\thanks{This package has version
%    \ltshipoutversion\ dated \ltshipoutdate, \copyright\ \LaTeX\
%    Project.}}
%
% \author{Frank Mittelbach}
%
% \maketitle
%
%
% \tableofcontents
%
% \section{Introduction}
%
%    The code provides an interface to the \cs{shipout} primitive of
%    \TeX{} which is called when a finished pages is finally
%    \enquote{shipped out} to the target output file, e.g., the
%    \texttt{.dvi} or \texttt{.pdf} file.
%    A good portion of the code is based on ideas by Heiko Oberdiek
%    implemented in his packages \pkg{atbegshi} and \pkg{atenddvi}
%    even though the interfaces are somewhat
%    different.\footnote{Heiko's interfaces are emulated by the kernel
%    code, if a document requests his packages, so older documents
%    will continue to work.}
%
%  \subsection{Overloading the \cs{shipout} primitive}
%
%
% \begin{function}{\shipout}
%    With this implementation \TeX's shipout primitive is no longer
%    available for direct use. Instead \cs{shipout} is running some
%    (complicated) code that picks up the box to be shipped out
%    regardless of how that is done, i.e., as a constructed \cs{vbox}
%    or \cs{hbox} or as a box register.
%
%    It then stores it in a named box register.  This box can then be
%    manipulated through a set of hooks after which it is shipped out
%    for real.
% \end{function}

%  \begin{variable}{\ShipoutBox,\l_shipout_box}
%    This box register is called \cs{ShipoutBox} (alternatively available via the
%    L3 name \cs{l_shipout_box}).
%  \end{variable}
%

%  \begin{variable}{\l_shipout_box_ht_dim,
%                   \l_shipout_box_dp_dim,\l_shipout_box_wd_dim,
%                   \l_shipout_box_ht_plus_dp_dim}
%    The shipout box dimensions are available in the L3 registers
%    \cs{l_shipout_box_ht_dim}, etc.\ (there are no \LaTeXe{}
%    names).\footnotemark{} These variables can be used
%    inside the hook code for \hook{shipout/before},
%    \hook{shipout/foreground} and \hook{shipout/background} if needed.
%  \end{variable}
%  \footnotetext{Might need changing, but HO's version as strings
%    is not really helpful I think).}
%
%
%
%
% \subsection{Provided hooks}
%
%  \begin{variable}{shipout/before,
%                   shipout/foreground,shipout/background,
%                   shipout/firstpage,
%                   shipout/lastpage}
%    The code offers a number of hooks into which packages (or the
%    user) can add code to support different use cases.
%    These are:
%    \begin{description}
%    \item[\hook{shipout/before}]
%
%       This hook is executed after the finished page has been stored in
%       \cs{ShipoutBox} / \cs{l_shipout_box}).
%       It can be used to alter that box content or to discard it
%       completely (see \cs{DiscardShipoutBox} below).
%
%    \item[\hook{shipout/background}]
%
%       This hook adds a picture environment into the background of
%       the page (with the \texttt{(0,0)} coordinate in the top-left
%       corner using a \cs{unitlength} of \texttt{1pt}.
%
%       It should therefore only receive \cs{put} commands or other
%       commands suitable in a \texttt{picture} environment and the
%       vertical coordinate values would normally be
%       negative.
%
%    \fmi{Again, mainly \pkg{atbegshi} compatibility. Not
%       sure it is best to have  to always use negative
%       coordinates.}
%
%       Technically this is implemented by adding a zero-sized
%       \cs{hbox} as the very first item into the \cs{ShipoutBox}
%       containing that \texttt{picture} environment. Thus the rest of
%       the box content will overprint what ever is typeset by that hook.
%
%
%    \item[\hook{shipout/foreground}]
%
%       This hook adds a picture environment into the foreground of
%       the page (with the \texttt{(0,0)} coordinate in the top-left
%       corner using a \cs{unitlength} of \texttt{1pt}.
%
%       Technically this is implemented by adding a zero-sized
%       \cs{hbox} as the very last item into the \cs{ShipoutBox} and
%       raising it up so that it still has its \texttt{(0,0)} point in
%       the top-left corner.
%       But being placed after the main box content it will be typeset
%       later and thus overprints it (i.e., is in the foreground).
%
%
%    \item[\hook{shipout/firstpage}]
%
%       The material from this hook is executed only once at the very
%       beginning of the first output page. It should only contain
%       \cs{special} commands needed to direct post processors
%       handling the \texttt{.dvi} or \texttt{.pdf} output.\fmi{not
%       sure it has to be that restrictive.}
%
%       In \LaTeXe{} that was already existing but implemented as a box
%       register \cs{@begindvibox}.
%       \fmi{drop or at least mark the obsolete code in latex.ltx}
%
%
%    \item[\hook{shipout/lastpage}]
%
%       The corresponding hook to add \cs{special}s at the very end of
%       the output file. It is only entered on the very last page.
%
%       It may not be possible for \LaTeX{} to correctly determine which page is
%       the last one without several reruns. If this happens and the
%       hook is non-empty then \LaTeX{} will add an extra page to
%       place the material and also request a rerun to get the correct
%       placement sorted out.
%
%    \end{description}
%  \end{variable}
%
%    As mentioned above the hook \hook{shipout/before} is executed
%    first and can manipulate the prepared shipout box stored in
%    \cs{ShipoutBox} or set things up for use in \cs{write} during the
%    actual shipout. The other hooks are added inside hboxes to the
%    box being shipped out in the following
%    order:
%    \begin{center}
%    \begin{tabular}{ll}
%       \hook{shipout/firstpage}   & only on the first page \\
%       \hook{shipout/background}  &                        \\
%       \meta{boxed content of \cs{ShipoutBox}} &             \\
%       \hook{shipout/foreground}  &                       \\
%       \hook{shipout/lastpage}    & only on the last page \\
%    \end{tabular}
%    \end{center}
%    If any of the hooks has no code then that particular no box is
%    added at that point.
%
%    In a document that doesn't produce pages, e.g., only makes
%    \cs{typeout}s, none of the hooks are executed (as there is no
%    \cs{shipout}) not even the \hook{shipout/lastpage} hook.
%
% \begin{function}{\AtBeginDvi,\AtEndDvi}
%    \cs{AtBeginDvi} is the existing \LaTeXe{} interface to fill the
%    \hook{shipout/firstpage} hook. This is not really a good name
%    as it is not just supporting \texttt{.dvi} but also \texttt{.pdf}
%    output or \texttt{.dvx}.
%
%    \cs{AtEndDvi} is the counterpart that was not available in the
%    kernel but only through the package \pkg{atenddvi}. It fills the
%    \hook{shipout/lastpage} hook.
%
%    \fmi{better names? Any suggestions?}
%
% \end{function}
%
% \subsection{Special commands for use inside the hooks}
%
% \begin{function}{\DiscardShipoutBox,\shipout_discard_box:}
%   \begin{syntax}
%     \cs{AddToHookNext} \texttt{\{shipout/before\} \{...\cs{DiscardShipoutBox}...\}}
%   \end{syntax}
%    The \cs{DiscardShipoutBox} declaration (L3 name
%    \cs{shipout_discard_box:})
%    requests that on the next
%    shipout the page box is thrown away instead of being shipped to
%    the \texttt{.dvi} or \texttt{.pdf} file.
%
%    Typical applications wouldn't do this unconditionally, but have
%    some processing logic that decides to use or not to use the page.
%
%    Note that if this declaration is used directly in the document it
%    may depend on the placement to which page it applies, given that
%    \LaTeX{} output routine is called in an asynchronous manner!
%
%    \fmi{Once we have a new mark mechanism available we can improve
%    on that and make sure that the declaration applies to the page
%    that contains it.}
%  \end{function}
%
%    In the \pkg{atbegshi} package there are a number of additional
%    commands for use inside the \hook{shipout/before} hook. They
%    should normally not be needed any more as one can instead simply
%    add code to the hooks \hook{shipout/before},
%    \hook{shipout/background} or
%    \hook{shipout/foreground}.\footnote{If that assumption turns out to
%    be wrong it would be trivial to change them to public functions
%    (right now they are private).} If \pkg{atbegshi} gets loaded then
%    those commands become available as public functions with their original
%    names as given below.
%
%
% \subsection{Information counters}
%
%
%  \begin{variable}{\ReadonlyShipoutCounter,\g_shipout_readonly_int}
%   \begin{syntax}
%     \cs{ifnum}\cs{ReadOnlyShipoutCounter}\texttt{=...}
%     \cs{int_use:N} \cs{g_shipout_readonly_int} \texttt{\% expl3 usage}
%   \end{syntax}
%    This integer holds the number of pages shipped out up to now
%    (including the one to be shipped out when inside the output
%    routine). More precisely, it is incremented only after it is
%    clear that a page will be shipped out, i.e., after the
%    \hook{shipout/before} hook (because that might discard the page)!

%    Just like with the \texttt{page} counter its value is
%    only accurate within the output routine. In the body of the
%    document it may be off by one as the output routine is called
%    asynchronously!
%
%    Also important: it \emph{must not} be set, only read. There are
%    no provisions to prevent that but if you do, chaos will be the
%    result. To emphasize this fact it is not provided as a \LaTeX{}
%    counter but as a \TeX{} counter (i.e., a command), so
%    \cs{Alph}\Arg{\cs{ReadonlyShipoutCounter}} etc, would not work.
%  \end{variable}
%
%  \begin{variable}{totalpages,\g_shipout_totalpages_int}
%   \begin{syntax}
%     \cs{arabic}\texttt{\{totalpages\}}
%     \cs{int_use:N} \cs{g_shipout_totalpage_int} \texttt{\% expl3 usage}
%   \end{syntax}
%    In contrast to \cs{ReadonlyShipoutCounter}, the
%    \texttt{totalpages} counter is a \LaTeX{} counter and incremented
%    for each shipout attempt including those pages that are discarded
%    for one or the other reason. Again \hook{shipout/before} sees
%    the counter before it is incremented).
%
%    Furthermore, while it is incremented for each page, its value is
%    never used by \LaTeX. It can therefore be freely reset or changed by user
%    code, for example, to additionally count a number of pages that
%    are not build by \LaTeX\ but are added in a later part of the
%    process, e.g., cover pages or picture pages made externally.
%
%    Important: as this is a page-related counter its value is only
%    reliable inside the output routine!
%  \end{variable}
%

%
% \subsection{Debugging shipout code}
%
% \begin{function}{\DebugShipoutOn,\DebugShipoutOff,
%         \shipout_debug_on:,\shipout_debug_off:}
%   \begin{syntax}
%     \cs{DebugShipoutOn}
%   \end{syntax}
%    Turn the debugging of shipout code on or off. This displays
%    changes made to the shipout data structures.  \fmi{This needs
%    some rationalizing and will probably not stay this way.}
% \end{function}
%
%

%
% \section{Emulating commands from other packages}
%
%    The packages in this section are no longer necessary but as they
%    are used in other packages they are emulated when they are loaded
%    via \cs{usepackage} or \cs{RequirePackage}.
%
%
% \subsection{Emulating \pkg{atbegshi}}
%
%
% \begin{function}{\AtBeginShipoutUpperLeft,\AtBeginShipoutUpperLeftForeground}
%   \begin{syntax}
%     \cs{AddToHook} \texttt{\{shipout/before\} \{...\cs{AtBeginShipoutUpperLeft}}\Arg{code}\texttt{...\}}
%   \end{syntax}
%    This adds a \texttt{picture} environment into the background of the shipout
%    box expecting \meta{code} to contain \texttt{picture}
%    commands. The same effect can be obtained by simply using kernel
%    features as follows:
%    \begin{quote}
%      \cs{AddToHook}\texttt{\{shipout/background\}}\Arg{code}
%    \end{quote}
%    There is one technical difference: if
%    \cs{AtBeginShipoutUpperLeft} is used several times each
%    invocation is put into its own box inside the shipout box whereas
%    all \meta{code} going into \hook{shipout/background} ends up
%    all in the same box in the order it is added or sorted based on
%    the rules for the hook chunks.
%
%    \cs{AtBeginShipoutUpperLeftForeground} is similar with the
%    difference that the \texttt{picture} environment is placed in the
%    foreground. To model it with the kernel functions use the hook
%    \hook{shipout/foreground} instead.
% \end{function}
%
%
% \begin{function}{\AtBeginShipoutAddToBox,\AtBeginShipoutAddToBoxForeground}
%   \begin{syntax}
%     \cs{AddToHook} \texttt{\{shipout/before\} \{...\cs{AtBeginShipoutAddToBox}}\Arg{code}\texttt{...\}}
%   \end{syntax}
%    These work like \cs{AtBeginShipoutUpperLeft} and
%    \cs{AtBeginShipoutUpperLeftForeground} with the difference that
%    \meta{code} is directly placed into an \cs{hbox} inside the
%    shipout box and not surrounded by a \texttt{picture} environment.
%
%    To emulate them using \hook{shipout/background} or
%    \hook{shipout/foreground} you may have to wrap \meta{code} into
%    a \cs{put} statement but if the code is not doing any typesetting
%    just adding it to the hook should be sufficient.
% \end{function}
%


% \begin{function}{\AtBeginShipoutBox}
%    This is the name of the shipout box as \pkg{atbegshi} knows it.
% \end{function}
%
% \begin{function}{\AtBeginShipoutInit}
%   By default \pkg{atbegshi} delayed its action until
%    \verb=\beg{document}=.  This command was forcing it in an earlier
%    place. With the new concept it does nothing.
% \end{function}
%
% \begin{function}{\AtBeginShipout,\AtBeginShipoutNext}
%   \begin{syntax}
% \cs{AtBeginShipout}\Arg{code} $\equiv$ \cs{AddToHook}\texttt{\{shipout/before\}}\Arg{code}
% \cs{AtBeginShipoutNext}\Arg{code} $\equiv$ \cs{AddToHookNext}\texttt{\{shipout/before\}}\Arg{code}
%   \end{syntax}
%   This is equivalent to filling the \hook{shipout/before} hook
%    by  either using \cs{AddToHook} or \cs{AddToHookNext}, respectively.
% \end{function}
%
% \begin{function}{\AtBeginShipoutFirst,\AtBeginShipoutDiscard}
%   The \pkg{atbegshi} names for \cs{AtBeginDvi} and \cs{DiscardShipoutBox}.
% \end{function}
%



% \subsection{Emulating \pkg{everyshi}}
%


% \begin{function}{\EveryShipout}
%   \begin{syntax}
% \cs{EveryShipout}\Arg{code} $\equiv$ \cs{AddToHook}\texttt{\{shipout/before\}}\Arg{code}
%   \end{syntax}
% \end{function}
%
% \begin{function}{\AtNextShipout}
%   \begin{syntax}
% \cs{AtNextShipout}\Arg{code} $\equiv$ \cs{AddToHookNext}\texttt{\{shipout/before\}}\Arg{code}
%   \end{syntax}
% \end{function}
%


% \subsection{Emulating \pkg{atenddvi}}
%
% The \pkg{atenddvi} package implemented only a single command:
%    \cs{AtEndDvi} and that is now available out of the box.



% \subsection{Emulating \pkg{everypage}}
%
%    This page takes over the original \cs{@begindvi} hook and replaces 
%    it. It should be all covered by the hooks offered here (details
%    need checking) and thus could simply use the provided hooks
%    rather than defining its own.
%
%
%
% \StopEventually{\setlength\IndexMin{200pt}  \PrintIndex  }
%
%
% \section{The Implementation}
%    \begin{macrocode}
%<*2ekernel>
%<@@=shipout>
\ExplSyntaxOn
%    \end{macrocode}
%
%
%  \subsection{Debugging}
%
%  \begin{macro}{\g_@@_debug_bool}
%    Holds the current debugging state.
%    \begin{macrocode}
\bool_new:N \g_@@_debug_bool
%    \end{macrocode}
%  \end{macro}
%
%  \begin{macro}{\shipout_debug_on:,\shipout_debug_off:}
%  \begin{macro}{\@@_debug:n}
%  \begin{macro}{\@@_debug_gset:}
%    Turns debugging on and off by redefining \cs{@@_debug:n}.
%    \begin{macrocode}
\cs_new_eq:NN \@@_debug:n  \use_none:n
\cs_new_protected:Npn \shipout_debug_on:
  {
    \bool_gset_true:N \g_@@_debug_bool
    \@@_debug_gset:
  }
\cs_new_protected:Npn \shipout_debug_off:
  {
    \bool_gset_false:N \g_@@_debug_bool
    \@@_debug_gset:
  }
\cs_new_protected:Npn \@@_debug_gset:
  {
    \cs_gset_protected:Npx \@@_debug:n ##1
      { \bool_if:NT \g_@@_debug_bool {##1} }
  }
%    \end{macrocode}
%  \end{macro}
%  \end{macro}
%  \end{macro}
%

%  \begin{macro}{shipout/before,
%                   shipout/foreground,shipout/background,
%                   shipout/firstpage,
%                   shipout/lastpage}
%    Declaring all hooks for the shipout code.
%    \begin{macrocode}
\hook_new:n{shipout/before}
\hook_new:n{shipout/foreground}
\hook_new:n{shipout/background}
\hook_new:n{shipout/firstpage}
\hook_new:n{shipout/lastpage}
%    \end{macrocode}
%  \end{macro}




%  \begin{macro}{\ShipoutBox,\l_shipout_box}
%    The box filled with the page to be shipped out (both L3 and
%       \LaTeXe{} name).
%    \begin{macrocode}
\box_new:N  \l_shipout_box
%    \end{macrocode}
%    
%    \begin{macrocode}
\cs_set_eq:NN \ShipoutBox \l_shipout_box
%    \end{macrocode}
%  \end{macro}




%  \begin{macro}{\@@_execute:}
%    This is going to the be the code run by \cs{shipout}. The code
%    follows closely the  ideas from \pkg{atbegshi}, so not
%    documenting that here for now.
%    \begin{macrocode}
\cs_set:Npn\@@_execute: {
  \tl_set:Nx \l_@@_group_level_tl
     { \int_value:w \tex_currentgrouplevel:D }
  \tex_afterassignment:D \@@_execute_test_level:
  \tex_setbox:D \l_shipout_box
}
%    \end{macrocode}
%  \end{macro}


%  \begin{macro}{\shipout}
%    Overloading the \cs{shipout} primitive:
%    \begin{macrocode}
\cs_gset_eq:NN \shipout \@@_execute:
%    \end{macrocode}
%  \end{macro}


%  \begin{macro}{\l_@@_group_level_tl}
%    Helper token list to record the group level at which
%    \cs{@@_execute:} is encountered.  \begin{macrocode}
\tl_new:N \l_@@_group_level_tl
%    \end{macrocode}
%  \end{macro}

 

%  \begin{macro}{\@@_execute_test_level:}
%    If the group level has changed then we are still constructing
%    \cs{l_shipout_box} and to continue we need to wait until the
%    current group has finished, hence the \cs{tex_aftergroup:D}.
%    \begin{macrocode}
\cs_new:Npn \@@_execute_test_level: {
  \int_compare:nNnT
     \l_@@_group_level_tl < \tex_currentgrouplevel:D 
     \tex_aftergroup:D
  \@@_execute_cont:
}
%    \end{macrocode}
%  \end{macro}


%  \begin{macro}{\@@_execute_cont:}
%    When we have reached this point the shipout box has been
%    processed and is available in \cs{l_shipout_box} and ready for
%    real ship out (perhaps)..
%
%    First we quickly check if it is void (can't happen in the
%    standard \LaTeX{} output routine but \cs{shipout} might be called
%    from a package that has some special processing logic). If it is
%    void we aren't shipping anything out and processing ends.\footnote{In that
%    case we don't reset the deadcyles, that would be up to the OR
%    processing logic to do.}
%    \begin{macrocode}
\cs_new:Npn \@@_execute_cont: {
  \box_if_empty:NTF \l_shipout_box
    { \PackageWarning{ltshipout}{Ignoring~ void~ shipout~ box} }
    {
%    \end{macrocode}
%    Otherwise we assume that we will ship something and prepare for
%    final adjustments (in particular setting the state of
%    \cs{protect} while we are running the hook code).
%    \begin{macrocode}
      \bool_gset_false:N \g_@@_discard_bool
      \set@typeset@protect
%    \end{macrocode}
%    We also store the current shipout box dimension in registers, so that
%    they can be used in the hook code.\footnote{This is not really
%    necessary as the code could access them via \cs{box_ht:N}, etc.,
%    but it is perhaps convenient.}
%    \begin{macrocode}
      \@@_get_box_size:N \l_shipout_box
%    \end{macrocode}
%    Then we execute the \hook{shipout/before} hook.
%    \begin{macrocode}
      \hook_use:n {shipout/before}
%    \end{macrocode}
%    In \cs{g_shipout_totalpages_int} we count all shipout attempts so
%    we increment that counter already here (the other one is
%    incremented later when we know for sure that we do a
%    \cs{shipout}.
%
%     We increment it after running the above hook so that the values
%    for \cs{g_shipout_totalpages_int} and \cs{} are in sync while the
%    hook is executed (in the case that totalpages isn't manually
%    altered or through discarding pages that is).
%    \begin{macrocode}
      \int_gincr:N \g_shipout_totalpages_int
%    \end{macrocode}
%    The above hook might contain code that requests the page to be discarded so
%    we now test for it.
%    \begin{macrocode}
      \bool_if:NTF \g_@@_discard_bool
        { \PackageInfo{ltshipout}{Completed~ page~ discarded}
          \bool_gset_false:N \g_@@_discard_bool
%    \end{macrocode}
%    As we are discarding the page box and not shipping anything out,
%    we need to do some house cleaning and reset \TeX's deadcycles so
%    that it doesn't complain about too many calls to the OR without
%    any shipout.
%    \begin{macrocode}
          \tex_deadcycles:D \c_zero_int
%    \end{macrocode}
%    \fmi{In \pkg{atbegshi} the box was dropped but is that actually
%    needed? Or the resetting of \cs{protect} to its kernel value?}
%    \begin{macrocode}
%          \group_begin:
%            \box_set_eq_drop:NN \l_shipout_box \l_shipout_box
%          \group_end:
%          \cs_set_eq:NN \protect \exp_not:N
        }
%    \end{macrocode}
%    Even if there was no explicit request to discard the box it is
%    possible that the code for the hook \hook{shipout/before} has
%    voided the box (by mistake or deliberately). We therefore test
%    once more but this time make it a warning, because the best
%    practice way is to use the request mechanism.
%    \begin{macrocode}
        { \box_if_empty:NTF \l_shipout_box
            { \PackageWarning{ltshipout}{
                Shipout~ box~ was~ voided~ by~ hook,\MessageBreak
                ignoring~ shipout~ box  }
            }
%    \end{macrocode}
%    Finally, if the box is still non-empty we are nearly ready to
%    ship it out.
%    First we increment the total page counter so that we can later
%    test if we have reached the final page according to our available
%    information.\footnote{Doing that earlier would be wrong because we
%    might end up with the last page counted but discard and then we
%    have no place to add the final objects into the output file.}
%    \begin{macrocode}
            {
              \int_gincr:N \g_shipout_readonly_int
              \@@_debug:n {
                \typeout{Absolute~ page~ =~ \int_use:N \g_shipout_readonly_int
                         \space (target:~ \@abspage@last)}
              }
%    \end{macrocode}
%    Then we store the box sizes again (as they may have
%    changed) and then look at the hooks \hook{shipout/foreground}
%    and \hook{shipout/background}. If either or both are non-empty
%    we add a \texttt{picture} environment to the box (in the
%    foreground and or in the background) and execute the hook code
%    inside that environment.
%    
%    \begin{macrocode}
              \@@_get_box_size:N \l_shipout_box
              \@kernel@before@shipout@foreground
              \hook_if_empty:nF {shipout/foreground}
                   { \@@_add_foreground_picture:n
                     { \hook_use:n {shipout/foreground} } }
              \hook_if_empty:nF {shipout/background}
                   { \@@_add_background_picture:n
                     { \hook_use:n {shipout/background} } }
%    \end{macrocode}
%    We then run \cs{@@_execute_firstpage_hook:} that adds
%    the content of the hook \hook{shipout/firstpage} to the
%    start of the first page (if non-empty). It is then redefined to
%    do nothing on later pages.
%    \begin{macrocode}
              \@@_execute_firstpage_hook:
%    \end{macrocode}
%    The we check if we have to add the \hook{shipout/lastpage} hook
%    because we have reached the last page. This test will be false for
%    all but one (and hopefully the correct) page.
%    \begin{macrocode}
              \int_compare:nNnT \@abspage@last = \g_shipout_readonly_int
                { \hook_if_empty:nF {shipout/lastpage}
                    { \@@_debug:n { \typeout{Executing~ lastpage~ hook~
                          on~ page~ \int_use:N \g_shipout_readonly_int }        }
                      \_@@_add_foreground_box:n { \UseHook{shipout/lastpage}
                                                  \@kernel@after@shipout@lastpage }
                    }
                    \bool_gset_true:N \g_@@_lastpage_handled_bool
                }
%    \end{macrocode}
%    Finally we run the actual \TeX{} primitive for shipout. As that will
%    expand delayed \cs{write} statements inside the page in which
%    protected commands should not expand we first change \cs{protect}
%    to the appropriate definition for that case.
%    \begin{macrocode}
              \cs_set_eq:NN \protect \exp_not:N
              \tex_shipout:D \box_use:N \l_shipout_box
            }
        }
    }
}
%    \end{macrocode}
%  \end{macro}



%  \begin{macro}{\@kernel@after@shipout@lastpage,\@kernel@before@shipout@foreground}
%    
%    \begin{macrocode}
\let\@kernel@after@shipout@lastpage\@empty
\let\@kernel@before@shipout@foreground\@empty
%    \end{macrocode}
%  \end{macro}


%  \begin{macro}{\@@_execute_firstpage_hook: }
%    This command adds any specials into a box and adds that to the
%    very beginning of the first box shipped out. After that we
%    redefine it to do nothing on later pages.
%    \begin{macrocode}
\cs_new:Npn \@@_execute_firstpage_hook: {
%    \end{macrocode}
%    Adding something to the beginning means adding it to the
%    background as that  layer is done first in the output. Of course
%    that is only needed if the hook actually contains anything.
%    \begin{macrocode}
  \hook_if_empty:nF {shipout/firstpage}
       { \@@_add_background_box:n { \UseHook{shipout/firstpage} } }
%    \end{macrocode}
%    Once we are here we change the definition to do nothing next time
%    and we also change the command used to implement \cs{AtBeginDvi}
%    to become a warning and not  add further material to a hook that
%    is never used again.
%    \begin{macrocode}
  \cs_gset_eq:NN \@@_execute_firstpage_hook: \prg_do_nothing:
  \cs_gset:Npn \@@_add_firstpage_material:Nn ##1 ##2 {
    \PackageWarning{ltshipout}{
        First~ page~ is~ already~ shipped~ out,~ ignoring\MessageBreak
        \string##1 }
  }
}
%    \end{macrocode}
%  \end{macro}


%  \begin{macro}{\g_@@_lastpage_handled_bool}
%    A boolean to signal if we have already handled the
%    \hook{shipout/lastpage} hook.
%    \begin{macrocode}
\bool_new:N \g_@@_lastpage_handled_bool
%    \end{macrocode}
%  \end{macro}



%  \begin{macro}{\@@_add_firstpage_material:Nn}
%    This command adds material to the
%    \hook{shipout/firstpage} hook. It is used in
%    \cs{AtBeginDvi}, etc. The first argument is the
%    command through which is it called. Initially this is ignored but
%    once we are passed the first page it can be used to generate a
%    warning message mentioning the right user command.
%    \begin{macrocode}
\cs_new:Npn \@@_add_firstpage_material:Nn #1#2 {
   \AddToHook{shipout/firstpage}{#2}
}
%    \end{macrocode}
%  \end{macro}







%  \begin{macro}{\@@_get_box_size:N}
%    Store the box dimensions in dimen registers.
%    \fmi{This could/should perhaps be generalized to set height depth and
%    width given an arbitrary box.}
%    \begin{macrocode}
\cs_new:Npn \@@_get_box_size:N #1 {
  \dim_set:Nn \l_shipout_box_ht_dim { \box_ht:N #1 }
  \dim_set:Nn \l_shipout_box_dp_dim { \box_dp:N #1 }
  \dim_set:Nn \l_shipout_box_wd_dim { \box_wd:N #1 }
  \dim_set:Nn \l_shipout_box_ht_plus_dp_dim { \l_shipout_box_ht_dim +
                                         \l_shipout_box_dp_dim }
}
%    \end{macrocode}
%  \end{macro}

%  \begin{macro}{\l_shipout_box_ht_dim,
%                   \l_shipout_box_dp_dim,\l_shipout_box_wd_dim,
%                   \l_shipout_box_ht_plus_dp_dim}
%    And here are the variables set by \cs{@@_get_box_size:N}.
%    \begin{macrocode}
\dim_new:N \l_shipout_box_ht_dim
\dim_new:N \l_shipout_box_dp_dim
\dim_new:N \l_shipout_box_wd_dim
\dim_new:N \l_shipout_box_ht_plus_dp_dim
%    \end{macrocode}
%  \end{macro}





%  \begin{macro}{\g_@@_discard_bool}
%    Indicate whether or not the current page box should be discarded
%    \begin{macrocode}
\bool_new:N \g_@@_discard_bool
%    \end{macrocode}
%  \end{macro}



%  \begin{macro}{\l_@@_tmp_box,\l_@@_saved_badness_tl}
%    We need a box for the background and foreground material and a
%    token register to remember badness settings as we disable  them
%    during te buildup below.
%    \begin{macrocode}
\box_new:N \l_@@_tmp_box
\tl_new:N  \l_@@_saved_badness_tl
%    \end{macrocode}
%  \end{macro}


%  \begin{macro}{\@@_add_background_box:n}
%    In standard \LaTeX{} the shipout box is always a \cs{vbox} but
%    here we are allow for other usage as well, in case some package
%    has its own output routine.
%    \begin{macrocode}
\cs_new:Npn \@@_add_background_box:n #1
{ \@@_get_box_size:N \l_shipout_box
%    \end{macrocode}
%    But we start testing for a vertical box as that should be the
%    normal case.
%    \begin{macrocode}
  \box_if_vertical:NTF \l_shipout_box
      {
%    \end{macrocode}
%    Save current values of \cs{vfuzz} and \cs{vbadness} then change
%    them to allow box manipulations without warnings.
%    \begin{macrocode}
        \tl_set:Nx \l_@@_saved_badness_tl
           { \vfuzz=\the\vfuzz\relax
             \vbadness=\the\vbadness\relax }
        \vfuzz=\c_max_dim
        \vbadness=\c_max_int
%    \end{macrocode}
%    Then we reconstruct \cs{l_shipout_box} \ldots
%    \begin{macrocode}
        \vbox_set_to_ht:Nnn \l_shipout_box \l_shipout_box_ht_plus_dp_dim 
             {
%    \end{macrocode}
%    \ldots{} the material in \verb=#1= is placed into a horizontal
%    box with zero dimensions.
%    \begin{macrocode}
               \hbox_set:Nn \l_@@_tmp_box
                    { \l_@@_saved_badness_tl #1 }
               \box_set_wd:Nn \l_@@_tmp_box \c_zero_dim
               \box_set_ht:Nn \l_@@_tmp_box \c_zero_dim
               \box_set_dp:Nn \l_@@_tmp_box \c_zero_dim
%    \end{macrocode}
%    The we typeset that box followed by whatever was in
%    \cs{l_shipout_box} before (unpacked).
%    \begin{macrocode}
               \skip_zero:N \baselineskip
               \skip_zero:N \lineskip
               \skip_zero:N \lineskiplimit
               \box_use:N \l_@@_tmp_box
               \vbox_unpack:N \l_shipout_box
%    \end{macrocode}
%    The \cs{kern} ensures that the box has no depth which is
%    afterwards explicitly corrected.
%    \begin{macrocode}
               \kern \c_zero_dim
             }
        \box_set_ht:Nn \l_shipout_box \l_shipout_box_ht_dim
        \box_set_dp:Nn \l_shipout_box \l_shipout_box_dp_dim 
%    \end{macrocode}
%    \fmi{The whole boxing maneuver looks a bit like overkill to me, but for
%    the moment I leave.}               
%    \begin{macrocode}
        \l_@@_saved_badness_tl
      }
      {
%    \end{macrocode}
%    A horizontal box is handled in a similar way. The last case would
%    be a void box in which case we do nothing hence the missing
%    \texttt{F} branch. 
%    \begin{macrocode}
        \box_if_horizontal:NT \l_shipout_box
            {
              \tl_set:Nx \l_@@_saved_badness_tl
                 { \hfuzz=\the\hfuzz\relax
                   \hbadness=\the\hbadness\relax }
              \hfuzz=\c_max_dim
              \hbadness=\c_max_int
              \hbox_set_to_wd:Nnn \l_shipout_box \l_shipout_box_wd_dim
                   {
                     \hbox_set:Nn \l_@@_tmp_box
                          { \l_@@_saved_badness_tl #1 }
                     \box_set_wd:Nn \l_@@_tmp_box \c_zero_dim
                     \box_set_ht:Nn \l_@@_tmp_box \c_zero_dim
                     \box_set_dp:Nn \l_@@_tmp_box \c_zero_dim
                     \box_move_up:nn
                         \l_shipout_box_ht_dim 
                         { \box_use:N \l_@@_tmp_box }
                     \hbox_unpack:N \l_shipout_box
                   }
              \l_@@_saved_badness_tl
            }
      }
}
%    \end{macrocode}
%  \end{macro}




%  \begin{macro}{\@@_add_foreground_box:n}
%    Foreground boxes are done in the same way, only the order and
%    placement of boxes has to be done differently.
%    \begin{macrocode}
\cs_new:Npn \@@_add_foreground_box:n #1
{
  \box_if_vertical:NTF \l_shipout_box
    {
      \tl_set:Nx \l_@@_saved_badness_tl
         { \vfuzz=\the\vfuzz\relax
           \vbadness=\the\vbadness\relax }
      \vfuzz=\c_max_dim
      \vbadness=\c_max_int
      \vbox_set_to_ht:Nnn \l_shipout_box \l_shipout_box_ht_plus_dp_dim
           {
             \hbox_set:Nn \l_@@_tmp_box
                  { \l_@@_saved_badness_tl #1 }
             \box_set_wd:Nn \l_@@_tmp_box \c_zero_dim
             \box_set_ht:Nn \l_@@_tmp_box \c_zero_dim
             \box_set_dp:Nn \l_@@_tmp_box \c_zero_dim
             \skip_zero:N \baselineskip
             \skip_zero:N \lineskip
             \skip_zero:N \lineskiplimit
             \vbox_unpack:N \l_shipout_box
             \kern -\l_shipout_box_ht_plus_dp_dim
             \box_use:N \l_@@_tmp_box
             \kern  \l_shipout_box_ht_plus_dp_dim
           }
      \l_@@_saved_badness_tl
      \box_set_ht:Nn \l_shipout_box \l_shipout_box_ht_dim
      \box_set_dp:Nn \l_shipout_box \l_shipout_box_dp_dim 
    }
    {
      \box_if_horizontal:NT \l_shipout_box
        {
          \tl_set:Nx \l_@@_saved_badness_tl
            { \hfuzz=\the\hfuzz\relax
              \hbadness=\the\hbadness\relax }
          \hfuzz=\c_max_dim
          \hbadness=\c_max_int
          \hbox_set_to_wd:Nnn \l_shipout_box \l_shipout_box_wd_dim
               {
                 \hbox_unpack:N \l_shipout_box
                 \kern -\box_wd:N \l_shipout_box
                 \hbox_set:Nn \l_@@_tmp_box
                     { \l_@@_saved_badness_tl #1 }
                 \box_set_wd:Nn \l_@@_tmp_box \c_zero_dim
                 \box_set_ht:Nn \l_@@_tmp_box \c_zero_dim
                 \box_set_dp:Nn \l_@@_tmp_box \c_zero_dim
                 \box_move_up:nn { \box_ht:N \l_shipout_box }
                               { \box_use:N \l_@@_tmp_box }
                 \kern \box_wd:N \l_shipout_box
               }%
               \l_@@_saved_badness_tl
        }
    }
}
%    \end{macrocode}
%  \end{macro}




%  \begin{macro}{\c_@@_horigin_tl,\c_@@_vorigin_tl}
%    Two constants holding the offset of the top-left with respect to
%    the media box.
%
%    Setting the constants this way is courtesy of Bruno.
%    \begin{macrocode}
\tl_gput_right:Nn \@kernel@after@begindocument {
  \tl_const:Nx \c_@@_horigin_tl
     {
       \cs_if_exist_use:NTF \pdfvariable { horigin }
          { \cs_if_exist_use:NF \pdfhorigin { 1in } }
     }
  \tl_const:Nx \c_@@_vorigin_tl
     {
       \cs_if_exist_use:NTF \pdfvariable { vorigin }
          { \cs_if_exist_use:NF \pdfvorigin { 1in } }
     }
}     
%    \end{macrocode}
%  \end{macro}


%  \begin{macro}{\@@_picture_overlay:n}
%    Put the argument into a \texttt{picture} environment that doesn't take up
%    any size and uses \texttt{1pt} for \cs{unitlength}.
%    \fmi{Could perhaps be generalized as it might be useful elsewhere. For
%    not it is not.}
%    \begin{macrocode}
\cs_new:Npn \@@_picture_overlay:n #1 {
    \kern -\c_@@_horigin_tl \scan_stop:
    \vbox_to_zero:n {
      \kern -\c_@@_vorigin_tl \scan_stop:
      \unitlength 1pt \scan_stop:
      \hbox_set_to_wd:Nnn \l_@@_tmp_box \c_zero_dim { \ignorespaces #1 }
%      \box_set_wd:Nn \l_@@_tmp_box \c_zero_dim
      \box_set_ht:Nn \l_@@_tmp_box \c_zero_dim
      \box_set_dp:Nn \l_@@_tmp_box \c_zero_dim
      \box_use:N \l_@@_tmp_box
      \tex_vss:D
    }
}
%    \end{macrocode}
%  \end{macro}


%  \begin{macro}{\@@_add_background_picture:n}
%    Put a \texttt{picture} env in  the background of the shipout box
%    with its reference point in the top-left corner.
%    \begin{macrocode}
\cs_new:Npn \@@_add_background_picture:n #1 {
   \@@_add_background_box:n { \@@_picture_overlay:n {#1} }
}
%    \end{macrocode}
%  \end{macro}



%  \begin{macro}{\@@_add_foreground_picture:n}
%    
%    Put a \texttt{picture} env in  the foreground of the shipout box
%    with its reference point in the top-left corner.
%    \begin{macrocode}
\cs_new:Npn \@@_add_foreground_picture:n #1 {
   \@@_add_foreground_box:n { \@@_picture_overlay:n {#1} }
}
%    \end{macrocode}
%  \end{macro}


%  \begin{macro}{\shipout_discard:}
%    Request that the next shipout box should be discarded. At the
%    moment this is just setting a boolean, but we may want to augment
%    this behavior that the position of the call is taken into account
%    (in case \LaTeX{} looks ahead and is not using the position for
%    on the next page).
%    \begin{macrocode}
\cs_new:Npn \shipout_discard: {
  \bool_gset_true:N \g_@@_discard_bool
}
%    \end{macrocode}
%  \end{macro}




% \subsection{Handling the end of job hook}

%    At the moment this is partly solved by using the existing hooks.
%    But rather than putting the code into these hooks it should be
%    moved to the right place directly as we shouldn't prefill hooks
%    with material unless it needs to interact with other code. 
%

%    
%  \begin{macro}{\g_shipout_readonly_int,\ReadonlyShipoutCounter}
%    We count every shipout activity that makes a page (but not those
%    that are discarded) in order to know how many pages got produced.
%    \begin{macrocode}
\int_new:N \g_shipout_readonly_int
%    \end{macrocode}
%    For \LaTeXe{} it is available as a command (i.e., a \TeX{}
%    counter only.
%    \begin{macrocode}
\cs_new_eq:NN \ReadonlyShipoutCounter  \g_shipout_readonly_int
%    \end{macrocode}
%  \end{macro}

%  \begin{macro}{\g_shipout_totalpages_int,\c@totalpages}
%    We count every shipout attempt (even those that are discarded) in
%    tis counter. It is not used in the code but may get used in user
%    code.
%    \begin{macrocode}
\int_new:N \g_shipout_totalpages_int
%    \end{macrocode}
%    For \LaTeXe{} this is offered as a \LaTeX{} counter so can be
%    easily typeset inside the output routine to display things like
%    \enquote{\cs{thepage}\texttt{/}\cs{thetotalpages}}, etc.
%    \begin{macrocode}
\cs_new_eq:NN \c@totalpages \g_shipout_totalpages_int
\cs_new:Npn \thetotalpages { \arabic{totalpages} }
%    \end{macrocode}
%  \end{macro}
%



%  \begin{macro}{\@abspage@last}
%    In \cs{@abspage@last} record the number of pages from the last
%    run. This is written to the \texttt{.aux} and this way made
%    available to the next run. In case there is no \texttt{.aux} file
%    or the statement is missing from it we initialize it with the
%    largest possible number in \TeX{}. We use this a s the default
%    because then we are inserting the \hook{shipout/lastpage} on
%    the last page (or after the last page but not on page 1 for a
%    multipage document.
%    \begin{macrocode}
\xdef\@abspage@last{\number\maxdimen}
%    \end{macrocode}
%  \end{macro}


% \begin{macro}{\enddocument}
%
%    Instead of using the hooks \hook{enddocument} and
%    \hook{enddocument/afterlastpage} we add this code to private
%    kernel hooks to be 100\% when it is executed and to avoid
%    cluttering the hooks with data that is always there.
%
%    Inside \cs{enddocument} there is a \cs{clearpage}. Just before
%    that we execute this code here. There is a good change that we
%    are on the last page. Therefore, if we don't know the value from
%    the last run, we assume that the current page is the right
%    one. So we set \cs{@abspage@last} and as a result the next
%    shipout will run the \hook{shipout/lastpage} code. Of course,
%    if there are floats that still need a placement this guess will
%    be wrong but then rerunning the document will give us the correct
%    value next time around.
%
% \begin{macro}{\@kernel@after@enddocument}
%    \begin{macrocode}
\g@addto@macro \@kernel@after@enddocument {
  \int_compare:nNnT \@abspage@last = \maxdimen
    {
%    \end{macrocode}
%    We use \LaTeXe{} coding as \cs{@abspage@last} is not an L3 name.
%    \begin{macrocode}
      \xdef\@abspage@last{ \int_eval:n {\g_shipout_readonly_int + 1} }
    }
}
%    \end{macrocode}
% \end{macro}
%
% \begin{macro}{\@kernel@after@enddocument@afterlastpage}
%    Once the \cs{clearpage} has done its work inside \cs{enddocument}
%    we know for sure how many pages this document has, so we record
%    that in the \texttt{.aux} file for the next run.
%
%    \begin{macrocode}
\g@addto@macro \@kernel@after@enddocument@afterlastpage {
%    \end{macrocode}
%    There is one special case: If no output is produced then there is
%    no point in a) recording the number as 0 will never match the
%    page number of a real page and b) adding an extra page to ran the
%    \hook{shipout/lastpage} is pointless as well (as it would
%    remain forever). So we test for this and run the code only if
%    there have been pages.
%    \begin{macrocode}
  \int_compare:nNnF \g_shipout_readonly_int = 0
    {
%    \end{macrocode}
%     This ends up in the \texttt{.aux} so we use \LaTeXe{} names here.
%     \fmi{this needs an interface for \cs{nofiles} in expl3, doesn't at the moment!}
%    \begin{macrocode}
     \if@filesw
        \iow_now:Nx \@auxout {
          \gdef\string\@abspage@last {\int_use:N \g_shipout_readonly_int}}
     \fi
%    \end{macrocode}
%    But we may have guessed wrongly earlier and we still have to run the
%    \hook{shipout/lastpage} even though there is no page to place
%    it into. If that is the case we make a trivial extra page and put
%    it there. This temporary page will then vanish again on the next
%    run but helps to keep pdf viewers happy.
%    \begin{macrocode}
      \bool_if:NF \g_@@_lastpage_handled_bool
        { \tex_shipout:D\vbox to\textheight{
            \hbox{\UseHook{shipout/lastpage} \@kernel@after@shipout@lastpage }
%    \end{macrocode}
%    This extra page could be totally empty except for the hook
%    content, but to help the user understanding why it is there we
%    put some text into it.
%    \begin{macrocode}
            \@@_excuse_extra_page:
            \null
          }
%    \end{macrocode}
%    At this point we should also signal to \LaTeX{} that a rerun is
%    necessary so that an appropriate message can be shown on the
%    terminal.
%
%    \fmi{integrate warning ``Rerun \ldots'' }
%    \begin{macrocode}
      }
    }
}
%    \end{macrocode}
% \end{macro}
% \end{macro}



%  \begin{macro}{\@@_excuse_extra_page:}
%    Say mea culpa \ldots
%    \begin{macrocode}
\cs_new:Npn \@@_excuse_extra_page: {
  \vfil
  \begin{center}
    \bfseries Temporary~ page! 
  \end{center}
    \LaTeX{}~ was~ unable~ to~ guess~ the~ total~ number~ of~ pages~
    correctly.~ ~ As~ there~ was~ some~ unprocessed~ data~ that~
    should~ have~ been~ added~ to~ the~ final~ page~ this~ extra~
    page~ has~ been~ added~ to~ receive~ it.
    \par
    If~ you~ rerun~ the~ document~ (without~ altering~ it)~ this~
    surplus~ page~ will~ go~ away,~ because~ \LaTeX{}~ now~ knows~
    how~ many~ pages~ to~ expect~ for~ this~ document.
  \vfil
}
%    \end{macrocode}
%  \end{macro}




% \section{Legacy \LaTeXe{} interfaces}



%  \begin{macro}{\DiscardShipoutBox}
%    Request that the next shipout box is to be discarded.
%    \begin{macrocode}
\cs_new_eq:NN \DiscardShipoutBox \shipout_discard:
%    \end{macrocode}
%  \end{macro}


%  \begin{macro}{\AtBeginDvi}
%    
%    \begin{macrocode}
\renewcommand \AtBeginDvi {\@@_add_firstpage_material:Nn \AtBeginDvi}
%    \end{macrocode}
%  \end{macro}


%  \begin{macro}{\DebugShipoutOn,\DebugShipoutOff}
%    
%    \begin{macrocode}
\cs_new_eq:NN \DebugShipoutOn  \shipout_debug_on:
\cs_new_eq:NN \DebugShipoutOff \shipout_debug_off:
%    \end{macrocode}
%  \end{macro}





% \section{Package emulation for compatibility}


% \subsection{Package \pkg{atbegshi} emulation}

%  \begin{macro}{\AtBeginShipoutBox}
%    \begin{macrocode}
\cs_new_eq:NN \AtBeginShipoutBox \ShipoutBox
%    \end{macrocode}
%  \end{macro}



%  \begin{macro}{\AtBeginShipoutInit}
%    Compatibility only, we aren't delaying \ldots
%    \begin{macrocode}
\cs_set_eq:NN\AtBeginShipoutInit\@empty 
%    \end{macrocode}
%  \end{macro}



%  \begin{macro}{\AtBeginShipout,\AtBeginShipoutNext}
%    Filling hooks
%    \begin{macrocode}
\newcommand\AtBeginShipout     {\AddToHook{shipout/before}}
\newcommand\AtBeginShipoutNext {\AddToHookNext{shipout/before}}
%    \end{macrocode}
%  \end{macro}



%  \begin{macro}{\AtBeginShipoutFirst}
%    Slightly more complex as we need to know the name of the command under which the
%    \hook{shipout/firstpage} hook is filled.
%    \begin{macrocode}
\newcommand\AtBeginShipoutFirst{\@@_add_firstpage_material:Nn \AtBeginShipoutFirst}
%    \end{macrocode}
%  \end{macro}



% This is somewhat different from the original where
% \cs{ShipoutBoxHeight} etc.\ only holds the \verb=\the\ht<box>= value. This
% may has some implications in some use cases and if that is a problem
%    then it might need changing.
%    
%    \begin{macrocode}
\cs_new:Npn \ShipoutBoxHeight { \dim_use:N \l_shipout_box_ht_dim }
\cs_new:Npn \ShipoutBoxDepth  { \dim_use:N \l_shipout_box_dp_dim }
\cs_new:Npn \ShipoutBoxWidth  { \dim_use:N \l_shipout_box_wd_dim }
%    \end{macrocode}


%  \begin{macro}{\AtBeginShipoutDiscard}
%    Just a different name.
%    \begin{macrocode}
\cs_new_eq:NN \AtBeginShipoutDiscard \DiscardShipoutBox
%    \end{macrocode}
%  \end{macro}


%  \begin{macro}{\AtBeginShipoutAddToBox,\AtBeginShipoutAddToBoxForeground,
%                \AtBeginShipoutUpperLeft,\AtBeginShipoutUpperLeftForeground}
%    We don't expose them.
%    \begin{macrocode}
\cs_new_eq:NN \AtBeginShipoutAddToBox           \@@_add_background_box:n
\cs_new_eq:NN \AtBeginShipoutAddToBoxForeground \@@_add_foreground_box:n 
%    \end{macrocode}
%    
%    \begin{macrocode}
\cs_new_eq:NN\AtBeginShipoutUpperLeft           \@@_add_background_picture:n
\cs_new_eq:NN\AtBeginShipoutUpperLeftForeground \@@_add_foreground_picture:n
%    \end{macrocode}
%  \end{macro}

%
%
%    We prevent the package \pkg{atbegshi} from loading:
%    \begin{macrocode}
\expandafter\cs_set_eq:NN\csname ver@atbegshi.sty\endcsname\fmtversion
%    \end{macrocode}
%    \pkg{hyperref} code (and \pkg{ltxcmds}) doesn't understand 2020-10-01
%       and thinks it is before 1994, so for now \ldots
%    \begin{macrocode}
\@namedef {ver@atbegshi.sty}{2020/10/01}
%    \end{macrocode}



% \subsection{Package \pkg{everyshi} emulation}
%
%  \begin{macro}{\EveryShipout,\AtNextShipout}
%    This package has only two public commands to simulating it is easy:
%    \begin{macrocode}
\cs_new_eq:NN\EveryShipout\AtBeginShipout
\cs_new_eq:NN\AtNextShipout\AtBeginShipoutNext
%    \end{macrocode}
%  \end{macro}
%
%    \begin{macrocode}
\expandafter\cs_set_eq:NN\csname ver@everyshi.sty\endcsname\fmtversion
\@namedef {ver@everyshi.sty}{2020/10/01}
%    \end{macrocode}



% \subsection{Package \pkg{atenddvi} emulation}
%
%  \begin{macro}{\AtEndDvi}
%    This package has only one public command to simulating it is easy:
%    \begin{macrocode}
\cs_new:Npn \AtEndDvi {\AddToHook{shipout/lastpage}}
%    \end{macrocode}
%  \end{macro}
%
%
%    \begin{macrocode}
\expandafter\cs_set_eq:NN\csname ver@atenddvi.sty\endcsname\fmtversion
\@namedef {ver@atenddvi.sty}{2020/10/01}
%    \end{macrocode}




%    \begin{macrocode}
\ExplSyntaxOff
%</2ekernel>
%    \end{macrocode}

%    \Finale
%


%%%%%%%%%%%%%%%%%%%%%%%%%%%%%%%%%%%%%%%%%%%%%  
\endinput
%%%%%%%%%%%%%%%%%%%%%%%%%%%%%%%%%%%%%%%%%%%%%  



                     % \iffalse meta-comment
%
%% File: ltfilehook.dtx (C) Copyright 2020 Frank Mittelbach, LaTeX Team
%
% It may be distributed and/or modified under the conditions of the
% LaTeX Project Public License (LPPL), either version 1.3c of this
% license or (at your option) any later version.  The latest version
% of this license is in the file
%
%    https://www.latex-project.org/lppl.txt
%
%
% The development version of the bundle can be found below
%
%    https://github.com/FrankMittelbach/...
%
% for those people who are interested or want to report an issue.
%
%    \begin{macrocode}
\providecommand\ltfilehookversion{v0.6g}
\providecommand\ltfilehookdate{2020/07/16}
%    \end{macrocode}
%
%<*driver>

\RequirePackage{lthooks}        % needed for structedlog if running old format

\documentclass{l3doc}

% bug fix fo l3doc.cls
\ExplSyntaxOn
\cs_set_protected:Npn \__codedoc_macro_typeset_one:nN #1#2
  {
    \vbox_set:Nn \l__codedoc_macro_box
      {
        \vbox_unpack_drop:N \l__codedoc_macro_box
        \hbox { \llap { \__codedoc_print_macroname:nN {#1} #2
            \MacroFont       % <----- without it the \ is in lmr10 if a link is made
            \      
        } }
      }
    \int_incr:N \l__codedoc_macro_int
  }
\ExplSyntaxOff

\usepackage{structuredlog}  % for demonstration

\EnableCrossrefs
\CodelineIndex
\begin{document}
  \DocInput{ltfilehook.dtx}
\end{document}
%</driver>
%
% \fi
%
%
% \long\def\fmi#1{\begin{quote}\itshape Todo: #1\end{quote}}
%
% \let\hook\texttt
%
% \title{The \texttt{ltfilehook} package\thanks{This package has version
%    \ltfilehookversion\ dated \ltfilehookdate, \copyright\ \LaTeX\
%    Project.}}
%
% \author{Frank Mittelbach}
%
% \maketitle
%
%
%
% \tableofcontents
%
% \section{Introduction}
%
%
%
% \subsection{Provided hooks}
%
%    The code offers a number of hooks into which packages (or the
%    user) can add code to support different use cases.
%    Many hooks are offered as pairs (i.e., the second hook is
%    reversed. Also important to know is that these pairs are
%    properly nested with respect to other pairs of hooks.
%
%    There are hooks that are executed for all files of a certain type
%    (if they contain code), e.g., for all \enquote{include files} or
%    all \enquote{packages},
%    and there are also hooks that are specific to a single file,
%    e.g., do something after the package \texttt{foo.sty} has been
%    loaded.
%
%
% \subsection{General hooks for file reading}
% \label{sec:general-file-hooks}
%
%    There are four hooks that are called for each file that is read
%    using document-level commands such as \cs{input}, \cs{include},
%    \cs{usepackage}, etc.  They are not called for files read using
%    internal low-level methods, such as \cs{@input} or \cs{openin}.
%
%  \begin{variable}{file/before,file/before/...,
%                   file/after/...,file/after,
%                   }
%    These are:
%    \begin{description}
%    \item[\texttt{file/before}, \texttt{file/before/\meta{file-name}}]
%
%       These hooks are executed in that order just before the file is
%       loaded for reading. The code of the first hook is used is used
%       with every file, while the second is executed only for the
%       file with matching \meta{file-name} allowing you to specify
%       code that only applies to one file.
%
%    \item[\texttt{file/after/\meta{file-name}}, \texttt{file/after}]
%
%       These hooks are after the file with name \meta{file-name} has
%       been fully consumed. The order is swapped (the specific one
%       comes first) so that the \texttt{before} and \texttt{after}
%       hooks nest properly, which is important if any of them involve
%       grouping (e.g., contain environments, for example).
%       furthermore both hooks are reversed hooks to support correct
%       nesting of different packages add code to both
%       \texttt{/before} and \texttt{/after} hooks.
%
%    \end{description}
%  \end{variable}
%
%
%     So the overall sequence of hook processing for any file read
%     through the user interface commands of \LaTeX{} is:
%
%  \begin{tabbing}
%    mm\=mm\=mm\=mm\=\kill
%    \>\cs{UseHook}\marg{\hook{file/before}} \\
%    \>\cs{UseHook}\marg{\hook{file/before/\meta{file name}}} \\
%    \>\> \meta{file contents} \\
%    \>\cs{UseHook}\marg{\hook{file/after/\meta{file name}}} \\
%    \>\cs{UseHook}\marg{\hook{file/after}}
%  \end{tabbing}
%
%     \fmi{With a higher-level interface that doesn't matter, but it
%    is a bit weird, if you use \cs{AddToHook} or \cs{hook_gput:nnn}
%    directly, so I guess that has to be done differently!}
%
%    The file hooks only refer to the file by its name and extension,
%    so the \meta{file name} should be the file name as it is on the
%    filesystem with extension (if any) and without paths.  Different
%    from \cs{input} and similar commands, for hooks the \texttt{.tex}
%    extension is not assumed, so \texttt{.tex} files must also be given
%    with their extension.
%    Files within subfolders should also be addressed by their name and
%    extension only.
%
%    Extensionless files also work, and should then be given without
%    extension.  Note however that \TeX{} prioritizes \texttt{.tex}
%    files, so if two files \texttt{foo} and \texttt{foo.tex} exist in
%    the search path, only the latter will be seen.
%
%    When a file is input, the \meta{file name} is available in
%    \cs{CurrentFile}, which is then used when accessing the
%    \hook{file/before/\meta{file name}} and
%    \hook{file/after/\meta{file name}}.
%
%  \begin{variable}{\CurrentFile}
%    The name of the file about to be read (or just finished) is
%    available to the hooks through \cs{CurrentFile} (there is no
%    \texttt{expl3} name for it for now).  The file is always provided
%    with its extension, i.e., how it appears on your hard drive, but
%    without any specified path to it. For example,
%    \verb=\input{sample}= and \verb=\input{app/sample.tex}= would
%    both have \cs{CurrentFile} being \texttt{sample.tex}.
%  \end{variable}
%
%  \begin{variable}{\CurrentFilePath}
%    The path to the current file (complement to \cs{CurrentFile}) is
%    available in \cs{CurrentFilePath} if needed.
%    The paths returned in \cs{CurrentFilePath} are only user paths,
%    given through \cs{input@path} (or \pkg{expl3}'s equivalent
%    \cs{l_file_search_path_seq}) or by directly typing in the path
%    in the \cs{input} command or equivalent.  Files located by
%    \texttt{kpsewhich} get the path added internally by the \TeX{}
%    implementation, so at the macro level it looks as if the file were
%    in the current folder, so the path in \cs{CurrentFilePath} is empty
%    in these cases (package and class files, mostly).
%  \end{variable}
%
% \subsection{Hooks for package and class files}
%
%    Commands to load package and class files (e.g., \cs{usepackage},
%    \cs{RequirePackage}, \cs{LoadPackageWithOptions}, etc.) offer the
%    hooks from section~\ref{sec:general-file-hooks} when they are
%    used to load a  package or class file, e.g.,
%    \texttt{file/after/array.sty} would be called after the
%    \pkg{array} package got loaded. But as packages and classes form as special group
%    of files, there are some additional hooks available that only
%    apply when a package or class is loaded.
%
%
%  \begin{variable}{
%                   package/before,package/after,
%                   package/before/...,package/after/...,
%                   class/before,class/after,
%                   class/before/...,class/after/...,
%                   }
%    These are:
%    \begin{description}
%    \item[\texttt{package/before}, \texttt{package/after}]
%
%      These hooks are called for each package being loaded.
%
%    \item[\texttt{package/before/\meta{name}},
%    \texttt{package/after/\meta{name}}]
%
%      These hooks are additionally called if the package name is
%      \meta{name} (without extension).
%
%    \item[\texttt{class/before}, \texttt{class/after}]
%
%      These hooks are called for each class being loaded.
%
%    \item[\texttt{class/before/\meta{name}}, \texttt{class/after/\meta{name}}]
%
%      These hooks are additionally called if the class name is
%      \meta{name} (without extension).
%
%    \end{description}
%  \end{variable}
%     All \hook{/after} hooks are implemented as reversed hooks.
%
%     \noindent The overall sequence of execution for \cs{usepackage}
%     and friends is therefore:
%  \begin{tabbing}
%    mm\=mm\=mm\=mm\=\kill
%    \>\cs{UseHook}\marg{\hook{package/before}} \\
%    \>\cs{UseHook}\marg{\hook{package/before/\meta{package name}}} \\[5pt]
%    \>\>\cs{UseHook}\marg{\hook{file/before}} \\
%    \>\>\cs{UseHook}\marg{\hook{file/before/\meta{package name}.sty}} \\
%    \>\>\> \meta{package contents} \\
%    \>\>\cs{UseHook}\marg{\hook{file/after/\meta{package name}.sty}} \\
%    \>\>\cs{UseHook}\marg{\hook{file/after}} \\[5pt]
%    \>\>\emph{code from \cs{AtEndOfPackage} if
%                used inside the package} \\[5pt]
%    \>\cs{UseHook}\marg{\hook{package/after/\meta{package name}}} \\
%    \>\cs{UseHook}\marg{\hook{package/after}} 
%  \end{tabbing}
%    and similar for class file loading, except that \texttt{package/}
%    is replaced by \texttt{class/} and \cs{AtEndOfPackage} by
%    \cs{AtEndOfClass}.
%
%    If a package or class is not loaded (or it was loaded before the
%    hooks were set) none of the hooks are executed!
%
% \subsection{Hooks for \cs{include} files}
%
%    To manage \cs{include} files, \LaTeX{} issues a \cs{clearpage}
%    before and after loading such a file. Depending on the use case
%    one may want to execute code before or after these
%    \cs{clearpage}s especially for the one that is issued at the end.
%
%    Executing code before the final \cs{clearpage}, means that the
%    code is processed while the last page of the included material is
%    still under construction.  Executing code after it means that all
%    floats from inside the include file are placed (which
%    might have added further pages) and the final page has finished.
%
%    Because of these different scenarios we offer hooks in three
%    places.\footnote{If you want to execute code before the first
%     \cs{clearpage} there is no need to use a hook---you can write it
%     directly in front of the \cs{include}.}
%    None of the hooks are executed when an \cs{include} file is
%    bypassed because of an \cs{includeonly} declaration. They are,
%    however, all executed if \LaTeX{} makes an attempt to load the
%    \cs{include} file (even if it doesn't exist and all that happens
%    is \enquote{\texttt{No file \meta{filename}.tex}}).
%
%
%  \begin{variable}{include/before,include/before/...,
%                   include/end,include/end/...,
%                   include/after,include/after/...,
%                  }
%    These are:
%    \begin{description}
%
%    \item[\texttt{include/before}, \texttt{include/before/\meta{name}}]
%
%      These hooks are executed one after another after the initial
%      \cs{clearpage} and after \texttt{.aux} file is changed to use
%      \texttt{\meta{name}.aux}, but before the
%      \texttt{\meta{name}.tex} file is loaded. In other words they are executed
%      at the very beginning of the first page of the \cs{include}
%      file.
%
%
%    \item[\texttt{include/end/\meta{name}}, \texttt{include/end}]
%
%      These hooks are executed (in that order) after \LaTeX{} has
%      stopped reading from the \cs{include} file, but before it has
%      issued a \cs{clearpage} to output any deferred floats.
%
%
%    \item[\texttt{include/after/\meta{name}}, \texttt{include/after}]
%
%      These hooks are executed (in that order) after \LaTeX{} has
%      issued the \cs{clearpage} but before is has switched back
%      writing to the main \texttt{.aux} file. Thus technically we are
%      still inside the \cs{include} and if the hooks generate any
%      further typeset material including anything that writes to the
%      \texttt{.aux} file, then it would be considered part of the
%      included material and bypassed if it is not loaded because of
%      some \cs{includeonly} statement.\footnotemark
%
%    \end{description}
%  \end{variable}\footnotetext{For that reason
%      another \cs{clearpage} is executed after these hooks which
%      normally does nothing, but starts a new page if further material
%      got added this way.}
%
% \subsection{High-level interfaces for \LaTeX{}}
%
%    We do not provide any high-level \LaTeX{} commands (like
%    \pkg{filehook} or \pkg{scrlfile} do) but think that for package
%    writers the commands from for hook management are sufficient.
%
% \subsection{A sample package for structuring the log output}
%
%    As an application we provide the package \pkg{structuredlog} that
%    adds lines to the \texttt{.log} when a file is opened and closed
%    for reading keeping track of nesting level es well.
%    For example, for the current document it adds the lines
%\begin{verbatim}
%    = (LEVEL 1 START) t1lmr.fd
%    = (LEVEL 1 STOP) t1lmr.fd
%    = (LEVEL 1 START) supp-pdf.mkii
%    = (LEVEL 1 STOP) supp-pdf.mkii
%    = (LEVEL 1 START) nameref.sty
%    == (LEVEL 2 START) refcount.sty
%    == (LEVEL 2 STOP) refcount.sty
%    == (LEVEL 2 START) gettitlestring.sty
%    == (LEVEL 2 STOP) gettitlestring.sty
%    = (LEVEL 1 STOP) nameref.sty
%    = (LEVEL 1 START) ltfilehook-doc.out
%    = (LEVEL 1 STOP) ltfilehook-doc.out
%    = (LEVEL 1 START) ltfilehook-doc.out
%    = (LEVEL 1 STOP) ltfilehook-doc.out
%    = (LEVEL 1 START) ltfilehook-doc.hd
%    = (LEVEL 1 STOP) ltfilehook-doc.hd
%    = (LEVEL 1 START) ltfilehook.dtx
%    == (LEVEL 2 START) ot1lmr.fd
%    == (LEVEL 2 STOP) ot1lmr.fd
%    == (LEVEL 2 START) omllmm.fd
%    == (LEVEL 2 STOP) omllmm.fd
%    == (LEVEL 2 START) omslmsy.fd
%    == (LEVEL 2 STOP) omslmsy.fd
%    == (LEVEL 2 START) omxlmex.fd
%    == (LEVEL 2 STOP) omxlmex.fd
%    == (LEVEL 2 START) umsa.fd
%    == (LEVEL 2 STOP) umsa.fd
%    == (LEVEL 2 START) umsb.fd
%    == (LEVEL 2 STOP) umsb.fd
%    == (LEVEL 2 START) ts1lmr.fd
%    == (LEVEL 2 STOP) ts1lmr.fd
%    == (LEVEL 2 START) t1lmss.fd
%    == (LEVEL 2 STOP) t1lmss.fd
%    = (LEVEL 1 STOP) ltfilehook.dtx
%\end{verbatim}
%    Thus if you inspect an issue in the \texttt{.log} it is easy to
%    figure out in which file it occurred, simply by searching back for
%    \texttt{LEVEL} and if it is a \texttt{STOP} then remove 1 from
%    the level value and search further for \texttt{LEVEL} with that value
%    which should then be the \texttt{START}  level of the file you are in.
%
% \StopEventually{\setlength\IndexMin{200pt}  \PrintIndex  }
%
%
% \section{The Implementation}
%    \begin{macrocode}
%<*2ekernel>
%    \end{macrocode}
%
% \subsection{\pkg{expl3} helpers}
%
%    \begin{macrocode}
%<@@=filehook>
%    \end{macrocode}
%
% \begin{macro}{\CurrentFile,\CurrentFilePath}
%   User-level macros that hold the current file name and file path.
%   These are used internally as well because the code takes care to
%   protect against a possible redefinition of these macros in the loaded
%   file (it's necessary anyway to make hooks work with nested
%   \cs{input}).
%    \begin{macrocode}
\def\CurrentFile{}
\def\CurrentFilePath{}
%    \end{macrocode}
% \end{macro}
%
% \begin{macro}{\l_@@_internal_tl}
% \begin{macro}{\@filehook@set@curr@file,\@@_normalise_file_name:n}
%   When inputting a file, \cs{@filehook@set@curr@file} does a file
%   lookup (in \cs{input@path} and \cs{l_file_search_path_seq}) and
%   returns the actual file name (\meta{base} plus \meta{ext}) in
%   \cs{CurrentFile}.  Only the base and extension are returned,
%   regardless of the input (both \texttt{path/to/file.tex} and
%   \texttt{file.tex} end up as \texttt{file.tex} in \cs{CurrentFile}).
%   The path is returned in \cs{CurrentFilePath}, in case it's needed.
%   \cs{CurrentFile} is then used to run the file hooks with
%   \texttt{file/before/\cs{CurrentFile}} and
%   \texttt{file/after/\cs{CurrentFile}}.
%    \begin{macrocode}
\ExplSyntaxOn
\tl_new:N \l_@@_internal_tl
\cs_new_protected:Npn \@filehook@set@curr@file #1
  { \exp_args:NV \@@_normalise_file_name:n #1 }
\cs_new_protected:Npn \@@_normalise_file_name:n #1
  {
    \file_if_exist:nTF {#1}
      {
        \exp_args:Nx \file_parse_full_name:nNNN
          { \file_full_name:n {#1} }
      }
      { \file_parse_full_name:nNNN {#1} }
      \CurrentFilePath \CurrentFile \l_@@_internal_tl
    \tl_set:Nx \CurrentFile { \CurrentFile \l_@@_internal_tl }
  }
%    \end{macrocode}
% \end{macro}
% \end{macro}
%
% \begin{macro}{\g_@@_input_file_seq}
% \begin{macro}{\@filehook@file@push,\@filehook@file@pop}
% \begin{macro}{\@@_file_pop_assign:nn}
%   Yet another stack, to keep track of \cs{CurrentFile} and
%   \cs{CurrentFilePath} with nested \cs{input}s.  At the beginning of
%   \cs{InputIfFileExists}, the current value of \cs{CurrentFilePath}
%   and \cs{CurrentFile} is pushed to \cs{g_@@_input_file_seq}, and
%   at the end, it is popped and the value reassigned.
%   \cs{IfFileExists} does \cs{set@curr@file} internally, which changes
%   \cs{CurrentFile}, so \cs{@filehook@file@push} has to be executed
%   before \cs{IfFileExists}.
%    \begin{macrocode}
\seq_new:N \g_@@_input_file_seq
\cs_new_protected:Npn \@filehook@file@push
  {
    \seq_gpush:Nx \g_@@_input_file_seq
      { { \CurrentFilePath } { \CurrentFile } }
  }
\cs_new_protected:Npn \@filehook@file@pop
  {
    \seq_gpop:NNTF \g_@@_input_file_seq \l_@@_internal_tl
      { \exp_after:wN \@@_file_pop_assign:nn \l_@@_internal_tl }
      { \ERROR_should_not_happen }
  }
\cs_new_protected:Npn \@@_file_pop_assign:nn #1 #2
  {
    \tl_set:Nn \CurrentFilePath {#1}
    \tl_set:Nn \CurrentFile {#2}
  }
\ExplSyntaxOff
%    \end{macrocode}
% \end{macro}
% \end{macro}
% \end{macro}
%
%
% \subsection{Declaring the file-related hooks}
%
%  All hooks starting with \texttt{file/} \texttt{include/},
%  \texttt{class/} or \texttt{package/} are generic and will be
%  allocated if code is added to them. Thus there is no need to
%  explicitly declare any hook in the code below.
%
%  Furthermore, those named \texttt{.../after} or \texttt{.../end} are
%  automatically declared as reversed hooks if filled with code, so this
%  is also automatically taken care of.
%
% \subsection{Patching \LaTeX{} commands (need proper integration later)}
%
%    Most of what we have to do is adding \cs{UseHook} into several
%  \LaTeXe{} core commands, which is done for now by patching them.
%    \begin{macrocode}
%<@@=>
%    \end{macrocode}
%
% \begin{macro}{\InputIfFileExists}
%    \cs{InputIfFileExists} loads any file if it is available so we
%    have to add the hooks \texttt{file/before} and
%    \texttt{file/after} in the right places. If the file doesn't
%    exist no hooks should be executed.
%    \begin{macrocode}
\let\InputIfFileExists\@undefined
\DeclareRobustCommand \InputIfFileExists[3]{%
  \@filehook@file@push
  \IfFileExists{#1}%
    {%
%    \end{macrocode}
%    If the file exists then \cs{@curr@file} holds its name. But we
%    can't rely on that still being true after the file has been
%    processed. Thus for using the name in the file hooks we need to
%    preserve the name and then restored it for the
%    \texttt{file/after/...} hook.
%
%    The hook always refers to the \emph{actual} file that will be
%    operated on, regardless of how the user had it written in the
%    document.  \pkg{expl3}'s \cs{file_full_name:n} normalizes the file
%    name (to factor out differences in the |.tex| extension), and
%    then does a file lookup to take into account a possible path from
%    \cs{l_file_search_path_seq} and \cs{input@path}.  However only
%    the file name and extension are returned so that file hooks can
%    refer to the file by their name only.  The path to the file is
%    returned in \cs{CurrentFilePath}.
%    \begin{macrocode}
      \edef\reserved@a{\@filef@und
        \def\noexpand\CurrentFile{\CurrentFile}%
        \def\noexpand\CurrentFilePath{\CurrentFilePath}%
      }%
      \expandafter\@swaptwoargs\expandafter
        {\reserved@a}%
        {%
          #2%
          \@addtofilelist{#1}%
          \UseHook{file/before}%
%    \end{macrocode}
%    The current file name is available in \cs{@curr@file} so we use
%    that in the specific hook.
%    \begin{macrocode}
          \UseHook{file/before/\CurrentFile}%
          \@@input
        }%
%    \end{macrocode}
%    And it is restored here so we can use it once more.
%    \begin{macrocode}
      \UseHook{file/after/\CurrentFile}%
      \UseHook{file/after}%
      \@filehook@file@pop
    }%
    {\@filehook@file@pop #3}%
}
%    \end{macrocode}
% \end{macro}
%
% \begin{macro}{\set@curr@file}
%   Now we just hook into \cs{set@curr@file} to add
%   \cs{@filehook@set@curr@file} at the end, after \cs{@curr@file} is
%   set.
% \changes{0.69}{2020-70-16}}{When the argument is empty return an
%    empty macro instead of \texttt{csnameendcsname} (gh.121)}
%    \begin{macrocode}
\def\set@curr@file#1{%
  \begingroup
    \def\reserved@a{#1}%
    \ifx\reserved@a\@empty
      \xdef\@curr@file{}%
    \else
      \escapechar\m@ne
      \xdef\@curr@file{%
        \expandafter\expandafter\expandafter\unquote@name
        \expandafter\expandafter\expandafter{%
        \expandafter\string
        \csname\@firstofone#1\@empty\endcsname}}%
    \fi
  \endgroup
  \@filehook@set@curr@file{\@curr@file}%
}
%    \end{macrocode}
%  \end{macro}
%
%  \begin{macro}{\load@onefilewithoptions}
%    This macro is used when loading packages or classes.
%    \begin{macrocode}
\def\load@onefilewithoptions#1[#2][#3]#4{%
  \@pushfilename
  \xdef\@currname{#1}%
  \global\let\@currext#4%
  \let\CurrentOption\@empty
  \@reset@ptions
  \makeatletter
  \def\reserved@a{%
    \@ifl@aded\@currext{#1}%
      {\@if@ptions\@currext{#1}{#2}{}%
        {\@latex@error
            {Option clash for \@cls@pkg\space #1}%
            {The package #1 has already been loaded
             with options:\MessageBreak
             \space\space[\@ptionlist{#1.\@currext}]\MessageBreak
             There has now been an attempt to load it
              with options\MessageBreak
             \space\space[#2]\MessageBreak
             Adding the global options:\MessageBreak
             \space\space
                  \@ptionlist{#1.\@currext},#2\MessageBreak
             to your \noexpand\documentclass declaration may fix this.%
             \MessageBreak
             Try typing \space <return> \space to proceed.}}}%
      {\@pass@ptions\@currext{#2}{#1}%
       \global\expandafter
       \let\csname ver@\@currname.\@currext\endcsname\@empty
       \expandafter\let\csname\@currname.\@currext-h@@k\endcsname\@empty
       \InputIfFileExists
         {\@currname.\@currext}%
         {%
%-----------------------------------------
%    \end{macrocode}
%    When the current extension is \cs{@pkgextension} we are loading a
%    package otherwise, if it is \cs{@clsextension}, a class, so
%    depending on that we execute different hooks.  If the extension is
%    neither, then it is another type of file without special hooks.
%    \begin{macrocode}
           \ifx\@currext\@pkgextension
             \UseHook{package/before}%
             \UseHook{package/before/\@currname}%
           \else
             \ifx\@currext\@clsextension
               \UseHook{class/before}%
               \UseHook{class/before/\@currname}%
             \fi
           \fi
%    \end{macrocode}
%    The value of \cs{CurrentFile} holds during \cs{InputIfFileExists},
%    so the hooks above have that available for using.  However at this
%    point \cs{CurrentFile} is reset to its previous value by
%    \cs{@filehook@file@pop} because it doesn't know that we'll have
%    more hooks ahead.  So here (still in the \meta{true} branch of
%    \cs{InputIfFileExists}, right after actually reading the file in)
%    we'll cheat:  use \cs{@filehook@file@push} once more, so there are
%    two entries for the current file in the name stack, so that when
%    \cs{InputIfFileExists} pops it, there's still one (identical) left.
%    \begin{macrocode}
           \@filehook@file@push
%-----------------------------------------
         }%
         {\@missingfileerror\@currname\@currext}%
    \let\@unprocessedoptions\@@unprocessedoptions
    \csname\@currname.\@currext-h@@k\endcsname
    \expandafter\let\csname\@currname.\@currext-h@@k\endcsname
              \@undefined
%-----------------------------------------
%    \end{macrocode}
%    And same procedure, James, when we are finished loading, except
%    that the hook order is now reversed.
%    \begin{macrocode}
    \ifx\@currext\@pkgextension
      \UseHook{package/after/\@currname}%
      \UseHook{package/after}%
    \else
      \ifx\@currext\@clsextension
        \UseHook{class/after/\@currname}%
        \UseHook{class/after}%
      \fi
    \fi
%    \end{macrocode}
%    Now here we do \cs{@filehook@file@pop} to restore the
%    \cs{CurrentFile} before this file being loaded and fix what we've
%    done in the stack right above.
%    \begin{macrocode}
    \@filehook@file@pop
%-----------------------------------------
    \@unprocessedoptions}%
    \@ifl@ter\@currext{#1}{#3}{}%
      {\@latex@warning@no@line
         {You have requested,\on@line,
          version\MessageBreak
            `#3' of \@cls@pkg\space #1,\MessageBreak
          but only version\MessageBreak
           `\csname ver@#1.\@currext\endcsname'\MessageBreak
          is available}}%
    \ifx\@currext\@clsextension\let\LoadClass\@twoloadclasserror\fi
    \@popfilename
    \@reset@ptions}%
  \reserved@a}
%    \end{macrocode}
%
%    The code for this macro has changed between 2020/02/02 and
%    2020/10/01 so the never version is this:
%    \begin{macrocode}
\@ifl@t@r\fmtversion{2020/10/01}
{%
\def\load@onefilewithoptions#1[#2][#3]#4{%
  \@pushfilename
  \xdef\@currname{#1}%
  \global\let\@currext#4%
  \let\CurrentOption\@empty
  \@reset@ptions
  \makeatletter
  \def\reserved@a{%
    \@ifl@aded\@currext{#1}%
      {\@if@ptions\@currext{#1}{#2}{}%
        {\@latex@error
            {Option clash for \@cls@pkg\space #1}%
            {The package #1 has already been loaded
             with options:\MessageBreak
             \space\space[\@ptionlist{#1.\@currext}]\MessageBreak
             There has now been an attempt to load it
              with options\MessageBreak
             \space\space[#2]\MessageBreak
             Adding the global options:\MessageBreak
             \space\space
                  \@ptionlist{#1.\@currext},#2\MessageBreak
             to your \noexpand\documentclass declaration may fix this.%
             \MessageBreak
             Try typing \space <return> \space to proceed.}}}%
      {\@pass@ptions\@currext{#2}{#1}%
       \global\expandafter
       \let\csname ver@\@currname.\@currext\endcsname\@empty
       \expandafter\let\csname\@currname.\@currext-h@@k\endcsname\@empty
       \InputIfFileExists
         {\@currname.\@currext}%
         {%
%-----------------------------------------
           \ifx\@currext\@pkgextension
             \UseHook{package/before}%
             \UseHook{package/before/\@currname}%
           \else
             \ifx\@currext\@clsextension
               \UseHook{class/before}%
               \UseHook{class/before/\@currname}%
             \fi
           \fi
           \@filehook@file@push
%-----------------------------------------
         }%
         {\@missingfileerror\@currname\@currext}%
    \expandafter\let\csname unprocessedoptions-\@currname.\@currext\endcsname
                    \@@unprocessedoptions
    \csname\@currname.\@currext-h@@k\endcsname
    \expandafter\let\csname\@currname.\@currext-h@@k\endcsname
              \@undefined
    \ifx\@unprocessedoptions\relax
      \let\@unprocessedoptions\@undefined
    \else
      \csname unprocessedoptions-\@currname.\@currext\endcsname
    \fi
    \expandafter\let
        \csname unprocessedoptions-\@currname.\@currext\endcsname
       \@undefined
%-----------------------------------------
    \ifx\@currext\@pkgextension
      \UseHook{package/after/\@currname}%
      \UseHook{package/after}%
    \else
      \ifx\@currext\@clsextension
        \UseHook{class/after/\@currname}%
        \UseHook{class/after}%
      \fi
    \fi
    \@filehook@file@pop
%-----------------------------------------
    }%
    \@ifl@ter\@currext{#1}{#3}{}%
      {\@latex@warning@no@line
         {You have requested,\on@line,
          version\MessageBreak
            `#3' of \@cls@pkg\space #1,\MessageBreak
          but only version\MessageBreak
           `\csname ver@#1.\@currext\endcsname'\MessageBreak
          is available}}%
    \ifx\@currext\@clsextension\let\LoadClass\@twoloadclasserror\fi
    \@popfilename
    \@reset@ptions}%
  \reserved@a}
}{}%
%    \end{macrocode}
%  \end{macro}




%  \begin{macro}{\@include}
%    \begin{macrocode}
\def\@include#1 {%
  \clearpage
  \if@filesw
    \immediate\write\@mainaux{\string\@input{#1.aux}}%
  \fi
  \@tempswatrue
  \if@partsw
    \@tempswafalse
    \edef\reserved@b{#1}%
    \@for\reserved@a:=\@partlist\do
      {\ifx\reserved@a\reserved@b\@tempswatrue\fi}%
  \fi
  \if@tempswa
    \let\@auxout\@partaux
    \if@filesw
      \immediate\openout\@partaux #1.aux
      \immediate\write\@partaux{\relax}%
    \fi
%-----------------------------------------
%    \end{macrocode}
%    First we need to fix \cs{CurrentFile}, because due the pesky
%    space-delimited \cs{@include}, \cs{CurrentFile} contains the
%    \cs{include}'d file with a space.  To fix that, we re-do
%    \cs{set@curr@file}.  The 2020/10/01 release doesn't need this as
%    \cs{include} was changed to do \cs{set@curr@file} on the correct
%    file name, rather than one with a trailing space.
%    \begin{macrocode}
    \set@curr@file{#1}%
%    \end{macrocode}
%    Execute the \texttt{before} hooks just after we switched the
%    \texttt{.aux} file \ldots
%    \begin{macrocode}
    \UseHook{include/before}%
    \UseHook{include/before/#1}%
%-----------------------------------------
    \@input@{#1.tex}%
%    \end{macrocode}
%    \ldots{} then \texttt{end} hooks \ldots
%    \begin{macrocode}
%-----------------------------------------
    \UseHook{include/end/#1}%
    \UseHook{include/end}%
%-----------------------------------------
    \clearpage
%    \end{macrocode}
%    \ldots{} and after the \cs{clearpage} the \texttt{after} hooks
%    followed by another \cs{clearpage} just in case new material got
%    added (after all we need to be in  well defined state after the
%    \cs{include}).
%    \begin{macrocode}
%-----------------------------------------
    \UseHook{include/after/#1}%
    \UseHook{include/after}%
%    \end{macrocode}
%    The additional \cs{clearpage} is needed to ensure that switching
%    the \texttt{.aux} files happen at a defined point even if the
%    above hooks add further material.
%    \begin{macrocode}
    \clearpage
%-----------------------------------------
    \@writeckpt{#1}%
    \if@filesw
      \immediate\closeout\@partaux
    \fi
  \else
    \deadcycles\z@
    \@nameuse{cp@#1}%
  \fi
  \let\@auxout\@mainaux
}
%    \end{macrocode}
%
%    The code for this macro has changed between 2020/02/02 and
%    2020/10/01 so the never version is this:
%    \begin{macrocode}
\@ifl@t@r\fmtversion{2020/10/01}
{%
\def\@include#1 {%
  \clearpage
  \if@filesw
    \immediate\write\@mainaux{\string\@input{"#1.aux"}}%
  \fi
  \@tempswatrue
  \if@partsw
    \@tempswafalse
    \edef\reserved@b{#1}%
    \@for\reserved@a:=\@partlist\do
      {\ifx\reserved@a\reserved@b\@tempswatrue\fi}%
  \fi
  \if@tempswa
    \let\@auxout\@partaux
    \if@filesw
      \immediate\openout\@partaux "#1.aux"
      \immediate\write\@partaux{\relax}%
    \fi
%-----------------------------------------
    \UseHook{include/before}%
    \UseHook{include/before/#1}%
%-----------------------------------------
    \@input@{#1.tex}%
%-----------------------------------------
    \UseHook{include/end/#1}%
    \UseHook{include/end}%
%-----------------------------------------
    \clearpage
%-----------------------------------------
    \UseHook{include/after/#1}%
    \UseHook{include/after}%
    \clearpage
%-----------------------------------------
    \@writeckpt{#1}%
    \if@filesw
      \immediate\closeout\@partaux
    \fi
  \else
    \deadcycles\z@
    \@nameuse{cp@#1}%
  \fi
  \let\@auxout\@mainaux}
}{}
%    \end{macrocode}
%  \end{macro}
%
%
%
% \subsection{High-level interfaces for \LaTeX{}}
%
%    None so far and the general feeling for now is that the hooks are
%    enough. Packages like \pkg{filehook}, etc., may use them to set
%    up their interfaces (samples are given below) but for the now the
%    kernel will not provide any.
%
%
%
%    \begin{macrocode}
%</2ekernel>
%    \end{macrocode}
%
%
%
% \section{Package emulation for compatibility}
%
%
% \subsection{Package \pkg{filehook} emulation}
%
%    This is a partial implementation of the  \pkg{filehook}
%    interfaces. It is only meant for guidance in case that package
%    gets updated to use the hook management.
%
%    Not implemented are:
%\begin{verbatim}
%  \AtBeginOfFiles
%  \AtEndOfFiles
%  \AtBeginOfInputs
%  \AtEndOfInputs
%  \AtBeginOfInputFile
%  \AtEndOfInputFile
%\end{verbatim}
%
%    \begin{macrocode}
%<*filehook-draft>
%    \end{macrocode}
%
%    \begin{macrocode}
\newcommand\AtBeginOfEveryFile [1]
  {\AddToHook{file/before}{#1}}
\newcommand\AtEndOfEveryFile [1]
  {\AddToHook{file/after}{#1}}
%    \end{macrocode}
%    
%    \begin{macrocode}
\newcommand\AtBeginOfIncludes [1]
  {\AddToHook{include/before}{#1}}
\newcommand\AtEndOfIncludes [1]
  {\AddToHook{include/end}{#1}}
\newcommand\AfterIncludes [1]
  {\AddToHook{include/after}{#1}}
%    \end{macrocode}
%    
%    \begin{macrocode}
\newcommand\AtBeginOfPackages [1]
  {\AddToHook{package/before}{#1}}
\newcommand\AtEndOfPackages [1]
  {\AddToHook{package/after}{#1}}
%    \end{macrocode}
%    
%    \begin{macrocode}
\newcommand\AtBeginOfClasses [1]
  {\AddToHook{class/before}{#1}}
\newcommand\AtEndOfClasses [1]
  {\AddToHook{class/after}{#1}}
%    \end{macrocode}
%
%    For normal files we drop the \texttt{.tex} extension for now:
%    \begin{macrocode}
\newcommand\AtBeginOfFile [2]
  {\AddToHook{file/before/#1}{#2}}
\newcommand\AtEndOfFile [2]
  {\AddToHook{file/after/#1}{#2}}
%    \end{macrocode}
%    
%    \begin{macrocode}
\DeclareDocumentCommand \AtBeginOfPackageFile {smm}
   {\IfBooleanTF{#1}%
     {\@ifpackageloaded{#2}%
         {#3}%
         {\AddToHook{package/before/#2}{#3}}}%
     {\AddToHook{package/before/#2}{#3}}%
   }
\DeclareDocumentCommand \AtEndOfPackageFile {smm}
   {\IfBooleanTF{#1}%
     {\@ifpackageloaded{#2}%
         {#3}%
         {\AddToHook{package/after/#2}{#3}}}%
     {\AddToHook{package/after/#2}{#3}}%
   }
%    \end{macrocode}
%
%    Are the * forms here of any use? I know they are use 3--4 times
%    on CTAN but I wonder if those are real or mistaken usages.  
%    \begin{macrocode}
\DeclareDocumentCommand \AtBeginOfClassFile {smm}
   {\IfBooleanTF{#1}%
     {\@ifclassloaded{#2}%
         {#3}%
         {\AddToHook{class/before/#2}{#3}}}%
     {\AddToHook{class/before/#2}{#3}}%
   }
\DeclareDocumentCommand \AtEndOfClassFile {smm}
   {\IfBooleanTF{#1}%
     {\@ifclassloaded{#2}%
         {#3}%
         {\AddToHook{class/after/#2}{#3}}}%
     {\AddToHook{class/after/#2}{#3}}%
   }
%    \end{macrocode}
%    
%    \begin{macrocode}
\newcommand\AtBeginOfIncludeFile [2]
  {\AddToHook{include/before/#1}{#2}}
\newcommand\AtEndOfIncludeFile [2]
  {\AddToHook{include/end/#1}{#2}}
\newcommand\AfterIncludeFile [2]
  {\AddToHook{include/after/#1}{#2}}
%    \end{macrocode}
%
%
%    This is missing some interfaces so disabling the package isn't
%    really correct, but then this code above is not supposed to stay
%    like this anyway.
%    \begin{macrocode}
\expandafter\let\csname ver@filehook.sty\endcsname\fmtversion
\@namedef {ver@filehook.sty}{2020/10/01}
%    \end{macrocode}
%
%
%    \begin{macrocode}
%</filehook-draft>
%    \end{macrocode}
%
%
%
%
% \subsection{Package \pkg{scrlfile} emulation}
%
%    This is a partial implementation of the  \pkg{scrlfile}
%    interfaces. It is only meant for guidance in case that package
%    gets updated to use the hook management.
%
%    \begin{macrocode}
%<*scrlfile-draft>
%    \end{macrocode}
%
%    I think this is roughly correct (using the \texttt{file/...}
%    hooks rather than the class or package hooks at least for the
%    \cs{After...} commands but it needs some further verification.
%
%    The star and plus variants haven't been implemented so far, this
%    is only a rough draft.
%    \begin{macrocode}
\newcommand\BeforeClass[2]
  {\AddToHook{file/before/#1.cls}{#2}}
\newcommand\AfterClass [2]
  {\AddToHook{file/after/#1.cls}{#2}}
\newcommand\AfterAtEndOfClass [2]
  {\AddToHook{class/after/#1}{#2}}
%    \end{macrocode}
%
%    \begin{macrocode}
\newcommand\BeforePackage [2]
  {\AddToHook{package/before/#1.sty}{#2}}
\newcommand\AfterPackage [2]
  {\AddToHook{file/after/#1.sty}{#2}}
\newcommand\AfterEndOfPackage [2]
  {\AddToHook{package/after/#1}{#2}}
%    \end{macrocode}
%
%    \begin{macrocode}
\newcommand\BeforeFile [2]
  {%
  \typeout{BeforeFile: #1!!!}%
   \AddToHook{file/before/#1}{#2}}
\newcommand\AfterFile [2]
  {%
    \typeout{AfterFile: #1!!!}%
    \AddToHook{file/after/#1}{#2}}
%    \end{macrocode}
%
%    This is missing some interfaces so disabling the package isn't
%    really correct, but then this code above is not supposed to stay
%    like this anyway.
%    \begin{macrocode}
\expandafter\let\csname ver@scrlfile.sty\endcsname\fmtversion
\@namedef {ver@scrlfile.sty}{2020/10/01}
%    \end{macrocode}
%
%    \begin{macrocode}
%</scrlfile-draft>
%    \end{macrocode}
%
%
%
% \section{A sample package for structuring the log output}
%
%    \begin{macrocode}
%<*structuredlog>
%<@@=filehook>
%    \end{macrocode}
%
%    \begin{macrocode}
\ProvidesExplPackage
    {structuredlog}{\ltfilehookdate}{\ltfilehookversion}
    {Structuring the TeX transcript file}
%    \end{macrocode}



%    \begin{macrocode}
\int_new:N \g_@@_nesting_level_int
%    \end{macrocode}

%    \begin{macrocode}
\tl_new:N   \g_@@_nesting_prefix_tl
\tl_gset:Nn \g_@@_nesting_prefix_tl { }
%    \end{macrocode}
%    
%    \begin{macrocode}
\AddToHook{file/before}{
  \int_gincr:N \g_@@_nesting_level_int
  \tl_gput_right:Nn\g_@@_nesting_prefix_tl {=}
  \iow_term:x {
    \g_@@_nesting_prefix_tl \space
    ( LEVEL~ \int_use:N \g_@@_nesting_level_int \space START )~
    \CurrentFile  ^^J
  }
}
%    \end{macrocode}
%    We don't want to install the \hook{file/after} hook immediately,
%    because that would mean it is the first time executed when the
%    package finishes. We therefore put the declaration inside
%    \cs{AddToHookNext} so that it gets only installed when we have
%    left the package. 
%    \begin{macrocode}
\AddToHookNext{file/after}{
  \AddToHook{file/after}{
    \iow_term:x {
      \g_@@_nesting_prefix_tl \space
      ( LEVEL~ \int_use:N \g_@@_nesting_level_int \space STOP )~
      \CurrentFile  ^^J
    }
    \int_gdecr:N \g_@@_nesting_level_int
    \tl_gset:Nx \g_@@_nesting_prefix_tl
       {\exp_after:wN \use_none:n \g_@@_nesting_prefix_tl}
  }
}
%    \end{macrocode}
%
%    We have to manually increment the level because now that we have
%    installed the code in \texttt{file/after} it gets decremented
%    when we leave the package without ever being incremented upon
%    entry.
%    \begin{macrocode}
%\int_incr:N\g_@@_nesting_level_int
%    \end{macrocode}
%
%    \begin{macrocode}
%</structuredlog>
%    \end{macrocode}

%    \Finale
%


%%%%%%%%%%%%%%%%%%%%%%%%%%%%%%%%%%%%%%%%%%%%%  
\endinput
%%%%%%%%%%%%%%%%%%%%%%%%%%%%%%%%%%%%%%%%%%%%%  



                    }
%    \end{macrocode}
%
%  \subsection{Package options}
%
%    For now we offer a simple debug option which turns on a lot of
%    strange \cs{typeout} messages, nothing fancy.
%    \begin{macrocode}
\ExplSyntaxOn
%    \end{macrocode}
%    
%    \begin{macrocode}
\hook_debug_off:
\DeclareOption { debug } { \hook_debug_on:
                           \shipout_debug_on: }
%    \end{macrocode}
%
%
%    For now we offer a simple debug option which turns on a lot of
%    strange \cs{typeout} messages, nothing fancy.
%    \begin{macrocode}
\shipout_debug_off:
\DeclareOption { debug-shipout } { \shipout_debug_on: }
%    \end{macrocode}
%    
%    \begin{macrocode}
\ProcessOptions
%    \end{macrocode}
%
%
%  \subsection{Temporarily patching package until changed}
%
%
%
%    \pkg{filehook} support until that package is patched:
%    \begin{macrocode}
\RequirePackage{filehook-ltx}
%    \end{macrocode}
%
%
%
%    \begin{macrocode}
%</package>
%    \end{macrocode}
%
%
%
% \Finale
%



                     % \iffalse meta-comment
%
%% File: ltshipout.dtx (C) Copyright 2020 Frank Mittelbach, LaTeX Team
%
% It may be distributed and/or modified under the conditions of the
% LaTeX Project Public License (LPPL), either version 1.3c of this
% license or (at your option) any later version.  The latest version
% of this license is in the file
%
%    https://www.latex-project.org/lppl.txt
%
%
% The development version of the bundle can be found below
%
%    https://github.com/FrankMittelbach/...
%
% for those people who are interested or want to report an issue.
%
%    \begin{macrocode}
\providecommand\ltshipoutversion{v0.9a}
\providecommand\ltshipoutdate{2020/07/15}
%    \end{macrocode}
%
%<*driver>

\documentclass{l3doc}

% bug fix fo l3doc.cls
\ExplSyntaxOn
\cs_set_protected:Npn \__codedoc_macro_typeset_one:nN #1#2
  {
    \vbox_set:Nn \l__codedoc_macro_box
      {
        \vbox_unpack_drop:N \l__codedoc_macro_box
        \hbox { \llap { \__codedoc_print_macroname:nN {#1} #2
            \MacroFont       % <----- without it the \ is in lmr10 if a link is made
            \      
        } }
      }
    \int_incr:N \l__codedoc_macro_int
  }
\ExplSyntaxOff

\EnableCrossrefs
\CodelineIndex
\begin{document}
  \DocInput{ltshipout.dtx}
\end{document}
%</driver>
%
% \fi
%
%
% \long\def\fmi#1{\begin{quote}\itshape Todo: #1\end{quote}}
%
% \newcommand\hook[1]{\texttt{#1}}
%
%
% \title{The \texttt{ltshipout} package\thanks{This package has version
%    \ltshipoutversion\ dated \ltshipoutdate, \copyright\ \LaTeX\
%    Project.}}
%
% \author{Frank Mittelbach}
%
% \maketitle
%
%
% \tableofcontents
%
% \section{Introduction}
%
%    The code provides an interface to the \cs{shipout} primitive of
%    \TeX{} which is called when a finished pages is finally
%    \enquote{shipped out} to the target output file, e.g., the
%    \texttt{.dvi} or \texttt{.pdf} file.
%    A good portion of the code is based on ideas by Heiko Oberdiek
%    implemented in his packages \pkg{atbegshi} and \pkg{atenddvi}
%    even though the interfaces are somewhat
%    different.\footnote{Heiko's interfaces are emulated by the kernel
%    code, if a document requests his packages, so older documents
%    will continue to work.}
%
%  \subsection{Overloading the \cs{shipout} primitive}
%
%
% \begin{function}{\shipout}
%    With this implementation \TeX's shipout primitive is no longer
%    available for direct use. Instead \cs{shipout} is running some
%    (complicated) code that picks up the box to be shipped out
%    regardless of how that is done, i.e., as a constructed \cs{vbox}
%    or \cs{hbox} or as a box register.
%
%    It then stores it in a named box register.  This box can then be
%    manipulated through a set of hooks after which it is shipped out
%    for real.
% \end{function}

%  \begin{variable}{\ShipoutBox,\l_shipout_box}
%    This box register is called \cs{ShipoutBox} (alternatively available via the
%    L3 name \cs{l_shipout_box}).
%  \end{variable}
%

%  \begin{variable}{\l_shipout_box_ht_dim,
%                   \l_shipout_box_dp_dim,\l_shipout_box_wd_dim,
%                   \l_shipout_box_ht_plus_dp_dim}
%    The shipout box dimensions are available in the L3 registers
%    \cs{l_shipout_box_ht_dim}, etc.\ (there are no \LaTeXe{}
%    names).\footnotemark{} These variables can be used
%    inside the hook code for \hook{shipout/before},
%    \hook{shipout/foreground} and \hook{shipout/background} if needed.
%  \end{variable}
%  \footnotetext{Might need changing, but HO's version as strings
%    is not really helpful I think).}
%
%
%
%
% \subsection{Provided hooks}
%
%  \begin{variable}{shipout/before,
%                   shipout/foreground,shipout/background,
%                   shipout/firstpage,
%                   shipout/lastpage}
%    The code offers a number of hooks into which packages (or the
%    user) can add code to support different use cases.
%    These are:
%    \begin{description}
%    \item[\hook{shipout/before}]
%
%       This hook is executed after the finished page has been stored in
%       \cs{ShipoutBox} / \cs{l_shipout_box}).
%       It can be used to alter that box content or to discard it
%       completely (see \cs{DiscardShipoutBox} below).
%
%    \item[\hook{shipout/background}]
%
%       This hook adds a picture environment into the background of
%       the page (with the \texttt{(0,0)} coordinate in the top-left
%       corner using a \cs{unitlength} of \texttt{1pt}.
%
%       It should therefore only receive \cs{put} commands or other
%       commands suitable in a \texttt{picture} environment and the
%       vertical coordinate values would normally be
%       negative.
%
%    \fmi{Again, mainly \pkg{atbegshi} compatibility. Not
%       sure it is best to have  to always use negative
%       coordinates.}
%
%       Technically this is implemented by adding a zero-sized
%       \cs{hbox} as the very first item into the \cs{ShipoutBox}
%       containing that \texttt{picture} environment. Thus the rest of
%       the box content will overprint what ever is typeset by that hook.
%
%
%    \item[\hook{shipout/foreground}]
%
%       This hook adds a picture environment into the foreground of
%       the page (with the \texttt{(0,0)} coordinate in the top-left
%       corner using a \cs{unitlength} of \texttt{1pt}.
%
%       Technically this is implemented by adding a zero-sized
%       \cs{hbox} as the very last item into the \cs{ShipoutBox} and
%       raising it up so that it still has its \texttt{(0,0)} point in
%       the top-left corner.
%       But being placed after the main box content it will be typeset
%       later and thus overprints it (i.e., is in the foreground).
%
%
%    \item[\hook{shipout/firstpage}]
%
%       The material from this hook is executed only once at the very
%       beginning of the first output page. It should only contain
%       \cs{special} commands needed to direct post processors
%       handling the \texttt{.dvi} or \texttt{.pdf} output.\fmi{not
%       sure it has to be that restrictive.}
%
%       In \LaTeXe{} that was already existing but implemented as a box
%       register \cs{@begindvibox}.
%       \fmi{drop or at least mark the obsolete code in latex.ltx}
%
%
%    \item[\hook{shipout/lastpage}]
%
%       The corresponding hook to add \cs{special}s at the very end of
%       the output file. It is only entered on the very last page.
%
%       It may not be possible for \LaTeX{} to correctly determine which page is
%       the last one without several reruns. If this happens and the
%       hook is non-empty then \LaTeX{} will add an extra page to
%       place the material and also request a rerun to get the correct
%       placement sorted out.
%
%    \end{description}
%  \end{variable}
%
%    As mentioned above the hook \hook{shipout/before} is executed
%    first and can manipulate the prepared shipout box stored in
%    \cs{ShipoutBox} or set things up for use in \cs{write} during the
%    actual shipout. The other hooks are added inside hboxes to the
%    box being shipped out in the following
%    order:
%    \begin{center}
%    \begin{tabular}{ll}
%       \hook{shipout/firstpage}   & only on the first page \\
%       \hook{shipout/background}  &                        \\
%       \meta{boxed content of \cs{ShipoutBox}} &             \\
%       \hook{shipout/foreground}  &                       \\
%       \hook{shipout/lastpage}    & only on the last page \\
%    \end{tabular}
%    \end{center}
%    If any of the hooks has no code then that particular no box is
%    added at that point.
%
%    In a document that doesn't produce pages, e.g., only makes
%    \cs{typeout}s, none of the hooks are executed (as there is no
%    \cs{shipout}) not even the \hook{shipout/lastpage} hook.
%
% \begin{function}{\AtBeginDvi,\AtEndDvi}
%    \cs{AtBeginDvi} is the existing \LaTeXe{} interface to fill the
%    \hook{shipout/firstpage} hook. This is not really a good name
%    as it is not just supporting \texttt{.dvi} but also \texttt{.pdf}
%    output or \texttt{.dvx}.
%
%    \cs{AtEndDvi} is the counterpart that was not available in the
%    kernel but only through the package \pkg{atenddvi}. It fills the
%    \hook{shipout/lastpage} hook.
%
%    \fmi{better names? Any suggestions?}
%
% \end{function}
%
% \subsection{Special commands for use inside the hooks}
%
% \begin{function}{\DiscardShipoutBox,\shipout_discard_box:}
%   \begin{syntax}
%     \cs{AddToHookNext} \texttt{\{shipout/before\} \{...\cs{DiscardShipoutBox}...\}}
%   \end{syntax}
%    The \cs{DiscardShipoutBox} declaration (L3 name
%    \cs{shipout_discard_box:})
%    requests that on the next
%    shipout the page box is thrown away instead of being shipped to
%    the \texttt{.dvi} or \texttt{.pdf} file.
%
%    Typical applications wouldn't do this unconditionally, but have
%    some processing logic that decides to use or not to use the page.
%
%    Note that if this declaration is used directly in the document it
%    may depend on the placement to which page it applies, given that
%    \LaTeX{} output routine is called in an asynchronous manner!
%
%    \fmi{Once we have a new mark mechanism available we can improve
%    on that and make sure that the declaration applies to the page
%    that contains it.}
%  \end{function}
%
%    In the \pkg{atbegshi} package there are a number of additional
%    commands for use inside the \hook{shipout/before} hook. They
%    should normally not be needed any more as one can instead simply
%    add code to the hooks \hook{shipout/before},
%    \hook{shipout/background} or
%    \hook{shipout/foreground}.\footnote{If that assumption turns out to
%    be wrong it would be trivial to change them to public functions
%    (right now they are private).} If \pkg{atbegshi} gets loaded then
%    those commands become available as public functions with their original
%    names as given below.
%
%
% \subsection{Information counters}
%
%
%  \begin{variable}{\ReadonlyShipoutCounter,\g_shipout_readonly_int}
%   \begin{syntax}
%     \cs{ifnum}\cs{ReadOnlyShipoutCounter}\texttt{=...}
%     \cs{int_use:N} \cs{g_shipout_readonly_int} \texttt{\% expl3 usage}
%   \end{syntax}
%    This integer holds the number of pages shipped out up to now
%    (including the one to be shipped out when inside the output
%    routine). More precisely, it is incremented only after it is
%    clear that a page will be shipped out, i.e., after the
%    \hook{shipout/before} hook (because that might discard the page)!

%    Just like with the \texttt{page} counter its value is
%    only accurate within the output routine. In the body of the
%    document it may be off by one as the output routine is called
%    asynchronously!
%
%    Also important: it \emph{must not} be set, only read. There are
%    no provisions to prevent that but if you do, chaos will be the
%    result. To emphasize this fact it is not provided as a \LaTeX{}
%    counter but as a \TeX{} counter (i.e., a command), so
%    \cs{Alph}\Arg{\cs{ReadonlyShipoutCounter}} etc, would not work.
%  \end{variable}
%
%  \begin{variable}{totalpages,\g_shipout_totalpages_int}
%   \begin{syntax}
%     \cs{arabic}\texttt{\{totalpages\}}
%     \cs{int_use:N} \cs{g_shipout_totalpage_int} \texttt{\% expl3 usage}
%   \end{syntax}
%    In contrast to \cs{ReadonlyShipoutCounter}, the
%    \texttt{totalpages} counter is a \LaTeX{} counter and incremented
%    for each shipout attempt including those pages that are discarded
%    for one or the other reason. Again \hook{shipout/before} sees
%    the counter before it is incremented).
%
%    Furthermore, while it is incremented for each page, its value is
%    never used by \LaTeX. It can therefore be freely reset or changed by user
%    code, for example, to additionally count a number of pages that
%    are not build by \LaTeX\ but are added in a later part of the
%    process, e.g., cover pages or picture pages made externally.
%
%    Important: as this is a page-related counter its value is only
%    reliable inside the output routine!
%  \end{variable}
%

%
% \subsection{Debugging shipout code}
%
% \begin{function}{\DebugShipoutOn,\DebugShipoutOff,
%         \shipout_debug_on:,\shipout_debug_off:}
%   \begin{syntax}
%     \cs{DebugShipoutOn}
%   \end{syntax}
%    Turn the debugging of shipout code on or off. This displays
%    changes made to the shipout data structures.  \fmi{This needs
%    some rationalizing and will probably not stay this way.}
% \end{function}
%
%

%
% \section{Emulating commands from other packages}
%
%    The packages in this section are no longer necessary but as they
%    are used in other packages they are emulated when they are loaded
%    via \cs{usepackage} or \cs{RequirePackage}.
%
%
% \subsection{Emulating \pkg{atbegshi}}
%
%
% \begin{function}{\AtBeginShipoutUpperLeft,\AtBeginShipoutUpperLeftForeground}
%   \begin{syntax}
%     \cs{AddToHook} \texttt{\{shipout/before\} \{...\cs{AtBeginShipoutUpperLeft}}\Arg{code}\texttt{...\}}
%   \end{syntax}
%    This adds a \texttt{picture} environment into the background of the shipout
%    box expecting \meta{code} to contain \texttt{picture}
%    commands. The same effect can be obtained by simply using kernel
%    features as follows:
%    \begin{quote}
%      \cs{AddToHook}\texttt{\{shipout/background\}}\Arg{code}
%    \end{quote}
%    There is one technical difference: if
%    \cs{AtBeginShipoutUpperLeft} is used several times each
%    invocation is put into its own box inside the shipout box whereas
%    all \meta{code} going into \hook{shipout/background} ends up
%    all in the same box in the order it is added or sorted based on
%    the rules for the hook chunks.
%
%    \cs{AtBeginShipoutUpperLeftForeground} is similar with the
%    difference that the \texttt{picture} environment is placed in the
%    foreground. To model it with the kernel functions use the hook
%    \hook{shipout/foreground} instead.
% \end{function}
%
%
% \begin{function}{\AtBeginShipoutAddToBox,\AtBeginShipoutAddToBoxForeground}
%   \begin{syntax}
%     \cs{AddToHook} \texttt{\{shipout/before\} \{...\cs{AtBeginShipoutAddToBox}}\Arg{code}\texttt{...\}}
%   \end{syntax}
%    These work like \cs{AtBeginShipoutUpperLeft} and
%    \cs{AtBeginShipoutUpperLeftForeground} with the difference that
%    \meta{code} is directly placed into an \cs{hbox} inside the
%    shipout box and not surrounded by a \texttt{picture} environment.
%
%    To emulate them using \hook{shipout/background} or
%    \hook{shipout/foreground} you may have to wrap \meta{code} into
%    a \cs{put} statement but if the code is not doing any typesetting
%    just adding it to the hook should be sufficient.
% \end{function}
%


% \begin{function}{\AtBeginShipoutBox}
%    This is the name of the shipout box as \pkg{atbegshi} knows it.
% \end{function}
%
% \begin{function}{\AtBeginShipoutInit}
%   By default \pkg{atbegshi} delayed its action until
%    \verb=\beg{document}=.  This command was forcing it in an earlier
%    place. With the new concept it does nothing.
% \end{function}
%
% \begin{function}{\AtBeginShipout,\AtBeginShipoutNext}
%   \begin{syntax}
% \cs{AtBeginShipout}\Arg{code} $\equiv$ \cs{AddToHook}\texttt{\{shipout/before\}}\Arg{code}
% \cs{AtBeginShipoutNext}\Arg{code} $\equiv$ \cs{AddToHookNext}\texttt{\{shipout/before\}}\Arg{code}
%   \end{syntax}
%   This is equivalent to filling the \hook{shipout/before} hook
%    by  either using \cs{AddToHook} or \cs{AddToHookNext}, respectively.
% \end{function}
%
% \begin{function}{\AtBeginShipoutFirst,\AtBeginShipoutDiscard}
%   The \pkg{atbegshi} names for \cs{AtBeginDvi} and \cs{DiscardShipoutBox}.
% \end{function}
%



% \subsection{Emulating \pkg{everyshi}}
%


% \begin{function}{\EveryShipout}
%   \begin{syntax}
% \cs{EveryShipout}\Arg{code} $\equiv$ \cs{AddToHook}\texttt{\{shipout/before\}}\Arg{code}
%   \end{syntax}
% \end{function}
%
% \begin{function}{\AtNextShipout}
%   \begin{syntax}
% \cs{AtNextShipout}\Arg{code} $\equiv$ \cs{AddToHookNext}\texttt{\{shipout/before\}}\Arg{code}
%   \end{syntax}
% \end{function}
%


% \subsection{Emulating \pkg{atenddvi}}
%
% The \pkg{atenddvi} package implemented only a single command:
%    \cs{AtEndDvi} and that is now available out of the box.



% \subsection{Emulating \pkg{everypage}}
%
%    This page takes over the original \cs{@begindvi} hook and replaces 
%    it. It should be all covered by the hooks offered here (details
%    need checking) and thus could simply use the provided hooks
%    rather than defining its own.
%
%
%
% \StopEventually{\setlength\IndexMin{200pt}  \PrintIndex  }
%
%
% \section{The Implementation}
%    \begin{macrocode}
%<*2ekernel>
%<@@=shipout>
\ExplSyntaxOn
%    \end{macrocode}
%
%
%  \subsection{Debugging}
%
%  \begin{macro}{\g_@@_debug_bool}
%    Holds the current debugging state.
%    \begin{macrocode}
\bool_new:N \g_@@_debug_bool
%    \end{macrocode}
%  \end{macro}
%
%  \begin{macro}{\shipout_debug_on:,\shipout_debug_off:}
%  \begin{macro}{\@@_debug:n}
%  \begin{macro}{\@@_debug_gset:}
%    Turns debugging on and off by redefining \cs{@@_debug:n}.
%    \begin{macrocode}
\cs_new_eq:NN \@@_debug:n  \use_none:n
\cs_new_protected:Npn \shipout_debug_on:
  {
    \bool_gset_true:N \g_@@_debug_bool
    \@@_debug_gset:
  }
\cs_new_protected:Npn \shipout_debug_off:
  {
    \bool_gset_false:N \g_@@_debug_bool
    \@@_debug_gset:
  }
\cs_new_protected:Npn \@@_debug_gset:
  {
    \cs_gset_protected:Npx \@@_debug:n ##1
      { \bool_if:NT \g_@@_debug_bool {##1} }
  }
%    \end{macrocode}
%  \end{macro}
%  \end{macro}
%  \end{macro}
%

%  \begin{macro}{shipout/before,
%                   shipout/foreground,shipout/background,
%                   shipout/firstpage,
%                   shipout/lastpage}
%    Declaring all hooks for the shipout code.
%    \begin{macrocode}
\hook_new:n{shipout/before}
\hook_new:n{shipout/foreground}
\hook_new:n{shipout/background}
\hook_new:n{shipout/firstpage}
\hook_new:n{shipout/lastpage}
%    \end{macrocode}
%  \end{macro}




%  \begin{macro}{\ShipoutBox,\l_shipout_box}
%    The box filled with the page to be shipped out (both L3 and
%       \LaTeXe{} name).
%    \begin{macrocode}
\box_new:N  \l_shipout_box
%    \end{macrocode}
%    
%    \begin{macrocode}
\cs_set_eq:NN \ShipoutBox \l_shipout_box
%    \end{macrocode}
%  \end{macro}




%  \begin{macro}{\@@_execute:}
%    This is going to the be the code run by \cs{shipout}. The code
%    follows closely the  ideas from \pkg{atbegshi}, so not
%    documenting that here for now.
%    \begin{macrocode}
\cs_set:Npn\@@_execute: {
  \tl_set:Nx \l_@@_group_level_tl
     { \int_value:w \tex_currentgrouplevel:D }
  \tex_afterassignment:D \@@_execute_test_level:
  \tex_setbox:D \l_shipout_box
}
%    \end{macrocode}
%  \end{macro}


%  \begin{macro}{\shipout}
%    Overloading the \cs{shipout} primitive:
%    \begin{macrocode}
\cs_gset_eq:NN \shipout \@@_execute:
%    \end{macrocode}
%  \end{macro}


%  \begin{macro}{\l_@@_group_level_tl}
%    Helper token list to record the group level at which
%    \cs{@@_execute:} is encountered.  \begin{macrocode}
\tl_new:N \l_@@_group_level_tl
%    \end{macrocode}
%  \end{macro}

 

%  \begin{macro}{\@@_execute_test_level:}
%    If the group level has changed then we are still constructing
%    \cs{l_shipout_box} and to continue we need to wait until the
%    current group has finished, hence the \cs{tex_aftergroup:D}.
%    \begin{macrocode}
\cs_new:Npn \@@_execute_test_level: {
  \int_compare:nNnT
     \l_@@_group_level_tl < \tex_currentgrouplevel:D 
     \tex_aftergroup:D
  \@@_execute_cont:
}
%    \end{macrocode}
%  \end{macro}


%  \begin{macro}{\@@_execute_cont:}
%    When we have reached this point the shipout box has been
%    processed and is available in \cs{l_shipout_box} and ready for
%    real ship out (perhaps)..
%
%    First we quickly check if it is void (can't happen in the
%    standard \LaTeX{} output routine but \cs{shipout} might be called
%    from a package that has some special processing logic). If it is
%    void we aren't shipping anything out and processing ends.\footnote{In that
%    case we don't reset the deadcyles, that would be up to the OR
%    processing logic to do.}
%    \begin{macrocode}
\cs_new:Npn \@@_execute_cont: {
  \box_if_empty:NTF \l_shipout_box
    { \PackageWarning{ltshipout}{Ignoring~ void~ shipout~ box} }
    {
%    \end{macrocode}
%    Otherwise we assume that we will ship something and prepare for
%    final adjustments (in particular setting the state of
%    \cs{protect} while we are running the hook code).
%    \begin{macrocode}
      \bool_gset_false:N \g_@@_discard_bool
      \set@typeset@protect
%    \end{macrocode}
%    We also store the current shipout box dimension in registers, so that
%    they can be used in the hook code.\footnote{This is not really
%    necessary as the code could access them via \cs{box_ht:N}, etc.,
%    but it is perhaps convenient.}
%    \begin{macrocode}
      \@@_get_box_size:N \l_shipout_box
%    \end{macrocode}
%    Then we execute the \hook{shipout/before} hook.
%    \begin{macrocode}
      \hook_use:n {shipout/before}
%    \end{macrocode}
%    In \cs{g_shipout_totalpages_int} we count all shipout attempts so
%    we increment that counter already here (the other one is
%    incremented later when we know for sure that we do a
%    \cs{shipout}.
%
%     We increment it after running the above hook so that the values
%    for \cs{g_shipout_totalpages_int} and \cs{} are in sync while the
%    hook is executed (in the case that totalpages isn't manually
%    altered or through discarding pages that is).
%    \begin{macrocode}
      \int_gincr:N \g_shipout_totalpages_int
%    \end{macrocode}
%    The above hook might contain code that requests the page to be discarded so
%    we now test for it.
%    \begin{macrocode}
      \bool_if:NTF \g_@@_discard_bool
        { \PackageInfo{ltshipout}{Completed~ page~ discarded}
          \bool_gset_false:N \g_@@_discard_bool
%    \end{macrocode}
%    As we are discarding the page box and not shipping anything out,
%    we need to do some house cleaning and reset \TeX's deadcycles so
%    that it doesn't complain about too many calls to the OR without
%    any shipout.
%    \begin{macrocode}
          \tex_deadcycles:D \c_zero_int
%    \end{macrocode}
%    \fmi{In \pkg{atbegshi} the box was dropped but is that actually
%    needed? Or the resetting of \cs{protect} to its kernel value?}
%    \begin{macrocode}
%          \group_begin:
%            \box_set_eq_drop:NN \l_shipout_box \l_shipout_box
%          \group_end:
%          \cs_set_eq:NN \protect \exp_not:N
        }
%    \end{macrocode}
%    Even if there was no explicit request to discard the box it is
%    possible that the code for the hook \hook{shipout/before} has
%    voided the box (by mistake or deliberately). We therefore test
%    once more but this time make it a warning, because the best
%    practice way is to use the request mechanism.
%    \begin{macrocode}
        { \box_if_empty:NTF \l_shipout_box
            { \PackageWarning{ltshipout}{
                Shipout~ box~ was~ voided~ by~ hook,\MessageBreak
                ignoring~ shipout~ box  }
            }
%    \end{macrocode}
%    Finally, if the box is still non-empty we are nearly ready to
%    ship it out.
%    First we increment the total page counter so that we can later
%    test if we have reached the final page according to our available
%    information.\footnote{Doing that earlier would be wrong because we
%    might end up with the last page counted but discard and then we
%    have no place to add the final objects into the output file.}
%    \begin{macrocode}
            {
              \int_gincr:N \g_shipout_readonly_int
              \@@_debug:n {
                \typeout{Absolute~ page~ =~ \int_use:N \g_shipout_readonly_int
                         \space (target:~ \@abspage@last)}
              }
%    \end{macrocode}
%    Then we store the box sizes again (as they may have
%    changed) and then look at the hooks \hook{shipout/foreground}
%    and \hook{shipout/background}. If either or both are non-empty
%    we add a \texttt{picture} environment to the box (in the
%    foreground and or in the background) and execute the hook code
%    inside that environment.
%    
%    \begin{macrocode}
              \@@_get_box_size:N \l_shipout_box
              \@kernel@before@shipout@foreground
              \hook_if_empty:nF {shipout/foreground}
                   { \@@_add_foreground_picture:n
                     { \hook_use:n {shipout/foreground} } }
              \hook_if_empty:nF {shipout/background}
                   { \@@_add_background_picture:n
                     { \hook_use:n {shipout/background} } }
%    \end{macrocode}
%    We then run \cs{@@_execute_firstpage_hook:} that adds
%    the content of the hook \hook{shipout/firstpage} to the
%    start of the first page (if non-empty). It is then redefined to
%    do nothing on later pages.
%    \begin{macrocode}
              \@@_execute_firstpage_hook:
%    \end{macrocode}
%    The we check if we have to add the \hook{shipout/lastpage} hook
%    because we have reached the last page. This test will be false for
%    all but one (and hopefully the correct) page.
%    \begin{macrocode}
              \int_compare:nNnT \@abspage@last = \g_shipout_readonly_int
                { \hook_if_empty:nF {shipout/lastpage}
                    { \@@_debug:n { \typeout{Executing~ lastpage~ hook~
                          on~ page~ \int_use:N \g_shipout_readonly_int }        }
                      \_@@_add_foreground_box:n { \UseHook{shipout/lastpage}
                                                  \@kernel@after@shipout@lastpage }
                    }
                    \bool_gset_true:N \g_@@_lastpage_handled_bool
                }
%    \end{macrocode}
%    Finally we run the actual \TeX{} primitive for shipout. As that will
%    expand delayed \cs{write} statements inside the page in which
%    protected commands should not expand we first change \cs{protect}
%    to the appropriate definition for that case.
%    \begin{macrocode}
              \cs_set_eq:NN \protect \exp_not:N
              \tex_shipout:D \box_use:N \l_shipout_box
            }
        }
    }
}
%    \end{macrocode}
%  \end{macro}



%  \begin{macro}{\@kernel@after@shipout@lastpage,\@kernel@before@shipout@foreground}
%    
%    \begin{macrocode}
\let\@kernel@after@shipout@lastpage\@empty
\let\@kernel@before@shipout@foreground\@empty
%    \end{macrocode}
%  \end{macro}


%  \begin{macro}{\@@_execute_firstpage_hook: }
%    This command adds any specials into a box and adds that to the
%    very beginning of the first box shipped out. After that we
%    redefine it to do nothing on later pages.
%    \begin{macrocode}
\cs_new:Npn \@@_execute_firstpage_hook: {
%    \end{macrocode}
%    Adding something to the beginning means adding it to the
%    background as that  layer is done first in the output. Of course
%    that is only needed if the hook actually contains anything.
%    \begin{macrocode}
  \hook_if_empty:nF {shipout/firstpage}
       { \@@_add_background_box:n { \UseHook{shipout/firstpage} } }
%    \end{macrocode}
%    Once we are here we change the definition to do nothing next time
%    and we also change the command used to implement \cs{AtBeginDvi}
%    to become a warning and not  add further material to a hook that
%    is never used again.
%    \begin{macrocode}
  \cs_gset_eq:NN \@@_execute_firstpage_hook: \prg_do_nothing:
  \cs_gset:Npn \@@_add_firstpage_material:Nn ##1 ##2 {
    \PackageWarning{ltshipout}{
        First~ page~ is~ already~ shipped~ out,~ ignoring\MessageBreak
        \string##1 }
  }
}
%    \end{macrocode}
%  \end{macro}


%  \begin{macro}{\g_@@_lastpage_handled_bool}
%    A boolean to signal if we have already handled the
%    \hook{shipout/lastpage} hook.
%    \begin{macrocode}
\bool_new:N \g_@@_lastpage_handled_bool
%    \end{macrocode}
%  \end{macro}



%  \begin{macro}{\@@_add_firstpage_material:Nn}
%    This command adds material to the
%    \hook{shipout/firstpage} hook. It is used in
%    \cs{AtBeginDvi}, etc. The first argument is the
%    command through which is it called. Initially this is ignored but
%    once we are passed the first page it can be used to generate a
%    warning message mentioning the right user command.
%    \begin{macrocode}
\cs_new:Npn \@@_add_firstpage_material:Nn #1#2 {
   \AddToHook{shipout/firstpage}{#2}
}
%    \end{macrocode}
%  \end{macro}







%  \begin{macro}{\@@_get_box_size:N}
%    Store the box dimensions in dimen registers.
%    \fmi{This could/should perhaps be generalized to set height depth and
%    width given an arbitrary box.}
%    \begin{macrocode}
\cs_new:Npn \@@_get_box_size:N #1 {
  \dim_set:Nn \l_shipout_box_ht_dim { \box_ht:N #1 }
  \dim_set:Nn \l_shipout_box_dp_dim { \box_dp:N #1 }
  \dim_set:Nn \l_shipout_box_wd_dim { \box_wd:N #1 }
  \dim_set:Nn \l_shipout_box_ht_plus_dp_dim { \l_shipout_box_ht_dim +
                                         \l_shipout_box_dp_dim }
}
%    \end{macrocode}
%  \end{macro}

%  \begin{macro}{\l_shipout_box_ht_dim,
%                   \l_shipout_box_dp_dim,\l_shipout_box_wd_dim,
%                   \l_shipout_box_ht_plus_dp_dim}
%    And here are the variables set by \cs{@@_get_box_size:N}.
%    \begin{macrocode}
\dim_new:N \l_shipout_box_ht_dim
\dim_new:N \l_shipout_box_dp_dim
\dim_new:N \l_shipout_box_wd_dim
\dim_new:N \l_shipout_box_ht_plus_dp_dim
%    \end{macrocode}
%  \end{macro}





%  \begin{macro}{\g_@@_discard_bool}
%    Indicate whether or not the current page box should be discarded
%    \begin{macrocode}
\bool_new:N \g_@@_discard_bool
%    \end{macrocode}
%  \end{macro}



%  \begin{macro}{\l_@@_tmp_box,\l_@@_saved_badness_tl}
%    We need a box for the background and foreground material and a
%    token register to remember badness settings as we disable  them
%    during te buildup below.
%    \begin{macrocode}
\box_new:N \l_@@_tmp_box
\tl_new:N  \l_@@_saved_badness_tl
%    \end{macrocode}
%  \end{macro}


%  \begin{macro}{\@@_add_background_box:n}
%    In standard \LaTeX{} the shipout box is always a \cs{vbox} but
%    here we are allow for other usage as well, in case some package
%    has its own output routine.
%    \begin{macrocode}
\cs_new:Npn \@@_add_background_box:n #1
{ \@@_get_box_size:N \l_shipout_box
%    \end{macrocode}
%    But we start testing for a vertical box as that should be the
%    normal case.
%    \begin{macrocode}
  \box_if_vertical:NTF \l_shipout_box
      {
%    \end{macrocode}
%    Save current values of \cs{vfuzz} and \cs{vbadness} then change
%    them to allow box manipulations without warnings.
%    \begin{macrocode}
        \tl_set:Nx \l_@@_saved_badness_tl
           { \vfuzz=\the\vfuzz\relax
             \vbadness=\the\vbadness\relax }
        \vfuzz=\c_max_dim
        \vbadness=\c_max_int
%    \end{macrocode}
%    Then we reconstruct \cs{l_shipout_box} \ldots
%    \begin{macrocode}
        \vbox_set_to_ht:Nnn \l_shipout_box \l_shipout_box_ht_plus_dp_dim 
             {
%    \end{macrocode}
%    \ldots{} the material in \verb=#1= is placed into a horizontal
%    box with zero dimensions.
%    \begin{macrocode}
               \hbox_set:Nn \l_@@_tmp_box
                    { \l_@@_saved_badness_tl #1 }
               \box_set_wd:Nn \l_@@_tmp_box \c_zero_dim
               \box_set_ht:Nn \l_@@_tmp_box \c_zero_dim
               \box_set_dp:Nn \l_@@_tmp_box \c_zero_dim
%    \end{macrocode}
%    The we typeset that box followed by whatever was in
%    \cs{l_shipout_box} before (unpacked).
%    \begin{macrocode}
               \skip_zero:N \baselineskip
               \skip_zero:N \lineskip
               \skip_zero:N \lineskiplimit
               \box_use:N \l_@@_tmp_box
               \vbox_unpack:N \l_shipout_box
%    \end{macrocode}
%    The \cs{kern} ensures that the box has no depth which is
%    afterwards explicitly corrected.
%    \begin{macrocode}
               \kern \c_zero_dim
             }
        \box_set_ht:Nn \l_shipout_box \l_shipout_box_ht_dim
        \box_set_dp:Nn \l_shipout_box \l_shipout_box_dp_dim 
%    \end{macrocode}
%    \fmi{The whole boxing maneuver looks a bit like overkill to me, but for
%    the moment I leave.}               
%    \begin{macrocode}
        \l_@@_saved_badness_tl
      }
      {
%    \end{macrocode}
%    A horizontal box is handled in a similar way. The last case would
%    be a void box in which case we do nothing hence the missing
%    \texttt{F} branch. 
%    \begin{macrocode}
        \box_if_horizontal:NT \l_shipout_box
            {
              \tl_set:Nx \l_@@_saved_badness_tl
                 { \hfuzz=\the\hfuzz\relax
                   \hbadness=\the\hbadness\relax }
              \hfuzz=\c_max_dim
              \hbadness=\c_max_int
              \hbox_set_to_wd:Nnn \l_shipout_box \l_shipout_box_wd_dim
                   {
                     \hbox_set:Nn \l_@@_tmp_box
                          { \l_@@_saved_badness_tl #1 }
                     \box_set_wd:Nn \l_@@_tmp_box \c_zero_dim
                     \box_set_ht:Nn \l_@@_tmp_box \c_zero_dim
                     \box_set_dp:Nn \l_@@_tmp_box \c_zero_dim
                     \box_move_up:nn
                         \l_shipout_box_ht_dim 
                         { \box_use:N \l_@@_tmp_box }
                     \hbox_unpack:N \l_shipout_box
                   }
              \l_@@_saved_badness_tl
            }
      }
}
%    \end{macrocode}
%  \end{macro}




%  \begin{macro}{\@@_add_foreground_box:n}
%    Foreground boxes are done in the same way, only the order and
%    placement of boxes has to be done differently.
%    \begin{macrocode}
\cs_new:Npn \@@_add_foreground_box:n #1
{
  \box_if_vertical:NTF \l_shipout_box
    {
      \tl_set:Nx \l_@@_saved_badness_tl
         { \vfuzz=\the\vfuzz\relax
           \vbadness=\the\vbadness\relax }
      \vfuzz=\c_max_dim
      \vbadness=\c_max_int
      \vbox_set_to_ht:Nnn \l_shipout_box \l_shipout_box_ht_plus_dp_dim
           {
             \hbox_set:Nn \l_@@_tmp_box
                  { \l_@@_saved_badness_tl #1 }
             \box_set_wd:Nn \l_@@_tmp_box \c_zero_dim
             \box_set_ht:Nn \l_@@_tmp_box \c_zero_dim
             \box_set_dp:Nn \l_@@_tmp_box \c_zero_dim
             \skip_zero:N \baselineskip
             \skip_zero:N \lineskip
             \skip_zero:N \lineskiplimit
             \vbox_unpack:N \l_shipout_box
             \kern -\l_shipout_box_ht_plus_dp_dim
             \box_use:N \l_@@_tmp_box
             \kern  \l_shipout_box_ht_plus_dp_dim
           }
      \l_@@_saved_badness_tl
      \box_set_ht:Nn \l_shipout_box \l_shipout_box_ht_dim
      \box_set_dp:Nn \l_shipout_box \l_shipout_box_dp_dim 
    }
    {
      \box_if_horizontal:NT \l_shipout_box
        {
          \tl_set:Nx \l_@@_saved_badness_tl
            { \hfuzz=\the\hfuzz\relax
              \hbadness=\the\hbadness\relax }
          \hfuzz=\c_max_dim
          \hbadness=\c_max_int
          \hbox_set_to_wd:Nnn \l_shipout_box \l_shipout_box_wd_dim
               {
                 \hbox_unpack:N \l_shipout_box
                 \kern -\box_wd:N \l_shipout_box
                 \hbox_set:Nn \l_@@_tmp_box
                     { \l_@@_saved_badness_tl #1 }
                 \box_set_wd:Nn \l_@@_tmp_box \c_zero_dim
                 \box_set_ht:Nn \l_@@_tmp_box \c_zero_dim
                 \box_set_dp:Nn \l_@@_tmp_box \c_zero_dim
                 \box_move_up:nn { \box_ht:N \l_shipout_box }
                               { \box_use:N \l_@@_tmp_box }
                 \kern \box_wd:N \l_shipout_box
               }%
               \l_@@_saved_badness_tl
        }
    }
}
%    \end{macrocode}
%  \end{macro}




%  \begin{macro}{\c_@@_horigin_tl,\c_@@_vorigin_tl}
%    Two constants holding the offset of the top-left with respect to
%    the media box.
%
%    Setting the constants this way is courtesy of Bruno.
%    \begin{macrocode}
\tl_gput_right:Nn \@kernel@after@begindocument {
  \tl_const:Nx \c_@@_horigin_tl
     {
       \cs_if_exist_use:NTF \pdfvariable { horigin }
          { \cs_if_exist_use:NF \pdfhorigin { 1in } }
     }
  \tl_const:Nx \c_@@_vorigin_tl
     {
       \cs_if_exist_use:NTF \pdfvariable { vorigin }
          { \cs_if_exist_use:NF \pdfvorigin { 1in } }
     }
}     
%    \end{macrocode}
%  \end{macro}


%  \begin{macro}{\@@_picture_overlay:n}
%    Put the argument into a \texttt{picture} environment that doesn't take up
%    any size and uses \texttt{1pt} for \cs{unitlength}.
%    \fmi{Could perhaps be generalized as it might be useful elsewhere. For
%    not it is not.}
%    \begin{macrocode}
\cs_new:Npn \@@_picture_overlay:n #1 {
    \kern -\c_@@_horigin_tl \scan_stop:
    \vbox_to_zero:n {
      \kern -\c_@@_vorigin_tl \scan_stop:
      \unitlength 1pt \scan_stop:
      \hbox_set_to_wd:Nnn \l_@@_tmp_box \c_zero_dim { \ignorespaces #1 }
%      \box_set_wd:Nn \l_@@_tmp_box \c_zero_dim
      \box_set_ht:Nn \l_@@_tmp_box \c_zero_dim
      \box_set_dp:Nn \l_@@_tmp_box \c_zero_dim
      \box_use:N \l_@@_tmp_box
      \tex_vss:D
    }
}
%    \end{macrocode}
%  \end{macro}


%  \begin{macro}{\@@_add_background_picture:n}
%    Put a \texttt{picture} env in  the background of the shipout box
%    with its reference point in the top-left corner.
%    \begin{macrocode}
\cs_new:Npn \@@_add_background_picture:n #1 {
   \@@_add_background_box:n { \@@_picture_overlay:n {#1} }
}
%    \end{macrocode}
%  \end{macro}



%  \begin{macro}{\@@_add_foreground_picture:n}
%    
%    Put a \texttt{picture} env in  the foreground of the shipout box
%    with its reference point in the top-left corner.
%    \begin{macrocode}
\cs_new:Npn \@@_add_foreground_picture:n #1 {
   \@@_add_foreground_box:n { \@@_picture_overlay:n {#1} }
}
%    \end{macrocode}
%  \end{macro}


%  \begin{macro}{\shipout_discard:}
%    Request that the next shipout box should be discarded. At the
%    moment this is just setting a boolean, but we may want to augment
%    this behavior that the position of the call is taken into account
%    (in case \LaTeX{} looks ahead and is not using the position for
%    on the next page).
%    \begin{macrocode}
\cs_new:Npn \shipout_discard: {
  \bool_gset_true:N \g_@@_discard_bool
}
%    \end{macrocode}
%  \end{macro}




% \subsection{Handling the end of job hook}

%    At the moment this is partly solved by using the existing hooks.
%    But rather than putting the code into these hooks it should be
%    moved to the right place directly as we shouldn't prefill hooks
%    with material unless it needs to interact with other code. 
%

%    
%  \begin{macro}{\g_shipout_readonly_int,\ReadonlyShipoutCounter}
%    We count every shipout activity that makes a page (but not those
%    that are discarded) in order to know how many pages got produced.
%    \begin{macrocode}
\int_new:N \g_shipout_readonly_int
%    \end{macrocode}
%    For \LaTeXe{} it is available as a command (i.e., a \TeX{}
%    counter only.
%    \begin{macrocode}
\cs_new_eq:NN \ReadonlyShipoutCounter  \g_shipout_readonly_int
%    \end{macrocode}
%  \end{macro}

%  \begin{macro}{\g_shipout_totalpages_int,\c@totalpages}
%    We count every shipout attempt (even those that are discarded) in
%    tis counter. It is not used in the code but may get used in user
%    code.
%    \begin{macrocode}
\int_new:N \g_shipout_totalpages_int
%    \end{macrocode}
%    For \LaTeXe{} this is offered as a \LaTeX{} counter so can be
%    easily typeset inside the output routine to display things like
%    \enquote{\cs{thepage}\texttt{/}\cs{thetotalpages}}, etc.
%    \begin{macrocode}
\cs_new_eq:NN \c@totalpages \g_shipout_totalpages_int
\cs_new:Npn \thetotalpages { \arabic{totalpages} }
%    \end{macrocode}
%  \end{macro}
%



%  \begin{macro}{\@abspage@last}
%    In \cs{@abspage@last} record the number of pages from the last
%    run. This is written to the \texttt{.aux} and this way made
%    available to the next run. In case there is no \texttt{.aux} file
%    or the statement is missing from it we initialize it with the
%    largest possible number in \TeX{}. We use this a s the default
%    because then we are inserting the \hook{shipout/lastpage} on
%    the last page (or after the last page but not on page 1 for a
%    multipage document.
%    \begin{macrocode}
\xdef\@abspage@last{\number\maxdimen}
%    \end{macrocode}
%  \end{macro}


% \begin{macro}{\enddocument}
%
%    Instead of using the hooks \hook{enddocument} and
%    \hook{enddocument/afterlastpage} we add this code to private
%    kernel hooks to be 100\% when it is executed and to avoid
%    cluttering the hooks with data that is always there.
%
%    Inside \cs{enddocument} there is a \cs{clearpage}. Just before
%    that we execute this code here. There is a good change that we
%    are on the last page. Therefore, if we don't know the value from
%    the last run, we assume that the current page is the right
%    one. So we set \cs{@abspage@last} and as a result the next
%    shipout will run the \hook{shipout/lastpage} code. Of course,
%    if there are floats that still need a placement this guess will
%    be wrong but then rerunning the document will give us the correct
%    value next time around.
%
% \begin{macro}{\@kernel@after@enddocument}
%    \begin{macrocode}
\g@addto@macro \@kernel@after@enddocument {
  \int_compare:nNnT \@abspage@last = \maxdimen
    {
%    \end{macrocode}
%    We use \LaTeXe{} coding as \cs{@abspage@last} is not an L3 name.
%    \begin{macrocode}
      \xdef\@abspage@last{ \int_eval:n {\g_shipout_readonly_int + 1} }
    }
}
%    \end{macrocode}
% \end{macro}
%
% \begin{macro}{\@kernel@after@enddocument@afterlastpage}
%    Once the \cs{clearpage} has done its work inside \cs{enddocument}
%    we know for sure how many pages this document has, so we record
%    that in the \texttt{.aux} file for the next run.
%
%    \begin{macrocode}
\g@addto@macro \@kernel@after@enddocument@afterlastpage {
%    \end{macrocode}
%    There is one special case: If no output is produced then there is
%    no point in a) recording the number as 0 will never match the
%    page number of a real page and b) adding an extra page to ran the
%    \hook{shipout/lastpage} is pointless as well (as it would
%    remain forever). So we test for this and run the code only if
%    there have been pages.
%    \begin{macrocode}
  \int_compare:nNnF \g_shipout_readonly_int = 0
    {
%    \end{macrocode}
%     This ends up in the \texttt{.aux} so we use \LaTeXe{} names here.
%     \fmi{this needs an interface for \cs{nofiles} in expl3, doesn't at the moment!}
%    \begin{macrocode}
     \if@filesw
        \iow_now:Nx \@auxout {
          \gdef\string\@abspage@last {\int_use:N \g_shipout_readonly_int}}
     \fi
%    \end{macrocode}
%    But we may have guessed wrongly earlier and we still have to run the
%    \hook{shipout/lastpage} even though there is no page to place
%    it into. If that is the case we make a trivial extra page and put
%    it there. This temporary page will then vanish again on the next
%    run but helps to keep pdf viewers happy.
%    \begin{macrocode}
      \bool_if:NF \g_@@_lastpage_handled_bool
        { \tex_shipout:D\vbox to\textheight{
            \hbox{\UseHook{shipout/lastpage} \@kernel@after@shipout@lastpage }
%    \end{macrocode}
%    This extra page could be totally empty except for the hook
%    content, but to help the user understanding why it is there we
%    put some text into it.
%    \begin{macrocode}
            \@@_excuse_extra_page:
            \null
          }
%    \end{macrocode}
%    At this point we should also signal to \LaTeX{} that a rerun is
%    necessary so that an appropriate message can be shown on the
%    terminal.
%
%    \fmi{integrate warning ``Rerun \ldots'' }
%    \begin{macrocode}
      }
    }
}
%    \end{macrocode}
% \end{macro}
% \end{macro}



%  \begin{macro}{\@@_excuse_extra_page:}
%    Say mea culpa \ldots
%    \begin{macrocode}
\cs_new:Npn \@@_excuse_extra_page: {
  \vfil
  \begin{center}
    \bfseries Temporary~ page! 
  \end{center}
    \LaTeX{}~ was~ unable~ to~ guess~ the~ total~ number~ of~ pages~
    correctly.~ ~ As~ there~ was~ some~ unprocessed~ data~ that~
    should~ have~ been~ added~ to~ the~ final~ page~ this~ extra~
    page~ has~ been~ added~ to~ receive~ it.
    \par
    If~ you~ rerun~ the~ document~ (without~ altering~ it)~ this~
    surplus~ page~ will~ go~ away,~ because~ \LaTeX{}~ now~ knows~
    how~ many~ pages~ to~ expect~ for~ this~ document.
  \vfil
}
%    \end{macrocode}
%  \end{macro}




% \section{Legacy \LaTeXe{} interfaces}



%  \begin{macro}{\DiscardShipoutBox}
%    Request that the next shipout box is to be discarded.
%    \begin{macrocode}
\cs_new_eq:NN \DiscardShipoutBox \shipout_discard:
%    \end{macrocode}
%  \end{macro}


%  \begin{macro}{\AtBeginDvi}
%    
%    \begin{macrocode}
\renewcommand \AtBeginDvi {\@@_add_firstpage_material:Nn \AtBeginDvi}
%    \end{macrocode}
%  \end{macro}


%  \begin{macro}{\DebugShipoutOn,\DebugShipoutOff}
%    
%    \begin{macrocode}
\cs_new_eq:NN \DebugShipoutOn  \shipout_debug_on:
\cs_new_eq:NN \DebugShipoutOff \shipout_debug_off:
%    \end{macrocode}
%  \end{macro}





% \section{Package emulation for compatibility}


% \subsection{Package \pkg{atbegshi} emulation}

%  \begin{macro}{\AtBeginShipoutBox}
%    \begin{macrocode}
\cs_new_eq:NN \AtBeginShipoutBox \ShipoutBox
%    \end{macrocode}
%  \end{macro}



%  \begin{macro}{\AtBeginShipoutInit}
%    Compatibility only, we aren't delaying \ldots
%    \begin{macrocode}
\cs_set_eq:NN\AtBeginShipoutInit\@empty 
%    \end{macrocode}
%  \end{macro}



%  \begin{macro}{\AtBeginShipout,\AtBeginShipoutNext}
%    Filling hooks
%    \begin{macrocode}
\newcommand\AtBeginShipout     {\AddToHook{shipout/before}}
\newcommand\AtBeginShipoutNext {\AddToHookNext{shipout/before}}
%    \end{macrocode}
%  \end{macro}



%  \begin{macro}{\AtBeginShipoutFirst}
%    Slightly more complex as we need to know the name of the command under which the
%    \hook{shipout/firstpage} hook is filled.
%    \begin{macrocode}
\newcommand\AtBeginShipoutFirst{\@@_add_firstpage_material:Nn \AtBeginShipoutFirst}
%    \end{macrocode}
%  \end{macro}



% This is somewhat different from the original where
% \cs{ShipoutBoxHeight} etc.\ only holds the \verb=\the\ht<box>= value. This
% may has some implications in some use cases and if that is a problem
%    then it might need changing.
%    
%    \begin{macrocode}
\cs_new:Npn \ShipoutBoxHeight { \dim_use:N \l_shipout_box_ht_dim }
\cs_new:Npn \ShipoutBoxDepth  { \dim_use:N \l_shipout_box_dp_dim }
\cs_new:Npn \ShipoutBoxWidth  { \dim_use:N \l_shipout_box_wd_dim }
%    \end{macrocode}


%  \begin{macro}{\AtBeginShipoutDiscard}
%    Just a different name.
%    \begin{macrocode}
\cs_new_eq:NN \AtBeginShipoutDiscard \DiscardShipoutBox
%    \end{macrocode}
%  \end{macro}


%  \begin{macro}{\AtBeginShipoutAddToBox,\AtBeginShipoutAddToBoxForeground,
%                \AtBeginShipoutUpperLeft,\AtBeginShipoutUpperLeftForeground}
%    We don't expose them.
%    \begin{macrocode}
\cs_new_eq:NN \AtBeginShipoutAddToBox           \@@_add_background_box:n
\cs_new_eq:NN \AtBeginShipoutAddToBoxForeground \@@_add_foreground_box:n 
%    \end{macrocode}
%    
%    \begin{macrocode}
\cs_new_eq:NN\AtBeginShipoutUpperLeft           \@@_add_background_picture:n
\cs_new_eq:NN\AtBeginShipoutUpperLeftForeground \@@_add_foreground_picture:n
%    \end{macrocode}
%  \end{macro}

%
%
%    We prevent the package \pkg{atbegshi} from loading:
%    \begin{macrocode}
\expandafter\cs_set_eq:NN\csname ver@atbegshi.sty\endcsname\fmtversion
%    \end{macrocode}
%    \pkg{hyperref} code (and \pkg{ltxcmds}) doesn't understand 2020-10-01
%       and thinks it is before 1994, so for now \ldots
%    \begin{macrocode}
\@namedef {ver@atbegshi.sty}{2020/10/01}
%    \end{macrocode}



% \subsection{Package \pkg{everyshi} emulation}
%
%  \begin{macro}{\EveryShipout,\AtNextShipout}
%    This package has only two public commands to simulating it is easy:
%    \begin{macrocode}
\cs_new_eq:NN\EveryShipout\AtBeginShipout
\cs_new_eq:NN\AtNextShipout\AtBeginShipoutNext
%    \end{macrocode}
%  \end{macro}
%
%    \begin{macrocode}
\expandafter\cs_set_eq:NN\csname ver@everyshi.sty\endcsname\fmtversion
\@namedef {ver@everyshi.sty}{2020/10/01}
%    \end{macrocode}



% \subsection{Package \pkg{atenddvi} emulation}
%
%  \begin{macro}{\AtEndDvi}
%    This package has only one public command to simulating it is easy:
%    \begin{macrocode}
\cs_new:Npn \AtEndDvi {\AddToHook{shipout/lastpage}}
%    \end{macrocode}
%  \end{macro}
%
%
%    \begin{macrocode}
\expandafter\cs_set_eq:NN\csname ver@atenddvi.sty\endcsname\fmtversion
\@namedef {ver@atenddvi.sty}{2020/10/01}
%    \end{macrocode}




%    \begin{macrocode}
\ExplSyntaxOff
%</2ekernel>
%    \end{macrocode}

%    \Finale
%


%%%%%%%%%%%%%%%%%%%%%%%%%%%%%%%%%%%%%%%%%%%%%  
\endinput
%%%%%%%%%%%%%%%%%%%%%%%%%%%%%%%%%%%%%%%%%%%%%  



                     % \iffalse meta-comment
%
%% File: ltfilehook.dtx (C) Copyright 2020 Frank Mittelbach, LaTeX Team
%
% It may be distributed and/or modified under the conditions of the
% LaTeX Project Public License (LPPL), either version 1.3c of this
% license or (at your option) any later version.  The latest version
% of this license is in the file
%
%    https://www.latex-project.org/lppl.txt
%
%
% The development version of the bundle can be found below
%
%    https://github.com/FrankMittelbach/...
%
% for those people who are interested or want to report an issue.
%
%    \begin{macrocode}
\providecommand\ltfilehookversion{v0.6g}
\providecommand\ltfilehookdate{2020/07/16}
%    \end{macrocode}
%
%<*driver>

\RequirePackage{lthooks}        % needed for structedlog if running old format

\documentclass{l3doc}

% bug fix fo l3doc.cls
\ExplSyntaxOn
\cs_set_protected:Npn \__codedoc_macro_typeset_one:nN #1#2
  {
    \vbox_set:Nn \l__codedoc_macro_box
      {
        \vbox_unpack_drop:N \l__codedoc_macro_box
        \hbox { \llap { \__codedoc_print_macroname:nN {#1} #2
            \MacroFont       % <----- without it the \ is in lmr10 if a link is made
            \      
        } }
      }
    \int_incr:N \l__codedoc_macro_int
  }
\ExplSyntaxOff

\usepackage{structuredlog}  % for demonstration

\EnableCrossrefs
\CodelineIndex
\begin{document}
  \DocInput{ltfilehook.dtx}
\end{document}
%</driver>
%
% \fi
%
%
% \long\def\fmi#1{\begin{quote}\itshape Todo: #1\end{quote}}
%
% \let\hook\texttt
%
% \title{The \texttt{ltfilehook} package\thanks{This package has version
%    \ltfilehookversion\ dated \ltfilehookdate, \copyright\ \LaTeX\
%    Project.}}
%
% \author{Frank Mittelbach}
%
% \maketitle
%
%
%
% \tableofcontents
%
% \section{Introduction}
%
%
%
% \subsection{Provided hooks}
%
%    The code offers a number of hooks into which packages (or the
%    user) can add code to support different use cases.
%    Many hooks are offered as pairs (i.e., the second hook is
%    reversed. Also important to know is that these pairs are
%    properly nested with respect to other pairs of hooks.
%
%    There are hooks that are executed for all files of a certain type
%    (if they contain code), e.g., for all \enquote{include files} or
%    all \enquote{packages},
%    and there are also hooks that are specific to a single file,
%    e.g., do something after the package \texttt{foo.sty} has been
%    loaded.
%
%
% \subsection{General hooks for file reading}
% \label{sec:general-file-hooks}
%
%    There are four hooks that are called for each file that is read
%    using document-level commands such as \cs{input}, \cs{include},
%    \cs{usepackage}, etc.  They are not called for files read using
%    internal low-level methods, such as \cs{@input} or \cs{openin}.
%
%  \begin{variable}{file/before,file/before/...,
%                   file/after/...,file/after,
%                   }
%    These are:
%    \begin{description}
%    \item[\texttt{file/before}, \texttt{file/before/\meta{file-name}}]
%
%       These hooks are executed in that order just before the file is
%       loaded for reading. The code of the first hook is used is used
%       with every file, while the second is executed only for the
%       file with matching \meta{file-name} allowing you to specify
%       code that only applies to one file.
%
%    \item[\texttt{file/after/\meta{file-name}}, \texttt{file/after}]
%
%       These hooks are after the file with name \meta{file-name} has
%       been fully consumed. The order is swapped (the specific one
%       comes first) so that the \texttt{before} and \texttt{after}
%       hooks nest properly, which is important if any of them involve
%       grouping (e.g., contain environments, for example).
%       furthermore both hooks are reversed hooks to support correct
%       nesting of different packages add code to both
%       \texttt{/before} and \texttt{/after} hooks.
%
%    \end{description}
%  \end{variable}
%
%
%     So the overall sequence of hook processing for any file read
%     through the user interface commands of \LaTeX{} is:
%
%  \begin{tabbing}
%    mm\=mm\=mm\=mm\=\kill
%    \>\cs{UseHook}\marg{\hook{file/before}} \\
%    \>\cs{UseHook}\marg{\hook{file/before/\meta{file name}}} \\
%    \>\> \meta{file contents} \\
%    \>\cs{UseHook}\marg{\hook{file/after/\meta{file name}}} \\
%    \>\cs{UseHook}\marg{\hook{file/after}}
%  \end{tabbing}
%
%     \fmi{With a higher-level interface that doesn't matter, but it
%    is a bit weird, if you use \cs{AddToHook} or \cs{hook_gput:nnn}
%    directly, so I guess that has to be done differently!}
%
%    The file hooks only refer to the file by its name and extension,
%    so the \meta{file name} should be the file name as it is on the
%    filesystem with extension (if any) and without paths.  Different
%    from \cs{input} and similar commands, for hooks the \texttt{.tex}
%    extension is not assumed, so \texttt{.tex} files must also be given
%    with their extension.
%    Files within subfolders should also be addressed by their name and
%    extension only.
%
%    Extensionless files also work, and should then be given without
%    extension.  Note however that \TeX{} prioritizes \texttt{.tex}
%    files, so if two files \texttt{foo} and \texttt{foo.tex} exist in
%    the search path, only the latter will be seen.
%
%    When a file is input, the \meta{file name} is available in
%    \cs{CurrentFile}, which is then used when accessing the
%    \hook{file/before/\meta{file name}} and
%    \hook{file/after/\meta{file name}}.
%
%  \begin{variable}{\CurrentFile}
%    The name of the file about to be read (or just finished) is
%    available to the hooks through \cs{CurrentFile} (there is no
%    \texttt{expl3} name for it for now).  The file is always provided
%    with its extension, i.e., how it appears on your hard drive, but
%    without any specified path to it. For example,
%    \verb=\input{sample}= and \verb=\input{app/sample.tex}= would
%    both have \cs{CurrentFile} being \texttt{sample.tex}.
%  \end{variable}
%
%  \begin{variable}{\CurrentFilePath}
%    The path to the current file (complement to \cs{CurrentFile}) is
%    available in \cs{CurrentFilePath} if needed.
%    The paths returned in \cs{CurrentFilePath} are only user paths,
%    given through \cs{input@path} (or \pkg{expl3}'s equivalent
%    \cs{l_file_search_path_seq}) or by directly typing in the path
%    in the \cs{input} command or equivalent.  Files located by
%    \texttt{kpsewhich} get the path added internally by the \TeX{}
%    implementation, so at the macro level it looks as if the file were
%    in the current folder, so the path in \cs{CurrentFilePath} is empty
%    in these cases (package and class files, mostly).
%  \end{variable}
%
% \subsection{Hooks for package and class files}
%
%    Commands to load package and class files (e.g., \cs{usepackage},
%    \cs{RequirePackage}, \cs{LoadPackageWithOptions}, etc.) offer the
%    hooks from section~\ref{sec:general-file-hooks} when they are
%    used to load a  package or class file, e.g.,
%    \texttt{file/after/array.sty} would be called after the
%    \pkg{array} package got loaded. But as packages and classes form as special group
%    of files, there are some additional hooks available that only
%    apply when a package or class is loaded.
%
%
%  \begin{variable}{
%                   package/before,package/after,
%                   package/before/...,package/after/...,
%                   class/before,class/after,
%                   class/before/...,class/after/...,
%                   }
%    These are:
%    \begin{description}
%    \item[\texttt{package/before}, \texttt{package/after}]
%
%      These hooks are called for each package being loaded.
%
%    \item[\texttt{package/before/\meta{name}},
%    \texttt{package/after/\meta{name}}]
%
%      These hooks are additionally called if the package name is
%      \meta{name} (without extension).
%
%    \item[\texttt{class/before}, \texttt{class/after}]
%
%      These hooks are called for each class being loaded.
%
%    \item[\texttt{class/before/\meta{name}}, \texttt{class/after/\meta{name}}]
%
%      These hooks are additionally called if the class name is
%      \meta{name} (without extension).
%
%    \end{description}
%  \end{variable}
%     All \hook{/after} hooks are implemented as reversed hooks.
%
%     \noindent The overall sequence of execution for \cs{usepackage}
%     and friends is therefore:
%  \begin{tabbing}
%    mm\=mm\=mm\=mm\=\kill
%    \>\cs{UseHook}\marg{\hook{package/before}} \\
%    \>\cs{UseHook}\marg{\hook{package/before/\meta{package name}}} \\[5pt]
%    \>\>\cs{UseHook}\marg{\hook{file/before}} \\
%    \>\>\cs{UseHook}\marg{\hook{file/before/\meta{package name}.sty}} \\
%    \>\>\> \meta{package contents} \\
%    \>\>\cs{UseHook}\marg{\hook{file/after/\meta{package name}.sty}} \\
%    \>\>\cs{UseHook}\marg{\hook{file/after}} \\[5pt]
%    \>\>\emph{code from \cs{AtEndOfPackage} if
%                used inside the package} \\[5pt]
%    \>\cs{UseHook}\marg{\hook{package/after/\meta{package name}}} \\
%    \>\cs{UseHook}\marg{\hook{package/after}} 
%  \end{tabbing}
%    and similar for class file loading, except that \texttt{package/}
%    is replaced by \texttt{class/} and \cs{AtEndOfPackage} by
%    \cs{AtEndOfClass}.
%
%    If a package or class is not loaded (or it was loaded before the
%    hooks were set) none of the hooks are executed!
%
% \subsection{Hooks for \cs{include} files}
%
%    To manage \cs{include} files, \LaTeX{} issues a \cs{clearpage}
%    before and after loading such a file. Depending on the use case
%    one may want to execute code before or after these
%    \cs{clearpage}s especially for the one that is issued at the end.
%
%    Executing code before the final \cs{clearpage}, means that the
%    code is processed while the last page of the included material is
%    still under construction.  Executing code after it means that all
%    floats from inside the include file are placed (which
%    might have added further pages) and the final page has finished.
%
%    Because of these different scenarios we offer hooks in three
%    places.\footnote{If you want to execute code before the first
%     \cs{clearpage} there is no need to use a hook---you can write it
%     directly in front of the \cs{include}.}
%    None of the hooks are executed when an \cs{include} file is
%    bypassed because of an \cs{includeonly} declaration. They are,
%    however, all executed if \LaTeX{} makes an attempt to load the
%    \cs{include} file (even if it doesn't exist and all that happens
%    is \enquote{\texttt{No file \meta{filename}.tex}}).
%
%
%  \begin{variable}{include/before,include/before/...,
%                   include/end,include/end/...,
%                   include/after,include/after/...,
%                  }
%    These are:
%    \begin{description}
%
%    \item[\texttt{include/before}, \texttt{include/before/\meta{name}}]
%
%      These hooks are executed one after another after the initial
%      \cs{clearpage} and after \texttt{.aux} file is changed to use
%      \texttt{\meta{name}.aux}, but before the
%      \texttt{\meta{name}.tex} file is loaded. In other words they are executed
%      at the very beginning of the first page of the \cs{include}
%      file.
%
%
%    \item[\texttt{include/end/\meta{name}}, \texttt{include/end}]
%
%      These hooks are executed (in that order) after \LaTeX{} has
%      stopped reading from the \cs{include} file, but before it has
%      issued a \cs{clearpage} to output any deferred floats.
%
%
%    \item[\texttt{include/after/\meta{name}}, \texttt{include/after}]
%
%      These hooks are executed (in that order) after \LaTeX{} has
%      issued the \cs{clearpage} but before is has switched back
%      writing to the main \texttt{.aux} file. Thus technically we are
%      still inside the \cs{include} and if the hooks generate any
%      further typeset material including anything that writes to the
%      \texttt{.aux} file, then it would be considered part of the
%      included material and bypassed if it is not loaded because of
%      some \cs{includeonly} statement.\footnotemark
%
%    \end{description}
%  \end{variable}\footnotetext{For that reason
%      another \cs{clearpage} is executed after these hooks which
%      normally does nothing, but starts a new page if further material
%      got added this way.}
%
% \subsection{High-level interfaces for \LaTeX{}}
%
%    We do not provide any high-level \LaTeX{} commands (like
%    \pkg{filehook} or \pkg{scrlfile} do) but think that for package
%    writers the commands from for hook management are sufficient.
%
% \subsection{A sample package for structuring the log output}
%
%    As an application we provide the package \pkg{structuredlog} that
%    adds lines to the \texttt{.log} when a file is opened and closed
%    for reading keeping track of nesting level es well.
%    For example, for the current document it adds the lines
%\begin{verbatim}
%    = (LEVEL 1 START) t1lmr.fd
%    = (LEVEL 1 STOP) t1lmr.fd
%    = (LEVEL 1 START) supp-pdf.mkii
%    = (LEVEL 1 STOP) supp-pdf.mkii
%    = (LEVEL 1 START) nameref.sty
%    == (LEVEL 2 START) refcount.sty
%    == (LEVEL 2 STOP) refcount.sty
%    == (LEVEL 2 START) gettitlestring.sty
%    == (LEVEL 2 STOP) gettitlestring.sty
%    = (LEVEL 1 STOP) nameref.sty
%    = (LEVEL 1 START) ltfilehook-doc.out
%    = (LEVEL 1 STOP) ltfilehook-doc.out
%    = (LEVEL 1 START) ltfilehook-doc.out
%    = (LEVEL 1 STOP) ltfilehook-doc.out
%    = (LEVEL 1 START) ltfilehook-doc.hd
%    = (LEVEL 1 STOP) ltfilehook-doc.hd
%    = (LEVEL 1 START) ltfilehook.dtx
%    == (LEVEL 2 START) ot1lmr.fd
%    == (LEVEL 2 STOP) ot1lmr.fd
%    == (LEVEL 2 START) omllmm.fd
%    == (LEVEL 2 STOP) omllmm.fd
%    == (LEVEL 2 START) omslmsy.fd
%    == (LEVEL 2 STOP) omslmsy.fd
%    == (LEVEL 2 START) omxlmex.fd
%    == (LEVEL 2 STOP) omxlmex.fd
%    == (LEVEL 2 START) umsa.fd
%    == (LEVEL 2 STOP) umsa.fd
%    == (LEVEL 2 START) umsb.fd
%    == (LEVEL 2 STOP) umsb.fd
%    == (LEVEL 2 START) ts1lmr.fd
%    == (LEVEL 2 STOP) ts1lmr.fd
%    == (LEVEL 2 START) t1lmss.fd
%    == (LEVEL 2 STOP) t1lmss.fd
%    = (LEVEL 1 STOP) ltfilehook.dtx
%\end{verbatim}
%    Thus if you inspect an issue in the \texttt{.log} it is easy to
%    figure out in which file it occurred, simply by searching back for
%    \texttt{LEVEL} and if it is a \texttt{STOP} then remove 1 from
%    the level value and search further for \texttt{LEVEL} with that value
%    which should then be the \texttt{START}  level of the file you are in.
%
% \StopEventually{\setlength\IndexMin{200pt}  \PrintIndex  }
%
%
% \section{The Implementation}
%    \begin{macrocode}
%<*2ekernel>
%    \end{macrocode}
%
% \subsection{\pkg{expl3} helpers}
%
%    \begin{macrocode}
%<@@=filehook>
%    \end{macrocode}
%
% \begin{macro}{\CurrentFile,\CurrentFilePath}
%   User-level macros that hold the current file name and file path.
%   These are used internally as well because the code takes care to
%   protect against a possible redefinition of these macros in the loaded
%   file (it's necessary anyway to make hooks work with nested
%   \cs{input}).
%    \begin{macrocode}
\def\CurrentFile{}
\def\CurrentFilePath{}
%    \end{macrocode}
% \end{macro}
%
% \begin{macro}{\l_@@_internal_tl}
% \begin{macro}{\@filehook@set@curr@file,\@@_normalise_file_name:n}
%   When inputting a file, \cs{@filehook@set@curr@file} does a file
%   lookup (in \cs{input@path} and \cs{l_file_search_path_seq}) and
%   returns the actual file name (\meta{base} plus \meta{ext}) in
%   \cs{CurrentFile}.  Only the base and extension are returned,
%   regardless of the input (both \texttt{path/to/file.tex} and
%   \texttt{file.tex} end up as \texttt{file.tex} in \cs{CurrentFile}).
%   The path is returned in \cs{CurrentFilePath}, in case it's needed.
%   \cs{CurrentFile} is then used to run the file hooks with
%   \texttt{file/before/\cs{CurrentFile}} and
%   \texttt{file/after/\cs{CurrentFile}}.
%    \begin{macrocode}
\ExplSyntaxOn
\tl_new:N \l_@@_internal_tl
\cs_new_protected:Npn \@filehook@set@curr@file #1
  { \exp_args:NV \@@_normalise_file_name:n #1 }
\cs_new_protected:Npn \@@_normalise_file_name:n #1
  {
    \file_if_exist:nTF {#1}
      {
        \exp_args:Nx \file_parse_full_name:nNNN
          { \file_full_name:n {#1} }
      }
      { \file_parse_full_name:nNNN {#1} }
      \CurrentFilePath \CurrentFile \l_@@_internal_tl
    \tl_set:Nx \CurrentFile { \CurrentFile \l_@@_internal_tl }
  }
%    \end{macrocode}
% \end{macro}
% \end{macro}
%
% \begin{macro}{\g_@@_input_file_seq}
% \begin{macro}{\@filehook@file@push,\@filehook@file@pop}
% \begin{macro}{\@@_file_pop_assign:nn}
%   Yet another stack, to keep track of \cs{CurrentFile} and
%   \cs{CurrentFilePath} with nested \cs{input}s.  At the beginning of
%   \cs{InputIfFileExists}, the current value of \cs{CurrentFilePath}
%   and \cs{CurrentFile} is pushed to \cs{g_@@_input_file_seq}, and
%   at the end, it is popped and the value reassigned.
%   \cs{IfFileExists} does \cs{set@curr@file} internally, which changes
%   \cs{CurrentFile}, so \cs{@filehook@file@push} has to be executed
%   before \cs{IfFileExists}.
%    \begin{macrocode}
\seq_new:N \g_@@_input_file_seq
\cs_new_protected:Npn \@filehook@file@push
  {
    \seq_gpush:Nx \g_@@_input_file_seq
      { { \CurrentFilePath } { \CurrentFile } }
  }
\cs_new_protected:Npn \@filehook@file@pop
  {
    \seq_gpop:NNTF \g_@@_input_file_seq \l_@@_internal_tl
      { \exp_after:wN \@@_file_pop_assign:nn \l_@@_internal_tl }
      { \ERROR_should_not_happen }
  }
\cs_new_protected:Npn \@@_file_pop_assign:nn #1 #2
  {
    \tl_set:Nn \CurrentFilePath {#1}
    \tl_set:Nn \CurrentFile {#2}
  }
\ExplSyntaxOff
%    \end{macrocode}
% \end{macro}
% \end{macro}
% \end{macro}
%
%
% \subsection{Declaring the file-related hooks}
%
%  All hooks starting with \texttt{file/} \texttt{include/},
%  \texttt{class/} or \texttt{package/} are generic and will be
%  allocated if code is added to them. Thus there is no need to
%  explicitly declare any hook in the code below.
%
%  Furthermore, those named \texttt{.../after} or \texttt{.../end} are
%  automatically declared as reversed hooks if filled with code, so this
%  is also automatically taken care of.
%
% \subsection{Patching \LaTeX{} commands (need proper integration later)}
%
%    Most of what we have to do is adding \cs{UseHook} into several
%  \LaTeXe{} core commands, which is done for now by patching them.
%    \begin{macrocode}
%<@@=>
%    \end{macrocode}
%
% \begin{macro}{\InputIfFileExists}
%    \cs{InputIfFileExists} loads any file if it is available so we
%    have to add the hooks \texttt{file/before} and
%    \texttt{file/after} in the right places. If the file doesn't
%    exist no hooks should be executed.
%    \begin{macrocode}
\let\InputIfFileExists\@undefined
\DeclareRobustCommand \InputIfFileExists[3]{%
  \@filehook@file@push
  \IfFileExists{#1}%
    {%
%    \end{macrocode}
%    If the file exists then \cs{@curr@file} holds its name. But we
%    can't rely on that still being true after the file has been
%    processed. Thus for using the name in the file hooks we need to
%    preserve the name and then restored it for the
%    \texttt{file/after/...} hook.
%
%    The hook always refers to the \emph{actual} file that will be
%    operated on, regardless of how the user had it written in the
%    document.  \pkg{expl3}'s \cs{file_full_name:n} normalizes the file
%    name (to factor out differences in the |.tex| extension), and
%    then does a file lookup to take into account a possible path from
%    \cs{l_file_search_path_seq} and \cs{input@path}.  However only
%    the file name and extension are returned so that file hooks can
%    refer to the file by their name only.  The path to the file is
%    returned in \cs{CurrentFilePath}.
%    \begin{macrocode}
      \edef\reserved@a{\@filef@und
        \def\noexpand\CurrentFile{\CurrentFile}%
        \def\noexpand\CurrentFilePath{\CurrentFilePath}%
      }%
      \expandafter\@swaptwoargs\expandafter
        {\reserved@a}%
        {%
          #2%
          \@addtofilelist{#1}%
          \UseHook{file/before}%
%    \end{macrocode}
%    The current file name is available in \cs{@curr@file} so we use
%    that in the specific hook.
%    \begin{macrocode}
          \UseHook{file/before/\CurrentFile}%
          \@@input
        }%
%    \end{macrocode}
%    And it is restored here so we can use it once more.
%    \begin{macrocode}
      \UseHook{file/after/\CurrentFile}%
      \UseHook{file/after}%
      \@filehook@file@pop
    }%
    {\@filehook@file@pop #3}%
}
%    \end{macrocode}
% \end{macro}
%
% \begin{macro}{\set@curr@file}
%   Now we just hook into \cs{set@curr@file} to add
%   \cs{@filehook@set@curr@file} at the end, after \cs{@curr@file} is
%   set.
% \changes{0.69}{2020-70-16}}{When the argument is empty return an
%    empty macro instead of \texttt{csnameendcsname} (gh.121)}
%    \begin{macrocode}
\def\set@curr@file#1{%
  \begingroup
    \def\reserved@a{#1}%
    \ifx\reserved@a\@empty
      \xdef\@curr@file{}%
    \else
      \escapechar\m@ne
      \xdef\@curr@file{%
        \expandafter\expandafter\expandafter\unquote@name
        \expandafter\expandafter\expandafter{%
        \expandafter\string
        \csname\@firstofone#1\@empty\endcsname}}%
    \fi
  \endgroup
  \@filehook@set@curr@file{\@curr@file}%
}
%    \end{macrocode}
%  \end{macro}
%
%  \begin{macro}{\load@onefilewithoptions}
%    This macro is used when loading packages or classes.
%    \begin{macrocode}
\def\load@onefilewithoptions#1[#2][#3]#4{%
  \@pushfilename
  \xdef\@currname{#1}%
  \global\let\@currext#4%
  \let\CurrentOption\@empty
  \@reset@ptions
  \makeatletter
  \def\reserved@a{%
    \@ifl@aded\@currext{#1}%
      {\@if@ptions\@currext{#1}{#2}{}%
        {\@latex@error
            {Option clash for \@cls@pkg\space #1}%
            {The package #1 has already been loaded
             with options:\MessageBreak
             \space\space[\@ptionlist{#1.\@currext}]\MessageBreak
             There has now been an attempt to load it
              with options\MessageBreak
             \space\space[#2]\MessageBreak
             Adding the global options:\MessageBreak
             \space\space
                  \@ptionlist{#1.\@currext},#2\MessageBreak
             to your \noexpand\documentclass declaration may fix this.%
             \MessageBreak
             Try typing \space <return> \space to proceed.}}}%
      {\@pass@ptions\@currext{#2}{#1}%
       \global\expandafter
       \let\csname ver@\@currname.\@currext\endcsname\@empty
       \expandafter\let\csname\@currname.\@currext-h@@k\endcsname\@empty
       \InputIfFileExists
         {\@currname.\@currext}%
         {%
%-----------------------------------------
%    \end{macrocode}
%    When the current extension is \cs{@pkgextension} we are loading a
%    package otherwise, if it is \cs{@clsextension}, a class, so
%    depending on that we execute different hooks.  If the extension is
%    neither, then it is another type of file without special hooks.
%    \begin{macrocode}
           \ifx\@currext\@pkgextension
             \UseHook{package/before}%
             \UseHook{package/before/\@currname}%
           \else
             \ifx\@currext\@clsextension
               \UseHook{class/before}%
               \UseHook{class/before/\@currname}%
             \fi
           \fi
%    \end{macrocode}
%    The value of \cs{CurrentFile} holds during \cs{InputIfFileExists},
%    so the hooks above have that available for using.  However at this
%    point \cs{CurrentFile} is reset to its previous value by
%    \cs{@filehook@file@pop} because it doesn't know that we'll have
%    more hooks ahead.  So here (still in the \meta{true} branch of
%    \cs{InputIfFileExists}, right after actually reading the file in)
%    we'll cheat:  use \cs{@filehook@file@push} once more, so there are
%    two entries for the current file in the name stack, so that when
%    \cs{InputIfFileExists} pops it, there's still one (identical) left.
%    \begin{macrocode}
           \@filehook@file@push
%-----------------------------------------
         }%
         {\@missingfileerror\@currname\@currext}%
    \let\@unprocessedoptions\@@unprocessedoptions
    \csname\@currname.\@currext-h@@k\endcsname
    \expandafter\let\csname\@currname.\@currext-h@@k\endcsname
              \@undefined
%-----------------------------------------
%    \end{macrocode}
%    And same procedure, James, when we are finished loading, except
%    that the hook order is now reversed.
%    \begin{macrocode}
    \ifx\@currext\@pkgextension
      \UseHook{package/after/\@currname}%
      \UseHook{package/after}%
    \else
      \ifx\@currext\@clsextension
        \UseHook{class/after/\@currname}%
        \UseHook{class/after}%
      \fi
    \fi
%    \end{macrocode}
%    Now here we do \cs{@filehook@file@pop} to restore the
%    \cs{CurrentFile} before this file being loaded and fix what we've
%    done in the stack right above.
%    \begin{macrocode}
    \@filehook@file@pop
%-----------------------------------------
    \@unprocessedoptions}%
    \@ifl@ter\@currext{#1}{#3}{}%
      {\@latex@warning@no@line
         {You have requested,\on@line,
          version\MessageBreak
            `#3' of \@cls@pkg\space #1,\MessageBreak
          but only version\MessageBreak
           `\csname ver@#1.\@currext\endcsname'\MessageBreak
          is available}}%
    \ifx\@currext\@clsextension\let\LoadClass\@twoloadclasserror\fi
    \@popfilename
    \@reset@ptions}%
  \reserved@a}
%    \end{macrocode}
%
%    The code for this macro has changed between 2020/02/02 and
%    2020/10/01 so the never version is this:
%    \begin{macrocode}
\@ifl@t@r\fmtversion{2020/10/01}
{%
\def\load@onefilewithoptions#1[#2][#3]#4{%
  \@pushfilename
  \xdef\@currname{#1}%
  \global\let\@currext#4%
  \let\CurrentOption\@empty
  \@reset@ptions
  \makeatletter
  \def\reserved@a{%
    \@ifl@aded\@currext{#1}%
      {\@if@ptions\@currext{#1}{#2}{}%
        {\@latex@error
            {Option clash for \@cls@pkg\space #1}%
            {The package #1 has already been loaded
             with options:\MessageBreak
             \space\space[\@ptionlist{#1.\@currext}]\MessageBreak
             There has now been an attempt to load it
              with options\MessageBreak
             \space\space[#2]\MessageBreak
             Adding the global options:\MessageBreak
             \space\space
                  \@ptionlist{#1.\@currext},#2\MessageBreak
             to your \noexpand\documentclass declaration may fix this.%
             \MessageBreak
             Try typing \space <return> \space to proceed.}}}%
      {\@pass@ptions\@currext{#2}{#1}%
       \global\expandafter
       \let\csname ver@\@currname.\@currext\endcsname\@empty
       \expandafter\let\csname\@currname.\@currext-h@@k\endcsname\@empty
       \InputIfFileExists
         {\@currname.\@currext}%
         {%
%-----------------------------------------
           \ifx\@currext\@pkgextension
             \UseHook{package/before}%
             \UseHook{package/before/\@currname}%
           \else
             \ifx\@currext\@clsextension
               \UseHook{class/before}%
               \UseHook{class/before/\@currname}%
             \fi
           \fi
           \@filehook@file@push
%-----------------------------------------
         }%
         {\@missingfileerror\@currname\@currext}%
    \expandafter\let\csname unprocessedoptions-\@currname.\@currext\endcsname
                    \@@unprocessedoptions
    \csname\@currname.\@currext-h@@k\endcsname
    \expandafter\let\csname\@currname.\@currext-h@@k\endcsname
              \@undefined
    \ifx\@unprocessedoptions\relax
      \let\@unprocessedoptions\@undefined
    \else
      \csname unprocessedoptions-\@currname.\@currext\endcsname
    \fi
    \expandafter\let
        \csname unprocessedoptions-\@currname.\@currext\endcsname
       \@undefined
%-----------------------------------------
    \ifx\@currext\@pkgextension
      \UseHook{package/after/\@currname}%
      \UseHook{package/after}%
    \else
      \ifx\@currext\@clsextension
        \UseHook{class/after/\@currname}%
        \UseHook{class/after}%
      \fi
    \fi
    \@filehook@file@pop
%-----------------------------------------
    }%
    \@ifl@ter\@currext{#1}{#3}{}%
      {\@latex@warning@no@line
         {You have requested,\on@line,
          version\MessageBreak
            `#3' of \@cls@pkg\space #1,\MessageBreak
          but only version\MessageBreak
           `\csname ver@#1.\@currext\endcsname'\MessageBreak
          is available}}%
    \ifx\@currext\@clsextension\let\LoadClass\@twoloadclasserror\fi
    \@popfilename
    \@reset@ptions}%
  \reserved@a}
}{}%
%    \end{macrocode}
%  \end{macro}




%  \begin{macro}{\@include}
%    \begin{macrocode}
\def\@include#1 {%
  \clearpage
  \if@filesw
    \immediate\write\@mainaux{\string\@input{#1.aux}}%
  \fi
  \@tempswatrue
  \if@partsw
    \@tempswafalse
    \edef\reserved@b{#1}%
    \@for\reserved@a:=\@partlist\do
      {\ifx\reserved@a\reserved@b\@tempswatrue\fi}%
  \fi
  \if@tempswa
    \let\@auxout\@partaux
    \if@filesw
      \immediate\openout\@partaux #1.aux
      \immediate\write\@partaux{\relax}%
    \fi
%-----------------------------------------
%    \end{macrocode}
%    First we need to fix \cs{CurrentFile}, because due the pesky
%    space-delimited \cs{@include}, \cs{CurrentFile} contains the
%    \cs{include}'d file with a space.  To fix that, we re-do
%    \cs{set@curr@file}.  The 2020/10/01 release doesn't need this as
%    \cs{include} was changed to do \cs{set@curr@file} on the correct
%    file name, rather than one with a trailing space.
%    \begin{macrocode}
    \set@curr@file{#1}%
%    \end{macrocode}
%    Execute the \texttt{before} hooks just after we switched the
%    \texttt{.aux} file \ldots
%    \begin{macrocode}
    \UseHook{include/before}%
    \UseHook{include/before/#1}%
%-----------------------------------------
    \@input@{#1.tex}%
%    \end{macrocode}
%    \ldots{} then \texttt{end} hooks \ldots
%    \begin{macrocode}
%-----------------------------------------
    \UseHook{include/end/#1}%
    \UseHook{include/end}%
%-----------------------------------------
    \clearpage
%    \end{macrocode}
%    \ldots{} and after the \cs{clearpage} the \texttt{after} hooks
%    followed by another \cs{clearpage} just in case new material got
%    added (after all we need to be in  well defined state after the
%    \cs{include}).
%    \begin{macrocode}
%-----------------------------------------
    \UseHook{include/after/#1}%
    \UseHook{include/after}%
%    \end{macrocode}
%    The additional \cs{clearpage} is needed to ensure that switching
%    the \texttt{.aux} files happen at a defined point even if the
%    above hooks add further material.
%    \begin{macrocode}
    \clearpage
%-----------------------------------------
    \@writeckpt{#1}%
    \if@filesw
      \immediate\closeout\@partaux
    \fi
  \else
    \deadcycles\z@
    \@nameuse{cp@#1}%
  \fi
  \let\@auxout\@mainaux
}
%    \end{macrocode}
%
%    The code for this macro has changed between 2020/02/02 and
%    2020/10/01 so the never version is this:
%    \begin{macrocode}
\@ifl@t@r\fmtversion{2020/10/01}
{%
\def\@include#1 {%
  \clearpage
  \if@filesw
    \immediate\write\@mainaux{\string\@input{"#1.aux"}}%
  \fi
  \@tempswatrue
  \if@partsw
    \@tempswafalse
    \edef\reserved@b{#1}%
    \@for\reserved@a:=\@partlist\do
      {\ifx\reserved@a\reserved@b\@tempswatrue\fi}%
  \fi
  \if@tempswa
    \let\@auxout\@partaux
    \if@filesw
      \immediate\openout\@partaux "#1.aux"
      \immediate\write\@partaux{\relax}%
    \fi
%-----------------------------------------
    \UseHook{include/before}%
    \UseHook{include/before/#1}%
%-----------------------------------------
    \@input@{#1.tex}%
%-----------------------------------------
    \UseHook{include/end/#1}%
    \UseHook{include/end}%
%-----------------------------------------
    \clearpage
%-----------------------------------------
    \UseHook{include/after/#1}%
    \UseHook{include/after}%
    \clearpage
%-----------------------------------------
    \@writeckpt{#1}%
    \if@filesw
      \immediate\closeout\@partaux
    \fi
  \else
    \deadcycles\z@
    \@nameuse{cp@#1}%
  \fi
  \let\@auxout\@mainaux}
}{}
%    \end{macrocode}
%  \end{macro}
%
%
%
% \subsection{High-level interfaces for \LaTeX{}}
%
%    None so far and the general feeling for now is that the hooks are
%    enough. Packages like \pkg{filehook}, etc., may use them to set
%    up their interfaces (samples are given below) but for the now the
%    kernel will not provide any.
%
%
%
%    \begin{macrocode}
%</2ekernel>
%    \end{macrocode}
%
%
%
% \section{Package emulation for compatibility}
%
%
% \subsection{Package \pkg{filehook} emulation}
%
%    This is a partial implementation of the  \pkg{filehook}
%    interfaces. It is only meant for guidance in case that package
%    gets updated to use the hook management.
%
%    Not implemented are:
%\begin{verbatim}
%  \AtBeginOfFiles
%  \AtEndOfFiles
%  \AtBeginOfInputs
%  \AtEndOfInputs
%  \AtBeginOfInputFile
%  \AtEndOfInputFile
%\end{verbatim}
%
%    \begin{macrocode}
%<*filehook-draft>
%    \end{macrocode}
%
%    \begin{macrocode}
\newcommand\AtBeginOfEveryFile [1]
  {\AddToHook{file/before}{#1}}
\newcommand\AtEndOfEveryFile [1]
  {\AddToHook{file/after}{#1}}
%    \end{macrocode}
%    
%    \begin{macrocode}
\newcommand\AtBeginOfIncludes [1]
  {\AddToHook{include/before}{#1}}
\newcommand\AtEndOfIncludes [1]
  {\AddToHook{include/end}{#1}}
\newcommand\AfterIncludes [1]
  {\AddToHook{include/after}{#1}}
%    \end{macrocode}
%    
%    \begin{macrocode}
\newcommand\AtBeginOfPackages [1]
  {\AddToHook{package/before}{#1}}
\newcommand\AtEndOfPackages [1]
  {\AddToHook{package/after}{#1}}
%    \end{macrocode}
%    
%    \begin{macrocode}
\newcommand\AtBeginOfClasses [1]
  {\AddToHook{class/before}{#1}}
\newcommand\AtEndOfClasses [1]
  {\AddToHook{class/after}{#1}}
%    \end{macrocode}
%
%    For normal files we drop the \texttt{.tex} extension for now:
%    \begin{macrocode}
\newcommand\AtBeginOfFile [2]
  {\AddToHook{file/before/#1}{#2}}
\newcommand\AtEndOfFile [2]
  {\AddToHook{file/after/#1}{#2}}
%    \end{macrocode}
%    
%    \begin{macrocode}
\DeclareDocumentCommand \AtBeginOfPackageFile {smm}
   {\IfBooleanTF{#1}%
     {\@ifpackageloaded{#2}%
         {#3}%
         {\AddToHook{package/before/#2}{#3}}}%
     {\AddToHook{package/before/#2}{#3}}%
   }
\DeclareDocumentCommand \AtEndOfPackageFile {smm}
   {\IfBooleanTF{#1}%
     {\@ifpackageloaded{#2}%
         {#3}%
         {\AddToHook{package/after/#2}{#3}}}%
     {\AddToHook{package/after/#2}{#3}}%
   }
%    \end{macrocode}
%
%    Are the * forms here of any use? I know they are use 3--4 times
%    on CTAN but I wonder if those are real or mistaken usages.  
%    \begin{macrocode}
\DeclareDocumentCommand \AtBeginOfClassFile {smm}
   {\IfBooleanTF{#1}%
     {\@ifclassloaded{#2}%
         {#3}%
         {\AddToHook{class/before/#2}{#3}}}%
     {\AddToHook{class/before/#2}{#3}}%
   }
\DeclareDocumentCommand \AtEndOfClassFile {smm}
   {\IfBooleanTF{#1}%
     {\@ifclassloaded{#2}%
         {#3}%
         {\AddToHook{class/after/#2}{#3}}}%
     {\AddToHook{class/after/#2}{#3}}%
   }
%    \end{macrocode}
%    
%    \begin{macrocode}
\newcommand\AtBeginOfIncludeFile [2]
  {\AddToHook{include/before/#1}{#2}}
\newcommand\AtEndOfIncludeFile [2]
  {\AddToHook{include/end/#1}{#2}}
\newcommand\AfterIncludeFile [2]
  {\AddToHook{include/after/#1}{#2}}
%    \end{macrocode}
%
%
%    This is missing some interfaces so disabling the package isn't
%    really correct, but then this code above is not supposed to stay
%    like this anyway.
%    \begin{macrocode}
\expandafter\let\csname ver@filehook.sty\endcsname\fmtversion
\@namedef {ver@filehook.sty}{2020/10/01}
%    \end{macrocode}
%
%
%    \begin{macrocode}
%</filehook-draft>
%    \end{macrocode}
%
%
%
%
% \subsection{Package \pkg{scrlfile} emulation}
%
%    This is a partial implementation of the  \pkg{scrlfile}
%    interfaces. It is only meant for guidance in case that package
%    gets updated to use the hook management.
%
%    \begin{macrocode}
%<*scrlfile-draft>
%    \end{macrocode}
%
%    I think this is roughly correct (using the \texttt{file/...}
%    hooks rather than the class or package hooks at least for the
%    \cs{After...} commands but it needs some further verification.
%
%    The star and plus variants haven't been implemented so far, this
%    is only a rough draft.
%    \begin{macrocode}
\newcommand\BeforeClass[2]
  {\AddToHook{file/before/#1.cls}{#2}}
\newcommand\AfterClass [2]
  {\AddToHook{file/after/#1.cls}{#2}}
\newcommand\AfterAtEndOfClass [2]
  {\AddToHook{class/after/#1}{#2}}
%    \end{macrocode}
%
%    \begin{macrocode}
\newcommand\BeforePackage [2]
  {\AddToHook{package/before/#1.sty}{#2}}
\newcommand\AfterPackage [2]
  {\AddToHook{file/after/#1.sty}{#2}}
\newcommand\AfterEndOfPackage [2]
  {\AddToHook{package/after/#1}{#2}}
%    \end{macrocode}
%
%    \begin{macrocode}
\newcommand\BeforeFile [2]
  {%
  \typeout{BeforeFile: #1!!!}%
   \AddToHook{file/before/#1}{#2}}
\newcommand\AfterFile [2]
  {%
    \typeout{AfterFile: #1!!!}%
    \AddToHook{file/after/#1}{#2}}
%    \end{macrocode}
%
%    This is missing some interfaces so disabling the package isn't
%    really correct, but then this code above is not supposed to stay
%    like this anyway.
%    \begin{macrocode}
\expandafter\let\csname ver@scrlfile.sty\endcsname\fmtversion
\@namedef {ver@scrlfile.sty}{2020/10/01}
%    \end{macrocode}
%
%    \begin{macrocode}
%</scrlfile-draft>
%    \end{macrocode}
%
%
%
% \section{A sample package for structuring the log output}
%
%    \begin{macrocode}
%<*structuredlog>
%<@@=filehook>
%    \end{macrocode}
%
%    \begin{macrocode}
\ProvidesExplPackage
    {structuredlog}{\ltfilehookdate}{\ltfilehookversion}
    {Structuring the TeX transcript file}
%    \end{macrocode}



%    \begin{macrocode}
\int_new:N \g_@@_nesting_level_int
%    \end{macrocode}

%    \begin{macrocode}
\tl_new:N   \g_@@_nesting_prefix_tl
\tl_gset:Nn \g_@@_nesting_prefix_tl { }
%    \end{macrocode}
%    
%    \begin{macrocode}
\AddToHook{file/before}{
  \int_gincr:N \g_@@_nesting_level_int
  \tl_gput_right:Nn\g_@@_nesting_prefix_tl {=}
  \iow_term:x {
    \g_@@_nesting_prefix_tl \space
    ( LEVEL~ \int_use:N \g_@@_nesting_level_int \space START )~
    \CurrentFile  ^^J
  }
}
%    \end{macrocode}
%    We don't want to install the \hook{file/after} hook immediately,
%    because that would mean it is the first time executed when the
%    package finishes. We therefore put the declaration inside
%    \cs{AddToHookNext} so that it gets only installed when we have
%    left the package. 
%    \begin{macrocode}
\AddToHookNext{file/after}{
  \AddToHook{file/after}{
    \iow_term:x {
      \g_@@_nesting_prefix_tl \space
      ( LEVEL~ \int_use:N \g_@@_nesting_level_int \space STOP )~
      \CurrentFile  ^^J
    }
    \int_gdecr:N \g_@@_nesting_level_int
    \tl_gset:Nx \g_@@_nesting_prefix_tl
       {\exp_after:wN \use_none:n \g_@@_nesting_prefix_tl}
  }
}
%    \end{macrocode}
%
%    We have to manually increment the level because now that we have
%    installed the code in \texttt{file/after} it gets decremented
%    when we leave the package without ever being incremented upon
%    entry.
%    \begin{macrocode}
%\int_incr:N\g_@@_nesting_level_int
%    \end{macrocode}
%
%    \begin{macrocode}
%</structuredlog>
%    \end{macrocode}

%    \Finale
%


%%%%%%%%%%%%%%%%%%%%%%%%%%%%%%%%%%%%%%%%%%%%%  
\endinput
%%%%%%%%%%%%%%%%%%%%%%%%%%%%%%%%%%%%%%%%%%%%%  



                    }
%    \end{macrocode}
%
%  \subsection{Package options}
%
%    For now we offer a simple debug option which turns on a lot of
%    strange \cs{typeout} messages, nothing fancy.
%    \begin{macrocode}
\ExplSyntaxOn
%    \end{macrocode}
%    
%    \begin{macrocode}
\hook_debug_off:
\DeclareOption { debug } { \hook_debug_on:
                           \shipout_debug_on: }
%    \end{macrocode}
%
%
%    For now we offer a simple debug option which turns on a lot of
%    strange \cs{typeout} messages, nothing fancy.
%    \begin{macrocode}
\shipout_debug_off:
\DeclareOption { debug-shipout } { \shipout_debug_on: }
%    \end{macrocode}
%    
%    \begin{macrocode}
\ProcessOptions
%    \end{macrocode}
%
%
%  \subsection{Temporarily patching package until changed}
%
%
%
%    \pkg{filehook} support until that package is patched:
%    \begin{macrocode}
\RequirePackage{filehook-ltx}
%    \end{macrocode}
%
%
%
%    \begin{macrocode}
%</package>
%    \end{macrocode}
%
%
%
% \Finale
%



                     % \iffalse meta-comment
%
%% File: ltshipout.dtx (C) Copyright 2020 Frank Mittelbach, LaTeX Team
%
% It may be distributed and/or modified under the conditions of the
% LaTeX Project Public License (LPPL), either version 1.3c of this
% license or (at your option) any later version.  The latest version
% of this license is in the file
%
%    https://www.latex-project.org/lppl.txt
%
%
% The development version of the bundle can be found below
%
%    https://github.com/FrankMittelbach/...
%
% for those people who are interested or want to report an issue.
%
%    \begin{macrocode}
\providecommand\ltshipoutversion{v0.9a}
\providecommand\ltshipoutdate{2020/07/15}
%    \end{macrocode}
%
%<*driver>

\documentclass{l3doc}

% bug fix fo l3doc.cls
\ExplSyntaxOn
\cs_set_protected:Npn \__codedoc_macro_typeset_one:nN #1#2
  {
    \vbox_set:Nn \l__codedoc_macro_box
      {
        \vbox_unpack_drop:N \l__codedoc_macro_box
        \hbox { \llap { \__codedoc_print_macroname:nN {#1} #2
            \MacroFont       % <----- without it the \ is in lmr10 if a link is made
            \      
        } }
      }
    \int_incr:N \l__codedoc_macro_int
  }
\ExplSyntaxOff

\EnableCrossrefs
\CodelineIndex
\begin{document}
  \DocInput{ltshipout.dtx}
\end{document}
%</driver>
%
% \fi
%
%
% \long\def\fmi#1{\begin{quote}\itshape Todo: #1\end{quote}}
%
% \newcommand\hook[1]{\texttt{#1}}
%
%
% \title{The \texttt{ltshipout} package\thanks{This package has version
%    \ltshipoutversion\ dated \ltshipoutdate, \copyright\ \LaTeX\
%    Project.}}
%
% \author{Frank Mittelbach}
%
% \maketitle
%
%
% \tableofcontents
%
% \section{Introduction}
%
%    The code provides an interface to the \cs{shipout} primitive of
%    \TeX{} which is called when a finished pages is finally
%    \enquote{shipped out} to the target output file, e.g., the
%    \texttt{.dvi} or \texttt{.pdf} file.
%    A good portion of the code is based on ideas by Heiko Oberdiek
%    implemented in his packages \pkg{atbegshi} and \pkg{atenddvi}
%    even though the interfaces are somewhat
%    different.\footnote{Heiko's interfaces are emulated by the kernel
%    code, if a document requests his packages, so older documents
%    will continue to work.}
%
%  \subsection{Overloading the \cs{shipout} primitive}
%
%
% \begin{function}{\shipout}
%    With this implementation \TeX's shipout primitive is no longer
%    available for direct use. Instead \cs{shipout} is running some
%    (complicated) code that picks up the box to be shipped out
%    regardless of how that is done, i.e., as a constructed \cs{vbox}
%    or \cs{hbox} or as a box register.
%
%    It then stores it in a named box register.  This box can then be
%    manipulated through a set of hooks after which it is shipped out
%    for real.
% \end{function}

%  \begin{variable}{\ShipoutBox,\l_shipout_box}
%    This box register is called \cs{ShipoutBox} (alternatively available via the
%    L3 name \cs{l_shipout_box}).
%  \end{variable}
%

%  \begin{variable}{\l_shipout_box_ht_dim,
%                   \l_shipout_box_dp_dim,\l_shipout_box_wd_dim,
%                   \l_shipout_box_ht_plus_dp_dim}
%    The shipout box dimensions are available in the L3 registers
%    \cs{l_shipout_box_ht_dim}, etc.\ (there are no \LaTeXe{}
%    names).\footnotemark{} These variables can be used
%    inside the hook code for \hook{shipout/before},
%    \hook{shipout/foreground} and \hook{shipout/background} if needed.
%  \end{variable}
%  \footnotetext{Might need changing, but HO's version as strings
%    is not really helpful I think).}
%
%
%
%
% \subsection{Provided hooks}
%
%  \begin{variable}{shipout/before,
%                   shipout/foreground,shipout/background,
%                   shipout/firstpage,
%                   shipout/lastpage}
%    The code offers a number of hooks into which packages (or the
%    user) can add code to support different use cases.
%    These are:
%    \begin{description}
%    \item[\hook{shipout/before}]
%
%       This hook is executed after the finished page has been stored in
%       \cs{ShipoutBox} / \cs{l_shipout_box}).
%       It can be used to alter that box content or to discard it
%       completely (see \cs{DiscardShipoutBox} below).
%
%    \item[\hook{shipout/background}]
%
%       This hook adds a picture environment into the background of
%       the page (with the \texttt{(0,0)} coordinate in the top-left
%       corner using a \cs{unitlength} of \texttt{1pt}.
%
%       It should therefore only receive \cs{put} commands or other
%       commands suitable in a \texttt{picture} environment and the
%       vertical coordinate values would normally be
%       negative.
%
%    \fmi{Again, mainly \pkg{atbegshi} compatibility. Not
%       sure it is best to have  to always use negative
%       coordinates.}
%
%       Technically this is implemented by adding a zero-sized
%       \cs{hbox} as the very first item into the \cs{ShipoutBox}
%       containing that \texttt{picture} environment. Thus the rest of
%       the box content will overprint what ever is typeset by that hook.
%
%
%    \item[\hook{shipout/foreground}]
%
%       This hook adds a picture environment into the foreground of
%       the page (with the \texttt{(0,0)} coordinate in the top-left
%       corner using a \cs{unitlength} of \texttt{1pt}.
%
%       Technically this is implemented by adding a zero-sized
%       \cs{hbox} as the very last item into the \cs{ShipoutBox} and
%       raising it up so that it still has its \texttt{(0,0)} point in
%       the top-left corner.
%       But being placed after the main box content it will be typeset
%       later and thus overprints it (i.e., is in the foreground).
%
%
%    \item[\hook{shipout/firstpage}]
%
%       The material from this hook is executed only once at the very
%       beginning of the first output page. It should only contain
%       \cs{special} commands needed to direct post processors
%       handling the \texttt{.dvi} or \texttt{.pdf} output.\fmi{not
%       sure it has to be that restrictive.}
%
%       In \LaTeXe{} that was already existing but implemented as a box
%       register \cs{@begindvibox}.
%       \fmi{drop or at least mark the obsolete code in latex.ltx}
%
%
%    \item[\hook{shipout/lastpage}]
%
%       The corresponding hook to add \cs{special}s at the very end of
%       the output file. It is only entered on the very last page.
%
%       It may not be possible for \LaTeX{} to correctly determine which page is
%       the last one without several reruns. If this happens and the
%       hook is non-empty then \LaTeX{} will add an extra page to
%       place the material and also request a rerun to get the correct
%       placement sorted out.
%
%    \end{description}
%  \end{variable}
%
%    As mentioned above the hook \hook{shipout/before} is executed
%    first and can manipulate the prepared shipout box stored in
%    \cs{ShipoutBox} or set things up for use in \cs{write} during the
%    actual shipout. The other hooks are added inside hboxes to the
%    box being shipped out in the following
%    order:
%    \begin{center}
%    \begin{tabular}{ll}
%       \hook{shipout/firstpage}   & only on the first page \\
%       \hook{shipout/background}  &                        \\
%       \meta{boxed content of \cs{ShipoutBox}} &             \\
%       \hook{shipout/foreground}  &                       \\
%       \hook{shipout/lastpage}    & only on the last page \\
%    \end{tabular}
%    \end{center}
%    If any of the hooks has no code then that particular no box is
%    added at that point.
%
%    In a document that doesn't produce pages, e.g., only makes
%    \cs{typeout}s, none of the hooks are executed (as there is no
%    \cs{shipout}) not even the \hook{shipout/lastpage} hook.
%
% \begin{function}{\AtBeginDvi,\AtEndDvi}
%    \cs{AtBeginDvi} is the existing \LaTeXe{} interface to fill the
%    \hook{shipout/firstpage} hook. This is not really a good name
%    as it is not just supporting \texttt{.dvi} but also \texttt{.pdf}
%    output or \texttt{.dvx}.
%
%    \cs{AtEndDvi} is the counterpart that was not available in the
%    kernel but only through the package \pkg{atenddvi}. It fills the
%    \hook{shipout/lastpage} hook.
%
%    \fmi{better names? Any suggestions?}
%
% \end{function}
%
% \subsection{Special commands for use inside the hooks}
%
% \begin{function}{\DiscardShipoutBox,\shipout_discard_box:}
%   \begin{syntax}
%     \cs{AddToHookNext} \texttt{\{shipout/before\} \{...\cs{DiscardShipoutBox}...\}}
%   \end{syntax}
%    The \cs{DiscardShipoutBox} declaration (L3 name
%    \cs{shipout_discard_box:})
%    requests that on the next
%    shipout the page box is thrown away instead of being shipped to
%    the \texttt{.dvi} or \texttt{.pdf} file.
%
%    Typical applications wouldn't do this unconditionally, but have
%    some processing logic that decides to use or not to use the page.
%
%    Note that if this declaration is used directly in the document it
%    may depend on the placement to which page it applies, given that
%    \LaTeX{} output routine is called in an asynchronous manner!
%
%    \fmi{Once we have a new mark mechanism available we can improve
%    on that and make sure that the declaration applies to the page
%    that contains it.}
%  \end{function}
%
%    In the \pkg{atbegshi} package there are a number of additional
%    commands for use inside the \hook{shipout/before} hook. They
%    should normally not be needed any more as one can instead simply
%    add code to the hooks \hook{shipout/before},
%    \hook{shipout/background} or
%    \hook{shipout/foreground}.\footnote{If that assumption turns out to
%    be wrong it would be trivial to change them to public functions
%    (right now they are private).} If \pkg{atbegshi} gets loaded then
%    those commands become available as public functions with their original
%    names as given below.
%
%
% \subsection{Information counters}
%
%
%  \begin{variable}{\ReadonlyShipoutCounter,\g_shipout_readonly_int}
%   \begin{syntax}
%     \cs{ifnum}\cs{ReadOnlyShipoutCounter}\texttt{=...}
%     \cs{int_use:N} \cs{g_shipout_readonly_int} \texttt{\% expl3 usage}
%   \end{syntax}
%    This integer holds the number of pages shipped out up to now
%    (including the one to be shipped out when inside the output
%    routine). More precisely, it is incremented only after it is
%    clear that a page will be shipped out, i.e., after the
%    \hook{shipout/before} hook (because that might discard the page)!

%    Just like with the \texttt{page} counter its value is
%    only accurate within the output routine. In the body of the
%    document it may be off by one as the output routine is called
%    asynchronously!
%
%    Also important: it \emph{must not} be set, only read. There are
%    no provisions to prevent that but if you do, chaos will be the
%    result. To emphasize this fact it is not provided as a \LaTeX{}
%    counter but as a \TeX{} counter (i.e., a command), so
%    \cs{Alph}\Arg{\cs{ReadonlyShipoutCounter}} etc, would not work.
%  \end{variable}
%
%  \begin{variable}{totalpages,\g_shipout_totalpages_int}
%   \begin{syntax}
%     \cs{arabic}\texttt{\{totalpages\}}
%     \cs{int_use:N} \cs{g_shipout_totalpage_int} \texttt{\% expl3 usage}
%   \end{syntax}
%    In contrast to \cs{ReadonlyShipoutCounter}, the
%    \texttt{totalpages} counter is a \LaTeX{} counter and incremented
%    for each shipout attempt including those pages that are discarded
%    for one or the other reason. Again \hook{shipout/before} sees
%    the counter before it is incremented).
%
%    Furthermore, while it is incremented for each page, its value is
%    never used by \LaTeX. It can therefore be freely reset or changed by user
%    code, for example, to additionally count a number of pages that
%    are not build by \LaTeX\ but are added in a later part of the
%    process, e.g., cover pages or picture pages made externally.
%
%    Important: as this is a page-related counter its value is only
%    reliable inside the output routine!
%  \end{variable}
%

%
% \subsection{Debugging shipout code}
%
% \begin{function}{\DebugShipoutOn,\DebugShipoutOff,
%         \shipout_debug_on:,\shipout_debug_off:}
%   \begin{syntax}
%     \cs{DebugShipoutOn}
%   \end{syntax}
%    Turn the debugging of shipout code on or off. This displays
%    changes made to the shipout data structures.  \fmi{This needs
%    some rationalizing and will probably not stay this way.}
% \end{function}
%
%

%
% \section{Emulating commands from other packages}
%
%    The packages in this section are no longer necessary but as they
%    are used in other packages they are emulated when they are loaded
%    via \cs{usepackage} or \cs{RequirePackage}.
%
%
% \subsection{Emulating \pkg{atbegshi}}
%
%
% \begin{function}{\AtBeginShipoutUpperLeft,\AtBeginShipoutUpperLeftForeground}
%   \begin{syntax}
%     \cs{AddToHook} \texttt{\{shipout/before\} \{...\cs{AtBeginShipoutUpperLeft}}\Arg{code}\texttt{...\}}
%   \end{syntax}
%    This adds a \texttt{picture} environment into the background of the shipout
%    box expecting \meta{code} to contain \texttt{picture}
%    commands. The same effect can be obtained by simply using kernel
%    features as follows:
%    \begin{quote}
%      \cs{AddToHook}\texttt{\{shipout/background\}}\Arg{code}
%    \end{quote}
%    There is one technical difference: if
%    \cs{AtBeginShipoutUpperLeft} is used several times each
%    invocation is put into its own box inside the shipout box whereas
%    all \meta{code} going into \hook{shipout/background} ends up
%    all in the same box in the order it is added or sorted based on
%    the rules for the hook chunks.
%
%    \cs{AtBeginShipoutUpperLeftForeground} is similar with the
%    difference that the \texttt{picture} environment is placed in the
%    foreground. To model it with the kernel functions use the hook
%    \hook{shipout/foreground} instead.
% \end{function}
%
%
% \begin{function}{\AtBeginShipoutAddToBox,\AtBeginShipoutAddToBoxForeground}
%   \begin{syntax}
%     \cs{AddToHook} \texttt{\{shipout/before\} \{...\cs{AtBeginShipoutAddToBox}}\Arg{code}\texttt{...\}}
%   \end{syntax}
%    These work like \cs{AtBeginShipoutUpperLeft} and
%    \cs{AtBeginShipoutUpperLeftForeground} with the difference that
%    \meta{code} is directly placed into an \cs{hbox} inside the
%    shipout box and not surrounded by a \texttt{picture} environment.
%
%    To emulate them using \hook{shipout/background} or
%    \hook{shipout/foreground} you may have to wrap \meta{code} into
%    a \cs{put} statement but if the code is not doing any typesetting
%    just adding it to the hook should be sufficient.
% \end{function}
%


% \begin{function}{\AtBeginShipoutBox}
%    This is the name of the shipout box as \pkg{atbegshi} knows it.
% \end{function}
%
% \begin{function}{\AtBeginShipoutInit}
%   By default \pkg{atbegshi} delayed its action until
%    \verb=\beg{document}=.  This command was forcing it in an earlier
%    place. With the new concept it does nothing.
% \end{function}
%
% \begin{function}{\AtBeginShipout,\AtBeginShipoutNext}
%   \begin{syntax}
% \cs{AtBeginShipout}\Arg{code} $\equiv$ \cs{AddToHook}\texttt{\{shipout/before\}}\Arg{code}
% \cs{AtBeginShipoutNext}\Arg{code} $\equiv$ \cs{AddToHookNext}\texttt{\{shipout/before\}}\Arg{code}
%   \end{syntax}
%   This is equivalent to filling the \hook{shipout/before} hook
%    by  either using \cs{AddToHook} or \cs{AddToHookNext}, respectively.
% \end{function}
%
% \begin{function}{\AtBeginShipoutFirst,\AtBeginShipoutDiscard}
%   The \pkg{atbegshi} names for \cs{AtBeginDvi} and \cs{DiscardShipoutBox}.
% \end{function}
%



% \subsection{Emulating \pkg{everyshi}}
%


% \begin{function}{\EveryShipout}
%   \begin{syntax}
% \cs{EveryShipout}\Arg{code} $\equiv$ \cs{AddToHook}\texttt{\{shipout/before\}}\Arg{code}
%   \end{syntax}
% \end{function}
%
% \begin{function}{\AtNextShipout}
%   \begin{syntax}
% \cs{AtNextShipout}\Arg{code} $\equiv$ \cs{AddToHookNext}\texttt{\{shipout/before\}}\Arg{code}
%   \end{syntax}
% \end{function}
%


% \subsection{Emulating \pkg{atenddvi}}
%
% The \pkg{atenddvi} package implemented only a single command:
%    \cs{AtEndDvi} and that is now available out of the box.



% \subsection{Emulating \pkg{everypage}}
%
%    This page takes over the original \cs{@begindvi} hook and replaces 
%    it. It should be all covered by the hooks offered here (details
%    need checking) and thus could simply use the provided hooks
%    rather than defining its own.
%
%
%
% \StopEventually{\setlength\IndexMin{200pt}  \PrintIndex  }
%
%
% \section{The Implementation}
%    \begin{macrocode}
%<*2ekernel>
%<@@=shipout>
\ExplSyntaxOn
%    \end{macrocode}
%
%
%  \subsection{Debugging}
%
%  \begin{macro}{\g_@@_debug_bool}
%    Holds the current debugging state.
%    \begin{macrocode}
\bool_new:N \g_@@_debug_bool
%    \end{macrocode}
%  \end{macro}
%
%  \begin{macro}{\shipout_debug_on:,\shipout_debug_off:}
%  \begin{macro}{\@@_debug:n}
%  \begin{macro}{\@@_debug_gset:}
%    Turns debugging on and off by redefining \cs{@@_debug:n}.
%    \begin{macrocode}
\cs_new_eq:NN \@@_debug:n  \use_none:n
\cs_new_protected:Npn \shipout_debug_on:
  {
    \bool_gset_true:N \g_@@_debug_bool
    \@@_debug_gset:
  }
\cs_new_protected:Npn \shipout_debug_off:
  {
    \bool_gset_false:N \g_@@_debug_bool
    \@@_debug_gset:
  }
\cs_new_protected:Npn \@@_debug_gset:
  {
    \cs_gset_protected:Npx \@@_debug:n ##1
      { \bool_if:NT \g_@@_debug_bool {##1} }
  }
%    \end{macrocode}
%  \end{macro}
%  \end{macro}
%  \end{macro}
%

%  \begin{macro}{shipout/before,
%                   shipout/foreground,shipout/background,
%                   shipout/firstpage,
%                   shipout/lastpage}
%    Declaring all hooks for the shipout code.
%    \begin{macrocode}
\hook_new:n{shipout/before}
\hook_new:n{shipout/foreground}
\hook_new:n{shipout/background}
\hook_new:n{shipout/firstpage}
\hook_new:n{shipout/lastpage}
%    \end{macrocode}
%  \end{macro}




%  \begin{macro}{\ShipoutBox,\l_shipout_box}
%    The box filled with the page to be shipped out (both L3 and
%       \LaTeXe{} name).
%    \begin{macrocode}
\box_new:N  \l_shipout_box
%    \end{macrocode}
%    
%    \begin{macrocode}
\cs_set_eq:NN \ShipoutBox \l_shipout_box
%    \end{macrocode}
%  \end{macro}




%  \begin{macro}{\@@_execute:}
%    This is going to the be the code run by \cs{shipout}. The code
%    follows closely the  ideas from \pkg{atbegshi}, so not
%    documenting that here for now.
%    \begin{macrocode}
\cs_set:Npn\@@_execute: {
  \tl_set:Nx \l_@@_group_level_tl
     { \int_value:w \tex_currentgrouplevel:D }
  \tex_afterassignment:D \@@_execute_test_level:
  \tex_setbox:D \l_shipout_box
}
%    \end{macrocode}
%  \end{macro}


%  \begin{macro}{\shipout}
%    Overloading the \cs{shipout} primitive:
%    \begin{macrocode}
\cs_gset_eq:NN \shipout \@@_execute:
%    \end{macrocode}
%  \end{macro}


%  \begin{macro}{\l_@@_group_level_tl}
%    Helper token list to record the group level at which
%    \cs{@@_execute:} is encountered.  \begin{macrocode}
\tl_new:N \l_@@_group_level_tl
%    \end{macrocode}
%  \end{macro}

 

%  \begin{macro}{\@@_execute_test_level:}
%    If the group level has changed then we are still constructing
%    \cs{l_shipout_box} and to continue we need to wait until the
%    current group has finished, hence the \cs{tex_aftergroup:D}.
%    \begin{macrocode}
\cs_new:Npn \@@_execute_test_level: {
  \int_compare:nNnT
     \l_@@_group_level_tl < \tex_currentgrouplevel:D 
     \tex_aftergroup:D
  \@@_execute_cont:
}
%    \end{macrocode}
%  \end{macro}


%  \begin{macro}{\@@_execute_cont:}
%    When we have reached this point the shipout box has been
%    processed and is available in \cs{l_shipout_box} and ready for
%    real ship out (perhaps)..
%
%    First we quickly check if it is void (can't happen in the
%    standard \LaTeX{} output routine but \cs{shipout} might be called
%    from a package that has some special processing logic). If it is
%    void we aren't shipping anything out and processing ends.\footnote{In that
%    case we don't reset the deadcyles, that would be up to the OR
%    processing logic to do.}
%    \begin{macrocode}
\cs_new:Npn \@@_execute_cont: {
  \box_if_empty:NTF \l_shipout_box
    { \PackageWarning{ltshipout}{Ignoring~ void~ shipout~ box} }
    {
%    \end{macrocode}
%    Otherwise we assume that we will ship something and prepare for
%    final adjustments (in particular setting the state of
%    \cs{protect} while we are running the hook code).
%    \begin{macrocode}
      \bool_gset_false:N \g_@@_discard_bool
      \set@typeset@protect
%    \end{macrocode}
%    We also store the current shipout box dimension in registers, so that
%    they can be used in the hook code.\footnote{This is not really
%    necessary as the code could access them via \cs{box_ht:N}, etc.,
%    but it is perhaps convenient.}
%    \begin{macrocode}
      \@@_get_box_size:N \l_shipout_box
%    \end{macrocode}
%    Then we execute the \hook{shipout/before} hook.
%    \begin{macrocode}
      \hook_use:n {shipout/before}
%    \end{macrocode}
%    In \cs{g_shipout_totalpages_int} we count all shipout attempts so
%    we increment that counter already here (the other one is
%    incremented later when we know for sure that we do a
%    \cs{shipout}.
%
%     We increment it after running the above hook so that the values
%    for \cs{g_shipout_totalpages_int} and \cs{} are in sync while the
%    hook is executed (in the case that totalpages isn't manually
%    altered or through discarding pages that is).
%    \begin{macrocode}
      \int_gincr:N \g_shipout_totalpages_int
%    \end{macrocode}
%    The above hook might contain code that requests the page to be discarded so
%    we now test for it.
%    \begin{macrocode}
      \bool_if:NTF \g_@@_discard_bool
        { \PackageInfo{ltshipout}{Completed~ page~ discarded}
          \bool_gset_false:N \g_@@_discard_bool
%    \end{macrocode}
%    As we are discarding the page box and not shipping anything out,
%    we need to do some house cleaning and reset \TeX's deadcycles so
%    that it doesn't complain about too many calls to the OR without
%    any shipout.
%    \begin{macrocode}
          \tex_deadcycles:D \c_zero_int
%    \end{macrocode}
%    \fmi{In \pkg{atbegshi} the box was dropped but is that actually
%    needed? Or the resetting of \cs{protect} to its kernel value?}
%    \begin{macrocode}
%          \group_begin:
%            \box_set_eq_drop:NN \l_shipout_box \l_shipout_box
%          \group_end:
%          \cs_set_eq:NN \protect \exp_not:N
        }
%    \end{macrocode}
%    Even if there was no explicit request to discard the box it is
%    possible that the code for the hook \hook{shipout/before} has
%    voided the box (by mistake or deliberately). We therefore test
%    once more but this time make it a warning, because the best
%    practice way is to use the request mechanism.
%    \begin{macrocode}
        { \box_if_empty:NTF \l_shipout_box
            { \PackageWarning{ltshipout}{
                Shipout~ box~ was~ voided~ by~ hook,\MessageBreak
                ignoring~ shipout~ box  }
            }
%    \end{macrocode}
%    Finally, if the box is still non-empty we are nearly ready to
%    ship it out.
%    First we increment the total page counter so that we can later
%    test if we have reached the final page according to our available
%    information.\footnote{Doing that earlier would be wrong because we
%    might end up with the last page counted but discard and then we
%    have no place to add the final objects into the output file.}
%    \begin{macrocode}
            {
              \int_gincr:N \g_shipout_readonly_int
              \@@_debug:n {
                \typeout{Absolute~ page~ =~ \int_use:N \g_shipout_readonly_int
                         \space (target:~ \@abspage@last)}
              }
%    \end{macrocode}
%    Then we store the box sizes again (as they may have
%    changed) and then look at the hooks \hook{shipout/foreground}
%    and \hook{shipout/background}. If either or both are non-empty
%    we add a \texttt{picture} environment to the box (in the
%    foreground and or in the background) and execute the hook code
%    inside that environment.
%    
%    \begin{macrocode}
              \@@_get_box_size:N \l_shipout_box
              \@kernel@before@shipout@foreground
              \hook_if_empty:nF {shipout/foreground}
                   { \@@_add_foreground_picture:n
                     { \hook_use:n {shipout/foreground} } }
              \hook_if_empty:nF {shipout/background}
                   { \@@_add_background_picture:n
                     { \hook_use:n {shipout/background} } }
%    \end{macrocode}
%    We then run \cs{@@_execute_firstpage_hook:} that adds
%    the content of the hook \hook{shipout/firstpage} to the
%    start of the first page (if non-empty). It is then redefined to
%    do nothing on later pages.
%    \begin{macrocode}
              \@@_execute_firstpage_hook:
%    \end{macrocode}
%    The we check if we have to add the \hook{shipout/lastpage} hook
%    because we have reached the last page. This test will be false for
%    all but one (and hopefully the correct) page.
%    \begin{macrocode}
              \int_compare:nNnT \@abspage@last = \g_shipout_readonly_int
                { \hook_if_empty:nF {shipout/lastpage}
                    { \@@_debug:n { \typeout{Executing~ lastpage~ hook~
                          on~ page~ \int_use:N \g_shipout_readonly_int }        }
                      \_@@_add_foreground_box:n { \UseHook{shipout/lastpage}
                                                  \@kernel@after@shipout@lastpage }
                    }
                    \bool_gset_true:N \g_@@_lastpage_handled_bool
                }
%    \end{macrocode}
%    Finally we run the actual \TeX{} primitive for shipout. As that will
%    expand delayed \cs{write} statements inside the page in which
%    protected commands should not expand we first change \cs{protect}
%    to the appropriate definition for that case.
%    \begin{macrocode}
              \cs_set_eq:NN \protect \exp_not:N
              \tex_shipout:D \box_use:N \l_shipout_box
            }
        }
    }
}
%    \end{macrocode}
%  \end{macro}



%  \begin{macro}{\@kernel@after@shipout@lastpage,\@kernel@before@shipout@foreground}
%    
%    \begin{macrocode}
\let\@kernel@after@shipout@lastpage\@empty
\let\@kernel@before@shipout@foreground\@empty
%    \end{macrocode}
%  \end{macro}


%  \begin{macro}{\@@_execute_firstpage_hook: }
%    This command adds any specials into a box and adds that to the
%    very beginning of the first box shipped out. After that we
%    redefine it to do nothing on later pages.
%    \begin{macrocode}
\cs_new:Npn \@@_execute_firstpage_hook: {
%    \end{macrocode}
%    Adding something to the beginning means adding it to the
%    background as that  layer is done first in the output. Of course
%    that is only needed if the hook actually contains anything.
%    \begin{macrocode}
  \hook_if_empty:nF {shipout/firstpage}
       { \@@_add_background_box:n { \UseHook{shipout/firstpage} } }
%    \end{macrocode}
%    Once we are here we change the definition to do nothing next time
%    and we also change the command used to implement \cs{AtBeginDvi}
%    to become a warning and not  add further material to a hook that
%    is never used again.
%    \begin{macrocode}
  \cs_gset_eq:NN \@@_execute_firstpage_hook: \prg_do_nothing:
  \cs_gset:Npn \@@_add_firstpage_material:Nn ##1 ##2 {
    \PackageWarning{ltshipout}{
        First~ page~ is~ already~ shipped~ out,~ ignoring\MessageBreak
        \string##1 }
  }
}
%    \end{macrocode}
%  \end{macro}


%  \begin{macro}{\g_@@_lastpage_handled_bool}
%    A boolean to signal if we have already handled the
%    \hook{shipout/lastpage} hook.
%    \begin{macrocode}
\bool_new:N \g_@@_lastpage_handled_bool
%    \end{macrocode}
%  \end{macro}



%  \begin{macro}{\@@_add_firstpage_material:Nn}
%    This command adds material to the
%    \hook{shipout/firstpage} hook. It is used in
%    \cs{AtBeginDvi}, etc. The first argument is the
%    command through which is it called. Initially this is ignored but
%    once we are passed the first page it can be used to generate a
%    warning message mentioning the right user command.
%    \begin{macrocode}
\cs_new:Npn \@@_add_firstpage_material:Nn #1#2 {
   \AddToHook{shipout/firstpage}{#2}
}
%    \end{macrocode}
%  \end{macro}







%  \begin{macro}{\@@_get_box_size:N}
%    Store the box dimensions in dimen registers.
%    \fmi{This could/should perhaps be generalized to set height depth and
%    width given an arbitrary box.}
%    \begin{macrocode}
\cs_new:Npn \@@_get_box_size:N #1 {
  \dim_set:Nn \l_shipout_box_ht_dim { \box_ht:N #1 }
  \dim_set:Nn \l_shipout_box_dp_dim { \box_dp:N #1 }
  \dim_set:Nn \l_shipout_box_wd_dim { \box_wd:N #1 }
  \dim_set:Nn \l_shipout_box_ht_plus_dp_dim { \l_shipout_box_ht_dim +
                                         \l_shipout_box_dp_dim }
}
%    \end{macrocode}
%  \end{macro}

%  \begin{macro}{\l_shipout_box_ht_dim,
%                   \l_shipout_box_dp_dim,\l_shipout_box_wd_dim,
%                   \l_shipout_box_ht_plus_dp_dim}
%    And here are the variables set by \cs{@@_get_box_size:N}.
%    \begin{macrocode}
\dim_new:N \l_shipout_box_ht_dim
\dim_new:N \l_shipout_box_dp_dim
\dim_new:N \l_shipout_box_wd_dim
\dim_new:N \l_shipout_box_ht_plus_dp_dim
%    \end{macrocode}
%  \end{macro}





%  \begin{macro}{\g_@@_discard_bool}
%    Indicate whether or not the current page box should be discarded
%    \begin{macrocode}
\bool_new:N \g_@@_discard_bool
%    \end{macrocode}
%  \end{macro}



%  \begin{macro}{\l_@@_tmp_box,\l_@@_saved_badness_tl}
%    We need a box for the background and foreground material and a
%    token register to remember badness settings as we disable  them
%    during te buildup below.
%    \begin{macrocode}
\box_new:N \l_@@_tmp_box
\tl_new:N  \l_@@_saved_badness_tl
%    \end{macrocode}
%  \end{macro}


%  \begin{macro}{\@@_add_background_box:n}
%    In standard \LaTeX{} the shipout box is always a \cs{vbox} but
%    here we are allow for other usage as well, in case some package
%    has its own output routine.
%    \begin{macrocode}
\cs_new:Npn \@@_add_background_box:n #1
{ \@@_get_box_size:N \l_shipout_box
%    \end{macrocode}
%    But we start testing for a vertical box as that should be the
%    normal case.
%    \begin{macrocode}
  \box_if_vertical:NTF \l_shipout_box
      {
%    \end{macrocode}
%    Save current values of \cs{vfuzz} and \cs{vbadness} then change
%    them to allow box manipulations without warnings.
%    \begin{macrocode}
        \tl_set:Nx \l_@@_saved_badness_tl
           { \vfuzz=\the\vfuzz\relax
             \vbadness=\the\vbadness\relax }
        \vfuzz=\c_max_dim
        \vbadness=\c_max_int
%    \end{macrocode}
%    Then we reconstruct \cs{l_shipout_box} \ldots
%    \begin{macrocode}
        \vbox_set_to_ht:Nnn \l_shipout_box \l_shipout_box_ht_plus_dp_dim 
             {
%    \end{macrocode}
%    \ldots{} the material in \verb=#1= is placed into a horizontal
%    box with zero dimensions.
%    \begin{macrocode}
               \hbox_set:Nn \l_@@_tmp_box
                    { \l_@@_saved_badness_tl #1 }
               \box_set_wd:Nn \l_@@_tmp_box \c_zero_dim
               \box_set_ht:Nn \l_@@_tmp_box \c_zero_dim
               \box_set_dp:Nn \l_@@_tmp_box \c_zero_dim
%    \end{macrocode}
%    The we typeset that box followed by whatever was in
%    \cs{l_shipout_box} before (unpacked).
%    \begin{macrocode}
               \skip_zero:N \baselineskip
               \skip_zero:N \lineskip
               \skip_zero:N \lineskiplimit
               \box_use:N \l_@@_tmp_box
               \vbox_unpack:N \l_shipout_box
%    \end{macrocode}
%    The \cs{kern} ensures that the box has no depth which is
%    afterwards explicitly corrected.
%    \begin{macrocode}
               \kern \c_zero_dim
             }
        \box_set_ht:Nn \l_shipout_box \l_shipout_box_ht_dim
        \box_set_dp:Nn \l_shipout_box \l_shipout_box_dp_dim 
%    \end{macrocode}
%    \fmi{The whole boxing maneuver looks a bit like overkill to me, but for
%    the moment I leave.}               
%    \begin{macrocode}
        \l_@@_saved_badness_tl
      }
      {
%    \end{macrocode}
%    A horizontal box is handled in a similar way. The last case would
%    be a void box in which case we do nothing hence the missing
%    \texttt{F} branch. 
%    \begin{macrocode}
        \box_if_horizontal:NT \l_shipout_box
            {
              \tl_set:Nx \l_@@_saved_badness_tl
                 { \hfuzz=\the\hfuzz\relax
                   \hbadness=\the\hbadness\relax }
              \hfuzz=\c_max_dim
              \hbadness=\c_max_int
              \hbox_set_to_wd:Nnn \l_shipout_box \l_shipout_box_wd_dim
                   {
                     \hbox_set:Nn \l_@@_tmp_box
                          { \l_@@_saved_badness_tl #1 }
                     \box_set_wd:Nn \l_@@_tmp_box \c_zero_dim
                     \box_set_ht:Nn \l_@@_tmp_box \c_zero_dim
                     \box_set_dp:Nn \l_@@_tmp_box \c_zero_dim
                     \box_move_up:nn
                         \l_shipout_box_ht_dim 
                         { \box_use:N \l_@@_tmp_box }
                     \hbox_unpack:N \l_shipout_box
                   }
              \l_@@_saved_badness_tl
            }
      }
}
%    \end{macrocode}
%  \end{macro}




%  \begin{macro}{\@@_add_foreground_box:n}
%    Foreground boxes are done in the same way, only the order and
%    placement of boxes has to be done differently.
%    \begin{macrocode}
\cs_new:Npn \@@_add_foreground_box:n #1
{
  \box_if_vertical:NTF \l_shipout_box
    {
      \tl_set:Nx \l_@@_saved_badness_tl
         { \vfuzz=\the\vfuzz\relax
           \vbadness=\the\vbadness\relax }
      \vfuzz=\c_max_dim
      \vbadness=\c_max_int
      \vbox_set_to_ht:Nnn \l_shipout_box \l_shipout_box_ht_plus_dp_dim
           {
             \hbox_set:Nn \l_@@_tmp_box
                  { \l_@@_saved_badness_tl #1 }
             \box_set_wd:Nn \l_@@_tmp_box \c_zero_dim
             \box_set_ht:Nn \l_@@_tmp_box \c_zero_dim
             \box_set_dp:Nn \l_@@_tmp_box \c_zero_dim
             \skip_zero:N \baselineskip
             \skip_zero:N \lineskip
             \skip_zero:N \lineskiplimit
             \vbox_unpack:N \l_shipout_box
             \kern -\l_shipout_box_ht_plus_dp_dim
             \box_use:N \l_@@_tmp_box
             \kern  \l_shipout_box_ht_plus_dp_dim
           }
      \l_@@_saved_badness_tl
      \box_set_ht:Nn \l_shipout_box \l_shipout_box_ht_dim
      \box_set_dp:Nn \l_shipout_box \l_shipout_box_dp_dim 
    }
    {
      \box_if_horizontal:NT \l_shipout_box
        {
          \tl_set:Nx \l_@@_saved_badness_tl
            { \hfuzz=\the\hfuzz\relax
              \hbadness=\the\hbadness\relax }
          \hfuzz=\c_max_dim
          \hbadness=\c_max_int
          \hbox_set_to_wd:Nnn \l_shipout_box \l_shipout_box_wd_dim
               {
                 \hbox_unpack:N \l_shipout_box
                 \kern -\box_wd:N \l_shipout_box
                 \hbox_set:Nn \l_@@_tmp_box
                     { \l_@@_saved_badness_tl #1 }
                 \box_set_wd:Nn \l_@@_tmp_box \c_zero_dim
                 \box_set_ht:Nn \l_@@_tmp_box \c_zero_dim
                 \box_set_dp:Nn \l_@@_tmp_box \c_zero_dim
                 \box_move_up:nn { \box_ht:N \l_shipout_box }
                               { \box_use:N \l_@@_tmp_box }
                 \kern \box_wd:N \l_shipout_box
               }%
               \l_@@_saved_badness_tl
        }
    }
}
%    \end{macrocode}
%  \end{macro}




%  \begin{macro}{\c_@@_horigin_tl,\c_@@_vorigin_tl}
%    Two constants holding the offset of the top-left with respect to
%    the media box.
%
%    Setting the constants this way is courtesy of Bruno.
%    \begin{macrocode}
\tl_gput_right:Nn \@kernel@after@begindocument {
  \tl_const:Nx \c_@@_horigin_tl
     {
       \cs_if_exist_use:NTF \pdfvariable { horigin }
          { \cs_if_exist_use:NF \pdfhorigin { 1in } }
     }
  \tl_const:Nx \c_@@_vorigin_tl
     {
       \cs_if_exist_use:NTF \pdfvariable { vorigin }
          { \cs_if_exist_use:NF \pdfvorigin { 1in } }
     }
}     
%    \end{macrocode}
%  \end{macro}


%  \begin{macro}{\@@_picture_overlay:n}
%    Put the argument into a \texttt{picture} environment that doesn't take up
%    any size and uses \texttt{1pt} for \cs{unitlength}.
%    \fmi{Could perhaps be generalized as it might be useful elsewhere. For
%    not it is not.}
%    \begin{macrocode}
\cs_new:Npn \@@_picture_overlay:n #1 {
    \kern -\c_@@_horigin_tl \scan_stop:
    \vbox_to_zero:n {
      \kern -\c_@@_vorigin_tl \scan_stop:
      \unitlength 1pt \scan_stop:
      \hbox_set_to_wd:Nnn \l_@@_tmp_box \c_zero_dim { \ignorespaces #1 }
%      \box_set_wd:Nn \l_@@_tmp_box \c_zero_dim
      \box_set_ht:Nn \l_@@_tmp_box \c_zero_dim
      \box_set_dp:Nn \l_@@_tmp_box \c_zero_dim
      \box_use:N \l_@@_tmp_box
      \tex_vss:D
    }
}
%    \end{macrocode}
%  \end{macro}


%  \begin{macro}{\@@_add_background_picture:n}
%    Put a \texttt{picture} env in  the background of the shipout box
%    with its reference point in the top-left corner.
%    \begin{macrocode}
\cs_new:Npn \@@_add_background_picture:n #1 {
   \@@_add_background_box:n { \@@_picture_overlay:n {#1} }
}
%    \end{macrocode}
%  \end{macro}



%  \begin{macro}{\@@_add_foreground_picture:n}
%    
%    Put a \texttt{picture} env in  the foreground of the shipout box
%    with its reference point in the top-left corner.
%    \begin{macrocode}
\cs_new:Npn \@@_add_foreground_picture:n #1 {
   \@@_add_foreground_box:n { \@@_picture_overlay:n {#1} }
}
%    \end{macrocode}
%  \end{macro}


%  \begin{macro}{\shipout_discard:}
%    Request that the next shipout box should be discarded. At the
%    moment this is just setting a boolean, but we may want to augment
%    this behavior that the position of the call is taken into account
%    (in case \LaTeX{} looks ahead and is not using the position for
%    on the next page).
%    \begin{macrocode}
\cs_new:Npn \shipout_discard: {
  \bool_gset_true:N \g_@@_discard_bool
}
%    \end{macrocode}
%  \end{macro}




% \subsection{Handling the end of job hook}

%    At the moment this is partly solved by using the existing hooks.
%    But rather than putting the code into these hooks it should be
%    moved to the right place directly as we shouldn't prefill hooks
%    with material unless it needs to interact with other code. 
%

%    
%  \begin{macro}{\g_shipout_readonly_int,\ReadonlyShipoutCounter}
%    We count every shipout activity that makes a page (but not those
%    that are discarded) in order to know how many pages got produced.
%    \begin{macrocode}
\int_new:N \g_shipout_readonly_int
%    \end{macrocode}
%    For \LaTeXe{} it is available as a command (i.e., a \TeX{}
%    counter only.
%    \begin{macrocode}
\cs_new_eq:NN \ReadonlyShipoutCounter  \g_shipout_readonly_int
%    \end{macrocode}
%  \end{macro}

%  \begin{macro}{\g_shipout_totalpages_int,\c@totalpages}
%    We count every shipout attempt (even those that are discarded) in
%    tis counter. It is not used in the code but may get used in user
%    code.
%    \begin{macrocode}
\int_new:N \g_shipout_totalpages_int
%    \end{macrocode}
%    For \LaTeXe{} this is offered as a \LaTeX{} counter so can be
%    easily typeset inside the output routine to display things like
%    \enquote{\cs{thepage}\texttt{/}\cs{thetotalpages}}, etc.
%    \begin{macrocode}
\cs_new_eq:NN \c@totalpages \g_shipout_totalpages_int
\cs_new:Npn \thetotalpages { \arabic{totalpages} }
%    \end{macrocode}
%  \end{macro}
%



%  \begin{macro}{\@abspage@last}
%    In \cs{@abspage@last} record the number of pages from the last
%    run. This is written to the \texttt{.aux} and this way made
%    available to the next run. In case there is no \texttt{.aux} file
%    or the statement is missing from it we initialize it with the
%    largest possible number in \TeX{}. We use this a s the default
%    because then we are inserting the \hook{shipout/lastpage} on
%    the last page (or after the last page but not on page 1 for a
%    multipage document.
%    \begin{macrocode}
\xdef\@abspage@last{\number\maxdimen}
%    \end{macrocode}
%  \end{macro}


% \begin{macro}{\enddocument}
%
%    Instead of using the hooks \hook{enddocument} and
%    \hook{enddocument/afterlastpage} we add this code to private
%    kernel hooks to be 100\% when it is executed and to avoid
%    cluttering the hooks with data that is always there.
%
%    Inside \cs{enddocument} there is a \cs{clearpage}. Just before
%    that we execute this code here. There is a good change that we
%    are on the last page. Therefore, if we don't know the value from
%    the last run, we assume that the current page is the right
%    one. So we set \cs{@abspage@last} and as a result the next
%    shipout will run the \hook{shipout/lastpage} code. Of course,
%    if there are floats that still need a placement this guess will
%    be wrong but then rerunning the document will give us the correct
%    value next time around.
%
% \begin{macro}{\@kernel@after@enddocument}
%    \begin{macrocode}
\g@addto@macro \@kernel@after@enddocument {
  \int_compare:nNnT \@abspage@last = \maxdimen
    {
%    \end{macrocode}
%    We use \LaTeXe{} coding as \cs{@abspage@last} is not an L3 name.
%    \begin{macrocode}
      \xdef\@abspage@last{ \int_eval:n {\g_shipout_readonly_int + 1} }
    }
}
%    \end{macrocode}
% \end{macro}
%
% \begin{macro}{\@kernel@after@enddocument@afterlastpage}
%    Once the \cs{clearpage} has done its work inside \cs{enddocument}
%    we know for sure how many pages this document has, so we record
%    that in the \texttt{.aux} file for the next run.
%
%    \begin{macrocode}
\g@addto@macro \@kernel@after@enddocument@afterlastpage {
%    \end{macrocode}
%    There is one special case: If no output is produced then there is
%    no point in a) recording the number as 0 will never match the
%    page number of a real page and b) adding an extra page to ran the
%    \hook{shipout/lastpage} is pointless as well (as it would
%    remain forever). So we test for this and run the code only if
%    there have been pages.
%    \begin{macrocode}
  \int_compare:nNnF \g_shipout_readonly_int = 0
    {
%    \end{macrocode}
%     This ends up in the \texttt{.aux} so we use \LaTeXe{} names here.
%     \fmi{this needs an interface for \cs{nofiles} in expl3, doesn't at the moment!}
%    \begin{macrocode}
     \if@filesw
        \iow_now:Nx \@auxout {
          \gdef\string\@abspage@last {\int_use:N \g_shipout_readonly_int}}
     \fi
%    \end{macrocode}
%    But we may have guessed wrongly earlier and we still have to run the
%    \hook{shipout/lastpage} even though there is no page to place
%    it into. If that is the case we make a trivial extra page and put
%    it there. This temporary page will then vanish again on the next
%    run but helps to keep pdf viewers happy.
%    \begin{macrocode}
      \bool_if:NF \g_@@_lastpage_handled_bool
        { \tex_shipout:D\vbox to\textheight{
            \hbox{\UseHook{shipout/lastpage} \@kernel@after@shipout@lastpage }
%    \end{macrocode}
%    This extra page could be totally empty except for the hook
%    content, but to help the user understanding why it is there we
%    put some text into it.
%    \begin{macrocode}
            \@@_excuse_extra_page:
            \null
          }
%    \end{macrocode}
%    At this point we should also signal to \LaTeX{} that a rerun is
%    necessary so that an appropriate message can be shown on the
%    terminal.
%
%    \fmi{integrate warning ``Rerun \ldots'' }
%    \begin{macrocode}
      }
    }
}
%    \end{macrocode}
% \end{macro}
% \end{macro}



%  \begin{macro}{\@@_excuse_extra_page:}
%    Say mea culpa \ldots
%    \begin{macrocode}
\cs_new:Npn \@@_excuse_extra_page: {
  \vfil
  \begin{center}
    \bfseries Temporary~ page! 
  \end{center}
    \LaTeX{}~ was~ unable~ to~ guess~ the~ total~ number~ of~ pages~
    correctly.~ ~ As~ there~ was~ some~ unprocessed~ data~ that~
    should~ have~ been~ added~ to~ the~ final~ page~ this~ extra~
    page~ has~ been~ added~ to~ receive~ it.
    \par
    If~ you~ rerun~ the~ document~ (without~ altering~ it)~ this~
    surplus~ page~ will~ go~ away,~ because~ \LaTeX{}~ now~ knows~
    how~ many~ pages~ to~ expect~ for~ this~ document.
  \vfil
}
%    \end{macrocode}
%  \end{macro}




% \section{Legacy \LaTeXe{} interfaces}



%  \begin{macro}{\DiscardShipoutBox}
%    Request that the next shipout box is to be discarded.
%    \begin{macrocode}
\cs_new_eq:NN \DiscardShipoutBox \shipout_discard:
%    \end{macrocode}
%  \end{macro}


%  \begin{macro}{\AtBeginDvi}
%    
%    \begin{macrocode}
\renewcommand \AtBeginDvi {\@@_add_firstpage_material:Nn \AtBeginDvi}
%    \end{macrocode}
%  \end{macro}


%  \begin{macro}{\DebugShipoutOn,\DebugShipoutOff}
%    
%    \begin{macrocode}
\cs_new_eq:NN \DebugShipoutOn  \shipout_debug_on:
\cs_new_eq:NN \DebugShipoutOff \shipout_debug_off:
%    \end{macrocode}
%  \end{macro}





% \section{Package emulation for compatibility}


% \subsection{Package \pkg{atbegshi} emulation}

%  \begin{macro}{\AtBeginShipoutBox}
%    \begin{macrocode}
\cs_new_eq:NN \AtBeginShipoutBox \ShipoutBox
%    \end{macrocode}
%  \end{macro}



%  \begin{macro}{\AtBeginShipoutInit}
%    Compatibility only, we aren't delaying \ldots
%    \begin{macrocode}
\cs_set_eq:NN\AtBeginShipoutInit\@empty 
%    \end{macrocode}
%  \end{macro}



%  \begin{macro}{\AtBeginShipout,\AtBeginShipoutNext}
%    Filling hooks
%    \begin{macrocode}
\newcommand\AtBeginShipout     {\AddToHook{shipout/before}}
\newcommand\AtBeginShipoutNext {\AddToHookNext{shipout/before}}
%    \end{macrocode}
%  \end{macro}



%  \begin{macro}{\AtBeginShipoutFirst}
%    Slightly more complex as we need to know the name of the command under which the
%    \hook{shipout/firstpage} hook is filled.
%    \begin{macrocode}
\newcommand\AtBeginShipoutFirst{\@@_add_firstpage_material:Nn \AtBeginShipoutFirst}
%    \end{macrocode}
%  \end{macro}



% This is somewhat different from the original where
% \cs{ShipoutBoxHeight} etc.\ only holds the \verb=\the\ht<box>= value. This
% may has some implications in some use cases and if that is a problem
%    then it might need changing.
%    
%    \begin{macrocode}
\cs_new:Npn \ShipoutBoxHeight { \dim_use:N \l_shipout_box_ht_dim }
\cs_new:Npn \ShipoutBoxDepth  { \dim_use:N \l_shipout_box_dp_dim }
\cs_new:Npn \ShipoutBoxWidth  { \dim_use:N \l_shipout_box_wd_dim }
%    \end{macrocode}


%  \begin{macro}{\AtBeginShipoutDiscard}
%    Just a different name.
%    \begin{macrocode}
\cs_new_eq:NN \AtBeginShipoutDiscard \DiscardShipoutBox
%    \end{macrocode}
%  \end{macro}


%  \begin{macro}{\AtBeginShipoutAddToBox,\AtBeginShipoutAddToBoxForeground,
%                \AtBeginShipoutUpperLeft,\AtBeginShipoutUpperLeftForeground}
%    We don't expose them.
%    \begin{macrocode}
\cs_new_eq:NN \AtBeginShipoutAddToBox           \@@_add_background_box:n
\cs_new_eq:NN \AtBeginShipoutAddToBoxForeground \@@_add_foreground_box:n 
%    \end{macrocode}
%    
%    \begin{macrocode}
\cs_new_eq:NN\AtBeginShipoutUpperLeft           \@@_add_background_picture:n
\cs_new_eq:NN\AtBeginShipoutUpperLeftForeground \@@_add_foreground_picture:n
%    \end{macrocode}
%  \end{macro}

%
%
%    We prevent the package \pkg{atbegshi} from loading:
%    \begin{macrocode}
\expandafter\cs_set_eq:NN\csname ver@atbegshi.sty\endcsname\fmtversion
%    \end{macrocode}
%    \pkg{hyperref} code (and \pkg{ltxcmds}) doesn't understand 2020-10-01
%       and thinks it is before 1994, so for now \ldots
%    \begin{macrocode}
\@namedef {ver@atbegshi.sty}{2020/10/01}
%    \end{macrocode}



% \subsection{Package \pkg{everyshi} emulation}
%
%  \begin{macro}{\EveryShipout,\AtNextShipout}
%    This package has only two public commands to simulating it is easy:
%    \begin{macrocode}
\cs_new_eq:NN\EveryShipout\AtBeginShipout
\cs_new_eq:NN\AtNextShipout\AtBeginShipoutNext
%    \end{macrocode}
%  \end{macro}
%
%    \begin{macrocode}
\expandafter\cs_set_eq:NN\csname ver@everyshi.sty\endcsname\fmtversion
\@namedef {ver@everyshi.sty}{2020/10/01}
%    \end{macrocode}



% \subsection{Package \pkg{atenddvi} emulation}
%
%  \begin{macro}{\AtEndDvi}
%    This package has only one public command to simulating it is easy:
%    \begin{macrocode}
\cs_new:Npn \AtEndDvi {\AddToHook{shipout/lastpage}}
%    \end{macrocode}
%  \end{macro}
%
%
%    \begin{macrocode}
\expandafter\cs_set_eq:NN\csname ver@atenddvi.sty\endcsname\fmtversion
\@namedef {ver@atenddvi.sty}{2020/10/01}
%    \end{macrocode}




%    \begin{macrocode}
\ExplSyntaxOff
%</2ekernel>
%    \end{macrocode}

%    \Finale
%


%%%%%%%%%%%%%%%%%%%%%%%%%%%%%%%%%%%%%%%%%%%%%  
\endinput
%%%%%%%%%%%%%%%%%%%%%%%%%%%%%%%%%%%%%%%%%%%%%  



                     % \iffalse meta-comment
%
%% File: ltfilehook.dtx (C) Copyright 2020 Frank Mittelbach, LaTeX Team
%
% It may be distributed and/or modified under the conditions of the
% LaTeX Project Public License (LPPL), either version 1.3c of this
% license or (at your option) any later version.  The latest version
% of this license is in the file
%
%    https://www.latex-project.org/lppl.txt
%
%
% The development version of the bundle can be found below
%
%    https://github.com/FrankMittelbach/...
%
% for those people who are interested or want to report an issue.
%
%    \begin{macrocode}
\providecommand\ltfilehookversion{v0.6g}
\providecommand\ltfilehookdate{2020/07/16}
%    \end{macrocode}
%
%<*driver>

\RequirePackage{lthooks}        % needed for structedlog if running old format

\documentclass{l3doc}

% bug fix fo l3doc.cls
\ExplSyntaxOn
\cs_set_protected:Npn \__codedoc_macro_typeset_one:nN #1#2
  {
    \vbox_set:Nn \l__codedoc_macro_box
      {
        \vbox_unpack_drop:N \l__codedoc_macro_box
        \hbox { \llap { \__codedoc_print_macroname:nN {#1} #2
            \MacroFont       % <----- without it the \ is in lmr10 if a link is made
            \      
        } }
      }
    \int_incr:N \l__codedoc_macro_int
  }
\ExplSyntaxOff

\usepackage{structuredlog}  % for demonstration

\EnableCrossrefs
\CodelineIndex
\begin{document}
  \DocInput{ltfilehook.dtx}
\end{document}
%</driver>
%
% \fi
%
%
% \long\def\fmi#1{\begin{quote}\itshape Todo: #1\end{quote}}
%
% \let\hook\texttt
%
% \title{The \texttt{ltfilehook} package\thanks{This package has version
%    \ltfilehookversion\ dated \ltfilehookdate, \copyright\ \LaTeX\
%    Project.}}
%
% \author{Frank Mittelbach}
%
% \maketitle
%
%
%
% \tableofcontents
%
% \section{Introduction}
%
%
%
% \subsection{Provided hooks}
%
%    The code offers a number of hooks into which packages (or the
%    user) can add code to support different use cases.
%    Many hooks are offered as pairs (i.e., the second hook is
%    reversed. Also important to know is that these pairs are
%    properly nested with respect to other pairs of hooks.
%
%    There are hooks that are executed for all files of a certain type
%    (if they contain code), e.g., for all \enquote{include files} or
%    all \enquote{packages},
%    and there are also hooks that are specific to a single file,
%    e.g., do something after the package \texttt{foo.sty} has been
%    loaded.
%
%
% \subsection{General hooks for file reading}
% \label{sec:general-file-hooks}
%
%    There are four hooks that are called for each file that is read
%    using document-level commands such as \cs{input}, \cs{include},
%    \cs{usepackage}, etc.  They are not called for files read using
%    internal low-level methods, such as \cs{@input} or \cs{openin}.
%
%  \begin{variable}{file/before,file/before/...,
%                   file/after/...,file/after,
%                   }
%    These are:
%    \begin{description}
%    \item[\texttt{file/before}, \texttt{file/before/\meta{file-name}}]
%
%       These hooks are executed in that order just before the file is
%       loaded for reading. The code of the first hook is used is used
%       with every file, while the second is executed only for the
%       file with matching \meta{file-name} allowing you to specify
%       code that only applies to one file.
%
%    \item[\texttt{file/after/\meta{file-name}}, \texttt{file/after}]
%
%       These hooks are after the file with name \meta{file-name} has
%       been fully consumed. The order is swapped (the specific one
%       comes first) so that the \texttt{before} and \texttt{after}
%       hooks nest properly, which is important if any of them involve
%       grouping (e.g., contain environments, for example).
%       furthermore both hooks are reversed hooks to support correct
%       nesting of different packages add code to both
%       \texttt{/before} and \texttt{/after} hooks.
%
%    \end{description}
%  \end{variable}
%
%
%     So the overall sequence of hook processing for any file read
%     through the user interface commands of \LaTeX{} is:
%
%  \begin{tabbing}
%    mm\=mm\=mm\=mm\=\kill
%    \>\cs{UseHook}\marg{\hook{file/before}} \\
%    \>\cs{UseHook}\marg{\hook{file/before/\meta{file name}}} \\
%    \>\> \meta{file contents} \\
%    \>\cs{UseHook}\marg{\hook{file/after/\meta{file name}}} \\
%    \>\cs{UseHook}\marg{\hook{file/after}}
%  \end{tabbing}
%
%     \fmi{With a higher-level interface that doesn't matter, but it
%    is a bit weird, if you use \cs{AddToHook} or \cs{hook_gput:nnn}
%    directly, so I guess that has to be done differently!}
%
%    The file hooks only refer to the file by its name and extension,
%    so the \meta{file name} should be the file name as it is on the
%    filesystem with extension (if any) and without paths.  Different
%    from \cs{input} and similar commands, for hooks the \texttt{.tex}
%    extension is not assumed, so \texttt{.tex} files must also be given
%    with their extension.
%    Files within subfolders should also be addressed by their name and
%    extension only.
%
%    Extensionless files also work, and should then be given without
%    extension.  Note however that \TeX{} prioritizes \texttt{.tex}
%    files, so if two files \texttt{foo} and \texttt{foo.tex} exist in
%    the search path, only the latter will be seen.
%
%    When a file is input, the \meta{file name} is available in
%    \cs{CurrentFile}, which is then used when accessing the
%    \hook{file/before/\meta{file name}} and
%    \hook{file/after/\meta{file name}}.
%
%  \begin{variable}{\CurrentFile}
%    The name of the file about to be read (or just finished) is
%    available to the hooks through \cs{CurrentFile} (there is no
%    \texttt{expl3} name for it for now).  The file is always provided
%    with its extension, i.e., how it appears on your hard drive, but
%    without any specified path to it. For example,
%    \verb=\input{sample}= and \verb=\input{app/sample.tex}= would
%    both have \cs{CurrentFile} being \texttt{sample.tex}.
%  \end{variable}
%
%  \begin{variable}{\CurrentFilePath}
%    The path to the current file (complement to \cs{CurrentFile}) is
%    available in \cs{CurrentFilePath} if needed.
%    The paths returned in \cs{CurrentFilePath} are only user paths,
%    given through \cs{input@path} (or \pkg{expl3}'s equivalent
%    \cs{l_file_search_path_seq}) or by directly typing in the path
%    in the \cs{input} command or equivalent.  Files located by
%    \texttt{kpsewhich} get the path added internally by the \TeX{}
%    implementation, so at the macro level it looks as if the file were
%    in the current folder, so the path in \cs{CurrentFilePath} is empty
%    in these cases (package and class files, mostly).
%  \end{variable}
%
% \subsection{Hooks for package and class files}
%
%    Commands to load package and class files (e.g., \cs{usepackage},
%    \cs{RequirePackage}, \cs{LoadPackageWithOptions}, etc.) offer the
%    hooks from section~\ref{sec:general-file-hooks} when they are
%    used to load a  package or class file, e.g.,
%    \texttt{file/after/array.sty} would be called after the
%    \pkg{array} package got loaded. But as packages and classes form as special group
%    of files, there are some additional hooks available that only
%    apply when a package or class is loaded.
%
%
%  \begin{variable}{
%                   package/before,package/after,
%                   package/before/...,package/after/...,
%                   class/before,class/after,
%                   class/before/...,class/after/...,
%                   }
%    These are:
%    \begin{description}
%    \item[\texttt{package/before}, \texttt{package/after}]
%
%      These hooks are called for each package being loaded.
%
%    \item[\texttt{package/before/\meta{name}},
%    \texttt{package/after/\meta{name}}]
%
%      These hooks are additionally called if the package name is
%      \meta{name} (without extension).
%
%    \item[\texttt{class/before}, \texttt{class/after}]
%
%      These hooks are called for each class being loaded.
%
%    \item[\texttt{class/before/\meta{name}}, \texttt{class/after/\meta{name}}]
%
%      These hooks are additionally called if the class name is
%      \meta{name} (without extension).
%
%    \end{description}
%  \end{variable}
%     All \hook{/after} hooks are implemented as reversed hooks.
%
%     \noindent The overall sequence of execution for \cs{usepackage}
%     and friends is therefore:
%  \begin{tabbing}
%    mm\=mm\=mm\=mm\=\kill
%    \>\cs{UseHook}\marg{\hook{package/before}} \\
%    \>\cs{UseHook}\marg{\hook{package/before/\meta{package name}}} \\[5pt]
%    \>\>\cs{UseHook}\marg{\hook{file/before}} \\
%    \>\>\cs{UseHook}\marg{\hook{file/before/\meta{package name}.sty}} \\
%    \>\>\> \meta{package contents} \\
%    \>\>\cs{UseHook}\marg{\hook{file/after/\meta{package name}.sty}} \\
%    \>\>\cs{UseHook}\marg{\hook{file/after}} \\[5pt]
%    \>\>\emph{code from \cs{AtEndOfPackage} if
%                used inside the package} \\[5pt]
%    \>\cs{UseHook}\marg{\hook{package/after/\meta{package name}}} \\
%    \>\cs{UseHook}\marg{\hook{package/after}} 
%  \end{tabbing}
%    and similar for class file loading, except that \texttt{package/}
%    is replaced by \texttt{class/} and \cs{AtEndOfPackage} by
%    \cs{AtEndOfClass}.
%
%    If a package or class is not loaded (or it was loaded before the
%    hooks were set) none of the hooks are executed!
%
% \subsection{Hooks for \cs{include} files}
%
%    To manage \cs{include} files, \LaTeX{} issues a \cs{clearpage}
%    before and after loading such a file. Depending on the use case
%    one may want to execute code before or after these
%    \cs{clearpage}s especially for the one that is issued at the end.
%
%    Executing code before the final \cs{clearpage}, means that the
%    code is processed while the last page of the included material is
%    still under construction.  Executing code after it means that all
%    floats from inside the include file are placed (which
%    might have added further pages) and the final page has finished.
%
%    Because of these different scenarios we offer hooks in three
%    places.\footnote{If you want to execute code before the first
%     \cs{clearpage} there is no need to use a hook---you can write it
%     directly in front of the \cs{include}.}
%    None of the hooks are executed when an \cs{include} file is
%    bypassed because of an \cs{includeonly} declaration. They are,
%    however, all executed if \LaTeX{} makes an attempt to load the
%    \cs{include} file (even if it doesn't exist and all that happens
%    is \enquote{\texttt{No file \meta{filename}.tex}}).
%
%
%  \begin{variable}{include/before,include/before/...,
%                   include/end,include/end/...,
%                   include/after,include/after/...,
%                  }
%    These are:
%    \begin{description}
%
%    \item[\texttt{include/before}, \texttt{include/before/\meta{name}}]
%
%      These hooks are executed one after another after the initial
%      \cs{clearpage} and after \texttt{.aux} file is changed to use
%      \texttt{\meta{name}.aux}, but before the
%      \texttt{\meta{name}.tex} file is loaded. In other words they are executed
%      at the very beginning of the first page of the \cs{include}
%      file.
%
%
%    \item[\texttt{include/end/\meta{name}}, \texttt{include/end}]
%
%      These hooks are executed (in that order) after \LaTeX{} has
%      stopped reading from the \cs{include} file, but before it has
%      issued a \cs{clearpage} to output any deferred floats.
%
%
%    \item[\texttt{include/after/\meta{name}}, \texttt{include/after}]
%
%      These hooks are executed (in that order) after \LaTeX{} has
%      issued the \cs{clearpage} but before is has switched back
%      writing to the main \texttt{.aux} file. Thus technically we are
%      still inside the \cs{include} and if the hooks generate any
%      further typeset material including anything that writes to the
%      \texttt{.aux} file, then it would be considered part of the
%      included material and bypassed if it is not loaded because of
%      some \cs{includeonly} statement.\footnotemark
%
%    \end{description}
%  \end{variable}\footnotetext{For that reason
%      another \cs{clearpage} is executed after these hooks which
%      normally does nothing, but starts a new page if further material
%      got added this way.}
%
% \subsection{High-level interfaces for \LaTeX{}}
%
%    We do not provide any high-level \LaTeX{} commands (like
%    \pkg{filehook} or \pkg{scrlfile} do) but think that for package
%    writers the commands from for hook management are sufficient.
%
% \subsection{A sample package for structuring the log output}
%
%    As an application we provide the package \pkg{structuredlog} that
%    adds lines to the \texttt{.log} when a file is opened and closed
%    for reading keeping track of nesting level es well.
%    For example, for the current document it adds the lines
%\begin{verbatim}
%    = (LEVEL 1 START) t1lmr.fd
%    = (LEVEL 1 STOP) t1lmr.fd
%    = (LEVEL 1 START) supp-pdf.mkii
%    = (LEVEL 1 STOP) supp-pdf.mkii
%    = (LEVEL 1 START) nameref.sty
%    == (LEVEL 2 START) refcount.sty
%    == (LEVEL 2 STOP) refcount.sty
%    == (LEVEL 2 START) gettitlestring.sty
%    == (LEVEL 2 STOP) gettitlestring.sty
%    = (LEVEL 1 STOP) nameref.sty
%    = (LEVEL 1 START) ltfilehook-doc.out
%    = (LEVEL 1 STOP) ltfilehook-doc.out
%    = (LEVEL 1 START) ltfilehook-doc.out
%    = (LEVEL 1 STOP) ltfilehook-doc.out
%    = (LEVEL 1 START) ltfilehook-doc.hd
%    = (LEVEL 1 STOP) ltfilehook-doc.hd
%    = (LEVEL 1 START) ltfilehook.dtx
%    == (LEVEL 2 START) ot1lmr.fd
%    == (LEVEL 2 STOP) ot1lmr.fd
%    == (LEVEL 2 START) omllmm.fd
%    == (LEVEL 2 STOP) omllmm.fd
%    == (LEVEL 2 START) omslmsy.fd
%    == (LEVEL 2 STOP) omslmsy.fd
%    == (LEVEL 2 START) omxlmex.fd
%    == (LEVEL 2 STOP) omxlmex.fd
%    == (LEVEL 2 START) umsa.fd
%    == (LEVEL 2 STOP) umsa.fd
%    == (LEVEL 2 START) umsb.fd
%    == (LEVEL 2 STOP) umsb.fd
%    == (LEVEL 2 START) ts1lmr.fd
%    == (LEVEL 2 STOP) ts1lmr.fd
%    == (LEVEL 2 START) t1lmss.fd
%    == (LEVEL 2 STOP) t1lmss.fd
%    = (LEVEL 1 STOP) ltfilehook.dtx
%\end{verbatim}
%    Thus if you inspect an issue in the \texttt{.log} it is easy to
%    figure out in which file it occurred, simply by searching back for
%    \texttt{LEVEL} and if it is a \texttt{STOP} then remove 1 from
%    the level value and search further for \texttt{LEVEL} with that value
%    which should then be the \texttt{START}  level of the file you are in.
%
% \StopEventually{\setlength\IndexMin{200pt}  \PrintIndex  }
%
%
% \section{The Implementation}
%    \begin{macrocode}
%<*2ekernel>
%    \end{macrocode}
%
% \subsection{\pkg{expl3} helpers}
%
%    \begin{macrocode}
%<@@=filehook>
%    \end{macrocode}
%
% \begin{macro}{\CurrentFile,\CurrentFilePath}
%   User-level macros that hold the current file name and file path.
%   These are used internally as well because the code takes care to
%   protect against a possible redefinition of these macros in the loaded
%   file (it's necessary anyway to make hooks work with nested
%   \cs{input}).
%    \begin{macrocode}
\def\CurrentFile{}
\def\CurrentFilePath{}
%    \end{macrocode}
% \end{macro}
%
% \begin{macro}{\l_@@_internal_tl}
% \begin{macro}{\@filehook@set@curr@file,\@@_normalise_file_name:n}
%   When inputting a file, \cs{@filehook@set@curr@file} does a file
%   lookup (in \cs{input@path} and \cs{l_file_search_path_seq}) and
%   returns the actual file name (\meta{base} plus \meta{ext}) in
%   \cs{CurrentFile}.  Only the base and extension are returned,
%   regardless of the input (both \texttt{path/to/file.tex} and
%   \texttt{file.tex} end up as \texttt{file.tex} in \cs{CurrentFile}).
%   The path is returned in \cs{CurrentFilePath}, in case it's needed.
%   \cs{CurrentFile} is then used to run the file hooks with
%   \texttt{file/before/\cs{CurrentFile}} and
%   \texttt{file/after/\cs{CurrentFile}}.
%    \begin{macrocode}
\ExplSyntaxOn
\tl_new:N \l_@@_internal_tl
\cs_new_protected:Npn \@filehook@set@curr@file #1
  { \exp_args:NV \@@_normalise_file_name:n #1 }
\cs_new_protected:Npn \@@_normalise_file_name:n #1
  {
    \file_if_exist:nTF {#1}
      {
        \exp_args:Nx \file_parse_full_name:nNNN
          { \file_full_name:n {#1} }
      }
      { \file_parse_full_name:nNNN {#1} }
      \CurrentFilePath \CurrentFile \l_@@_internal_tl
    \tl_set:Nx \CurrentFile { \CurrentFile \l_@@_internal_tl }
  }
%    \end{macrocode}
% \end{macro}
% \end{macro}
%
% \begin{macro}{\g_@@_input_file_seq}
% \begin{macro}{\@filehook@file@push,\@filehook@file@pop}
% \begin{macro}{\@@_file_pop_assign:nn}
%   Yet another stack, to keep track of \cs{CurrentFile} and
%   \cs{CurrentFilePath} with nested \cs{input}s.  At the beginning of
%   \cs{InputIfFileExists}, the current value of \cs{CurrentFilePath}
%   and \cs{CurrentFile} is pushed to \cs{g_@@_input_file_seq}, and
%   at the end, it is popped and the value reassigned.
%   \cs{IfFileExists} does \cs{set@curr@file} internally, which changes
%   \cs{CurrentFile}, so \cs{@filehook@file@push} has to be executed
%   before \cs{IfFileExists}.
%    \begin{macrocode}
\seq_new:N \g_@@_input_file_seq
\cs_new_protected:Npn \@filehook@file@push
  {
    \seq_gpush:Nx \g_@@_input_file_seq
      { { \CurrentFilePath } { \CurrentFile } }
  }
\cs_new_protected:Npn \@filehook@file@pop
  {
    \seq_gpop:NNTF \g_@@_input_file_seq \l_@@_internal_tl
      { \exp_after:wN \@@_file_pop_assign:nn \l_@@_internal_tl }
      { \ERROR_should_not_happen }
  }
\cs_new_protected:Npn \@@_file_pop_assign:nn #1 #2
  {
    \tl_set:Nn \CurrentFilePath {#1}
    \tl_set:Nn \CurrentFile {#2}
  }
\ExplSyntaxOff
%    \end{macrocode}
% \end{macro}
% \end{macro}
% \end{macro}
%
%
% \subsection{Declaring the file-related hooks}
%
%  All hooks starting with \texttt{file/} \texttt{include/},
%  \texttt{class/} or \texttt{package/} are generic and will be
%  allocated if code is added to them. Thus there is no need to
%  explicitly declare any hook in the code below.
%
%  Furthermore, those named \texttt{.../after} or \texttt{.../end} are
%  automatically declared as reversed hooks if filled with code, so this
%  is also automatically taken care of.
%
% \subsection{Patching \LaTeX{} commands (need proper integration later)}
%
%    Most of what we have to do is adding \cs{UseHook} into several
%  \LaTeXe{} core commands, which is done for now by patching them.
%    \begin{macrocode}
%<@@=>
%    \end{macrocode}
%
% \begin{macro}{\InputIfFileExists}
%    \cs{InputIfFileExists} loads any file if it is available so we
%    have to add the hooks \texttt{file/before} and
%    \texttt{file/after} in the right places. If the file doesn't
%    exist no hooks should be executed.
%    \begin{macrocode}
\let\InputIfFileExists\@undefined
\DeclareRobustCommand \InputIfFileExists[3]{%
  \@filehook@file@push
  \IfFileExists{#1}%
    {%
%    \end{macrocode}
%    If the file exists then \cs{@curr@file} holds its name. But we
%    can't rely on that still being true after the file has been
%    processed. Thus for using the name in the file hooks we need to
%    preserve the name and then restored it for the
%    \texttt{file/after/...} hook.
%
%    The hook always refers to the \emph{actual} file that will be
%    operated on, regardless of how the user had it written in the
%    document.  \pkg{expl3}'s \cs{file_full_name:n} normalizes the file
%    name (to factor out differences in the |.tex| extension), and
%    then does a file lookup to take into account a possible path from
%    \cs{l_file_search_path_seq} and \cs{input@path}.  However only
%    the file name and extension are returned so that file hooks can
%    refer to the file by their name only.  The path to the file is
%    returned in \cs{CurrentFilePath}.
%    \begin{macrocode}
      \edef\reserved@a{\@filef@und
        \def\noexpand\CurrentFile{\CurrentFile}%
        \def\noexpand\CurrentFilePath{\CurrentFilePath}%
      }%
      \expandafter\@swaptwoargs\expandafter
        {\reserved@a}%
        {%
          #2%
          \@addtofilelist{#1}%
          \UseHook{file/before}%
%    \end{macrocode}
%    The current file name is available in \cs{@curr@file} so we use
%    that in the specific hook.
%    \begin{macrocode}
          \UseHook{file/before/\CurrentFile}%
          \@@input
        }%
%    \end{macrocode}
%    And it is restored here so we can use it once more.
%    \begin{macrocode}
      \UseHook{file/after/\CurrentFile}%
      \UseHook{file/after}%
      \@filehook@file@pop
    }%
    {\@filehook@file@pop #3}%
}
%    \end{macrocode}
% \end{macro}
%
% \begin{macro}{\set@curr@file}
%   Now we just hook into \cs{set@curr@file} to add
%   \cs{@filehook@set@curr@file} at the end, after \cs{@curr@file} is
%   set.
% \changes{0.69}{2020-70-16}}{When the argument is empty return an
%    empty macro instead of \texttt{csnameendcsname} (gh.121)}
%    \begin{macrocode}
\def\set@curr@file#1{%
  \begingroup
    \def\reserved@a{#1}%
    \ifx\reserved@a\@empty
      \xdef\@curr@file{}%
    \else
      \escapechar\m@ne
      \xdef\@curr@file{%
        \expandafter\expandafter\expandafter\unquote@name
        \expandafter\expandafter\expandafter{%
        \expandafter\string
        \csname\@firstofone#1\@empty\endcsname}}%
    \fi
  \endgroup
  \@filehook@set@curr@file{\@curr@file}%
}
%    \end{macrocode}
%  \end{macro}
%
%  \begin{macro}{\load@onefilewithoptions}
%    This macro is used when loading packages or classes.
%    \begin{macrocode}
\def\load@onefilewithoptions#1[#2][#3]#4{%
  \@pushfilename
  \xdef\@currname{#1}%
  \global\let\@currext#4%
  \let\CurrentOption\@empty
  \@reset@ptions
  \makeatletter
  \def\reserved@a{%
    \@ifl@aded\@currext{#1}%
      {\@if@ptions\@currext{#1}{#2}{}%
        {\@latex@error
            {Option clash for \@cls@pkg\space #1}%
            {The package #1 has already been loaded
             with options:\MessageBreak
             \space\space[\@ptionlist{#1.\@currext}]\MessageBreak
             There has now been an attempt to load it
              with options\MessageBreak
             \space\space[#2]\MessageBreak
             Adding the global options:\MessageBreak
             \space\space
                  \@ptionlist{#1.\@currext},#2\MessageBreak
             to your \noexpand\documentclass declaration may fix this.%
             \MessageBreak
             Try typing \space <return> \space to proceed.}}}%
      {\@pass@ptions\@currext{#2}{#1}%
       \global\expandafter
       \let\csname ver@\@currname.\@currext\endcsname\@empty
       \expandafter\let\csname\@currname.\@currext-h@@k\endcsname\@empty
       \InputIfFileExists
         {\@currname.\@currext}%
         {%
%-----------------------------------------
%    \end{macrocode}
%    When the current extension is \cs{@pkgextension} we are loading a
%    package otherwise, if it is \cs{@clsextension}, a class, so
%    depending on that we execute different hooks.  If the extension is
%    neither, then it is another type of file without special hooks.
%    \begin{macrocode}
           \ifx\@currext\@pkgextension
             \UseHook{package/before}%
             \UseHook{package/before/\@currname}%
           \else
             \ifx\@currext\@clsextension
               \UseHook{class/before}%
               \UseHook{class/before/\@currname}%
             \fi
           \fi
%    \end{macrocode}
%    The value of \cs{CurrentFile} holds during \cs{InputIfFileExists},
%    so the hooks above have that available for using.  However at this
%    point \cs{CurrentFile} is reset to its previous value by
%    \cs{@filehook@file@pop} because it doesn't know that we'll have
%    more hooks ahead.  So here (still in the \meta{true} branch of
%    \cs{InputIfFileExists}, right after actually reading the file in)
%    we'll cheat:  use \cs{@filehook@file@push} once more, so there are
%    two entries for the current file in the name stack, so that when
%    \cs{InputIfFileExists} pops it, there's still one (identical) left.
%    \begin{macrocode}
           \@filehook@file@push
%-----------------------------------------
         }%
         {\@missingfileerror\@currname\@currext}%
    \let\@unprocessedoptions\@@unprocessedoptions
    \csname\@currname.\@currext-h@@k\endcsname
    \expandafter\let\csname\@currname.\@currext-h@@k\endcsname
              \@undefined
%-----------------------------------------
%    \end{macrocode}
%    And same procedure, James, when we are finished loading, except
%    that the hook order is now reversed.
%    \begin{macrocode}
    \ifx\@currext\@pkgextension
      \UseHook{package/after/\@currname}%
      \UseHook{package/after}%
    \else
      \ifx\@currext\@clsextension
        \UseHook{class/after/\@currname}%
        \UseHook{class/after}%
      \fi
    \fi
%    \end{macrocode}
%    Now here we do \cs{@filehook@file@pop} to restore the
%    \cs{CurrentFile} before this file being loaded and fix what we've
%    done in the stack right above.
%    \begin{macrocode}
    \@filehook@file@pop
%-----------------------------------------
    \@unprocessedoptions}%
    \@ifl@ter\@currext{#1}{#3}{}%
      {\@latex@warning@no@line
         {You have requested,\on@line,
          version\MessageBreak
            `#3' of \@cls@pkg\space #1,\MessageBreak
          but only version\MessageBreak
           `\csname ver@#1.\@currext\endcsname'\MessageBreak
          is available}}%
    \ifx\@currext\@clsextension\let\LoadClass\@twoloadclasserror\fi
    \@popfilename
    \@reset@ptions}%
  \reserved@a}
%    \end{macrocode}
%
%    The code for this macro has changed between 2020/02/02 and
%    2020/10/01 so the never version is this:
%    \begin{macrocode}
\@ifl@t@r\fmtversion{2020/10/01}
{%
\def\load@onefilewithoptions#1[#2][#3]#4{%
  \@pushfilename
  \xdef\@currname{#1}%
  \global\let\@currext#4%
  \let\CurrentOption\@empty
  \@reset@ptions
  \makeatletter
  \def\reserved@a{%
    \@ifl@aded\@currext{#1}%
      {\@if@ptions\@currext{#1}{#2}{}%
        {\@latex@error
            {Option clash for \@cls@pkg\space #1}%
            {The package #1 has already been loaded
             with options:\MessageBreak
             \space\space[\@ptionlist{#1.\@currext}]\MessageBreak
             There has now been an attempt to load it
              with options\MessageBreak
             \space\space[#2]\MessageBreak
             Adding the global options:\MessageBreak
             \space\space
                  \@ptionlist{#1.\@currext},#2\MessageBreak
             to your \noexpand\documentclass declaration may fix this.%
             \MessageBreak
             Try typing \space <return> \space to proceed.}}}%
      {\@pass@ptions\@currext{#2}{#1}%
       \global\expandafter
       \let\csname ver@\@currname.\@currext\endcsname\@empty
       \expandafter\let\csname\@currname.\@currext-h@@k\endcsname\@empty
       \InputIfFileExists
         {\@currname.\@currext}%
         {%
%-----------------------------------------
           \ifx\@currext\@pkgextension
             \UseHook{package/before}%
             \UseHook{package/before/\@currname}%
           \else
             \ifx\@currext\@clsextension
               \UseHook{class/before}%
               \UseHook{class/before/\@currname}%
             \fi
           \fi
           \@filehook@file@push
%-----------------------------------------
         }%
         {\@missingfileerror\@currname\@currext}%
    \expandafter\let\csname unprocessedoptions-\@currname.\@currext\endcsname
                    \@@unprocessedoptions
    \csname\@currname.\@currext-h@@k\endcsname
    \expandafter\let\csname\@currname.\@currext-h@@k\endcsname
              \@undefined
    \ifx\@unprocessedoptions\relax
      \let\@unprocessedoptions\@undefined
    \else
      \csname unprocessedoptions-\@currname.\@currext\endcsname
    \fi
    \expandafter\let
        \csname unprocessedoptions-\@currname.\@currext\endcsname
       \@undefined
%-----------------------------------------
    \ifx\@currext\@pkgextension
      \UseHook{package/after/\@currname}%
      \UseHook{package/after}%
    \else
      \ifx\@currext\@clsextension
        \UseHook{class/after/\@currname}%
        \UseHook{class/after}%
      \fi
    \fi
    \@filehook@file@pop
%-----------------------------------------
    }%
    \@ifl@ter\@currext{#1}{#3}{}%
      {\@latex@warning@no@line
         {You have requested,\on@line,
          version\MessageBreak
            `#3' of \@cls@pkg\space #1,\MessageBreak
          but only version\MessageBreak
           `\csname ver@#1.\@currext\endcsname'\MessageBreak
          is available}}%
    \ifx\@currext\@clsextension\let\LoadClass\@twoloadclasserror\fi
    \@popfilename
    \@reset@ptions}%
  \reserved@a}
}{}%
%    \end{macrocode}
%  \end{macro}




%  \begin{macro}{\@include}
%    \begin{macrocode}
\def\@include#1 {%
  \clearpage
  \if@filesw
    \immediate\write\@mainaux{\string\@input{#1.aux}}%
  \fi
  \@tempswatrue
  \if@partsw
    \@tempswafalse
    \edef\reserved@b{#1}%
    \@for\reserved@a:=\@partlist\do
      {\ifx\reserved@a\reserved@b\@tempswatrue\fi}%
  \fi
  \if@tempswa
    \let\@auxout\@partaux
    \if@filesw
      \immediate\openout\@partaux #1.aux
      \immediate\write\@partaux{\relax}%
    \fi
%-----------------------------------------
%    \end{macrocode}
%    First we need to fix \cs{CurrentFile}, because due the pesky
%    space-delimited \cs{@include}, \cs{CurrentFile} contains the
%    \cs{include}'d file with a space.  To fix that, we re-do
%    \cs{set@curr@file}.  The 2020/10/01 release doesn't need this as
%    \cs{include} was changed to do \cs{set@curr@file} on the correct
%    file name, rather than one with a trailing space.
%    \begin{macrocode}
    \set@curr@file{#1}%
%    \end{macrocode}
%    Execute the \texttt{before} hooks just after we switched the
%    \texttt{.aux} file \ldots
%    \begin{macrocode}
    \UseHook{include/before}%
    \UseHook{include/before/#1}%
%-----------------------------------------
    \@input@{#1.tex}%
%    \end{macrocode}
%    \ldots{} then \texttt{end} hooks \ldots
%    \begin{macrocode}
%-----------------------------------------
    \UseHook{include/end/#1}%
    \UseHook{include/end}%
%-----------------------------------------
    \clearpage
%    \end{macrocode}
%    \ldots{} and after the \cs{clearpage} the \texttt{after} hooks
%    followed by another \cs{clearpage} just in case new material got
%    added (after all we need to be in  well defined state after the
%    \cs{include}).
%    \begin{macrocode}
%-----------------------------------------
    \UseHook{include/after/#1}%
    \UseHook{include/after}%
%    \end{macrocode}
%    The additional \cs{clearpage} is needed to ensure that switching
%    the \texttt{.aux} files happen at a defined point even if the
%    above hooks add further material.
%    \begin{macrocode}
    \clearpage
%-----------------------------------------
    \@writeckpt{#1}%
    \if@filesw
      \immediate\closeout\@partaux
    \fi
  \else
    \deadcycles\z@
    \@nameuse{cp@#1}%
  \fi
  \let\@auxout\@mainaux
}
%    \end{macrocode}
%
%    The code for this macro has changed between 2020/02/02 and
%    2020/10/01 so the never version is this:
%    \begin{macrocode}
\@ifl@t@r\fmtversion{2020/10/01}
{%
\def\@include#1 {%
  \clearpage
  \if@filesw
    \immediate\write\@mainaux{\string\@input{"#1.aux"}}%
  \fi
  \@tempswatrue
  \if@partsw
    \@tempswafalse
    \edef\reserved@b{#1}%
    \@for\reserved@a:=\@partlist\do
      {\ifx\reserved@a\reserved@b\@tempswatrue\fi}%
  \fi
  \if@tempswa
    \let\@auxout\@partaux
    \if@filesw
      \immediate\openout\@partaux "#1.aux"
      \immediate\write\@partaux{\relax}%
    \fi
%-----------------------------------------
    \UseHook{include/before}%
    \UseHook{include/before/#1}%
%-----------------------------------------
    \@input@{#1.tex}%
%-----------------------------------------
    \UseHook{include/end/#1}%
    \UseHook{include/end}%
%-----------------------------------------
    \clearpage
%-----------------------------------------
    \UseHook{include/after/#1}%
    \UseHook{include/after}%
    \clearpage
%-----------------------------------------
    \@writeckpt{#1}%
    \if@filesw
      \immediate\closeout\@partaux
    \fi
  \else
    \deadcycles\z@
    \@nameuse{cp@#1}%
  \fi
  \let\@auxout\@mainaux}
}{}
%    \end{macrocode}
%  \end{macro}
%
%
%
% \subsection{High-level interfaces for \LaTeX{}}
%
%    None so far and the general feeling for now is that the hooks are
%    enough. Packages like \pkg{filehook}, etc., may use them to set
%    up their interfaces (samples are given below) but for the now the
%    kernel will not provide any.
%
%
%
%    \begin{macrocode}
%</2ekernel>
%    \end{macrocode}
%
%
%
% \section{Package emulation for compatibility}
%
%
% \subsection{Package \pkg{filehook} emulation}
%
%    This is a partial implementation of the  \pkg{filehook}
%    interfaces. It is only meant for guidance in case that package
%    gets updated to use the hook management.
%
%    Not implemented are:
%\begin{verbatim}
%  \AtBeginOfFiles
%  \AtEndOfFiles
%  \AtBeginOfInputs
%  \AtEndOfInputs
%  \AtBeginOfInputFile
%  \AtEndOfInputFile
%\end{verbatim}
%
%    \begin{macrocode}
%<*filehook-draft>
%    \end{macrocode}
%
%    \begin{macrocode}
\newcommand\AtBeginOfEveryFile [1]
  {\AddToHook{file/before}{#1}}
\newcommand\AtEndOfEveryFile [1]
  {\AddToHook{file/after}{#1}}
%    \end{macrocode}
%    
%    \begin{macrocode}
\newcommand\AtBeginOfIncludes [1]
  {\AddToHook{include/before}{#1}}
\newcommand\AtEndOfIncludes [1]
  {\AddToHook{include/end}{#1}}
\newcommand\AfterIncludes [1]
  {\AddToHook{include/after}{#1}}
%    \end{macrocode}
%    
%    \begin{macrocode}
\newcommand\AtBeginOfPackages [1]
  {\AddToHook{package/before}{#1}}
\newcommand\AtEndOfPackages [1]
  {\AddToHook{package/after}{#1}}
%    \end{macrocode}
%    
%    \begin{macrocode}
\newcommand\AtBeginOfClasses [1]
  {\AddToHook{class/before}{#1}}
\newcommand\AtEndOfClasses [1]
  {\AddToHook{class/after}{#1}}
%    \end{macrocode}
%
%    For normal files we drop the \texttt{.tex} extension for now:
%    \begin{macrocode}
\newcommand\AtBeginOfFile [2]
  {\AddToHook{file/before/#1}{#2}}
\newcommand\AtEndOfFile [2]
  {\AddToHook{file/after/#1}{#2}}
%    \end{macrocode}
%    
%    \begin{macrocode}
\DeclareDocumentCommand \AtBeginOfPackageFile {smm}
   {\IfBooleanTF{#1}%
     {\@ifpackageloaded{#2}%
         {#3}%
         {\AddToHook{package/before/#2}{#3}}}%
     {\AddToHook{package/before/#2}{#3}}%
   }
\DeclareDocumentCommand \AtEndOfPackageFile {smm}
   {\IfBooleanTF{#1}%
     {\@ifpackageloaded{#2}%
         {#3}%
         {\AddToHook{package/after/#2}{#3}}}%
     {\AddToHook{package/after/#2}{#3}}%
   }
%    \end{macrocode}
%
%    Are the * forms here of any use? I know they are use 3--4 times
%    on CTAN but I wonder if those are real or mistaken usages.  
%    \begin{macrocode}
\DeclareDocumentCommand \AtBeginOfClassFile {smm}
   {\IfBooleanTF{#1}%
     {\@ifclassloaded{#2}%
         {#3}%
         {\AddToHook{class/before/#2}{#3}}}%
     {\AddToHook{class/before/#2}{#3}}%
   }
\DeclareDocumentCommand \AtEndOfClassFile {smm}
   {\IfBooleanTF{#1}%
     {\@ifclassloaded{#2}%
         {#3}%
         {\AddToHook{class/after/#2}{#3}}}%
     {\AddToHook{class/after/#2}{#3}}%
   }
%    \end{macrocode}
%    
%    \begin{macrocode}
\newcommand\AtBeginOfIncludeFile [2]
  {\AddToHook{include/before/#1}{#2}}
\newcommand\AtEndOfIncludeFile [2]
  {\AddToHook{include/end/#1}{#2}}
\newcommand\AfterIncludeFile [2]
  {\AddToHook{include/after/#1}{#2}}
%    \end{macrocode}
%
%
%    This is missing some interfaces so disabling the package isn't
%    really correct, but then this code above is not supposed to stay
%    like this anyway.
%    \begin{macrocode}
\expandafter\let\csname ver@filehook.sty\endcsname\fmtversion
\@namedef {ver@filehook.sty}{2020/10/01}
%    \end{macrocode}
%
%
%    \begin{macrocode}
%</filehook-draft>
%    \end{macrocode}
%
%
%
%
% \subsection{Package \pkg{scrlfile} emulation}
%
%    This is a partial implementation of the  \pkg{scrlfile}
%    interfaces. It is only meant for guidance in case that package
%    gets updated to use the hook management.
%
%    \begin{macrocode}
%<*scrlfile-draft>
%    \end{macrocode}
%
%    I think this is roughly correct (using the \texttt{file/...}
%    hooks rather than the class or package hooks at least for the
%    \cs{After...} commands but it needs some further verification.
%
%    The star and plus variants haven't been implemented so far, this
%    is only a rough draft.
%    \begin{macrocode}
\newcommand\BeforeClass[2]
  {\AddToHook{file/before/#1.cls}{#2}}
\newcommand\AfterClass [2]
  {\AddToHook{file/after/#1.cls}{#2}}
\newcommand\AfterAtEndOfClass [2]
  {\AddToHook{class/after/#1}{#2}}
%    \end{macrocode}
%
%    \begin{macrocode}
\newcommand\BeforePackage [2]
  {\AddToHook{package/before/#1.sty}{#2}}
\newcommand\AfterPackage [2]
  {\AddToHook{file/after/#1.sty}{#2}}
\newcommand\AfterEndOfPackage [2]
  {\AddToHook{package/after/#1}{#2}}
%    \end{macrocode}
%
%    \begin{macrocode}
\newcommand\BeforeFile [2]
  {%
  \typeout{BeforeFile: #1!!!}%
   \AddToHook{file/before/#1}{#2}}
\newcommand\AfterFile [2]
  {%
    \typeout{AfterFile: #1!!!}%
    \AddToHook{file/after/#1}{#2}}
%    \end{macrocode}
%
%    This is missing some interfaces so disabling the package isn't
%    really correct, but then this code above is not supposed to stay
%    like this anyway.
%    \begin{macrocode}
\expandafter\let\csname ver@scrlfile.sty\endcsname\fmtversion
\@namedef {ver@scrlfile.sty}{2020/10/01}
%    \end{macrocode}
%
%    \begin{macrocode}
%</scrlfile-draft>
%    \end{macrocode}
%
%
%
% \section{A sample package for structuring the log output}
%
%    \begin{macrocode}
%<*structuredlog>
%<@@=filehook>
%    \end{macrocode}
%
%    \begin{macrocode}
\ProvidesExplPackage
    {structuredlog}{\ltfilehookdate}{\ltfilehookversion}
    {Structuring the TeX transcript file}
%    \end{macrocode}



%    \begin{macrocode}
\int_new:N \g_@@_nesting_level_int
%    \end{macrocode}

%    \begin{macrocode}
\tl_new:N   \g_@@_nesting_prefix_tl
\tl_gset:Nn \g_@@_nesting_prefix_tl { }
%    \end{macrocode}
%    
%    \begin{macrocode}
\AddToHook{file/before}{
  \int_gincr:N \g_@@_nesting_level_int
  \tl_gput_right:Nn\g_@@_nesting_prefix_tl {=}
  \iow_term:x {
    \g_@@_nesting_prefix_tl \space
    ( LEVEL~ \int_use:N \g_@@_nesting_level_int \space START )~
    \CurrentFile  ^^J
  }
}
%    \end{macrocode}
%    We don't want to install the \hook{file/after} hook immediately,
%    because that would mean it is the first time executed when the
%    package finishes. We therefore put the declaration inside
%    \cs{AddToHookNext} so that it gets only installed when we have
%    left the package. 
%    \begin{macrocode}
\AddToHookNext{file/after}{
  \AddToHook{file/after}{
    \iow_term:x {
      \g_@@_nesting_prefix_tl \space
      ( LEVEL~ \int_use:N \g_@@_nesting_level_int \space STOP )~
      \CurrentFile  ^^J
    }
    \int_gdecr:N \g_@@_nesting_level_int
    \tl_gset:Nx \g_@@_nesting_prefix_tl
       {\exp_after:wN \use_none:n \g_@@_nesting_prefix_tl}
  }
}
%    \end{macrocode}
%
%    We have to manually increment the level because now that we have
%    installed the code in \texttt{file/after} it gets decremented
%    when we leave the package without ever being incremented upon
%    entry.
%    \begin{macrocode}
%\int_incr:N\g_@@_nesting_level_int
%    \end{macrocode}
%
%    \begin{macrocode}
%</structuredlog>
%    \end{macrocode}

%    \Finale
%


%%%%%%%%%%%%%%%%%%%%%%%%%%%%%%%%%%%%%%%%%%%%%  
\endinput
%%%%%%%%%%%%%%%%%%%%%%%%%%%%%%%%%%%%%%%%%%%%%  



                    }
%    \end{macrocode}
%
%  \subsection{Package options}
%
%    For now we offer a simple debug option which turns on a lot of
%    strange \cs{typeout} messages, nothing fancy.
%    \begin{macrocode}
\ExplSyntaxOn
%    \end{macrocode}
%    
%    \begin{macrocode}
\hook_debug_off:
\DeclareOption { debug } { \hook_debug_on:
                           \shipout_debug_on: }
%    \end{macrocode}
%
%
%    For now we offer a simple debug option which turns on a lot of
%    strange \cs{typeout} messages, nothing fancy.
%    \begin{macrocode}
\shipout_debug_off:
\DeclareOption { debug-shipout } { \shipout_debug_on: }
%    \end{macrocode}
%    
%    \begin{macrocode}
\ProcessOptions
%    \end{macrocode}
%
%
%  \subsection{Temporarily patching package until changed}
%
%
%
%    \pkg{filehook} support until that package is patched:
%    \begin{macrocode}
\RequirePackage{filehook-ltx}
%    \end{macrocode}
%
%
%
%    \begin{macrocode}
%</package>
%    \end{macrocode}
%
%
%
% \Finale
%


